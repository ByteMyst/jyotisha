% !Tex program = xelatex
\documentclass[12pt]{article}
\usepackage[dvipsnames]{xcolor} 
\usepackage[paperwidth=135mm,paperheight=180mm,left=4mm,right=4mm,top=8mm,bottom=8mm]{geometry}
\usepackage[xetex]{graphicx}
\usepackage{array}
\usepackage{setspace}
\usepackage{multirow}
\usepackage{pxfonts}
\usepackage{bbding}
\usepackage{wasysym} 
\usepackage{fontspec}
\usepackage{multicol}
\usepackage{supertabular}
\usepackage{fancyhdr}
\pagestyle{fancy}
\fancyhf{}
\rhead{}
\lhead{}
\cfoot{}
\usepackage{../templates/listofitems}
\newcommand{\yearname}{2015}
%%%%%%%%%%%%%%%%%%%%%%%%%%%%%%%%%%%%%%%%%%%%%%%%%%%%%%%%%%%%%%%%%%%%%%%%%%%%%%%
%% MOONPHASE CODE 
%%%%%%%%%%%%%%%%%%%%%%%%%%%%%%%%%%%%%%%%%%%%%%%%%%%%%%%%%%%%%%%%%%%%%%%%%%%%%%%
%Credits: http://tex.stackexchange.com/questions/34785/how-to-typeset-moon-phase-symbols (Jake!)
\usepackage{tikz}
\usetikzlibrary{calendar,fpu}

\tikzset{
    moon colour/.style={
        moon fill/.style={
            fill=#1
        }
    },
    sky colour/.style={
        sky draw/.style={
            draw=#1
        },
        sky fill/.style={
            fill=#1
        }
    },
    southern hemisphere/.style={
        rotate=180
    }
}

\makeatletter
\pgfcalendardatetojulian{2010-01-15}{\c@pgf@counta} % 2010-01-15 07:11 UTC -- http://aa.usno.navy.mil/cgi-bin/aa_moonphases.pl?year=2010&ZZZ=END
\def\synodicmonth{29.530588853}
\newcommand{\moon}[2][]{%
    \edef\checkfordate{\noexpand\in@{-}{#2}}%
    \checkfordate%
    \ifin@%
        \pgfcalendardatetojulian{#2}{\c@pgf@countb}%
        \pgfkeys{/pgf/fpu=true,/pgf/fpu/output format=fixed}%
        \pgfmathsetmacro\dayssincenewmoon{\the\c@pgf@countb-\the\c@pgf@counta-(7/24+11/(24*60))}%
        \pgfmathsetmacro\lunarage{mod(\dayssincenewmoon,\synodicmonth)}
        \pgfkeys{/pgf/fpu=false}%%
    \else%
        \def\lunarage{#2}%
    \fi%
    \pgfmathsetmacro\leftside{ifthenelse(\lunarage<=\synodicmonth/2,cos(360*(\lunarage/\synodicmonth)),1)}%
    \pgfmathsetmacro\rightside{ifthenelse(\lunarage<=\synodicmonth/2,-1,-cos(360*(\lunarage/\synodicmonth))}%
    \tikz [moon colour=white,sky colour=black,#1]{
        \draw [moon fill, sky draw] (0,0) circle [radius=1ex];
        \draw [sky draw, sky fill] (0,1ex)
            arc (90:-90:\rightside ex and 1ex)
            arc (-90:90:\leftside ex and 1ex)
            -- cycle;
    }%
}
%%%%%%%%%%%%%%%%%%%%%%%%%%%%%%%%%%%%%%%%%%%%%%%%%%%%%%%%%%%%%%%%%%%%%%%%%%%%%%%
%% END MOONPHASE CODE
%%%%%%%%%%%%%%%%%%%%%%%%%%%%%%%%%%%%%%%%%%%%%%%%%%%%%%%%%%%%%%%%%%%%%%%%%%%%%%%
% \setlength{\footskip}{2mm}
% PDF SETUP
% ---- FILL IN HERE THE DOC TITLE AND AUTHOR
\defaultfontfeatures{Scale=MatchLowercase,Mapping=tex-text}
\setmainfont{siddhanta.ttf}[Path=../fonts/,Script=Devanagari] 
\setsansfont[Path=../fonts/,Scale=0.95,Numbers=Lining]{AlegreyaSans-Regular.ttf}
% \newfontfamily\noto[Path=../fonts/, Ligatures=TeX]{NotoSansUI-Regular}
%%%%%%% Numbers and counters %%%%%%%
\newcount\num
\newcount\tempone \newcount\temptwo
\newcommand{\devanumber}[1]{%
\num=#1\devanumberrecurse}
\setlength{\textheight}{150mm}
\newcommand{\devanumberrecurse}{%
{\tempone=\num
%  \showthe\tempone\ %
\ifnum\num > 0 
    \divide \num by 10%
    \temptwo=\num \multiply\temptwo by -10%
    \devanumberrecurse%
%   \\stage 2\ %
%   \showthe\temptwo\ %
%   temp1=\number\tempone\ %
%   num=\number\num\ %
    \advance\tempone by \temptwo%
    \devadigit
\fi
}}
\newcommand{\devadigit}{%
\ifcase\tempone०\or१\or२\or३\or४\or५\or६\or७\or८\or९\fi%\number\tempone%
}
\newcommand{\eventsep}{~\raisebox{1pt}{\scriptsize$\Diamondblack$} }
\newcommand{\TO}{\hspace{1pt}\raisebox{1pt}{\footnotesize\RIGHTarrow}\hspace{1pt}}
\newcommand{\To}{\hspace{1pt}\raisebox{1pt}{\footnotesize\RIGHTarrow}\hspace{1pt}}
\newcommand{\Too}{\hspace{1pt}\raisebox{1pt}{\footnotesize\RIGHTarrow\hspace{-5pt}\RIGHTarrow}\hspace{1pt}}
%%%%%%%% Calendar display stuff %%%%%%%%%%%
\newcommand{\samvatsaraName}{}
\newcommand{\solarMonthName}{}
\newcommand{\solarMonthEndTime}{}
\newcommand{\lunarMonthName}{}
\newcommand{\lunarRtu}{}
\newcommand{\solarMonthDate}{}
\newcommand{\vaaraName}{}
\newcommand{\rtuName}{}
\newcommand{\ayanamName}{}

\newcommand{\sunmonth}[9]{%
\renewcommand{\solarMonthName}{#1}
\renewcommand{\solarMonthDate}{#2}
\renewcommand{\solarMonthEndTime}{#3}
\renewcommand{\lunarMonthName}{#4}
\renewcommand{\lunarRtu}{#5}
\renewcommand{\vaaraName}{#6}
\renewcommand{\samvatsaraName}{#7}
\renewcommand{\ayanamName}{#8}
\renewcommand{\rtuName}{#9}
}
\newcommand{\tamil}[1]{%
{\fontspec[Scale=0.8,FakeStretch=0.9,Path=../fonts/]{NotoSansTamil-Regular.ttf} \footnotesize #1}%
}
\newcommand{\kalas}[1]{%
\setsepchar{ }
\readlist\arg{#1}
{\small{\mbox{ब्राह्म\,\textsf{\arg[1]}{\scriptsize\RIGHTarrow}\,सङ्गव\,\textsf{\arg[4]}{\scriptsize\RIGHTarrow}\,मध्याह्न\,\textsf{\arg[7]}{\scriptsize\RIGHTarrow}\,अपराह्ण\,\textsf{\arg[8]}{\scriptsize\RIGHTarrow}\,सायाह्न\,\textsf{\arg[9]}{\scriptsize\RIGHTarrow}}\hfill {दिनान्तः{\scriptsize\RIGHTarrow}\textsf{\arg[14]}}}}\\[-1ex]
{\small{\mbox{प्रातः सन्ध्या \textsf{\arg[2]}{\scriptsize\RIGHTarrow}\textsf{\arg[3]} माध्याह्निक \textsf{\arg[5]}{\scriptsize\RIGHTarrow}\textsf{\arg[6]} 
सायं \textsf{\arg[10]}{\scriptsize\RIGHTarrow}\textsf{\arg[11]}}\hfill शयन \textsf{\arg[12]}{\scriptsize\RIGHTarrow}\textsf{\arg[13]}}\\[-4.5ex]}
}
\newcommand{\sunmoonrsdata}[5]{%
\mbox{\large\sun{\small\UParrow}\textsf{#1}~\sun{\small\DOWNarrow}\textsf{#2}}\hfill
\mbox{\large\rightmoon{\small\UParrow}\textsf{#3}~\rightmoon{\small\DOWNarrow}\textsf{#4}}\\
#5
 }
\newcommand{\sunmoonsrdata}[5]{%
\mbox{\large\sun{\small\UParrow}\textsf{#1}~\sun{\small\DOWNarrow}\textsf{#2}}\hfill
\mbox{\large\rightmoon{\small\DOWNarrow}\textsf{#4}~\rightmoon{\small\UParrow}\textsf{#3}}\\
#5
 }
\newcommand{\ahorAtram}{अहोरात्रम्}
\newcommand{\tithi}[2]{\raisebox{-1pt}{\moon[scale=0.8]{#1}}\hspace{2pt}#2}
\newcommand{\tnykdata}[6]{\large%\fontsize{13pt}{16pt}\selectfont
{#1}\\%Tithi
{नक्षत्रम्–#2 (#3)}\\%Nakshatram and Rashi
{\setstretch{0.55}
\begin{tabular}{@{}r@{}p{108mm}@{}}
योगः–&#4\\[2pt]%Yogam
करणम्–&#5\\%Karanam
\end{tabular}}\mbox{}\\[3pt]
\parbox[c][2ex][c]{0.9\linewidth}{\footnotesize #6}%Lagna, if required
}
\newcommand{\avamA}{
    \raisebox{1.5pt}{\fcolorbox{white}{gray!40}{\scriptsize अवमा}}
}
\newcommand{\tridina}{
    \raisebox{1.5pt}{\fcolorbox{white}{gray!40}{\scriptsize त्रिदिनस्पृक्}}
}
\renewcommand{\time}[2]{#1 (#2)}
\newcommand{\anga}[2]{\mbox{#1\To{}\textsf{#2}}}
\newcommand{\fullanga}[1]{\mbox{#1\To{}\ahorAtram}}
\newcommand{\fulltithi}[1]{\mbox{#1\To{}\ahorAtram\tridina}}
\newcommand{\lagna}[2]{\mbox{#1\RIGHTarrow\textsf{#2}}}
\newcommand{\uanga}[1]{\mbox{#1\Too}}
\newcommand{\rygdata}[3]{%
\begin{minipage}{\linewidth}
\centering
\rule[-1ex]{0.7\textwidth}{.4pt}
\small राहु॰~\textsf{#1}~~यम॰~\textsf{#2}~~गुलिक॰~\textsf{#3}%Rahu Yama Gulika
\end{minipage}
}
\newcommand{\caldata}[7]{%
\clearpage
\begin{minipage}{\linewidth}
#3% Calls \sunmonth
\large% Fixes font size
{\centering\begin{tabular}{c|c}
\large \textsf{\yearname} & {\large\samvatsaraName}\\[-1ex]%YYYY
& {\footnotesize \ayanamName \hspace{6pt} \rtuName}\\[0.2ex]
\mbox{\sffamily\fontsize{20}{25}\selectfont {\uppercase{#1}}} & \parbox[c][14pt][c]{3cm}{\centering\LARGE\solarMonthName}\\[-4pt]%mmm
& {\mbox{\small \solarMonthEndTime}}\\[-2pt]
& {\parbox[c][15pt]{52mm}{\centering\lunarMonthName}}\\[-6pt]
& {\parbox[c][10pt]{52mm}{\centering\scriptsize(\lunarRtu)}}\\[-5pt]
\hspace{0.465\linewidth} & \hspace{0.465\linewidth} \\[-6pt]
\mbox{\sffamily\fontsize{96}{115}\selectfont #2} & \mbox{\fontsize{90}{24}\selectfont \devanumber{\solarMonthDate}}\\[1.6ex]%DD
\mbox{\sffamily\fontsize{24}{28}\selectfont\uppercase{#7}} & \parbox[c][24pt][t]{1cm}{\centering\LARGE\vaaraName}\\[1.2ex]%Day of the week
\hline
\end{tabular}
}\mbox{}\\[-4pt]
#4\\[0.5em]%Sun rise, kalas etc
#5\mbox{}\\[1em]%Tithi, Nakshatram, Varam, Yogam
% \vspace{\fill}
{\parbox[b]{0.95\linewidth}{\centering\normalsize\textcolor{RoyalBlue}{#6}}}%Festivals
\end{minipage}
}

\addtolength{\headsep}{-3ex}
\setlength\parindent{0pt}
\pagestyle{empty}
\begin{document}
\mbox{}
\renewcommand{\yearname}{2019}
\begin{center}
{\sffamily \fontsize{80}{80}\selectfont  2019\\[0.5cm]}
\mbox{\fontsize{48}{48}\selectfont विलम्बः–विकारी}\\
\mbox{\fontsize{32}{32}\selectfont कलि } %
{\sffamily \fontsize{43}{43}\selectfont  5119–5120\\[0.5cm]}
\hrule
\vspace{0.2cm}
{\sffamily \fontsize{50}{50}\selectfont  \uppercase{Chennai}\\[0.2cm]}
{\sffamily \fontsize{23}{23}\selectfont  {13.090°N, 80.270°E}\\[0.2cm]}
\hrule
\end{center}
\clearpage\pagestyle{fancy}
\caldata{JANUARY}{1}{\sunmonth{धनुः}{17}{}{मार्गशीर्षः}{हेमन्तऋतुः}{मङ्गलः}{विलम्बः}{दक्षिणायनम्}{हेमन्तऋतुः}}
{\sunmoonsrdata{06:34}{17:49}{03:08(+1)}{14:08}{12:12}
{\kalas{04:52 05:43 09:34 08:49 10:19 16:19 11:04 13:19 15:34 17:04 18:40 21:01 22:36 01:48(+1)}}}
{\tnykdata{\anga{\tithi{26}{कृष्ण-एकादशी}}{\time{47-14}{01:28(+1)}}\hspace{1ex}}%
{\anga{स्वाती}{\time{5-17}{08:42}}\hspace{1ex}}{चन्द्रराशिः—\mbox{तुला\RIGHTarrow\textsf{03:20(+1)}}}%
{\anga{धृतिः}{\time{50-6}{02:37(+1)}}\hspace{1ex}\uanga{शूलः}}%
{\anga{बवः}{\time{16-49}{13:18}}\hspace{1ex}\anga{बालवः}{\time{47-14}{01:28(+1)}}\hspace{1ex}\uanga{कौलवः}}{}
}
{सर्व-सफला-एकादशी}
{Tue} 
\cfoot{\rygdata{15:01--16:25}{09:23--10:48}{12:12--13:36}}
\caldata{JANUARY}{2}{\sunmonth{धनुः}{18}{}{मार्गशीर्षः}{हेमन्तऋतुः}{बुधः}{विलम्बः}{दक्षिणायनम्}{हेमन्तऋतुः}}
{\sunmoonsrdata{06:35}{17:50}{04:02(+1)}{14:54}{12:12}
{\kalas{04:53 05:44 09:35 08:50 10:20 16:20 11:05 13:20 15:35 17:05 18:41 21:01 22:37 01:48(+1)}}}
{\tnykdata{\anga{\tithi{27}{कृष्ण-द्वादशी}}{\time{48-59}{02:10(+1)}}\hspace{1ex}}%
{\anga{विशाखा}{\time{7-35}{09:37}}\hspace{1ex}}{चन्द्रराशिः—\mbox{वृश्चिकः}}%
{\anga{शूलः}{\time{48-30}{01:59(+1)}}\hspace{1ex}\uanga{गण्डः}}%
{\anga{कौलवः}{\time{17-57}{13:46}}\hspace{1ex}\anga{तैतिलः}{\time{48-59}{02:10(+1)}}\hspace{1ex}\uanga{गरः}}{}
}
{बुधानुराधा-पुण्यकालः\eventsep हरिवासरः{\RIGHTarrow}\textsf{07:36}\eventsep काञ्ची ६८ जगद्गुरु श्री-चन्द्रशेखरेन्द्र सरस्वती ७ आराधना~\#{२५}\eventsep पक्षवर्धिनी-महाद्वादशी\eventsep \tamil{உந்து~மதக்களிற்றன்}}
{Wed} 
\cfoot{\rygdata{12:12--13:37}{07:59--09:24}{10:48--12:12}}
\caldata{JANUARY}{3}{\sunmonth{धनुः}{19}{}{मार्गशीर्षः}{हेमन्तऋतुः}{गुरुः}{विलम्बः}{दक्षिणायनम्}{हेमन्तऋतुः}}
{\sunmoonsrdata{06:35}{17:50}{04:55(+1)}{15:43}{12:13}
{\kalas{04:53 05:44 09:35 08:50 10:20 16:20 11:05 13:20 15:35 17:05 18:41 21:02 22:37 01:49(+1)}}}
{\tnykdata{\anga{\tithi{28}{कृष्ण-त्रयोदशी}}{\time{51-54}{03:21(+1)}}\hspace{1ex}}%
{\anga{अनूराधा}{\time{11-3}{11:01}}\hspace{1ex}}{चन्द्रराशिः—\mbox{वृश्चिकः}}%
{\anga{गण्डः}{\time{47-48}{01:43(+1)}}\hspace{1ex}\uanga{वृद्धिः}}%
{\anga{गरः}{\time{20-17}{14:42}}\hspace{1ex}\anga{वणिजः}{\time{51-54}{03:21(+1)}}\hspace{1ex}\uanga{विष्टिः}}{}
}
{प्रदोष-व्रतम्}
{Thu} 
\cfoot{\rygdata{13:37--15:02}{06:35--08:00}{09:24--10:48}}
\caldata{JANUARY}{4}{\sunmonth{धनुः}{20}{}{मार्गशीर्षः}{हेमन्तऋतुः}{शुक्रः}{विलम्बः}{दक्षिणायनम्}{हेमन्तऋतुः}}
{\sunmoonsrdata{06:36}{17:51}{05:46(+1)}{16:33}{12:13}
{\kalas{04:54 05:45 09:36 08:51 10:21 16:21 11:06 13:21 15:36 17:06 18:42 21:02 22:38 01:49(+1)}}}
{\tnykdata{\anga{\tithi{29}{कृष्ण-चतुर्दशी}}{\time{55-54}{04:57(+1)}}\hspace{1ex}}%
{\anga{ज्येष्ठा}{\time{15-38}{12:51}}\hspace{1ex}}{चन्द्रराशिः—\mbox{वृश्चिकः\RIGHTarrow\textsf{12:51}}}%
{\anga{वृद्धिः}{\time{47-56}{01:46(+1)}}\hspace{1ex}\uanga{ध्रुवः}}%
{\anga{विष्टिः}{\time{23-46}{16:06}}\hspace{1ex}\anga{शकुनिः}{\time{55-54}{04:57(+1)}}\hspace{1ex}\uanga{चतुष्पात्}}{}
}
{मासशिवरात्रिः\eventsep \tamil{தொண்டரடிப்பொடியாழ்வார் திருநக்ஷத்திரம்}}
{Fri} 
\cfoot{\rygdata{10:49--12:13}{15:02--16:26}{08:00--09:24}}
\caldata{JANUARY}{5}{\sunmonth{धनुः}{21}{}{मार्गशीर्षः}{हेमन्तऋतुः}{शनिः}{विलम्बः}{दक्षिणायनम्}{हेमन्तऋतुः}}
{\sunmoonsrdata{06:36}{17:52}{06:36(+1)}{17:24}{12:14}
{\kalas{04:54 05:45 09:36 08:51 10:21 16:21 11:06 13:21 15:36 17:07 18:43 21:03 22:38 01:49(+1)}}}
{\tnykdata{\fulltithi{\tithi{30}{अमावास्या}}}%
{\anga{मूला}{\time{21-12}{15:05}}\hspace{1ex}}{चन्द्रराशिः—\mbox{धनुः}}%
{\anga{ध्रुवः}{\time{48-49}{02:08(+1)}}\hspace{1ex}\uanga{व्याघातः}}%
{\anga{चतुष्पात्}{\time{28-17}{17:55}}\hspace{1ex}\uanga{नाग}}{}
}
{काञ्ची १४ जगद्गुरु श्री-विद्याघनेन्द्र सरस्वती आराधना~\#{१७०२}\eventsep काञ्ची ३४ जगद्गुरु श्री-चन्द्रशेखरेन्द्र सरस्वती २ आराधना~\#{१३०९}\eventsep मार्गशीर्ष-अमावास्या\eventsep पार्वणव्रतम् अमावास्यायाम्\eventsep श्री-हनूमत्-जयन्ती}
{Sat} 
\cfoot{\rygdata{09:25--10:49}{13:38--15:03}{06:36--08:00}}
\caldata{JANUARY}{6}{\sunmonth{धनुः}{22}{}{मार्गशीर्षः}{हेमन्तऋतुः}{भानुः}{विलम्बः}{दक्षिणायनम्}{हेमन्तऋतुः}}
{\sunmoonsrdata{06:36}{17:52}{---}{18:15}{12:14}
{\kalas{04:54 05:45 09:36 08:51 10:21 16:22 11:07 13:22 15:37 17:07 18:43 21:03 22:39 01:50(+1)}}}
{\tnykdata{\anga{\tithi{30}{अमावास्या}}{\time{0-53}{06:58}}\hspace{1ex}}%
{\anga{पूर्वाषाढा}{\time{27-40}{17:40}}\hspace{1ex}}{चन्द्रराशिः—\mbox{धनुः\RIGHTarrow\textsf{00:22(+1)}}}%
{\anga{व्याघातः}{\time{50-20}{02:44(+1)}}\hspace{1ex}\uanga{हर्षणः}}%
{\anga{नाग}{\time{0-53}{06:58}}\hspace{1ex}\anga{किंस्तुघ्नः}{\time{33-43}{20:06}}\hspace{1ex}\uanga{बवः}}{}
}
{\tamil{சாக்கிய நாயனார் (33) குருபூஜை}\eventsep दर्शेष्टिः\eventsep स्थालीपाकः}
{Sun} 
\cfoot{\rygdata{16:28--17:52}{12:14--13:39}{15:03--16:28}}
\caldata{JANUARY}{7}{\sunmonth{धनुः}{23}{}{पौषः}{हेमन्तऋतुः}{सोमः}{विलम्बः}{दक्षिणायनम्}{हेमन्तऋतुः}}
{\sunmoonrsdata{06:37}{17:53}{07:22}{19:05}{12:15}
{\kalas{04:55 05:46 09:37 08:52 10:22 16:23 11:07 13:22 15:37 17:08 18:44 21:04 22:39 01:50(+1)}}}
{\tnykdata{\anga{\tithi{1}{शुक्ल-प्रथमा}}{\time{6-44}{09:18}}\hspace{1ex}}%
{\anga{उत्तराषाढा}{\time{34-51}{20:33}}\hspace{1ex}}{चन्द्रराशिः—\mbox{मकरः}}%
{\anga{हर्षणः}{\time{52-22}{03:33(+1)}}\hspace{1ex}\uanga{वज्रम्}}%
{\anga{बवः}{\time{6-44}{09:18}}\hspace{1ex}\anga{बालवः}{\time{39-55}{22:35}}\hspace{1ex}\uanga{कौलवः}}{}
}
{चन्द्र-दर्शनम्}
{Mon} 
\cfoot{\rygdata{08:01--09:26}{10:50--12:15}{13:39--15:04}}
\caldata{JANUARY}{8}{\sunmonth{धनुः}{24}{}{पौषः}{हेमन्तऋतुः}{मङ्गलः}{विलम्बः}{दक्षिणायनम्}{हेमन्तऋतुः}}
{\sunmoonrsdata{06:37}{17:53}{08:05}{19:54}{12:15}
{\kalas{04:55 05:46 09:37 08:52 10:22 16:23 11:07 13:23 15:38 17:08 18:44 21:04 22:40 01:51(+1)}}}
{\tnykdata{\anga{\tithi{2}{शुक्ल-द्वितीया}}{\time{13-12}{11:54}}\hspace{1ex}}%
{\anga{श्रवणः}{\time{42-32}{23:38}}\hspace{1ex}}{चन्द्रराशिः—\mbox{मकरः}}%
{\anga{वज्रम्}{\time{54-44}{04:31(+1)}}\hspace{1ex}\uanga{सिद्धिः}}%
{\anga{कौलवः}{\time{13-12}{11:54}}\hspace{1ex}\anga{तैतिलः}{\time{46-36}{01:15(+1)}}\hspace{1ex}\uanga{गरः}}{}
}
{श्रवण-व्रतम्}
{Tue} 
\cfoot{\rygdata{15:04--16:29}{09:26--10:50}{12:15--13:40}}
\caldata{JANUARY}{9}{\sunmonth{धनुः}{25}{}{पौषः}{हेमन्तऋतुः}{बुधः}{विलम्बः}{दक्षिणायनम्}{हेमन्तऋतुः}}
{\sunmoonrsdata{06:37}{17:54}{08:46}{20:41}{12:15}
{\kalas{04:55 05:46 09:38 08:52 10:23 16:24 11:08 13:23 15:39 17:09 18:45 21:05 22:40 01:51(+1)}}}
{\tnykdata{\anga{\tithi{3}{शुक्ल-तृतीया}}{\time{20-2}{14:38}}\hspace{1ex}}%
{\anga{श्रविष्ठा}{\time{50-25}{02:47(+1)}}\hspace{1ex}}{चन्द्रराशिः—\mbox{मकरः\RIGHTarrow\textsf{13:13}}}%
{\anga{सिद्धिः}{\time{57-12}{05:30(+1)}}\hspace{1ex}\uanga{व्यतीपातः}}%
{\anga{गरः}{\time{20-2}{14:38}}\hspace{1ex}\anga{वणिजः}{\time{53-28}{04:01(+1)}}\hspace{1ex}\uanga{विष्टिः}}{}
}
{}
{Wed} 
\cfoot{\rygdata{12:15--13:40}{08:02--09:26}{10:51--12:15}}
\caldata{JANUARY}{10}{\sunmonth{धनुः}{26}{}{पौषः}{हेमन्तऋतुः}{गुरुः}{विलम्बः}{दक्षिणायनम्}{हेमन्तऋतुः}}
{\sunmoonrsdata{06:37}{17:54}{09:24}{21:28}{12:16}
{\kalas{04:56 05:46 09:38 08:53 10:23 16:24 11:08 13:24 15:39 17:09 18:45 21:05 22:41 01:51(+1)}}}
{\tnykdata{\anga{\tithi{4}{शुक्ल-चतुर्थी}}{\time{26-51}{17:22}}\hspace{1ex}}%
{\anga{शतभिषक्}{\time{58-5}{05:52(+1)}}\hspace{1ex}}{चन्द्रराशिः—\mbox{कुम्भः}}%
{\anga{व्यतीपातः}{\time{59-30}{06:25(+1)}}\hspace{1ex}\uanga{वरीयान्}}%
{\anga{विष्टिः}{\time{26-51}{17:22}}\hspace{1ex}\uanga{बवः}}{}
}
{महाधनुर्व्यतीपात-स्नानम्\eventsep महाधनुर्व्यतीपात-श्राद्धम्}
{Thu} 
\cfoot{\rygdata{13:40--15:05}{06:37--08:02}{09:27--10:51}}
\caldata{JANUARY}{11}{\sunmonth{धनुः}{27}{}{पौषः}{हेमन्तऋतुः}{शुक्रः}{विलम्बः}{दक्षिणायनम्}{हेमन्तऋतुः}}
{\sunmoonrsdata{06:38}{17:55}{10:01}{22:13}{12:16}
{\kalas{04:56 05:47 09:38 08:53 10:23 16:25 11:09 13:24 15:40 17:10 18:46 21:06 22:41 01:52(+1)}}}
{\tnykdata{\anga{\tithi{5}{शुक्ल-पञ्चमी}}{\time{33-12}{19:54}}\hspace{1ex}}%
{\fullanga{पूर्वप्रोष्ठपदा}}{चन्द्रराशिः—\mbox{कुम्भः\RIGHTarrow\textsf{02:00(+1)}}}%
{\fullanga{वरीयान्}}%
{\anga{बवः}{\time{0-6}{06:40}}\hspace{1ex}\anga{बालवः}{\time{33-12}{19:54}}\hspace{1ex}\uanga{कौलवः}}{}
}
{\tamil{கூடாரவல்லீ}}
{Fri} 
\cfoot{\rygdata{10:52--12:16}{15:06--16:30}{08:02--09:27}}
\caldata{JANUARY}{12}{\sunmonth{धनुः}{28}{}{पौषः}{हेमन्तऋतुः}{शनिः}{विलम्बः}{दक्षिणायनम्}{हेमन्तऋतुः}}
{\sunmoonrsdata{06:38}{17:55}{10:38}{23:00}{12:17}
{\kalas{04:56 05:47 09:39 08:53 10:24 16:25 11:09 13:24 15:40 17:10 18:46 21:06 22:41 01:52(+1)}}}
{\tnykdata{\anga{\tithi{6}{शुक्ल-षष्ठी}}{\time{38-36}{22:05}}\hspace{1ex}}%
{\anga{पूर्वप्रोष्ठपदा}{\time{5-7}{08:41}}\hspace{1ex}}{चन्द्रराशिः—\mbox{मीनः}}%
{\anga{वरीयान्}{\time{1-15}{07:08}}\hspace{1ex}\uanga{परिघः}}%
{\anga{कौलवः}{\time{6-2}{09:03}}\hspace{1ex}\anga{तैतिलः}{\time{38-36}{22:05}}\hspace{1ex}\uanga{गरः}}{}
}
{षष्ठी-व्रतम्\eventsep \tamil{கறவைகள் பின்சென்று}}
{Sat} 
\cfoot{\rygdata{09:27--10:52}{13:41--15:06}{06:38--08:02}}
\caldata{JANUARY}{13}{\sunmonth{धनुः}{29}{}{पौषः}{हेमन्तऋतुः}{भानुः}{विलम्बः}{दक्षिणायनम्}{हेमन्तऋतुः}}
{\sunmoonrsdata{06:38}{17:56}{11:15}{23:47}{12:17}
{\kalas{04:56 05:47 09:39 08:54 10:24 16:26 11:09 13:25 15:40 17:11 18:47 21:07 22:42 01:52(+1)}}}
{\tnykdata{\anga{\tithi{7}{शुक्ल-सप्तमी}}{\time{42-39}{23:42}}\hspace{1ex}}%
{\anga{उत्तरप्रोष्ठपदा}{\time{11-3}{11:03}}\hspace{1ex}}{चन्द्रराशिः—\mbox{मीनः}}%
{\anga{परिघः}{\time{2-12}{07:31}}\hspace{1ex}\uanga{शिवः}}%
{\anga{गरः}{\time{10-50}{10:58}}\hspace{1ex}\anga{वणिजः}{\time{42-39}{23:42}}\hspace{1ex}\uanga{विष्टिः}}{}
}
{विजया-भानुसप्तमी}
{Sun} 
\cfoot{\rygdata{16:31--17:56}{12:17--13:42}{15:07--16:31}}
\caldata{JANUARY}{14}{\sunmonth{धनुः}{30}{\mbox{धनुः{\tiny\RIGHTarrow}\textsf{19:28}}}{पौषः}{हेमन्तऋतुः}{सोमः}{विलम्बः}{दक्षिणायनम्}{हेमन्तऋतुः}}
{\sunmoonrsdata{06:38}{17:57}{11:54}{00:36(+1)}{12:17}
{\kalas{04:57 05:47 09:39 08:54 10:24 16:26 11:10 13:25 15:41 17:11 18:47 21:07 22:42 01:53(+1)}}}
{\tnykdata{\anga{\tithi{8}{शुक्ल-अष्टमी}}{\time{44-57}{00:37(+1)}}\hspace{1ex}}%
{\anga{रेवती}{\time{15-29}{12:50}}\hspace{1ex}}{चन्द्रराशिः—\mbox{मीनः\RIGHTarrow\textsf{12:50}}}%
{\anga{शिवः}{\time{1-59}{07:26}}\hspace{1ex}\uanga{सिद्धः}}%
{\anga{विष्टिः}{\time{14-2}{12:15}}\hspace{1ex}\anga{बवः}{\time{44-57}{00:37(+1)}}\hspace{1ex}\uanga{बालवः}}{}
}
{\tamil{போகி}\eventsep मकर-सङ्क्रमण-पुण्यकालः~\textsf{19:28}{\RIGHTarrow}\textsf{03:28(+1)}\eventsep \tamil{வாயிலார் நாயனார் (49) குருபூஜை}}
{Mon} 
\cfoot{\rygdata{08:03--09:28}{10:53--12:17}{13:42--15:07}}
\caldata{JANUARY}{15}{\sunmonth{मकरः}{1}{}{पौषः}{हेमन्तऋतुः}{मङ्गलः}{विलम्बः}{उत्तरायणम्}{हेमन्तऋतुः}}
{\sunmoonrsdata{06:38}{17:57}{12:35}{01:29(+1)}{12:18}
{\kalas{04:57 05:48 09:39 08:54 10:25 16:27 11:10 13:26 15:41 17:12 18:48 21:08 22:43 01:53(+1)}}}
{\tnykdata{\anga{\tithi{9}{शुक्ल-नवमी}}{\time{45-16}{00:45(+1)}}\hspace{1ex}}%
{\anga{अश्विनी}{\time{18-7}{13:54}}\hspace{1ex}}{चन्द्रराशिः—\mbox{मेषः}}%
{\anga{सिद्धः}{\time{0-24}{06:48}}\hspace{1ex}\anga{साध्यः}{\time{57-17}{05:33(+1)}}\hspace{1ex}\uanga{शुभः}}%
{\anga{बालवः}{\time{15-21}{12:47}}\hspace{1ex}\anga{कौलवः}{\time{45-16}{00:45(+1)}}\hspace{1ex}\uanga{तैतिलः}}{}
}
{भौमाश्विनी-पुण्यकालः\eventsep \tamil{மதுரை மீனாக்ஷீ கோயிலில் கல் யானைக்கு கரும்பு கோடுத்த லீலை}\eventsep मकर-ज्योतिः\eventsep मकर-सङ्क्रान्तिः}
{Tue} 
\cfoot{\rygdata{15:07--16:32}{09:28--10:53}{12:18--13:43}}
\caldata{JANUARY}{16}{\sunmonth{मकरः}{2}{}{पौषः}{हेमन्तऋतुः}{बुधः}{विलम्बः}{उत्तरायणम्}{हेमन्तऋतुः}}
{\sunmoonrsdata{06:39}{17:58}{13:21}{02:25(+1)}{12:18}
{\kalas{04:57 05:48 09:40 08:54 10:25 16:27 11:10 13:26 15:42 17:12 18:48 21:08 22:43 01:53(+1)}}}
{\tnykdata{\anga{\tithi{10}{शुक्ल-दशमी}}{\time{43-31}{00:03(+1)}}\hspace{1ex}}%
{\anga{अपभरणी}{\time{18-47}{14:10}}\hspace{1ex}}{चन्द्रराशिः—\mbox{मेषः\RIGHTarrow\textsf{20:06}}}%
{\anga{शुभः}{\time{52-33}{03:40(+1)}}\hspace{1ex}\uanga{शुक्लः}}%
{\anga{तैतिलः}{\time{14-39}{12:30}}\hspace{1ex}\anga{गरः}{\time{43-31}{00:03(+1)}}\hspace{1ex}\uanga{वणिजः}}{}
}
{इन्द्र-पूजा/गो-पूजा\eventsep कृत्तिका-व्रतम्\eventsep \tamil{கனுப்~பொங்கல்}}
{Wed} 
\cfoot{\rygdata{12:18--13:43}{08:03--09:28}{10:53--12:18}}
\caldata{JANUARY}{17}{\sunmonth{मकरः}{3}{}{पौषः}{हेमन्तऋतुः}{गुरुः}{विलम्बः}{उत्तरायणम्}{हेमन्तऋतुः}}
{\sunmoonrsdata{06:39}{17:58}{14:12}{03:25(+1)}{12:19}
{\kalas{04:57 05:48 09:40 08:55 10:25 16:28 11:11 13:26 15:42 17:13 18:49 21:08 22:43 01:54(+1)}}}
{\tnykdata{\anga{\tithi{11}{शुक्ल-एकादशी}}{\time{39-48}{22:34}}\hspace{1ex}}%
{\anga{कृत्तिका}{\time{17-29}{13:38}}\hspace{1ex}}{चन्द्रराशिः—\mbox{वृषभः}}%
{\anga{शुक्लः}{\time{46-17}{01:10(+1)}}\hspace{1ex}\uanga{ब्रह्म}}%
{\anga{वणिजः}{\time{11-54}{11:24}}\hspace{1ex}\anga{विष्टिः}{\time{39-48}{22:34}}\hspace{1ex}\uanga{बवः}}{}
}
{काञ्ची ५५ जगद्गुरु श्री-चन्द्रचूडेन्द्र सरस्वती ३ आराधना~\#{४९५}\eventsep मन्वादिः-(चाक्षुषः-[६])\eventsep सर्व-पुत्रदा-एकादशी\eventsep \tamil{தை கிருத்திகை}\eventsep त्रैलङ्ग-स्वामि-जयन्ती}
{Thu} 
\cfoot{\rygdata{13:43--15:08}{06:39--08:04}{09:29--10:54}}
\caldata{JANUARY}{18}{\sunmonth{मकरः}{4}{}{पौषः}{हेमन्तऋतुः}{शुक्रः}{विलम्बः}{उत्तरायणम्}{हेमन्तऋतुः}}
{\sunmoonrsdata{06:39}{17:59}{15:08}{04:27(+1)}{12:19}
{\kalas{04:57 05:48 09:40 08:55 10:25 16:28 11:11 13:27 15:43 17:13 18:49 21:09 22:44 01:54(+1)}}}
{\tnykdata{\anga{\tithi{12}{शुक्ल-द्वादशी}}{\time{34-18}{20:22}}\hspace{1ex}}%
{\anga{रोहिणी}{\time{14-20}{12:23}}\hspace{1ex}}{चन्द्रराशिः—\mbox{वृषभः\RIGHTarrow\textsf{23:30}}}%
{\anga{ब्रह्म}{\time{38-36}{22:06}}\hspace{1ex}\uanga{इन्द्रः}}%
{\anga{बवः}{\time{7-15}{09:33}}\hspace{1ex}\anga{बालवः}{\time{34-18}{20:22}}\hspace{1ex}\uanga{कौलवः}}{}
}
{हरिवासरः{\RIGHTarrow}\textsf{04:05}\eventsep पापनाशिनी-महाद्वादशी\eventsep प्रदोष-व्रतम्\eventsep \tamil{தை~வெள்ளிக்கிழமை}}
{Fri} 
\cfoot{\rygdata{10:54--12:19}{15:09--16:34}{08:04--09:29}}
\caldata{JANUARY}{19}{\sunmonth{मकरः}{5}{}{पौषः}{हेमन्तऋतुः}{शनिः}{विलम्बः}{उत्तरायणम्}{हेमन्तऋतुः}}
{\sunmoonrsdata{06:39}{17:59}{16:10}{05:29(+1)}{12:19}
{\kalas{04:58 05:48 09:40 08:55 10:26 16:29 11:11 13:27 15:43 17:14 18:50 21:09 22:44 01:54(+1)}}}
{\tnykdata{\anga{\tithi{13}{शुक्ल-त्रयोदशी}}{\time{27-18}{17:34}}\hspace{1ex}}%
{\anga{मृगशीर्षम्}{\time{9-35}{10:29}}\hspace{1ex}}{चन्द्रराशिः—\mbox{मिथुनम्}}%
{\anga{इन्द्रः}{\time{29-45}{18:33}}\hspace{1ex}\uanga{वैधृतिः}}%
{\anga{कौलवः}{\time{0-57}{07:02}}\hspace{1ex}\anga{तैतिलः}{\time{27-18}{17:34}}\hspace{1ex}\anga{गरः}{\time{53-21}{03:59(+1)}}\hspace{1ex}\uanga{वणिजः}}{}
}
{\tamil{அரிவாட்டாய நாயனார் (12) குருபூஜை}\eventsep \tamil{கண்ணப்ப நாயனார் (9) குருபூஜை}\eventsep \tamil{கபாலீ தெப்போத்ஸவம்}}
{Sat} 
\cfoot{\rygdata{09:29--10:54}{13:44--15:09}{06:39--08:04}}
\caldata{JANUARY}{20}{\sunmonth{मकरः}{6}{}{पौषः}{हेमन्तऋतुः}{भानुः}{विलम्बः}{उत्तरायणम्}{हेमन्तऋतुः}}
{\sunmoonrsdata{06:39}{18:00}{17:14}{06:29(+1)}{12:19}
{\kalas{04:58 05:48 09:41 08:55 10:26 16:29 11:11 13:27 15:44 17:14 18:50 21:10 22:44 01:54(+1)}}}
{\tnykdata{\anga{\tithi{14}{शुक्ल-चतुर्दशी}}{\time{19-9}{14:19}}\hspace{1ex}}%
{\anga{आर्द्रा}{\time{3-35}{08:05}}\hspace{1ex}\anga{पुनर्वसुः}{\time{56-43}{05:20(+1)}}\hspace{1ex}}{चन्द्रराशिः—\mbox{मिथुनम्\RIGHTarrow\textsf{00:03(+1)}}}%
{\anga{वैधृतिः}{\time{19-59}{14:39}}\hspace{1ex}\uanga{विष्कम्भः}}%
{\anga{वणिजः}{\time{19-9}{14:19}}\hspace{1ex}\anga{विष्टिः}{\time{44-47}{00:34(+1)}}\hspace{1ex}\uanga{बवः}}{}
}
{काञ्ची ८ जगद्गुरु श्री-कैवल्यानन्दयोगेन्द्र सरस्वती आराधना~\#{१९९०}\eventsep \tamil{கபாலீ தெப்போத்ஸவம்}\eventsep पार्वणव्रतम् पूर्णिमायाम्\eventsep सहस्य-मासः/हेमन्तऋतुः{\RIGHTarrow}\textsf{14:29}\eventsep \tamil{தைப்பூசம்}\eventsep वेङ्कटाचले पूर्णिमा-गरुड-सेवा\eventsep वैधृति-श्राद्धम्\eventsep विष्णुपदी-पुण्यकालः~\textsf{08:05}{\RIGHTarrow}\textsf{20:53}}
{Sun} 
\cfoot{\rygdata{16:35--18:00}{12:19--13:44}{15:10--16:35}}
\caldata{JANUARY}{21}{\sunmonth{मकरः}{7}{}{पौषः}{हेमन्तऋतुः}{सोमः}{विलम्बः}{उत्तरायणम्}{हेमन्तऋतुः}}
{\sunmoonrsdata{06:39}{18:00}{18:20}{---}{12:20}
{\kalas{04:58 05:48 09:41 08:55 10:26 16:30 11:12 13:28 15:44 17:15 18:51 21:10 22:45 01:55(+1)}}}
{\tnykdata{\anga{\tithi{15}{पौर्णमासी}}{\time{10-16}{10:46}}\hspace{1ex}}%
{\anga{पुष्यः}{\time{49-25}{02:25(+1)}}\hspace{1ex}}{चन्द्रराशिः—\mbox{कटकः}}%
{\anga{विष्कम्भः}{\time{9-37}{10:30}}\hspace{1ex}\anga{प्रीतिः}{\time{58-58}{06:14(+1)}}\hspace{1ex}\uanga{आयुष्मान्}}%
{\anga{बवः}{\time{10-16}{10:46}}\hspace{1ex}\anga{बालवः}{\time{35-41}{20:55}}\hspace{1ex}\uanga{कौलवः}}{}
}
{बदरी ज्योतिर्मठ-प्रतिष्ठापन-जयन्ती~\#{२५०४}\eventsep \tamil{கபாலீ தெப்போத்ஸவம்}\eventsep पूर्र्णमासेष्टिः\eventsep पूर्णिमा-व्रतम्\eventsep स्थालीपाकः\eventsep शाकम्भरी-जयन्ती\eventsep शृङ्गेरी शारदामठ-प्रतिष्ठापन-जयन्ती~\#{२५०२}}
{Mon} 
\cfoot{\rygdata{08:04--09:29}{10:55--12:20}{13:45--15:10}}
\caldata{JANUARY}{22}{\sunmonth{मकरः}{8}{}{पौषः}{हेमन्तऋतुः}{मङ्गलः}{विलम्बः}{उत्तरायणम्}{हेमन्तऋतुः}}
{\sunmoonsrdata{06:39}{18:01}{19:24}{07:25}{12:20}
{\kalas{04:58 05:48 09:41 08:55 10:26 16:30 11:12 13:28 15:45 17:15 18:51 21:10 22:45 01:55(+1)}}}
{\tnykdata{\anga{\tithi{16}{कृष्ण-प्रथमा}}{\time{1-3}{07:05}}\hspace{1ex}\anga{\tithi{17}{कृष्ण-द्वितीया}}{\time{51-56}{03:26(+1)}}\hspace{1ex}}%
{\anga{आश्रेषा}{\time{42-6}{23:30}}\hspace{1ex}}{चन्द्रराशिः—\mbox{कटकः\RIGHTarrow\textsf{23:30}}}%
{\anga{आयुष्मान्}{\time{48-23}{02:00(+1)}}\hspace{1ex}\uanga{सौभाग्यः}}%
{\anga{कौलवः}{\time{1-3}{07:05}}\hspace{1ex}\anga{तैतिलः}{\time{26-27}{17:14}}\hspace{1ex}\anga{गरः}{\time{51-56}{03:26(+1)}}\hspace{1ex}\uanga{वणिजः}}{}
}
{काञ्ची ३७ जगद्गुरु श्री-विद्याघनेन्द्र सरस्वती ३ आराधना~\#{१२३१}\eventsep काञ्ची ६२ जगद्गुरु श्री-चन्द्रशेखरेन्द्र सरस्वती ४ आराधना~\#{२३६}}
{Tue} 
\cfoot{\rygdata{15:10--16:36}{09:29--10:55}{12:20--13:45}}
\caldata{JANUARY}{23}{\sunmonth{मकरः}{9}{}{पौषः}{हेमन्तऋतुः}{बुधः}{विलम्बः}{उत्तरायणम्}{हेमन्तऋतुः}}
{\sunmoonsrdata{06:39}{18:01}{20:25}{08:18}{12:20}
{\kalas{04:58 05:49 09:41 08:55 10:26 16:30 11:12 13:28 15:45 17:16 18:52 21:11 22:45 01:55(+1)}}}
{\tnykdata{\anga{\tithi{18}{कृष्ण-तृतीया}}{\time{43-19}{23:59}}\hspace{1ex}}%
{\anga{मघा}{\time{35-13}{20:45}}\hspace{1ex}}{चन्द्रराशिः—\mbox{सिंहः}}%
{\anga{सौभाग्यः}{\time{38-11}{21:56}}\hspace{1ex}\uanga{शोभनः}}%
{\anga{वणिजः}{\time{17-33}{13:40}}\hspace{1ex}\anga{विष्टिः}{\time{43-19}{23:59}}\hspace{1ex}\uanga{बवः}}{}
}
{\tamil{திருமழிசையாழ்வார் திருநக்ஷத்திரம்}}
{Wed} 
\cfoot{\rygdata{12:20--13:45}{08:04--09:30}{10:55--12:20}}
\caldata{JANUARY}{24}{\sunmonth{मकरः}{10}{}{पौषः}{हेमन्तऋतुः}{गुरुः}{विलम्बः}{उत्तरायणम्}{हेमन्तऋतुः}}
{\sunmoonsrdata{06:39}{18:02}{21:24}{09:06}{12:21}
{\kalas{04:58 05:49 09:41 08:56 10:27 16:31 11:12 13:29 15:45 17:16 18:52 21:11 22:46 01:55(+1)}}}
{\tnykdata{\anga{\tithi{19}{कृष्ण-चतुर्थी}}{\time{35-35}{20:53}}\hspace{1ex}}%
{\anga{पूर्वफल्गुनी}{\time{29-10}{18:19}}\hspace{1ex}}{चन्द्रराशिः—\mbox{सिंहः\RIGHTarrow\textsf{23:47}}}%
{\anga{शोभनः}{\time{28-41}{18:08}}\hspace{1ex}\uanga{अतिगण्डः}}%
{\anga{बवः}{\time{9-19}{10:23}}\hspace{1ex}\anga{बालवः}{\time{35-35}{20:53}}\hspace{1ex}\uanga{कौलवः}}{}
}
{लम्बोदर-महागणपति सङ्कटहर-चतुर्थी-व्रतम्}
{Thu} 
\cfoot{\rygdata{13:46--15:11}{06:39--08:04}{09:30--10:55}}
\caldata{JANUARY}{25}{\sunmonth{मकरः}{11}{}{पौषः}{हेमन्तऋतुः}{शुक्रः}{विलम्बः}{उत्तरायणम्}{हेमन्तऋतुः}}
{\sunmoonsrdata{06:39}{18:02}{22:21}{09:52}{12:21}
{\kalas{04:58 05:49 09:41 08:56 10:27 16:31 11:12 13:29 15:46 17:17 18:53 21:12 22:46 01:55(+1)}}}
{\tnykdata{\anga{\tithi{20}{कृष्ण-पञ्चमी}}{\time{29-6}{18:18}}\hspace{1ex}}%
{\anga{उत्तरफल्गुनी}{\time{24-19}{16:23}}\hspace{1ex}}{चन्द्रराशिः—\mbox{कन्या}}%
{\anga{अतिगण्डः}{\time{20-10}{14:43}}\hspace{1ex}\uanga{सुकर्म}}%
{\anga{कौलवः}{\time{2-11}{07:32}}\hspace{1ex}\anga{तैतिलः}{\time{29-6}{18:18}}\hspace{1ex}\anga{गरः}{\time{56-25}{05:13(+1)}}\hspace{1ex}\uanga{वणिजः}}{}
}
{बहुल-पञ्चमी\eventsep \tamil{சண்டேஶ்வர நாயனார் (19) குருபூஜை}\eventsep \tamil{தை~வெள்ளிக்கிழமை}\eventsep त्यागराज-आराधना~\#{१७२}}
{Fri} 
\cfoot{\rygdata{10:55--12:21}{15:12--16:37}{08:04--09:30}}
\caldata{JANUARY}{26}{\sunmonth{मकरः}{12}{}{पौषः}{हेमन्तऋतुः}{शनिः}{विलम्बः}{उत्तरायणम्}{हेमन्तऋतुः}}
{\sunmoonsrdata{06:39}{18:03}{23:16}{10:37}{12:21}
{\kalas{04:58 05:49 09:41 08:56 10:27 16:32 11:13 13:29 15:46 17:17 18:53 21:12 22:46 01:55(+1)}}}
{\tnykdata{\anga{\tithi{21}{कृष्ण-षष्ठी}}{\time{24-9}{16:19}}\hspace{1ex}}%
{\anga{हस्तः}{\time{20-57}{15:02}}\hspace{1ex}}{चन्द्रराशिः—\mbox{कन्या\RIGHTarrow\textsf{02:37(+1)}}}%
{\anga{सुकर्म}{\time{12-51}{11:48}}\hspace{1ex}\uanga{धृतिः}}%
{\anga{वणिजः}{\time{24-9}{16:19}}\hspace{1ex}\anga{विष्टिः}{\time{52-19}{03:35(+1)}}\hspace{1ex}\uanga{बवः}}{}
}
{श्री-शेषाद्रि-स्वामि-जयन्ती~\#{१५०}}
{Sat} 
\cfoot{\rygdata{09:30--10:55}{13:46--15:12}{06:39--08:04}}
\caldata{JANUARY}{27}{\sunmonth{मकरः}{13}{}{पौषः}{हेमन्तऋतुः}{भानुः}{विलम्बः}{उत्तरायणम्}{हेमन्तऋतुः}}
{\sunmoonsrdata{06:39}{18:03}{00:10(+1)}{11:21}{12:21}
{\kalas{04:58 05:49 09:42 08:56 10:27 16:32 11:13 13:30 15:46 17:18 18:54 21:12 22:47 01:56(+1)}}}
{\tnykdata{\anga{\tithi{22}{कृष्ण-सप्तमी}}{\time{20-57}{15:02}}\hspace{1ex}}%
{\anga{चित्रा}{\time{19-18}{14:22}}\hspace{1ex}}{चन्द्रराशिः—\mbox{तुला}}%
{\anga{धृतिः}{\time{6-54}{09:25}}\hspace{1ex}\uanga{शूलः}}%
{\anga{बवः}{\time{20-57}{15:02}}\hspace{1ex}\anga{बालवः}{\time{50-2}{02:40(+1)}}\hspace{1ex}\uanga{कौलवः}}{}
}
{भानुसप्तमी\eventsep पौष-अष्टका-पूर्वेद्युः\eventsep विवेकानन्द-जयन्ती~\#{१५७}}
{Sun} 
\cfoot{\rygdata{16:38--18:03}{12:21--13:47}{15:12--16:38}}
\caldata{JANUARY}{28}{\sunmonth{मकरः}{14}{}{पौषः}{हेमन्तऋतुः}{सोमः}{विलम्बः}{उत्तरायणम्}{हेमन्तऋतुः}}
{\sunmoonsrdata{06:39}{18:04}{01:04(+1)}{12:06}{12:21}
{\kalas{04:58 05:49 09:42 08:56 10:27 16:33 11:13 13:30 15:47 17:18 18:54 21:13 22:47 01:56(+1)}}}
{\tnykdata{\anga{\tithi{23}{कृष्ण-अष्टमी}}{\time{19-35}{14:29}}\hspace{1ex}}%
{\anga{स्वाती}{\time{19-27}{14:26}}\hspace{1ex}}{चन्द्रराशिः—\mbox{तुला}}%
{\anga{शूलः}{\time{2-24}{07:37}}\hspace{1ex}\anga{गण्डः}{\time{59-20}{06:23(+1)}}\hspace{1ex}\uanga{वृद्धिः}}%
{\anga{कौलवः}{\time{19-35}{14:29}}\hspace{1ex}\anga{तैतिलः}{\time{49-36}{02:29(+1)}}\hspace{1ex}\uanga{गरः}}{}
}
{पौष-अष्टका-श्राद्धम्}
{Mon} 
\cfoot{\rygdata{08:04--09:30}{10:56--12:21}{13:47--15:13}}
\caldata{JANUARY}{29}{\sunmonth{मकरः}{15}{}{पौषः}{हेमन्तऋतुः}{मङ्गलः}{विलम्बः}{उत्तरायणम्}{हेमन्तऋतुः}}
{\sunmoonsrdata{06:39}{18:04}{01:58(+1)}{12:52}{12:22}
{\kalas{04:58 05:49 09:42 08:56 10:27 16:33 11:13 13:30 15:47 17:19 18:55 21:13 22:47 01:56(+1)}}}
{\tnykdata{\anga{\tithi{24}{कृष्ण-नवमी}}{\time{20-4}{14:40}}\hspace{1ex}}%
{\anga{विशाखा}{\time{21-23}{15:12}}\hspace{1ex}}{चन्द्रराशिः—\mbox{तुला\RIGHTarrow\textsf{08:57}}}%
{\anga{वृद्धिः}{\time{57-38}{05:42(+1)}}\hspace{1ex}\uanga{ध्रुवः}}%
{\anga{गरः}{\time{20-4}{14:40}}\hspace{1ex}\anga{वणिजः}{\time{50-57}{03:02(+1)}}\hspace{1ex}\uanga{विष्टिः}}{}
}
{भीष्म-जयन्ती\eventsep पौष-अन्वष्टका-श्राद्धम्\eventsep \tamil{திருநீலகண்ட நாயனார் (1) குருபூஜை}}
{Tue} 
\cfoot{\rygdata{15:13--16:39}{09:30--10:56}{12:22--13:47}}
\caldata{JANUARY}{30}{\sunmonth{मकरः}{16}{}{पौषः}{हेमन्तऋतुः}{बुधः}{विलम्बः}{उत्तरायणम्}{हेमन्तऋतुः}}
{\sunmoonsrdata{06:39}{18:05}{02:51(+1)}{13:40}{12:22}
{\kalas{04:58 05:48 09:42 08:56 10:27 16:33 11:13 13:30 15:48 17:19 18:55 21:13 22:48 01:56(+1)}}}
{\tnykdata{\anga{\tithi{25}{कृष्ण-दशमी}}{\time{22-15}{15:33}}\hspace{1ex}}%
{\anga{अनूराधा}{\time{24-57}{16:38}}\hspace{1ex}}{चन्द्रराशिः—\mbox{वृश्चिकः}}%
{\anga{ध्रुवः}{\time{57-9}{05:30(+1)}}\hspace{1ex}\uanga{व्याघातः}}%
{\anga{विष्टिः}{\time{22-15}{15:33}}\hspace{1ex}\anga{बवः}{\time{53-56}{04:13(+1)}}\hspace{1ex}\uanga{बालवः}}{}
}
{बुधानुराधा-पुण्यकालः\eventsep त्रैलोक्य-गौरी-व्रतम्}
{Wed} 
\cfoot{\rygdata{12:22--13:48}{08:04--09:30}{10:56--12:22}}
\caldata{JANUARY}{31}{\sunmonth{मकरः}{17}{}{पौषः}{हेमन्तऋतुः}{गुरुः}{विलम्बः}{उत्तरायणम्}{हेमन्तऋतुः}}
{\sunmoonsrdata{06:39}{18:05}{03:43(+1)}{14:30}{12:22}
{\kalas{04:58 05:48 09:42 08:56 10:27 16:34 11:13 13:31 15:48 17:19 18:55 21:13 22:48 01:56(+1)}}}
{\tnykdata{\anga{\tithi{26}{कृष्ण-एकादशी}}{\time{25-57}{17:02}}\hspace{1ex}}%
{\anga{ज्येष्ठा}{\time{29-57}{18:38}}\hspace{1ex}}{चन्द्रराशिः—\mbox{वृश्चिकः\RIGHTarrow\textsf{18:38}}}%
{\anga{व्याघातः}{\time{57-40}{05:43(+1)}}\hspace{1ex}\uanga{हर्षणः}}%
{\anga{बालवः}{\time{25-57}{17:02}}\hspace{1ex}\anga{कौलवः}{\time{58-16}{05:57(+1)}}\hspace{1ex}\uanga{तैतिलः}}{}
}
{हरिवासरः{\RIGHTarrow}\textsf{23:29}\eventsep प्रदोष-व्रतम्\eventsep सर्व-षट्तिला-एकादशी}
{Thu} 
\cfoot{\rygdata{13:48--15:13}{06:39--08:04}{09:30--10:56}}
\caldata{FEBRUARY}{1}{\sunmonth{मकरः}{18}{}{पौषः}{हेमन्तऋतुः}{शुक्रः}{विलम्बः}{उत्तरायणम्}{हेमन्तऋतुः}}
{\sunmoonsrdata{06:39}{18:06}{04:32(+1)}{15:20}{12:22}
{\kalas{04:58 05:48 09:42 08:56 10:27 16:34 11:13 13:31 15:48 17:20 18:56 21:14 22:48 01:56(+1)}}}
{\tnykdata{\anga{\tithi{27}{कृष्ण-द्वादशी}}{\time{30-51}{18:59}}\hspace{1ex}}%
{\anga{मूला}{\time{36-6}{21:05}}\hspace{1ex}}{चन्द्रराशिः—\mbox{धनुः}}%
{\anga{हर्षणः}{\time{59-0}{06:15(+1)}}\hspace{1ex}\uanga{वज्रम्}}%
{\anga{तैतिलः}{\time{30-51}{18:59}}\hspace{1ex}\uanga{गरः}}{}
}
{सेङ्गालिपुरम्-मुत्तण्णावाळ्-आराधना~\#{१२६}\eventsep \tamil{தை~வெள்ளிக்கிழமை}}
{Fri} 
\cfoot{\rygdata{10:56--12:22}{15:14--16:40}{08:04--09:30}}
\caldata{FEBRUARY}{2}{\sunmonth{मकरः}{19}{}{पौषः}{हेमन्तऋतुः}{शनिः}{विलम्बः}{उत्तरायणम्}{हेमन्तऋतुः}}
{\sunmoonsrdata{06:38}{18:06}{05:19(+1)}{16:11}{12:22}
{\kalas{04:58 05:48 09:42 08:56 10:28 16:34 11:13 13:31 15:48 17:20 18:56 21:14 22:48 01:56(+1)}}}
{\tnykdata{\anga{\tithi{28}{कृष्ण-त्रयोदशी}}{\time{36-40}{21:19}}\hspace{1ex}}%
{\anga{पूर्वाषाढा}{\time{43-5}{23:52}}\hspace{1ex}}{चन्द्रराशिः—\mbox{धनुः\RIGHTarrow\textsf{06:37(+1)}}}%
{\fullanga{वज्रम्}}%
{\anga{गरः}{\time{3-41}{08:07}}\hspace{1ex}\anga{वणिजः}{\time{36-40}{21:19}}\hspace{1ex}\uanga{विष्टिः}}{}
}
{मासशिवरात्रिः\eventsep शनि-प्रदोष-व्रतम्}
{Sat} 
\cfoot{\rygdata{09:30--10:56}{13:48--15:14}{06:38--08:04}}
\caldata{FEBRUARY}{3}{\sunmonth{मकरः}{20}{}{पौषः}{हेमन्तऋतुः}{भानुः}{विलम्बः}{उत्तरायणम्}{हेमन्तऋतुः}}
{\sunmoonsrdata{06:38}{18:06}{06:03(+1)}{17:01}{12:22}
{\kalas{04:58 05:48 09:42 08:56 10:28 16:35 11:13 13:31 15:49 17:21 18:57 21:14 22:48 01:56(+1)}}}
{\tnykdata{\anga{\tithi{29}{कृष्ण-चतुर्दशी}}{\time{43-5}{23:52}}\hspace{1ex}}%
{\anga{उत्तराषाढा}{\time{50-36}{02:53(+1)}}\hspace{1ex}}{चन्द्रराशिः—\mbox{मकरः}}%
{\anga{वज्रम्}{\time{0-54}{07:00}}\hspace{1ex}\uanga{सिद्धिः}}%
{\anga{विष्टिः}{\time{9-50}{10:34}}\hspace{1ex}\anga{शकुनिः}{\time{43-5}{23:52}}\hspace{1ex}\uanga{चतुष्पात्}}{}
}
{}
{Sun} 
\cfoot{\rygdata{16:40--18:06}{12:22--13:48}{15:14--16:40}}
\caldata{FEBRUARY}{4}{\sunmonth{मकरः}{21}{}{पौषः}{हेमन्तऋतुः}{सोमः}{विलम्बः}{उत्तरायणम्}{हेमन्तऋतुः}}
{\sunmoonsrdata{06:38}{18:07}{---}{17:50}{12:22}
{\kalas{04:58 05:48 09:42 08:56 10:28 16:35 11:14 13:31 15:49 17:21 18:57 21:15 22:48 01:56(+1)}}}
{\tnykdata{\anga{\tithi{30}{अमावास्या}}{\time{49-48}{02:33(+1)}}\hspace{1ex}}%
{\anga{श्रवणः}{\time{58-22}{05:59(+1)}}\hspace{1ex}}{चन्द्रराशिः—\mbox{मकरः}}%
{\anga{सिद्धिः}{\time{3-10}{07:54}}\hspace{1ex}\uanga{व्यतीपातः}}%
{\anga{चतुष्पात्}{\time{16-25}{13:12}}\hspace{1ex}\anga{नाग}{\time{49-48}{02:33(+1)}}\hspace{1ex}\uanga{किंस्तुघ्नः}}{}
}
{महोदय-पुण्यकालः\eventsep मौनि (पौष/मकर) अमावास्या (अलभ्यम्–पुष्कला)\eventsep पार्वणव्रतम् अमावास्यायाम्\eventsep सोमवती अमावास्या\eventsep सोमश्रावणी-पुण्यकालः\eventsep \tamil{திருநெல்வேலி நெல்லையப்பர் பத்ர தீப திருவிழா}\eventsep व्यतीपात-श्राद्धम्\eventsep श्रवण-व्रतम्}
{Mon} 
\cfoot{\rygdata{08:04--09:30}{10:56--12:22}{13:49--15:15}}
\caldata{FEBRUARY}{5}{\sunmonth{मकरः}{22}{}{माघः}{शिशिरऋतुः}{मङ्गलः}{विलम्बः}{उत्तरायणम्}{हेमन्तऋतुः}}
{\sunmoonrsdata{06:38}{18:07}{06:45}{18:38}{12:23}
{\kalas{04:58 05:48 09:42 08:56 10:28 16:35 11:14 13:31 15:49 17:21 18:57 21:15 22:49 01:56(+1)}}}
{\tnykdata{\anga{\tithi{1}{शुक्ल-प्रथमा}}{\time{56-33}{05:15(+1)}}\hspace{1ex}}%
{\fullanga{श्रविष्ठा}}{चन्द्रराशिः—\mbox{मकरः\RIGHTarrow\textsf{19:33}}}%
{\anga{व्यतीपातः}{\time{5-35}{08:52}}\hspace{1ex}\uanga{वरीयान्}}%
{\anga{किंस्तुघ्नः}{\time{23-11}{15:54}}\hspace{1ex}\anga{बवः}{\time{56-33}{05:15(+1)}}\hspace{1ex}\uanga{बालवः}}{}
}
{दर्शेष्टिः\eventsep स्थालीपाकः\eventsep श्यामळानवरात्र-आरम्भः}
{Tue} 
\cfoot{\rygdata{15:15--16:41}{09:30--10:56}{12:23--13:49}}
\caldata{FEBRUARY}{6}{\sunmonth{मकरः}{23}{}{माघः}{शिशिरऋतुः}{बुधः}{विलम्बः}{उत्तरायणम्}{हेमन्तऋतुः}}
{\sunmoonrsdata{06:38}{18:08}{07:24}{19:25}{12:23}
{\kalas{04:57 05:48 09:42 08:56 10:28 16:36 11:14 13:32 15:50 17:22 18:58 21:15 22:49 01:56(+1)}}}
{\tnykdata{\fulltithi{\tithi{2}{शुक्ल-द्वितीया}}}%
{\anga{श्रविष्ठा}{\time{6-10}{09:06}}\hspace{1ex}}{चन्द्रराशिः—\mbox{कुम्भः}}%
{\anga{वरीयान्}{\time{8-1}{09:50}}\hspace{1ex}\uanga{परिघः}}%
{\anga{बालवः}{\time{29-52}{18:35}}\hspace{1ex}\uanga{कौलवः}}{}
}
{चन्द्र-दर्शनम्}
{Wed} 
\cfoot{\rygdata{12:23--13:49}{08:04--09:30}{10:56--12:23}}
\caldata{FEBRUARY}{7}{\sunmonth{मकरः}{24}{}{माघः}{शिशिरऋतुः}{गुरुः}{विलम्बः}{उत्तरायणम्}{हेमन्तऋतुः}}
{\sunmoonrsdata{06:37}{18:08}{08:01}{20:11}{12:23}
{\kalas{04:57 05:47 09:41 08:55 10:27 16:36 11:14 13:32 15:50 17:22 18:58 21:15 22:49 01:56(+1)}}}
{\tnykdata{\anga{\tithi{2}{शुक्ल-द्वितीया}}{\time{3-7}{07:52}}\hspace{1ex}}%
{\anga{शतभिषक्}{\time{13-44}{12:07}}\hspace{1ex}}{चन्द्रराशिः—\mbox{कुम्भः}}%
{\anga{परिघः}{\time{10-15}{10:43}}\hspace{1ex}\uanga{शिवः}}%
{\anga{कौलवः}{\time{3-7}{07:52}}\hspace{1ex}\anga{तैतिलः}{\time{36-13}{21:07}}\hspace{1ex}\uanga{गरः}}{}
}
{\tamil{அப்பூதியடிகள் நாயனார் (24) குருபூஜை}}
{Thu} 
\cfoot{\rygdata{13:49--15:15}{06:37--08:04}{09:30--10:56}}
\caldata{FEBRUARY}{8}{\sunmonth{मकरः}{25}{}{माघः}{शिशिरऋतुः}{शुक्रः}{विलम्बः}{उत्तरायणम्}{हेमन्तऋतुः}}
{\sunmoonrsdata{06:37}{18:08}{08:38}{20:56}{12:23}
{\kalas{04:57 05:47 09:41 08:55 10:27 16:36 11:14 13:32 15:50 17:22 18:58 21:15 22:49 01:56(+1)}}}
{\tnykdata{\anga{\tithi{3}{शुक्ल-तृतीया}}{\time{9-11}{10:18}}\hspace{1ex}}%
{\anga{पूर्वप्रोष्ठपदा}{\time{20-48}{14:56}}\hspace{1ex}}{चन्द्रराशिः—\mbox{कुम्भः\RIGHTarrow\textsf{08:15}}}%
{\anga{शिवः}{\time{12-7}{11:28}}\hspace{1ex}\uanga{सिद्धः}}%
{\anga{गरः}{\time{9-11}{10:18}}\hspace{1ex}\anga{वणिजः}{\time{41-58}{23:24}}\hspace{1ex}\uanga{विष्टिः}}{}
}
{\tamil{தை~வெள்ளிக்கிழமை}\eventsep \tamil{திருச்செந்தூர் முருகன் மாசித் திருவிழா தொடக்கம்}\eventsep वरकुन्द-चतुर्थी}
{Fri} 
\cfoot{\rygdata{10:56--12:23}{15:15--16:42}{08:03--09:30}}
\caldata{FEBRUARY}{9}{\sunmonth{मकरः}{26}{}{माघः}{शिशिरऋतुः}{शनिः}{विलम्बः}{उत्तरायणम्}{हेमन्तऋतुः}}
{\sunmoonrsdata{06:37}{18:09}{09:14}{21:43}{12:23}
{\kalas{04:57 05:47 09:41 08:55 10:27 16:36 11:14 13:32 15:50 17:23 18:59 21:16 22:49 01:56(+1)}}}
{\tnykdata{\anga{\tithi{4}{शुक्ल-चतुर्थी}}{\time{14-31}{12:26}}\hspace{1ex}}%
{\anga{उत्तरप्रोष्ठपदा}{\time{27-8}{17:28}}\hspace{1ex}}{चन्द्रराशिः—\mbox{मीनः}}%
{\anga{सिद्धः}{\time{13-27}{12:00}}\hspace{1ex}\uanga{साध्यः}}%
{\anga{विष्टिः}{\time{14-31}{12:26}}\hspace{1ex}\anga{बवः}{\time{46-49}{01:21(+1)}}\hspace{1ex}\uanga{बालवः}}{}
}
{मार्कण्डेय-जयन्ती\eventsep प्रोक्लस्-जन्म~\#{१६०७}\eventsep \tamil{திருச்செந்தூர் முருகன் மாசித் திருவிழா 2ம் நாள்}}
{Sat} 
\cfoot{\rygdata{09:30--10:56}{13:49--15:16}{06:37--08:03}}
\caldata{FEBRUARY}{10}{\sunmonth{मकरः}{27}{}{माघः}{शिशिरऋतुः}{भानुः}{विलम्बः}{उत्तरायणम्}{हेमन्तऋतुः}}
{\sunmoonrsdata{06:37}{18:09}{09:52}{22:31}{12:23}
{\kalas{04:57 05:47 09:41 08:55 10:27 16:37 11:14 13:32 15:50 17:23 18:59 21:16 22:49 01:56(+1)}}}
{\tnykdata{\anga{\tithi{5}{शुक्ल-पञ्चमी}}{\time{18-50}{14:09}}\hspace{1ex}}%
{\anga{रेवती}{\time{32-26}{19:35}}\hspace{1ex}}{चन्द्रराशिः—\mbox{मीनः\RIGHTarrow\textsf{19:35}}}%
{\anga{साध्यः}{\time{14-0}{12:13}}\hspace{1ex}\uanga{शुभः}}%
{\anga{बालवः}{\time{18-50}{14:09}}\hspace{1ex}\anga{कौलवः}{\time{50-30}{02:49(+1)}}\hspace{1ex}\uanga{तैतिलः}}{}
}
{\tamil{கலிக்கம்ப நாயனார் (42) குருபூஜை}\eventsep माघी-सरस्वती-पूजा\eventsep सर्प-पूजा\eventsep \tamil{திருச்செந்தூர் முருகன் மாசித் திருவிழா 3ம் நாள்—முருகன் பவனி}\eventsep वसन्त-श्री-पञ्चमी\eventsep श्रीराम-वनवास-गमनम्}
{Sun} 
\cfoot{\rygdata{16:42--18:09}{12:23--13:49}{15:16--16:42}}
\caldata{FEBRUARY}{11}{\sunmonth{मकरः}{28}{}{माघः}{शिशिरऋतुः}{सोमः}{विलम्बः}{उत्तरायणम्}{हेमन्तऋतुः}}
{\sunmoonrsdata{06:36}{18:09}{10:32}{23:21}{12:23}
{\kalas{04:57 05:46 09:41 08:55 10:27 16:37 11:13 13:32 15:51 17:23 18:59 21:16 22:49 01:56(+1)}}}
{\tnykdata{\anga{\tithi{6}{शुक्ल-षष्ठी}}{\time{21-50}{15:20}}\hspace{1ex}}%
{\anga{अश्विनी}{\time{36-25}{21:10}}\hspace{1ex}}{चन्द्रराशिः—\mbox{मेषः}}%
{\anga{शुभः}{\time{13-35}{12:03}}\hspace{1ex}\uanga{शुक्लः}}%
{\anga{तैतिलः}{\time{21-50}{15:20}}\hspace{1ex}\anga{गरः}{\time{52-45}{03:42(+1)}}\hspace{1ex}\uanga{वणिजः}}{}
}
{षष्ठी-व्रतम्\eventsep \tamil{திருச்செந்தூர் முருகன் மாசித் திருவிழா 4ம் நாள்}\eventsep \tamil{திருநெல்வேலி நெல்லையப்பர் நெல்லுக்கு வேலி கட்டிய லீலை}}
{Mon} 
\cfoot{\rygdata{08:03--09:29}{10:56--12:23}{13:49--15:16}}
\caldata{FEBRUARY}{12}{\sunmonth{मकरः}{29}{}{माघः}{शिशिरऋतुः}{मङ्गलः}{विलम्बः}{उत्तरायणम्}{हेमन्तऋतुः}}
{\sunmoonrsdata{06:36}{18:10}{11:15}{00:14(+1)}{12:23}
{\kalas{04:56 05:46 09:41 08:55 10:27 16:37 11:13 13:32 15:51 17:24 18:59 21:16 22:49 01:56(+1)}}}
{\tnykdata{\anga{\tithi{7}{शुक्ल-सप्तमी}}{\time{23-16}{15:54}}\hspace{1ex}}%
{\anga{अपभरणी}{\time{38-52}{22:09}}\hspace{1ex}}{चन्द्रराशिः—\mbox{मेषः\RIGHTarrow\textsf{04:17(+1)}}}%
{\anga{शुक्लः}{\time{12-1}{11:24}}\hspace{1ex}\uanga{ब्रह्म}}%
{\anga{वणिजः}{\time{23-16}{15:54}}\hspace{1ex}\anga{विष्टिः}{\time{53-20}{03:56(+1)}}\hspace{1ex}\uanga{बवः}}{}
}
{अचला-सप्तमी-व्रतम्\eventsep द्वारका-मठ-प्रतिष्ठापन-जयन्ती~\#{२५०९}\eventsep कुम्भ-रवि-सङ्क्रमण-विष्णुपदी-पुण्यकालः~\textsf{02:06(+1)}{\RIGHTarrow}\textsf{14:54(+1)}\eventsep मन्वादिः-(सावर्णिः-[८])\eventsep नर्मदा-जयन्ती\eventsep रथ-सप्तमी\eventsep \tamil{திருச்செந்தூர் முருகன் மாசித் திருவிழா 5ம் நாள்}}
{Tue} 
\cfoot{\rygdata{15:16--16:43}{09:29--10:56}{12:23--13:50}}
\caldata{FEBRUARY}{13}{\sunmonth{कुम्भः}{1}{\mbox{मकरः{\tiny\RIGHTarrow}\textsf{08:30}}}{माघः}{शिशिरऋतुः}{बुधः}{विलम्बः}{उत्तरायणम्}{शिशिरऋतुः}}
{\sunmoonrsdata{06:36}{18:10}{12:02}{01:10(+1)}{12:23}
{\kalas{04:56 05:46 09:41 08:54 10:27 16:37 11:13 13:32 15:51 17:24 19:00 21:16 22:49 01:56(+1)}}}
{\tnykdata{\anga{\tithi{8}{शुक्ल-अष्टमी}}{\time{22-57}{15:46}}\hspace{1ex}}%
{\anga{कृत्तिका}{\time{39-35}{22:26}}\hspace{1ex}}{चन्द्रराशिः—\mbox{वृषभः}}%
{\anga{ब्रह्म}{\time{9-7}{10:15}}\hspace{1ex}\uanga{इन्द्रः}}%
{\anga{बवः}{\time{22-57}{15:46}}\hspace{1ex}\anga{बालवः}{\time{52-5}{03:26(+1)}}\hspace{1ex}\uanga{कौलवः}}{}
}
{भीष्माष्टमी\eventsep बुधाष्टमी\eventsep कृत्तिका-व्रतम्\eventsep खोडियार-माता-जयन्ती\eventsep \tamil{திருச்செந்தூர் முருகன் மாசித் திருவிழா 6ம் நாள்—வெள்ளித் தேர் பவனி}}
{Wed} 
\cfoot{\rygdata{12:23--13:50}{08:02--09:29}{10:56--12:23}}
\caldata{FEBRUARY}{14}{\sunmonth{कुम्भः}{2}{}{माघः}{शिशिरऋतुः}{गुरुः}{विलम्बः}{उत्तरायणम्}{शिशिरऋतुः}}
{\sunmoonrsdata{06:35}{18:10}{12:54}{02:09(+1)}{12:23}
{\kalas{04:56 05:46 09:41 08:54 10:27 16:38 11:13 13:32 15:51 17:24 19:00 21:16 22:50 01:56(+1)}}}
{\tnykdata{\anga{\tithi{9}{शुक्ल-नवमी}}{\time{20-47}{14:54}}\hspace{1ex}}%
{\anga{रोहिणी}{\time{38-30}{21:59}}\hspace{1ex}}{चन्द्रराशिः—\mbox{वृषभः}}%
{\anga{इन्द्रः}{\time{4-48}{08:31}}\hspace{1ex}\anga{वैधृतिः}{\time{59-1}{06:12(+1)}}\hspace{1ex}\uanga{विष्कम्भः}}%
{\anga{कौलवः}{\time{20-47}{14:54}}\hspace{1ex}\anga{तैतिलः}{\time{49-1}{02:12(+1)}}\hspace{1ex}\uanga{गरः}}{}
}
{काञ्ची ११ जगद्गुरु श्री-शिवानन्द चिद्घनेन्द्र सरस्वती आराधना~\#{१८४७}\eventsep मध्व-नवमी\eventsep \tamil{திருச்செந்தூர் முருகன் மாசித் திருவிழா 7ம் நாள்—உருகு சத்தச் சேவை/சிகப்பு சாத்தி அலங்காரம்}\eventsep वैधृति-श्राद्धम्\eventsep श्यामळानवरात्र-समापनम्}
{Thu} 
\cfoot{\rygdata{13:50--15:16}{06:35--08:02}{09:29--10:56}}
\caldata{FEBRUARY}{15}{\sunmonth{कुम्भः}{3}{}{माघः}{शिशिरऋतुः}{शुक्रः}{विलम्बः}{उत्तरायणम्}{शिशिरऋतुः}}
{\sunmoonrsdata{06:35}{18:11}{13:51}{03:09(+1)}{12:23}
{\kalas{04:56 05:45 09:40 08:54 10:27 16:38 11:13 13:32 15:51 17:24 19:00 21:16 22:50 01:55(+1)}}}
{\tnykdata{\anga{\tithi{10}{शुक्ल-दशमी}}{\time{16-48}{13:18}}\hspace{1ex}}%
{\anga{मृगशीर्षम्}{\time{35-39}{20:51}}\hspace{1ex}}{चन्द्रराशिः—\mbox{वृषभः\RIGHTarrow\textsf{09:30}}}%
{\anga{विष्कम्भः}{\time{51-52}{03:20(+1)}}\hspace{1ex}\uanga{प्रीतिः}}%
{\anga{गरः}{\time{16-48}{13:18}}\hspace{1ex}\anga{वणिजः}{\time{44-9}{00:15(+1)}}\hspace{1ex}\uanga{विष्टिः}}{}
}
{साम्ब-दशमी (सूर्यपूजा)\eventsep \tamil{திருச்செந்தூர் முருகன் மாசித் திருவிழா 8ம் நாள்—பச்சை சாத்தி அலங்காரம்}}
{Fri} 
\cfoot{\rygdata{10:56--12:23}{15:17--16:44}{08:02--09:29}}
\caldata{FEBRUARY}{16}{\sunmonth{कुम्भः}{4}{}{माघः}{शिशिरऋतुः}{शनिः}{विलम्बः}{उत्तरायणम्}{शिशिरऋतुः}}
{\sunmoonrsdata{06:34}{18:11}{14:52}{04:09(+1)}{12:23}
{\kalas{04:55 05:45 09:40 08:54 10:27 16:38 11:13 13:32 15:52 17:25 19:00 21:17 22:50 01:55(+1)}}}
{\tnykdata{\anga{\tithi{11}{शुक्ल-एकादशी}}{\time{11-8}{11:02}}\hspace{1ex}}%
{\anga{आर्द्रा}{\time{31-13}{19:04}}\hspace{1ex}}{चन्द्रराशिः—\mbox{मिथुनम्}}%
{\anga{प्रीतिः}{\time{43-27}{23:57}}\hspace{1ex}\uanga{आयुष्मान्}}%
{\anga{विष्टिः}{\time{11-8}{11:02}}\hspace{1ex}\anga{बवः}{\time{37-43}{21:40}}\hspace{1ex}\uanga{बालवः}}{}
}
{हरिवासरः{\RIGHTarrow}\textsf{16:22}\eventsep सर्व-जया/भैमी-एकादशी\eventsep \tamil{திருச்செந்தூர் முருகன் மாசித் திருவிழா 9ம் நாள்—தங்க கைலாச வாஹனம்}}
{Sat} 
\cfoot{\rygdata{09:28--10:56}{13:50--15:17}{06:34--08:01}}
\caldata{FEBRUARY}{17}{\sunmonth{कुम्भः}{5}{}{माघः}{शिशिरऋतुः}{भानुः}{विलम्बः}{उत्तरायणम्}{शिशिरऋतुः}}
{\sunmoonrsdata{06:34}{18:11}{15:56}{05:06(+1)}{12:23}
{\kalas{04:55 05:44 09:40 08:53 10:26 16:38 11:13 13:32 15:52 17:25 19:01 21:17 22:50 01:55(+1)}}}
{\tnykdata{\anga{\tithi{12}{शुक्ल-द्वादशी}}{\time{3-59}{08:10}}\hspace{1ex}\anga{\tithi{13}{शुक्ल-त्रयोदशी}}{\time{55-40}{04:50(+1)}}\hspace{1ex}}%
{\anga{पुनर्वसुः}{\time{25-25}{16:44}}\hspace{1ex}}{चन्द्रराशिः—\mbox{मिथुनम्\RIGHTarrow\textsf{11:21}}}%
{\anga{आयुष्मान्}{\time{33-59}{20:10}}\hspace{1ex}\uanga{सौभाग्यः}}%
{\anga{बालवः}{\time{3-59}{08:10}}\hspace{1ex}\anga{कौलवः}{\time{29-57}{18:33}}\hspace{1ex}\anga{तैतिलः}{\time{55-40}{04:50(+1)}}\hspace{1ex}\uanga{गरः}}{}
}
{भीष्म-द्वादशी\eventsep जयन्ती-महाद्वादशी\eventsep \tamil{குலஶேகர ஆழ்வார் திருநக்ஷத்திரம்}\eventsep प्रदोष-व्रतम्\eventsep रविपुष्ययोग-पुण्यकालः\eventsep तिलपद्म-द्वादशी/तिलोत्पत्ति\eventsep \tamil{திருச்செந்தூர் முருகன் மாசித் திருவிழா 10ம் நாள்—தேர்}\eventsep वराह-द्वादशी\eventsep वराह-कल्पादिः}
{Sun} 
\cfoot{\rygdata{16:44--18:11}{12:23--13:50}{15:17--16:44}}
\caldata{FEBRUARY}{18}{\sunmonth{कुम्भः}{6}{}{माघः}{शिशिरऋतुः}{सोमः}{विलम्बः}{उत्तरायणम्}{शिशिरऋतुः}}
{\sunmoonrsdata{06:34}{18:11}{17:00}{06:01(+1)}{12:23}
{\kalas{04:55 05:44 09:40 08:53 10:26 16:38 11:13 13:32 15:52 17:25 19:01 21:17 22:50 01:55(+1)}}}
{\tnykdata{\anga{\tithi{14}{शुक्ल-चतुर्दशी}}{\time{46-33}{01:11(+1)}}\hspace{1ex}}%
{\anga{पुष्यः}{\time{18-36}{14:00}}\hspace{1ex}}{चन्द्रराशिः—\mbox{कटकः}}%
{\anga{सौभाग्यः}{\time{23-44}{16:03}}\hspace{1ex}\uanga{शोभनः}}%
{\anga{गरः}{\time{21-11}{15:02}}\hspace{1ex}\anga{वणिजः}{\time{46-33}{01:11(+1)}}\hspace{1ex}\uanga{विष्टिः}}{}
}
{षडशीति-पुण्यकालः~\textsf{04:33(+1)}{\RIGHTarrow}\textsf{28:33(+1)}\eventsep \tamil{நடராஜர் மஹாபிஷேகம்}\eventsep तपो-मासः{\RIGHTarrow}\textsf{04:33(+1)}\eventsep \tamil{திருச்செந்தூர் முருகன் தெப்பம்}}
{Mon} 
\cfoot{\rygdata{08:01--09:28}{10:55--12:23}{13:50--15:17}}
\caldata{FEBRUARY}{19}{\sunmonth{कुम्भः}{7}{}{माघः}{शिशिरऋतुः}{मङ्गलः}{विलम्बः}{उत्तरायणम्}{शिशिरऋतुः}}
{\sunmoonrsdata{06:33}{18:12}{18:03}{---}{12:22}
{\kalas{04:54 05:44 09:39 08:53 10:26 16:39 11:13 13:32 15:52 17:25 19:01 21:17 22:50 01:55(+1)}}}
{\tnykdata{\anga{\tithi{15}{पौर्णमासी}}{\time{37-4}{21:23}}\hspace{1ex}}%
{\anga{आश्रेषा}{\time{11-9}{11:01}}\hspace{1ex}}{चन्द्रराशिः—\mbox{कटकः\RIGHTarrow\textsf{11:01}}}%
{\anga{शोभनः}{\time{13-1}{11:46}}\hspace{1ex}\uanga{अतिगण्डः}}%
{\anga{विष्टिः}{\time{11-51}{11:18}}\hspace{1ex}\anga{बवः}{\time{37-4}{21:23}}\hspace{1ex}\uanga{बालवः}}{}
}
{ललिता-जयन्ती\eventsep \tamil{மாசி~செவ்வாய்}\eventsep माघ-पूर्णिमा\eventsep माघ-पूर्णिमा-स्नानम्\eventsep पार्वणव्रतम् पूर्णिमायाम्\eventsep पूर्णिमा-व्रतम्\eventsep \tamil{திருச்செந்தூர் மாசித் திருவிழா நிறைவு}\eventsep वेङ्कटाचले पूर्णिमा-गरुड-सेवा}
{Tue} 
\cfoot{\rygdata{15:17--16:44}{09:28--10:55}{12:22--13:50}}
\caldata{FEBRUARY}{20}{\sunmonth{कुम्भः}{8}{}{माघः}{शिशिरऋतुः}{बुधः}{विलम्बः}{उत्तरायणम्}{शिशिरऋतुः}}
{\sunmoonsrdata{06:33}{18:12}{19:05}{06:52}{12:22}
{\kalas{04:54 05:43 09:39 08:53 10:26 16:39 11:12 13:32 15:52 17:25 19:01 21:17 22:50 01:55(+1)}}}
{\tnykdata{\anga{\tithi{16}{कृष्ण-प्रथमा}}{\time{27-38}{17:36}}\hspace{1ex}}%
{\anga{मघा}{\time{3-33}{07:58}}\hspace{1ex}\anga{पूर्वफल्गुनी}{\time{56-13}{05:02(+1)}}\hspace{1ex}}{चन्द्रराशिः—\mbox{सिंहः}}%
{\anga{अतिगण्डः}{\time{2-9}{07:25}}\hspace{1ex}\anga{सुकर्म}{\time{51-31}{03:09(+1)}}\hspace{1ex}\uanga{धृतिः}}%
{\anga{बालवः}{\time{2-20}{07:29}}\hspace{1ex}\anga{कौलवः}{\time{27-38}{17:36}}\hspace{1ex}\anga{तैतिलः}{\time{53-4}{03:47(+1)}}\hspace{1ex}\uanga{गरः}}{}
}
{काञ्ची ५१ जगद्गुरु श्री-विद्यातीर्थेन्द्र सरस्वती आराधना~\#{६३४}\eventsep \tamil{மாசி~மகம்}\eventsep पूर्र्णमासेष्टिः\eventsep स्थालीपाकः}
{Wed} 
\cfoot{\rygdata{12:22--13:50}{08:00--09:28}{10:55--12:22}}
\caldata{FEBRUARY}{21}{\sunmonth{कुम्भः}{9}{}{माघः}{शिशिरऋतुः}{गुरुः}{विलम्बः}{उत्तरायणम्}{शिशिरऋतुः}}
{\sunmoonsrdata{06:32}{18:12}{20:04}{07:41}{12:22}
{\kalas{04:54 05:43 09:39 08:52 10:26 16:39 11:12 13:32 15:52 17:26 19:01 21:17 22:50 01:54(+1)}}}
{\tnykdata{\anga{\tithi{17}{कृष्ण-द्वितीया}}{\time{18-43}{14:02}}\hspace{1ex}}%
{\anga{उत्तरफल्गुनी}{\time{49-40}{02:24(+1)}}\hspace{1ex}}{चन्द्रराशिः—\mbox{सिंहः\RIGHTarrow\textsf{10:21}}}%
{\anga{धृतिः}{\time{41-30}{23:08}}\hspace{1ex}\uanga{शूलः}}%
{\anga{गरः}{\time{18-43}{14:02}}\hspace{1ex}\anga{वणिजः}{\time{44-34}{00:22(+1)}}\hspace{1ex}\uanga{विष्टिः}}{}
}
{}
{Thu} 
\cfoot{\rygdata{13:50--15:17}{06:32--08:00}{09:27--10:55}}
\caldata{FEBRUARY}{22}{\sunmonth{कुम्भः}{10}{}{माघः}{शिशिरऋतुः}{शुक्रः}{विलम्बः}{उत्तरायणम्}{शिशिरऋतुः}}
{\sunmoonsrdata{06:32}{18:12}{21:02}{08:27}{12:22}
{\kalas{04:53 05:43 09:39 08:52 10:25 16:39 11:12 13:32 15:52 17:26 19:02 21:17 22:50 01:54(+1)}}}
{\tnykdata{\anga{\tithi{18}{कृष्ण-तृतीया}}{\time{10-44}{10:50}}\hspace{1ex}}%
{\anga{हस्तः}{\time{44-19}{00:16(+1)}}\hspace{1ex}}{चन्द्रराशिः—\mbox{कन्या}}%
{\anga{शूलः}{\time{32-25}{19:30}}\hspace{1ex}\uanga{गण्डः}}%
{\anga{विष्टिः}{\time{10-44}{10:50}}\hspace{1ex}\anga{बवः}{\time{37-14}{21:25}}\hspace{1ex}\uanga{बालवः}}{}
}
{द्विजप्रिय-महागणपति सङ्कटहर-चतुर्थी-व्रतम्\eventsep \tamil{எறிபத்த நாயனார் (7) குருபூஜை}}
{Fri} 
\cfoot{\rygdata{10:55--12:22}{15:17--16:45}{07:59--09:27}}
\caldata{FEBRUARY}{23}{\sunmonth{कुम्भः}{11}{}{माघः}{शिशिरऋतुः}{शनिः}{विलम्बः}{उत्तरायणम्}{शिशिरऋतुः}}
{\sunmoonsrdata{06:31}{18:13}{21:59}{09:13}{12:22}
{\kalas{04:53 05:42 09:38 08:52 10:25 16:39 11:12 13:32 15:52 17:26 19:02 21:17 22:49 01:54(+1)}}}
{\tnykdata{\anga{\tithi{19}{कृष्ण-चतुर्थी}}{\time{4-8}{08:11}}\hspace{1ex}\anga{\tithi{20}{कृष्ण-पञ्चमी}}{\time{59-14}{06:13(+1)}}\hspace{1ex}}%
{\anga{चित्रा}{\time{40-34}{22:45}}\hspace{1ex}}{चन्द्रराशिः—\mbox{कन्या\RIGHTarrow\textsf{11:25}}}%
{\anga{गण्डः}{\time{24-34}{16:21}}\hspace{1ex}\uanga{वृद्धिः}}%
{\anga{बालवः}{\time{4-8}{08:11}}\hspace{1ex}\anga{कौलवः}{\time{31-27}{19:06}}\hspace{1ex}\anga{तैतिलः}{\time{59-14}{06:13(+1)}}\hspace{1ex}\uanga{गरः}}{}
}
{}
{Sat} 
\cfoot{\rygdata{09:27--10:54}{13:50--15:17}{06:31--07:59}}
\caldata{FEBRUARY}{24}{\sunmonth{कुम्भः}{12}{}{माघः}{शिशिरऋतुः}{भानुः}{विलम्बः}{उत्तरायणम्}{शिशिरऋतुः}}
{\sunmoonsrdata{06:31}{18:13}{22:55}{10:00}{12:22}
{\kalas{04:52 05:42 09:38 08:51 10:25 16:39 11:12 13:32 15:52 17:26 19:02 21:17 22:49 01:54(+1)}}}
{\tnykdata{\anga{\tithi{21}{कृष्ण-षष्ठी}}{\time{56-23}{05:04(+1)}}\hspace{1ex}}%
{\anga{स्वाती}{\time{38-44}{22:01}}\hspace{1ex}}{चन्द्रराशिः—\mbox{तुला}}%
{\anga{वृद्धिः}{\time{18-14}{13:49}}\hspace{1ex}\uanga{ध्रुवः}}%
{\anga{गरः}{\time{27-33}{17:32}}\hspace{1ex}\anga{वणिजः}{\time{56-23}{05:04(+1)}}\hspace{1ex}\uanga{विष्टिः}}{}
}
{यशोदा-जयन्ती}
{Sun} 
\cfoot{\rygdata{16:45--18:13}{12:22--13:50}{15:17--16:45}}
\caldata{FEBRUARY}{25}{\sunmonth{कुम्भः}{13}{}{माघः}{शिशिरऋतुः}{सोमः}{विलम्बः}{उत्तरायणम्}{शिशिरऋतुः}}
{\sunmoonsrdata{06:30}{18:13}{23:51}{10:47}{12:22}
{\kalas{04:52 05:41 09:38 08:51 10:25 16:39 11:11 13:32 15:52 17:26 19:02 21:17 22:49 01:54(+1)}}}
{\tnykdata{\anga{\tithi{22}{कृष्ण-सप्तमी}}{\time{55-40}{04:47(+1)}}\hspace{1ex}}%
{\anga{विशाखा}{\time{38-58}{22:06}}\hspace{1ex}}{चन्द्रराशिः—\mbox{तुला\RIGHTarrow\textsf{16:00}}}%
{\anga{ध्रुवः}{\time{13-32}{11:55}}\hspace{1ex}\uanga{व्याघातः}}%
{\anga{विष्टिः}{\time{25-46}{16:49}}\hspace{1ex}\anga{बवः}{\time{55-40}{04:47(+1)}}\hspace{1ex}\uanga{बालवः}}{}
}
{माघ-अष्टका-पूर्वेद्युः\eventsep निक्षुभार्क-सप्तमी\eventsep शबरी-जयन्ती}
{Mon} 
\cfoot{\rygdata{07:58--09:26}{10:54--12:22}{13:50--15:17}}
\caldata{FEBRUARY}{26}{\sunmonth{कुम्भः}{14}{}{माघः}{शिशिरऋतुः}{मङ्गलः}{विलम्बः}{उत्तरायणम्}{शिशिरऋतुः}}
{\sunmoonsrdata{06:30}{18:13}{00:46(+1)}{11:36}{12:21}
{\kalas{04:52 05:41 09:37 08:50 10:24 16:39 11:11 13:32 15:52 17:26 19:02 21:17 22:49 01:53(+1)}}}
{\tnykdata{\anga{\tithi{23}{कृष्ण-अष्टमी}}{\time{57-6}{05:20(+1)}}\hspace{1ex}}%
{\anga{अनूराधा}{\time{41-19}{23:01}}\hspace{1ex}}{चन्द्रराशिः—\mbox{वृश्चिकः}}%
{\anga{व्याघातः}{\time{10-32}{10:43}}\hspace{1ex}\uanga{हर्षणः}}%
{\anga{बालवः}{\time{26-8}{16:57}}\hspace{1ex}\anga{कौलवः}{\time{57-6}{05:20(+1)}}\hspace{1ex}\uanga{तैतिलः}}{}
}
{काञ्ची ६६ जगद्गुरु श्री-चन्द्रशेखरेन्द्र सरस्वती ६ आराधना~\#{११२}\eventsep \tamil{மாசி~செவ்வாய்}\eventsep माघ-अष्टका-श्राद्धम्}
{Tue} 
\cfoot{\rygdata{15:17--16:45}{09:26--10:54}{12:22--13:49}}
\caldata{FEBRUARY}{27}{\sunmonth{कुम्भः}{15}{}{माघः}{शिशिरऋतुः}{बुधः}{विलम्बः}{उत्तरायणम्}{शिशिरऋतुः}}
{\sunmoonsrdata{06:29}{18:13}{01:38(+1)}{12:26}{12:21}
{\kalas{04:51 05:40 09:37 08:50 10:24 16:40 11:11 13:32 15:53 17:27 19:02 21:17 22:49 01:53(+1)}}}
{\tnykdata{\fulltithi{\tithi{24}{कृष्ण-नवमी}}}%
{\anga{ज्येष्ठा}{\time{45-35}{00:43(+1)}}\hspace{1ex}}{चन्द्रराशिः—\mbox{वृश्चिकः\RIGHTarrow\textsf{00:43(+1)}}}%
{\anga{हर्षणः}{\time{9-10}{10:09}}\hspace{1ex}\uanga{वज्रम्}}%
{\anga{तैतिलः}{\time{28-34}{17:55}}\hspace{1ex}\uanga{गरः}}{}
}
{माघ-अन्वष्टका-श्राद्धम्}
{Wed} 
\cfoot{\rygdata{12:21--13:49}{07:57--09:25}{10:53--12:21}}
\caldata{FEBRUARY}{28}{\sunmonth{कुम्भः}{16}{}{माघः}{शिशिरऋतुः}{गुरुः}{विलम्बः}{उत्तरायणम्}{शिशिरऋतुः}}
{\sunmoonsrdata{06:29}{18:14}{02:29(+1)}{13:16}{12:21}
{\kalas{04:51 05:40 09:37 08:50 10:24 16:40 11:11 13:32 15:53 17:27 19:03 21:17 22:49 01:53(+1)}}}
{\tnykdata{\anga{\tithi{24}{कृष्ण-नवमी}}{\time{0-29}{06:41}}\hspace{1ex}}%
{\anga{मूला}{\time{51-28}{03:04(+1)}}\hspace{1ex}}{चन्द्रराशिः—\mbox{धनुः}}%
{\anga{वज्रम्}{\time{9-14}{10:10}}\hspace{1ex}\uanga{सिद्धिः}}%
{\anga{गरः}{\time{0-29}{06:41}}\hspace{1ex}\anga{वणिजः}{\time{32-47}{19:36}}\hspace{1ex}\uanga{विष्टिः}}{}
}
{}
{Thu} 
\cfoot{\rygdata{13:49--15:17}{06:29--07:57}{09:25--10:53}}
\caldata{MARCH}{1}{\sunmonth{कुम्भः}{17}{}{माघः}{शिशिरऋतुः}{शुक्रः}{विलम्बः}{उत्तरायणम्}{शिशिरऋतुः}}
{\sunmoonsrdata{06:28}{18:14}{03:17(+1)}{14:07}{12:21}
{\kalas{04:50 05:39 09:36 08:49 10:23 16:40 11:10 13:32 15:53 17:27 19:03 21:17 22:49 01:52(+1)}}}
{\tnykdata{\anga{\tithi{25}{कृष्ण-दशमी}}{\time{5-26}{08:39}}\hspace{1ex}}%
{\anga{पूर्वाषाढा}{\time{58-31}{05:53(+1)}}\hspace{1ex}}{चन्द्रराशिः—\mbox{धनुः}}%
{\anga{सिद्धिः}{\time{10-27}{10:39}}\hspace{1ex}\uanga{व्यतीपातः}}%
{\anga{विष्टिः}{\time{5-26}{08:39}}\hspace{1ex}\anga{बवः}{\time{38-22}{21:49}}\hspace{1ex}\uanga{बालवः}}{}
}
{\tamil{காரி நாயனார் (47) குருபூஜை}\eventsep व्यतीपात-श्राद्धम्}
{Fri} 
\cfoot{\rygdata{10:53--12:21}{15:17--16:46}{07:56--09:25}}
\caldata{MARCH}{2}{\sunmonth{कुम्भः}{18}{}{माघः}{शिशिरऋतुः}{शनिः}{विलम्बः}{उत्तरायणम्}{शिशिरऋतुः}}
{\sunmoonsrdata{06:28}{18:14}{04:02(+1)}{14:57}{12:21}
{\kalas{04:50 05:39 09:36 08:49 10:23 16:40 11:10 13:31 15:53 17:27 19:03 21:17 22:49 01:52(+1)}}}
{\tnykdata{\anga{\tithi{26}{कृष्ण-एकादशी}}{\time{11-31}{11:04}}\hspace{1ex}}%
{\fullanga{उत्तराषाढा}}{चन्द्रराशिः—\mbox{धनुः\RIGHTarrow\textsf{12:38}}}%
{\anga{व्यतीपातः}{\time{12-28}{11:27}}\hspace{1ex}\uanga{वरीयान्}}%
{\anga{बालवः}{\time{11-31}{11:04}}\hspace{1ex}\anga{कौलवः}{\time{44-48}{00:23(+1)}}\hspace{1ex}\uanga{तैतिलः}}{}
}
{हरिवासरः{\RIGHTarrow}\textsf{17:43}\eventsep सर्व-विजया-एकादशी}
{Sat} 
\cfoot{\rygdata{09:24--10:53}{13:49--15:17}{06:28--07:56}}
\caldata{MARCH}{3}{\sunmonth{कुम्भः}{19}{}{माघः}{शिशिरऋतुः}{भानुः}{विलम्बः}{उत्तरायणम्}{शिशिरऋतुः}}
{\sunmoonsrdata{06:27}{18:14}{04:44(+1)}{15:47}{12:21}
{\kalas{04:49 05:38 09:36 08:49 10:23 16:40 11:10 13:31 15:53 17:27 19:03 21:17 22:49 01:52(+1)}}}
{\tnykdata{\anga{\tithi{27}{कृष्ण-द्वादशी}}{\time{18-12}{13:44}}\hspace{1ex}}%
{\anga{उत्तराषाढा}{\time{6-16}{08:58}}\hspace{1ex}}{चन्द्रराशिः—\mbox{मकरः}}%
{\anga{वरीयान्}{\time{14-58}{12:26}}\hspace{1ex}\uanga{परिघः}}%
{\anga{तैतिलः}{\time{18-12}{13:44}}\hspace{1ex}\anga{गरः}{\time{51-38}{03:06(+1)}}\hspace{1ex}\uanga{वणिजः}}{}
}
{काञ्ची जगद्गुरु श्री-शङ्कर विजयेन्द्र सरस्वती जयन्ती~\#{५१}\eventsep प्रदोष-व्रतम्\eventsep विजया/श्रवण-महाद्वादशी\eventsep श्रवण-व्रतम्}
{Sun} 
\cfoot{\rygdata{16:46--18:14}{12:21--13:49}{15:17--16:46}}
\caldata{MARCH}{4}{\sunmonth{कुम्भः}{20}{}{माघः}{शिशिरऋतुः}{सोमः}{विलम्बः}{उत्तरायणम्}{शिशिरऋतुः}}
{\sunmoonsrdata{06:27}{18:14}{05:24(+1)}{16:35}{12:20}
{\kalas{04:49 05:38 09:35 08:48 10:22 16:40 11:10 13:31 15:53 17:27 19:03 21:17 22:49 01:52(+1)}}}
{\tnykdata{\anga{\tithi{28}{कृष्ण-त्रयोदशी}}{\time{25-4}{16:28}}\hspace{1ex}}%
{\anga{श्रवणः}{\time{14-13}{12:08}}\hspace{1ex}}{चन्द्रराशिः—\mbox{मकरः\RIGHTarrow\textsf{01:42(+1)}}}%
{\anga{परिघः}{\time{17-35}{13:29}}\hspace{1ex}\uanga{शिवः}}%
{\anga{वणिजः}{\time{25-4}{16:28}}\hspace{1ex}\anga{विष्टिः}{\time{58-25}{05:49(+1)}}\hspace{1ex}\uanga{शकुनिः}}{}
}
{मासशिवरात्रिः\eventsep महाशिवरात्रिः\eventsep सोमश्रावणी-पुण्यकालः}
{Mon} 
\cfoot{\rygdata{07:55--09:23}{10:52--12:20}{13:49--15:17}}
\caldata{MARCH}{5}{\sunmonth{कुम्भः}{21}{}{माघः}{शिशिरऋतुः}{मङ्गलः}{विलम्बः}{उत्तरायणम्}{शिशिरऋतुः}}
{\sunmoonsrdata{06:26}{18:14}{06:02(+1)}{17:22}{12:20}
{\kalas{04:48 05:37 09:35 08:48 10:22 16:40 11:09 13:31 15:53 17:27 19:03 21:17 22:49 01:51(+1)}}}
{\tnykdata{\anga{\tithi{29}{कृष्ण-चतुर्दशी}}{\time{31-42}{19:07}}\hspace{1ex}}%
{\anga{श्रविष्ठा}{\time{22-1}{15:15}}\hspace{1ex}}{चन्द्रराशिः—\mbox{कुम्भः}}%
{\anga{शिवः}{\time{20-5}{14:28}}\hspace{1ex}\uanga{सिद्धः}}%
{\anga{शकुनिः}{\time{31-42}{19:07}}\hspace{1ex}\uanga{चतुष्पात्}}{}
}
{कृष्णाङ्गारक-चतुर्दशी-पुण्यकालः/यमतर्पणम्\eventsep \tamil{மாசி~செவ்வாய்}}
{Tue} 
\cfoot{\rygdata{15:17--16:46}{09:23--10:52}{12:20--13:49}}
\caldata{MARCH}{6}{\sunmonth{कुम्भः}{22}{}{माघः}{शिशिरऋतुः}{बुधः}{विलम्बः}{उत्तरायणम्}{शिशिरऋतुः}}
{\sunmoonsrdata{06:25}{18:14}{---}{18:08}{12:20}
{\kalas{04:48 05:37 09:34 08:47 10:22 16:40 11:09 13:31 15:53 17:27 19:03 21:17 22:48 01:51(+1)}}}
{\tnykdata{\anga{\tithi{30}{अमावास्या}}{\time{37-50}{21:33}}\hspace{1ex}}%
{\anga{शतभिषक्}{\time{29-23}{18:11}}\hspace{1ex}}{चन्द्रराशिः—\mbox{कुम्भः}}%
{\anga{सिद्धः}{\time{22-15}{15:20}}\hspace{1ex}\uanga{साध्यः}}%
{\anga{चतुष्पात्}{\time{4-51}{08:22}}\hspace{1ex}\anga{नाग}{\time{37-50}{21:33}}\hspace{1ex}\uanga{किंस्तुघ्नः}}{}
}
{कलियुगादिः\eventsep \tamil{கொச்செங்கட் சோழ நாயனார் (59) குருபூஜை}\eventsep माघ-अमावास्या (अलभ्यम्–शतभिषक्)\eventsep माघ-स्नानपूर्तिः\eventsep पार्वणव्रतम् अमावास्यायाम्\eventsep पुरन्दरदास-आराधना~\#{४५५}}
{Wed} 
\cfoot{\rygdata{12:20--13:49}{07:54--09:23}{10:51--12:20}}
\caldata{MARCH}{7}{\sunmonth{कुम्भः}{23}{}{फाल्गुनः}{शिशिरऋतुः}{गुरुः}{विलम्बः}{उत्तरायणम्}{शिशिरऋतुः}}
{\sunmoonrsdata{06:25}{18:15}{06:39}{18:54}{12:20}
{\kalas{04:47 05:36 09:34 08:47 10:21 16:40 11:09 13:31 15:53 17:27 19:03 21:17 22:48 01:51(+1)}}}
{\tnykdata{\anga{\tithi{1}{शुक्ल-प्रथमा}}{\time{43-17}{23:44}}\hspace{1ex}}%
{\anga{पूर्वप्रोष्ठपदा}{\time{36-8}{20:52}}\hspace{1ex}}{चन्द्रराशिः—\mbox{कुम्भः\RIGHTarrow\textsf{14:13}}}%
{\anga{साध्यः}{\time{23-56}{15:59}}\hspace{1ex}\uanga{शुभः}}%
{\anga{किंस्तुघ्नः}{\time{10-40}{10:41}}\hspace{1ex}\anga{बवः}{\time{43-17}{23:44}}\hspace{1ex}\uanga{बालवः}}{}
}
{दर्शेष्टिः\eventsep काञ्ची ६७ जगद्गुरु श्री-महादेवेन्द्र सरस्वती ५ आराधना~\#{११२}\eventsep पयोव्रत-आरम्भः\eventsep स्थालीपाकः}
{Thu} 
\cfoot{\rygdata{13:48--15:17}{06:25--07:54}{09:22--10:51}}
\caldata{MARCH}{8}{\sunmonth{कुम्भः}{24}{}{फाल्गुनः}{शिशिरऋतुः}{शुक्रः}{विलम्बः}{उत्तरायणम्}{शिशिरऋतुः}}
{\sunmoonrsdata{06:24}{18:15}{07:15}{19:41}{12:19}
{\kalas{04:47 05:36 09:34 08:46 10:21 16:40 11:08 13:31 15:53 17:27 19:03 21:17 22:48 01:50(+1)}}}
{\tnykdata{\anga{\tithi{2}{शुक्ल-द्वितीया}}{\time{47-54}{01:34(+1)}}\hspace{1ex}}%
{\anga{उत्तरप्रोष्ठपदा}{\time{42-6}{23:15}}\hspace{1ex}}{चन्द्रराशिः—\mbox{मीनः}}%
{\anga{शुभः}{\time{25-1}{16:25}}\hspace{1ex}\uanga{शुक्लः}}%
{\anga{बालवः}{\time{15-43}{12:41}}\hspace{1ex}\anga{कौलवः}{\time{47-54}{01:34(+1)}}\hspace{1ex}\uanga{तैतिलः}}{}
}
{चन्द्र-दर्शनम्\eventsep फूलेरा-दूज्\eventsep रामकृष्ण-परमहंस-जयन्ती~\#{१८४}}
{Fri} 
\cfoot{\rygdata{10:51--12:20}{15:17--16:46}{07:53--09:22}}
\caldata{MARCH}{9}{\sunmonth{कुम्भः}{25}{}{फाल्गुनः}{शिशिरऋतुः}{शनिः}{विलम्बः}{उत्तरायणम्}{शिशिरऋतुः}}
{\sunmoonrsdata{06:24}{18:15}{07:53}{20:28}{12:19}
{\kalas{04:46 05:35 09:33 08:46 10:21 16:40 11:08 13:30 15:53 17:28 19:03 21:17 22:48 01:50(+1)}}}
{\tnykdata{\anga{\tithi{3}{शुक्ल-तृतीया}}{\time{51-37}{03:02(+1)}}\hspace{1ex}}%
{\anga{रेवती}{\time{47-12}{01:17(+1)}}\hspace{1ex}}{चन्द्रराशिः—\mbox{मीनः\RIGHTarrow\textsf{01:17(+1)}}}%
{\anga{शुक्लः}{\time{25-28}{16:35}}\hspace{1ex}\uanga{ब्रह्म}}%
{\anga{तैतिलः}{\time{19-53}{14:21}}\hspace{1ex}\anga{गरः}{\time{51-37}{03:02(+1)}}\hspace{1ex}\uanga{वणिजः}}{}
}
{}
{Sat} 
\cfoot{\rygdata{09:21--10:50}{13:48--15:17}{06:24--07:53}}
\caldata{MARCH}{10}{\sunmonth{कुम्भः}{26}{}{फाल्गुनः}{शिशिरऋतुः}{भानुः}{विलम्बः}{उत्तरायणम्}{शिशिरऋतुः}}
{\sunmoonrsdata{06:23}{18:15}{08:32}{21:18}{12:19}
{\kalas{04:46 05:34 09:33 08:45 10:20 16:40 11:08 13:30 15:53 17:28 19:03 21:17 22:48 01:50(+1)}}}
{\tnykdata{\anga{\tithi{4}{शुक्ल-चतुर्थी}}{\time{54-18}{04:06(+1)}}\hspace{1ex}}%
{\anga{अश्विनी}{\time{51-20}{02:55(+1)}}\hspace{1ex}}{चन्द्रराशिः—\mbox{मेषः}}%
{\anga{ब्रह्म}{\time{25-10}{16:27}}\hspace{1ex}\uanga{इन्द्रः}}%
{\anga{वणिजः}{\time{23-6}{15:38}}\hspace{1ex}\anga{विष्टिः}{\time{54-18}{04:06(+1)}}\hspace{1ex}\uanga{बवः}}{}
}
{पून्तानं-जयन्ती~\#{४७२}}
{Sun} 
\cfoot{\rygdata{16:46--18:15}{12:19--13:48}{15:17--16:46}}
\caldata{MARCH}{11}{\sunmonth{कुम्भः}{27}{}{फाल्गुनः}{शिशिरऋतुः}{सोमः}{विलम्बः}{उत्तरायणम्}{शिशिरऋतुः}}
{\sunmoonrsdata{06:22}{18:15}{09:13}{22:09}{12:19}
{\kalas{04:45 05:34 09:32 08:45 10:20 16:40 11:07 13:30 15:52 17:28 19:04 21:17 22:48 01:49(+1)}}}
{\tnykdata{\anga{\tithi{5}{शुक्ल-पञ्चमी}}{\time{55-51}{04:43(+1)}}\hspace{1ex}}%
{\anga{अपभरणी}{\time{54-24}{04:08(+1)}}\hspace{1ex}}{चन्द्रराशिः—\mbox{मेषः}}%
{\anga{इन्द्रः}{\time{24-3}{16:00}}\hspace{1ex}\uanga{वैधृतिः}}%
{\anga{बवः}{\time{25-14}{16:28}}\hspace{1ex}\anga{बालवः}{\time{55-51}{04:43(+1)}}\hspace{1ex}\uanga{कौलवः}}{}
}
{\tamil{கபாலீ த்வஜாரோஹணம்}}
{Mon} 
\cfoot{\rygdata{07:51--09:21}{10:50--12:19}{13:48--15:17}}
\caldata{MARCH}{12}{\sunmonth{कुम्भः}{28}{}{फाल्गुनः}{शिशिरऋतुः}{मङ्गलः}{विलम्बः}{उत्तरायणम्}{शिशिरऋतुः}}
{\sunmoonrsdata{06:22}{18:15}{09:58}{23:03}{12:18}
{\kalas{04:45 05:33 09:32 08:44 10:20 16:40 11:07 13:30 15:52 17:28 19:04 21:17 22:47 01:49(+1)}}}
{\tnykdata{\anga{\tithi{6}{शुक्ल-षष्ठी}}{\time{56-9}{04:50(+1)}}\hspace{1ex}}%
{\anga{कृत्तिका}{\time{56-14}{04:52(+1)}}\hspace{1ex}}{चन्द्रराशिः—\mbox{मेषः\RIGHTarrow\textsf{10:22}}}%
{\anga{वैधृतिः}{\time{22-1}{15:10}}\hspace{1ex}\uanga{विष्कम्भः}}%
{\anga{कौलवः}{\time{26-11}{16:50}}\hspace{1ex}\anga{तैतिलः}{\time{56-9}{04:50(+1)}}\hspace{1ex}\uanga{गरः}}{}
}
{षष्ठी-व्रतम्\eventsep कृत्तिका-व्रतम्\eventsep \tamil{கபாலீ ஸூர்ய~சந்த்ர~வட்டம்}\eventsep \tamil{மாசி~செவ்வாய்}\eventsep वैधृति-श्राद्धम्}
{Tue} 
\cfoot{\rygdata{15:17--16:46}{09:20--10:49}{12:19--13:48}}
\caldata{MARCH}{13}{\sunmonth{कुम्भः}{29}{}{फाल्गुनः}{शिशिरऋतुः}{बुधः}{विलम्बः}{उत्तरायणम्}{शिशिरऋतुः}}
{\sunmoonrsdata{06:21}{18:15}{10:47}{00:00(+1)}{12:18}
{\kalas{04:44 05:33 09:31 08:44 10:19 16:40 11:07 13:30 15:52 17:28 19:04 21:17 22:47 01:49(+1)}}}
{\tnykdata{\anga{\tithi{7}{शुक्ल-सप्तमी}}{\time{55-4}{04:23(+1)}}\hspace{1ex}}%
{\anga{रोहिणी}{\time{56-44}{05:03(+1)}}\hspace{1ex}}{चन्द्रराशिः—\mbox{वृषभः}}%
{\anga{विष्कम्भः}{\time{18-59}{13:57}}\hspace{1ex}\uanga{प्रीतिः}}%
{\anga{गरः}{\time{25-48}{16:40}}\hspace{1ex}\anga{वणिजः}{\time{55-4}{04:23(+1)}}\hspace{1ex}\uanga{विष्टिः}}{}
}
{\tamil{கபாலீ அதிகார நந்தி}\eventsep \tamil{கபாலீ பூதண் பூதகீ}\eventsep नन्दा-सप्तमी\eventsep श्री-राघवेन्द्र-स्वामि-जयन्ती~\#{४२५}}
{Wed} 
\cfoot{\rygdata{12:18--13:48}{07:50--09:20}{10:49--12:18}}
\caldata{MARCH}{14}{\sunmonth{कुम्भः}{30}{\mbox{कुम्भः{\tiny\RIGHTarrow}\textsf{05:24(+1)}}}{फाल्गुनः}{शिशिरऋतुः}{गुरुः}{विलम्बः}{उत्तरायणम्}{शिशिरऋतुः}}
{\sunmoonrsdata{06:20}{18:15}{11:41}{00:57(+1)}{12:18}
{\kalas{04:44 05:32 09:31 08:43 10:19 16:40 11:06 13:29 15:52 17:28 19:04 21:16 22:47 01:48(+1)}}}
{\tnykdata{\anga{\tithi{8}{शुक्ल-अष्टमी}}{\time{52-32}{03:21(+1)}}\hspace{1ex}}%
{\anga{मृगशीर्षम्}{\time{55-49}{04:40(+1)}}\hspace{1ex}}{चन्द्रराशिः—\mbox{वृषभः\RIGHTarrow\textsf{16:56}}}%
{\anga{प्रीतिः}{\time{14-50}{12:17}}\hspace{1ex}\uanga{आयुष्मान्}}%
{\anga{विष्टिः}{\time{24-0}{15:57}}\hspace{1ex}\anga{बवः}{\time{52-32}{03:21(+1)}}\hspace{1ex}\uanga{बालवः}}{}
}
{मीन-रवि-सङ्क्रमण-षडशीति-पुण्यकालः~\textsf{05:23(+1)}{\RIGHTarrow}\textsf{29:23(+1)}}
{Thu} 
\cfoot{\rygdata{13:47--15:17}{06:20--07:50}{09:19--10:49}}
\caldata{MARCH}{15}{\sunmonth{मीनः}{1}{}{फाल्गुनः}{शिशिरऋतुः}{शुक्रः}{विलम्बः}{उत्तरायणम्}{शिशिरऋतुः}}
{\sunmoonrsdata{06:20}{18:15}{12:38}{01:55(+1)}{12:18}
{\kalas{04:43 05:31 09:31 08:43 10:18 16:40 11:06 13:29 15:52 17:28 19:04 21:16 22:47 01:48(+1)}}}
{\tnykdata{\anga{\tithi{9}{शुक्ल-नवमी}}{\time{48-30}{01:44(+1)}}\hspace{1ex}}%
{\anga{आर्द्रा}{\time{53-26}{03:43(+1)}}\hspace{1ex}}{चन्द्रराशिः—\mbox{मिथुनम्}}%
{\anga{आयुष्मान्}{\time{9-31}{10:08}}\hspace{1ex}\uanga{सौभाग्यः}}%
{\anga{बालवः}{\time{20-43}{14:37}}\hspace{1ex}\anga{कौलवः}{\time{48-30}{01:44(+1)}}\hspace{1ex}\uanga{तैतिलः}}{}
}
{\tamil{காரடையான் நோன்பு}\eventsep \tamil{கபாலீ சவுடல் விமானம்}\eventsep \tamil{கபாலீ ரிஷப வாஹனம்}}
{Fri} 
\cfoot{\rygdata{10:48--12:18}{15:16--16:46}{07:49--09:19}}
\caldata{MARCH}{16}{\sunmonth{मीनः}{2}{}{फाल्गुनः}{शिशिरऋतुः}{शनिः}{विलम्बः}{उत्तरायणम्}{शिशिरऋतुः}}
{\sunmoonrsdata{06:19}{18:16}{13:39}{02:52(+1)}{12:17}
{\kalas{04:43 05:31 09:30 08:42 10:18 16:40 11:06 13:29 15:52 17:28 19:04 21:16 22:47 01:47(+1)}}}
{\tnykdata{\anga{\tithi{10}{शुक्ल-दशमी}}{\time{43-4}{23:33}}\hspace{1ex}}%
{\anga{पुनर्वसुः}{\time{49-40}{02:11(+1)}}\hspace{1ex}}{चन्द्रराशिः—\mbox{मिथुनम्\RIGHTarrow\textsf{20:37}}}%
{\anga{सौभाग्यः}{\time{3-2}{07:32}}\hspace{1ex}\anga{शोभनः}{\time{55-24}{04:29(+1)}}\hspace{1ex}\uanga{अतिगण्डः}}%
{\anga{तैतिलः}{\time{15-58}{12:43}}\hspace{1ex}\anga{गरः}{\time{43-4}{23:33}}\hspace{1ex}\uanga{वणिजः}}{}
}
{\tamil{கபாலீ பல்லக்கு விழா}\eventsep वेङ्कटाचले प्लवोत्सव-आरम्भः}
{Sat} 
\cfoot{\rygdata{09:18--10:48}{13:47--15:16}{06:19--07:49}}
\caldata{MARCH}{17}{\sunmonth{मीनः}{3}{}{फाल्गुनः}{शिशिरऋतुः}{भानुः}{विलम्बः}{उत्तरायणम्}{शिशिरऋतुः}}
{\sunmoonrsdata{06:19}{18:16}{14:41}{03:46(+1)}{12:17}
{\kalas{04:42 05:30 09:30 08:42 10:18 16:40 11:05 13:29 15:52 17:28 19:04 21:16 22:46 01:47(+1)}}}
{\tnykdata{\anga{\tithi{11}{शुक्ल-एकादशी}}{\time{36-19}{20:50}}\hspace{1ex}}%
{\anga{पुष्यः}{\time{44-38}{00:10(+1)}}\hspace{1ex}}{चन्द्रराशिः—\mbox{कटकः}}%
{\anga{अतिगण्डः}{\time{46-47}{01:01(+1)}}\hspace{1ex}\uanga{सुकर्म}}%
{\anga{वणिजः}{\time{9-51}{10:15}}\hspace{1ex}\anga{विष्टिः}{\time{36-19}{20:50}}\hspace{1ex}\uanga{बवः}}{}
}
{\tamil{கபாலீ தேர்}\eventsep \tamil{முனையடுவார் நாயனார் (50) குருபூஜை}\eventsep रंगभरी एकादशी\eventsep रविपुष्ययोग-पुण्यकालः\eventsep सर्व-आमलकी-एकादशी\eventsep वेङ्कटाचले प्लवोत्सवः}
{Sun} 
\cfoot{\rygdata{16:46--18:16}{12:17--13:47}{15:16--16:46}}
\caldata{MARCH}{18}{\sunmonth{मीनः}{4}{}{फाल्गुनः}{शिशिरऋतुः}{सोमः}{विलम्बः}{उत्तरायणम्}{शिशिरऋतुः}}
{\sunmoonrsdata{06:18}{18:16}{15:43}{04:37(+1)}{12:17}
{\kalas{04:42 05:30 09:29 08:41 10:17 16:40 11:05 13:29 15:52 17:28 19:04 21:16 22:46 01:47(+1)}}}
{\tnykdata{\anga{\tithi{12}{शुक्ल-द्वादशी}}{\time{28-33}{17:43}}\hspace{1ex}}%
{\anga{आश्रेषा}{\time{38-36}{21:45}}\hspace{1ex}}{चन्द्रराशिः—\mbox{कटकः\RIGHTarrow\textsf{21:45}}}%
{\anga{सुकर्म}{\time{37-22}{21:15}}\hspace{1ex}\uanga{धृतिः}}%
{\anga{बवः}{\time{2-34}{07:20}}\hspace{1ex}\anga{बालवः}{\time{28-33}{17:43}}\hspace{1ex}\anga{कौलवः}{\time{54-21}{04:02(+1)}}\hspace{1ex}\uanga{तैतिलः}}{}
}
{हरिवासरः{\RIGHTarrow}\textsf{02:06}\eventsep \tamil{கபாலீ அறுபத்து மூவர்}\eventsep नरसिंह-द्वादशी\eventsep पयोव्रत-समापनम्\eventsep सोम-प्रदोष-व्रतम्\eventsep वेङ्कटाचले प्लवोत्सवः}
{Mon} 
\cfoot{\rygdata{07:48--09:17}{10:47--12:17}{13:46--15:16}}
\caldata{MARCH}{19}{\sunmonth{मीनः}{5}{}{फाल्गुनः}{शिशिरऋतुः}{मङ्गलः}{विलम्बः}{उत्तरायणम्}{शिशिरऋतुः}}
{\sunmoonrsdata{06:17}{18:16}{16:44}{05:26(+1)}{12:16}
{\kalas{04:41 05:29 09:29 08:41 10:17 16:40 11:05 13:28 15:52 17:28 19:04 21:16 22:46 01:46(+1)}}}
{\tnykdata{\anga{\tithi{13}{शुक्ल-त्रयोदशी}}{\time{20-2}{14:18}}\hspace{1ex}}%
{\anga{मघा}{\time{31-55}{19:03}}\hspace{1ex}}{चन्द्रराशिः—\mbox{सिंहः}}%
{\anga{धृतिः}{\time{27-23}{17:14}}\hspace{1ex}\uanga{शूलः}}%
{\anga{तैतिलः}{\time{20-2}{14:18}}\hspace{1ex}\anga{गरः}{\time{45-36}{00:32(+1)}}\hspace{1ex}\uanga{वणिजः}}{}
}
{काञ्ची ६९ जगद्गुरु श्री-जयेन्द्र सरस्वती आराधना~\#{१}\eventsep वेङ्कटाचले प्लवोत्सवः}
{Tue} 
\cfoot{\rygdata{15:16--16:46}{09:17--10:47}{12:16--13:46}}
\caldata{MARCH}{20}{\sunmonth{मीनः}{6}{}{फाल्गुनः}{शिशिरऋतुः}{बुधः}{विलम्बः}{उत्तरायणम्}{शिशिरऋतुः}}
{\sunmoonrsdata{06:16}{18:16}{17:44}{06:14(+1)}{12:16}
{\kalas{04:40 05:28 09:28 08:40 10:16 16:40 11:04 13:28 15:52 17:28 19:04 21:16 22:46 01:46(+1)}}}
{\tnykdata{\anga{\tithi{14}{शुक्ल-चतुर्दशी}}{\time{11-10}{10:45}}\hspace{1ex}}%
{\anga{पूर्वफल्गुनी}{\time{24-57}{16:16}}\hspace{1ex}}{चन्द्रराशिः—\mbox{सिंहः\RIGHTarrow\textsf{21:34}}}%
{\anga{शूलः}{\time{17-9}{13:08}}\hspace{1ex}\uanga{गण्डः}}%
{\anga{वणिजः}{\time{11-10}{10:45}}\hspace{1ex}\anga{विष्टिः}{\time{36-42}{20:58}}\hspace{1ex}\uanga{बवः}}{}
}
{होलिका-पूर्णिमा\eventsep काम-दहनम्\eventsep \tamil{கற்பகாம்பாள்–கபாலீஶ்வரர் திருக்கல்யாணம்}\eventsep मेष-विषु-पुण्यकालः~\textsf{23:28}{\RIGHTarrow}\textsf{07:28(+1)}\eventsep मन्वादिः-(रुद्रः-[१२])\eventsep पार्वणव्रतम् पूर्णिमायाम्\eventsep तपस्य-मासः/शिशिरऋतुः{\RIGHTarrow}\textsf{03:28(+1)}\eventsep वेङ्कटाचले पूर्णिमा-गरुड-सेवा\eventsep वेङ्कटाचले प्लवोत्सव-समापनम्}
{Wed} 
\cfoot{\rygdata{12:16--13:46}{07:46--09:16}{10:46--12:16}}
\caldata{MARCH}{21}{\sunmonth{मीनः}{7}{}{फाल्गुनः}{शिशिरऋतुः}{गुरुः}{विलम्बः}{उत्तरायणम्}{शिशिरऋतुः}}
{\sunmoonrsdata{06:16}{18:16}{18:43}{---}{12:16}
{\kalas{04:40 05:28 09:28 08:40 10:16 16:40 11:04 13:28 15:52 17:28 19:04 21:16 22:46 01:45(+1)}}}
{\tnykdata{\anga{\tithi{15}{पौर्णमासी}}{\time{2-21}{07:12}}\hspace{1ex}\anga{\tithi{16}{कृष्ण-प्रथमा}}{\time{54-1}{03:52(+1)}}\hspace{1ex}}%
{\anga{उत्तरफल्गुनी}{\time{18-11}{13:32}}\hspace{1ex}}{चन्द्रराशिः—\mbox{कन्या}}%
{\anga{गण्डः}{\time{7-1}{09:04}}\hspace{1ex}\anga{वृद्धिः}{\time{57-19}{05:11(+1)}}\hspace{1ex}\uanga{ध्रुवः}}%
{\anga{बवः}{\time{2-21}{07:12}}\hspace{1ex}\anga{बालवः}{\time{28-5}{17:30}}\hspace{1ex}\anga{कौलवः}{\time{54-1}{03:52(+1)}}\hspace{1ex}\uanga{तैतिलः}}{}
}
{चैतन्य-महाप्रभु-जयन्ती~\#{५३४}\eventsep होलि\eventsep \tamil{கபாலீ உமா-மஹேஶ்வர தரிசனம்}\eventsep \tamil{கபாலீ விடையாற்றி தொடக்கம்}\eventsep पूर्र्णमासेष्टिः\eventsep पूर्णिमा-व्रतम्\eventsep \tamil{பங்குனி~உத்திரம்}\eventsep स्थालीपाकः\eventsep विषुवदिनम्}
{Thu} 
\cfoot{\rygdata{13:46--15:16}{06:16--07:46}{09:16--10:46}}
\caldata{MARCH}{22}{\sunmonth{मीनः}{8}{}{फाल्गुनः}{शिशिरऋतुः}{शुक्रः}{विलम्बः}{उत्तरायणम्}{शिशिरऋतुः}}
{\sunmoonsrdata{06:15}{18:16}{19:42}{07:01}{12:16}
{\kalas{04:39 05:27 09:27 08:39 10:15 16:40 11:03 13:28 15:52 17:28 19:04 21:16 22:45 01:45(+1)}}}
{\tnykdata{\anga{\tithi{17}{कृष्ण-द्वितीया}}{\time{46-40}{00:55(+1)}}\hspace{1ex}}%
{\anga{हस्तः}{\time{12-3}{11:04}}\hspace{1ex}}{चन्द्रराशिः—\mbox{कन्या\RIGHTarrow\textsf{22:00}}}%
{\anga{ध्रुवः}{\time{48-27}{01:38(+1)}}\hspace{1ex}\uanga{व्याघातः}}%
{\anga{तैतिलः}{\time{20-12}{14:20}}\hspace{1ex}\anga{गरः}{\time{46-40}{00:55(+1)}}\hspace{1ex}\uanga{वणिजः}}{}
}
{}
{Fri} 
\cfoot{\rygdata{10:45--12:16}{15:16--16:46}{07:45--09:15}}
\caldata{MARCH}{23}{\sunmonth{मीनः}{9}{}{फाल्गुनः}{शिशिरऋतुः}{शनिः}{विलम्बः}{उत्तरायणम्}{शिशिरऋतुः}}
{\sunmoonsrdata{06:14}{18:16}{20:40}{07:48}{12:15}
{\kalas{04:39 05:27 09:27 08:39 10:15 16:40 11:03 13:27 15:52 17:28 19:04 21:15 22:45 01:45(+1)}}}
{\tnykdata{\anga{\tithi{18}{कृष्ण-तृतीया}}{\time{40-43}{22:32}}\hspace{1ex}}%
{\anga{चित्रा}{\time{7-2}{09:04}}\hspace{1ex}}{चन्द्रराशिः—\mbox{तुला}}%
{\anga{व्याघातः}{\time{40-45}{22:33}}\hspace{1ex}\uanga{हर्षणः}}%
{\anga{वणिजः}{\time{13-30}{11:39}}\hspace{1ex}\anga{विष्टिः}{\time{40-43}{22:32}}\hspace{1ex}\uanga{बवः}}{}
}
{ब्रह्म-कल्पादिः\eventsep छत्रपति-शिवाजी-जयन्ती~\#{३९०}\eventsep काञ्ची जगद्गुरु श्री-जयेन्द्र सरस्वती आश्रम-स्वीकार-जयन्ती~\#{६६}\eventsep \tamil{காரைக்கால் அம்மையார் (23) குருபூஜை}}
{Sat} 
\cfoot{\rygdata{09:15--10:45}{13:45--15:16}{06:14--07:45}}
\caldata{MARCH}{24}{\sunmonth{मीनः}{10}{}{फाल्गुनः}{शिशिरऋतुः}{भानुः}{विलम्बः}{उत्तरायणम्}{शिशिरऋतुः}}
{\sunmoonsrdata{06:14}{18:16}{21:38}{08:36}{12:15}
{\kalas{04:38 05:26 09:26 08:38 10:15 16:40 11:03 13:27 15:52 17:28 19:04 21:15 22:45 01:44(+1)}}}
{\tnykdata{\anga{\tithi{19}{कृष्ण-चतुर्थी}}{\time{36-32}{20:51}}\hspace{1ex}}%
{\anga{स्वाती}{\time{3-34}{07:40}}\hspace{1ex}}{चन्द्रराशिः—\mbox{तुला\RIGHTarrow\textsf{01:06(+1)}}}%
{\anga{हर्षणः}{\time{34-29}{20:02}}\hspace{1ex}\uanga{वज्रम्}}%
{\anga{बवः}{\time{8-24}{09:36}}\hspace{1ex}\anga{बालवः}{\time{36-32}{20:51}}\hspace{1ex}\uanga{कौलवः}}{}
}
{भालचन्द्र-महागणपति सङ्कटहर-चतुर्थी-व्रतम्}
{Sun} 
\cfoot{\rygdata{16:46--18:16}{12:15--13:45}{15:16--16:46}}
\caldata{MARCH}{25}{\sunmonth{मीनः}{11}{}{फाल्गुनः}{शिशिरऋतुः}{सोमः}{विलम्बः}{उत्तरायणम्}{शिशिरऋतुः}}
{\sunmoonsrdata{06:13}{18:16}{22:35}{09:26}{12:15}
{\kalas{04:38 05:25 09:26 08:38 10:14 16:40 11:02 13:27 15:52 17:28 19:04 21:15 22:45 01:44(+1)}}}
{\tnykdata{\anga{\tithi{20}{कृष्ण-पञ्चमी}}{\time{34-26}{20:00}}\hspace{1ex}}%
{\anga{विशाखा}{\time{2-0}{07:01}}\hspace{1ex}}{चन्द्रराशिः—\mbox{वृश्चिकः}}%
{\anga{वज्रम्}{\time{29-52}{18:10}}\hspace{1ex}\uanga{सिद्धिः}}%
{\anga{कौलवः}{\time{5-14}{08:19}}\hspace{1ex}\anga{तैतिलः}{\time{34-26}{20:00}}\hspace{1ex}\uanga{गरः}}{}
}
{रङ्ग-पञ्चमी}
{Mon} 
\cfoot{\rygdata{07:44--09:14}{10:44--12:15}{13:45--15:15}}
\caldata{MARCH}{26}{\sunmonth{मीनः}{12}{}{फाल्गुनः}{शिशिरऋतुः}{मङ्गलः}{विलम्बः}{उत्तरायणम्}{शिशिरऋतुः}}
{\sunmoonsrdata{06:12}{18:16}{23:30}{10:17}{12:14}
{\kalas{04:37 05:25 09:26 08:37 10:14 16:40 11:02 13:27 15:51 17:28 19:04 21:15 22:45 01:43(+1)}}}
{\tnykdata{\anga{\tithi{21}{कृष्ण-षष्ठी}}{\time{34-32}{20:01}}\hspace{1ex}}%
{\anga{अनूराधा}{\time{2-32}{07:13}}\hspace{1ex}}{चन्द्रराशिः—\mbox{वृश्चिकः}}%
{\anga{सिद्धिः}{\time{27-0}{17:00}}\hspace{1ex}\uanga{व्यतीपातः}}%
{\anga{गरः}{\time{4-13}{07:54}}\hspace{1ex}\anga{वणिजः}{\time{34-32}{20:01}}\hspace{1ex}\uanga{विष्टिः}}{}
}
{}
{Tue} 
\cfoot{\rygdata{15:15--16:46}{09:13--10:44}{12:14--13:45}}
\caldata{MARCH}{27}{\sunmonth{मीनः}{13}{}{फाल्गुनः}{शिशिरऋतुः}{बुधः}{विलम्बः}{उत्तरायणम्}{शिशिरऋतुः}}
{\sunmoonsrdata{06:12}{18:16}{00:23(+1)}{11:09}{12:14}
{\kalas{04:36 05:24 09:25 08:37 10:13 16:40 11:02 13:26 15:51 17:28 19:04 21:15 22:44 01:43(+1)}}}
{\tnykdata{\anga{\tithi{22}{कृष्ण-सप्तमी}}{\time{36-47}{20:55}}\hspace{1ex}}%
{\anga{ज्येष्ठा}{\time{5-13}{08:17}}\hspace{1ex}}{चन्द्रराशिः—\mbox{वृश्चिकः\RIGHTarrow\textsf{08:17}}}%
{\anga{व्यतीपातः}{\time{25-48}{16:31}}\hspace{1ex}\uanga{वरीयान्}}%
{\anga{विष्टिः}{\time{5-25}{08:22}}\hspace{1ex}\anga{बवः}{\time{36-47}{20:55}}\hspace{1ex}\uanga{बालवः}}{}
}
{फाल्गुन-अष्टका-पूर्वेद्युः\eventsep व्यतीपात-श्राद्धम्}
{Wed} 
\cfoot{\rygdata{12:14--13:45}{07:42--09:13}{10:43--12:14}}
\caldata{MARCH}{28}{\sunmonth{मीनः}{14}{}{फाल्गुनः}{शिशिरऋतुः}{गुरुः}{विलम्बः}{उत्तरायणम्}{शिशिरऋतुः}}
{\sunmoonsrdata{06:11}{18:16}{01:12(+1)}{12:01}{12:14}
{\kalas{04:36 05:24 09:25 08:36 10:13 16:40 11:01 13:26 15:51 17:28 19:04 21:15 22:44 01:43(+1)}}}
{\tnykdata{\anga{\tithi{23}{कृष्ण-अष्टमी}}{\time{40-57}{22:34}}\hspace{1ex}}%
{\anga{मूला}{\time{9-53}{10:09}}\hspace{1ex}}{चन्द्रराशिः—\mbox{धनुः}}%
{\anga{वरीयान्}{\time{26-8}{16:38}}\hspace{1ex}\uanga{परिघः}}%
{\anga{बालवः}{\time{8-40}{09:39}}\hspace{1ex}\anga{कौलवः}{\time{40-57}{22:34}}\hspace{1ex}\uanga{तैतिलः}}{}
}
{फाल्गुन-अष्टका-श्राद्धम्}
{Thu} 
\cfoot{\rygdata{13:44--15:15}{06:11--07:42}{09:12--10:43}}
\caldata{MARCH}{29}{\sunmonth{मीनः}{15}{}{फाल्गुनः}{शिशिरऋतुः}{शुक्रः}{विलम्बः}{उत्तरायणम्}{शिशिरऋतुः}}
{\sunmoonsrdata{06:10}{18:16}{01:59(+1)}{12:52}{12:13}
{\kalas{04:35 05:23 09:24 08:36 10:13 16:40 11:01 13:26 15:51 17:28 19:04 21:15 22:44 01:42(+1)}}}
{\tnykdata{\anga{\tithi{24}{कृष्ण-नवमी}}{\time{46-33}{00:48(+1)}}\hspace{1ex}}%
{\anga{पूर्वाषाढा}{\time{16-11}{12:39}}\hspace{1ex}}{चन्द्रराशिः—\mbox{धनुः\RIGHTarrow\textsf{19:21}}}%
{\anga{परिघः}{\time{27-39}{17:14}}\hspace{1ex}\uanga{शिवः}}%
{\anga{तैतिलः}{\time{13-37}{11:37}}\hspace{1ex}\anga{गरः}{\time{46-33}{00:48(+1)}}\hspace{1ex}\uanga{वणिजः}}{}
}
{गुरु-सङ्क्रान्तिः~(वृश्चिकः\To{}धनुः)\eventsep फाल्गुन-अन्वष्टका-श्राद्धम्}
{Fri} 
\cfoot{\rygdata{10:43--12:13}{15:15--16:46}{07:41--09:12}}
\caldata{MARCH}{30}{\sunmonth{मीनः}{16}{}{फाल्गुनः}{शिशिरऋतुः}{शनिः}{विलम्बः}{उत्तरायणम्}{शिशिरऋतुः}}
{\sunmoonsrdata{06:10}{18:16}{02:42(+1)}{13:42}{12:13}
{\kalas{04:35 05:22 09:24 08:35 10:12 16:40 11:00 13:26 15:51 17:28 19:04 21:15 22:44 01:42(+1)}}}
{\tnykdata{\anga{\tithi{25}{कृष्ण-दशमी}}{\time{53-2}{03:23(+1)}}\hspace{1ex}}%
{\anga{उत्तराषाढा}{\time{23-34}{15:36}}\hspace{1ex}}{चन्द्रराशिः—\mbox{मकरः}}%
{\anga{शिवः}{\time{29-57}{18:09}}\hspace{1ex}\uanga{सिद्धः}}%
{\anga{वणिजः}{\time{19-44}{14:04}}\hspace{1ex}\anga{विष्टिः}{\time{53-2}{03:23(+1)}}\hspace{1ex}\uanga{बवः}}{}
}
{}
{Sat} 
\cfoot{\rygdata{09:11--10:42}{13:44--15:15}{06:10--07:41}}
\caldata{MARCH}{31}{\sunmonth{मीनः}{17}{}{फाल्गुनः}{शिशिरऋतुः}{भानुः}{विलम्बः}{उत्तरायणम्}{शिशिरऋतुः}}
{\sunmoonsrdata{06:09}{18:17}{03:22(+1)}{14:30}{12:13}
{\kalas{04:34 05:22 09:23 08:35 10:12 16:40 11:00 13:26 15:51 17:28 19:04 21:15 22:44 01:42(+1)}}}
{\tnykdata{\anga{\tithi{26}{कृष्ण-एकादशी}}{\time{59-47}{06:04(+1)}}\hspace{1ex}}%
{\anga{श्रवणः}{\time{31-28}{18:45}}\hspace{1ex}}{चन्द्रराशिः—\mbox{मकरः}}%
{\anga{सिद्धः}{\time{32-37}{19:12}}\hspace{1ex}\uanga{साध्यः}}%
{\anga{बवः}{\time{26-25}{16:43}}\hspace{1ex}\anga{बालवः}{\time{59-47}{06:04(+1)}}\hspace{1ex}\uanga{कौलवः}}{}
}
{\tamil{கபாலீ விடையாற்றி நிறைவு}\eventsep स्मार्त-पापमोचनी-एकादशी\eventsep श्रवण-व्रतम्}
{Sun} 
\cfoot{\rygdata{16:46--18:17}{12:13--13:44}{15:15--16:46}}
\caldata{APRIL}{1}{\sunmonth{मीनः}{18}{}{फाल्गुनः}{शिशिरऋतुः}{सोमः}{विलम्बः}{उत्तरायणम्}{शिशिरऋतुः}}
{\sunmoonsrdata{06:09}{18:17}{04:01(+1)}{15:18}{12:13}
{\kalas{04:34 05:21 09:23 08:34 10:11 16:40 11:00 13:25 15:51 17:28 19:04 21:14 22:43 01:41(+1)}}}
{\tnykdata{\fulltithi{\tithi{27}{कृष्ण-द्वादशी}}}%
{\anga{श्रविष्ठा}{\time{39-19}{21:52}}\hspace{1ex}}{चन्द्रराशिः—\mbox{मकरः\RIGHTarrow\textsf{08:19}}}%
{\anga{साध्यः}{\time{35-13}{20:14}}\hspace{1ex}\uanga{शुभः}}%
{\anga{कौलवः}{\time{33-5}{19:23}}\hspace{1ex}\uanga{तैतिलः}}{}
}
{हरिवासरः{\RIGHTarrow}\textsf{12:44}\eventsep वैष्णव-पापमोचनी-एकादशी\eventsep व्यञ्जुली-महाद्वादशी}
{Mon} 
\cfoot{\rygdata{07:40--09:11}{10:42--12:13}{13:44--15:15}}
\caldata{APRIL}{2}{\sunmonth{मीनः}{19}{}{फाल्गुनः}{शिशिरऋतुः}{मङ्गलः}{विलम्बः}{उत्तरायणम्}{शिशिरऋतुः}}
{\sunmoonsrdata{06:08}{18:17}{04:38(+1)}{16:04}{12:12}
{\kalas{04:33 05:20 09:22 08:34 10:11 16:40 10:59 13:25 15:51 17:28 19:04 21:14 22:43 01:41(+1)}}}
{\tnykdata{\anga{\tithi{27}{कृष्ण-द्वादशी}}{\time{6-16}{08:38}}\hspace{1ex}}%
{\anga{शतभिषक्}{\time{46-39}{00:47(+1)}}\hspace{1ex}}{चन्द्रराशिः—\mbox{कुम्भः}}%
{\anga{शुभः}{\time{37-27}{21:07}}\hspace{1ex}\uanga{शुक्लः}}%
{\anga{तैतिलः}{\time{6-16}{08:38}}\hspace{1ex}\anga{गरः}{\time{39-14}{21:50}}\hspace{1ex}\uanga{वणिजः}}{}
}
{\tamil{தண்டியடிகள் நாயனார் (30) குருபூஜை}\eventsep प्रदोष-व्रतम्}
{Tue} 
\cfoot{\rygdata{15:14--16:46}{09:10--10:41}{12:12--13:43}}
\caldata{APRIL}{3}{\sunmonth{मीनः}{20}{}{फाल्गुनः}{शिशिरऋतुः}{बुधः}{विलम्बः}{उत्तरायणम्}{शिशिरऋतुः}}
{\sunmoonsrdata{06:07}{18:17}{05:15(+1)}{16:51}{12:12}
{\kalas{04:33 05:20 09:22 08:33 10:10 16:40 10:59 13:25 15:51 17:28 19:04 21:14 22:43 01:40(+1)}}}
{\tnykdata{\anga{\tithi{28}{कृष्ण-त्रयोदशी}}{\time{12-2}{10:56}}\hspace{1ex}}%
{\anga{पूर्वप्रोष्ठपदा}{\time{53-9}{03:23(+1)}}\hspace{1ex}}{चन्द्रराशिः—\mbox{कुम्भः\RIGHTarrow\textsf{20:46}}}%
{\anga{शुक्लः}{\time{39-3}{21:45}}\hspace{1ex}\uanga{ब्रह्म}}%
{\anga{वणिजः}{\time{12-2}{10:56}}\hspace{1ex}\anga{विष्टिः}{\time{44-33}{23:57}}\hspace{1ex}\uanga{शकुनिः}}{}
}
{मासशिवरात्रिः}
{Wed} 
\cfoot{\rygdata{12:12--13:43}{07:38--09:10}{10:41--12:12}}
\caldata{APRIL}{4}{\sunmonth{मीनः}{21}{}{फाल्गुनः}{शिशिरऋतुः}{गुरुः}{विलम्बः}{उत्तरायणम्}{शिशिरऋतुः}}
{\sunmoonsrdata{06:07}{18:17}{05:53(+1)}{17:37}{12:12}
{\kalas{04:32 05:19 09:21 08:33 10:10 16:39 10:59 13:25 15:51 17:28 19:04 21:14 22:43 01:40(+1)}}}
{\tnykdata{\anga{\tithi{29}{कृष्ण-चतुर्दशी}}{\time{16-50}{12:51}}\hspace{1ex}}%
{\anga{उत्तरप्रोष्ठपदा}{\time{58-39}{05:34(+1)}}\hspace{1ex}}{चन्द्रराशिः—\mbox{मीनः}}%
{\anga{ब्रह्म}{\time{39-54}{22:04}}\hspace{1ex}\uanga{इन्द्रः}}%
{\anga{शकुनिः}{\time{16-50}{12:51}}\hspace{1ex}\anga{चतुष्पात्}{\time{48-50}{01:39(+1)}}\hspace{1ex}\uanga{नाग}}{}
}
{काञ्ची ६५ जगद्गुरु श्री-सुदर्शन महादेवेन्द्र सरस्वती आराधना~\#{१२९}\eventsep मन्वादिः-(रैवतः-[५])\eventsep फाल्गुन-अमावास्या (अलभ्यम्–पुष्कला)}
{Thu} 
\cfoot{\rygdata{13:43--15:14}{06:07--07:38}{09:09--10:40}}
\caldata{APRIL}{5}{\sunmonth{मीनः}{22}{}{फाल्गुनः}{शिशिरऋतुः}{शुक्रः}{विलम्बः}{उत्तरायणम्}{शिशिरऋतुः}}
{\sunmoonsrdata{06:06}{18:17}{---}{18:25}{12:11}
{\kalas{04:31 05:19 09:21 08:32 10:10 16:39 10:58 13:24 15:51 17:28 19:04 21:14 22:43 01:40(+1)}}}
{\tnykdata{\anga{\tithi{30}{अमावास्या}}{\time{20-35}{14:20}}\hspace{1ex}}%
{\fullanga{रेवती}}{चन्द्रराशिः—\mbox{मीनः}}%
{\anga{इन्द्रः}{\time{39-55}{22:04}}\hspace{1ex}\uanga{वैधृतिः}}%
{\anga{नाग}{\time{20-35}{14:20}}\hspace{1ex}\anga{किंस्तुघ्नः}{\time{52-2}{02:55(+1)}}\hspace{1ex}\uanga{बवः}}{}
}
{भृगुरेवती-पुण्यकालः\eventsep पार्वणव्रतम् अमावास्यायाम्}
{Fri} 
\cfoot{\rygdata{10:40--12:11}{15:14--16:45}{07:37--09:09}}
\caldata{APRIL}{6}{\sunmonth{मीनः}{23}{}{चैत्रः}{वसन्तऋतुः}{शनिः}{विलम्बः}{उत्तरायणम्}{शिशिरऋतुः}}
{\sunmoonrsdata{06:05}{18:17}{06:31}{19:14}{12:11}
{\kalas{04:31 05:18 09:20 08:32 10:09 16:39 10:58 13:24 15:51 17:28 19:04 21:14 22:42 01:39(+1)}}}
{\tnykdata{\anga{\tithi{1}{शुक्ल-प्रथमा}}{\time{23-14}{15:23}}\hspace{1ex}}%
{\anga{रेवती}{\time{3-8}{07:21}}\hspace{1ex}}{चन्द्रराशिः—\mbox{मीनः\RIGHTarrow\textsf{07:21}}}%
{\anga{वैधृतिः}{\time{39-6}{21:44}}\hspace{1ex}\uanga{विष्कम्भः}}%
{\anga{बवः}{\time{23-14}{15:23}}\hspace{1ex}\anga{बालवः}{\time{54-9}{03:45(+1)}}\hspace{1ex}\uanga{कौलवः}}{}
}
{चन्द्र-दर्शनम्\eventsep दर्शेष्टिः\eventsep काञ्ची १५ जगद्गुरु श्री-गङ्गाधरेन्द्र सरस्वती आराधना~\#{१६९१}\eventsep काञ्ची २७ जगद्गुरु श्री-चिद्विलासेन्द्र सरस्वती आराधना~\#{१४४३}\eventsep काञ्ची ५२ जगद्गुरु श्री-शङ्करानन्देन्द्र सरस्वती आराधना~\#{६०३}\eventsep स्थालीपाकः\eventsep वैधृति-श्राद्धम्\eventsep वसन्तनवरात्र-आरम्भः\eventsep युगादिः\eventsep श्वेत-कल्पादिः}
{Sat} 
\cfoot{\rygdata{09:08--10:40}{13:43--15:14}{06:05--07:37}}
\caldata{APRIL}{7}{\sunmonth{मीनः}{24}{}{चैत्रः}{वसन्तऋतुः}{भानुः}{विलम्बः}{उत्तरायणम्}{शिशिरऋतुः}}
{\sunmoonrsdata{06:05}{18:17}{07:13}{20:06}{12:11}
{\kalas{04:30 05:17 09:20 08:31 10:09 16:39 10:58 13:24 15:51 17:28 19:04 21:14 22:42 01:39(+1)}}}
{\tnykdata{\anga{\tithi{2}{शुक्ल-द्वितीया}}{\time{24-51}{16:01}}\hspace{1ex}}%
{\anga{अश्विनी}{\time{6-34}{08:43}}\hspace{1ex}}{चन्द्रराशिः—\mbox{मेषः}}%
{\anga{विष्कम्भः}{\time{37-30}{21:05}}\hspace{1ex}\uanga{प्रीतिः}}%
{\anga{कौलवः}{\time{24-51}{16:01}}\hspace{1ex}\anga{तैतिलः}{\time{55-16}{04:11(+1)}}\hspace{1ex}\uanga{गरः}}{}
}
{आन्दोलन-तृतीया\eventsep अरुन्धती-व्रत-आरम्भः\eventsep झूलेलाल-जयन्ती}
{Sun} 
\cfoot{\rygdata{16:45--18:17}{12:11--13:42}{15:14--16:45}}
\caldata{APRIL}{8}{\sunmonth{मीनः}{25}{}{चैत्रः}{वसन्तऋतुः}{सोमः}{विलम्बः}{उत्तरायणम्}{शिशिरऋतुः}}
{\sunmoonrsdata{06:04}{18:17}{07:57}{20:59}{12:11}
{\kalas{04:30 05:17 09:20 08:31 10:08 16:39 10:57 13:24 15:50 17:28 19:04 21:14 22:42 01:39(+1)}}}
{\tnykdata{\anga{\tithi{3}{शुक्ल-तृतीया}}{\time{25-28}{16:15}}\hspace{1ex}}%
{\anga{अपभरणी}{\time{9-3}{09:41}}\hspace{1ex}}{चन्द्रराशिः—\mbox{मेषः\RIGHTarrow\textsf{15:52}}}%
{\anga{प्रीतिः}{\time{35-9}{20:08}}\hspace{1ex}\uanga{आयुष्मान्}}%
{\anga{गरः}{\time{25-28}{16:15}}\hspace{1ex}\anga{वणिजः}{\time{55-24}{04:14(+1)}}\hspace{1ex}\uanga{विष्टिः}}{}
}
{गौरी-तृतीया/सौभाग्य-गौरी-व्रतम्\eventsep कृत्तिका-व्रतम्\eventsep मन्वादिः-(उत्तमः-[३])}
{Mon} 
\cfoot{\rygdata{07:36--09:07}{10:39--12:11}{13:42--15:14}}
\caldata{APRIL}{9}{\sunmonth{मीनः}{26}{}{चैत्रः}{वसन्तऋतुः}{मङ्गलः}{विलम्बः}{उत्तरायणम्}{शिशिरऋतुः}}
{\sunmoonrsdata{06:03}{18:17}{08:45}{21:55}{12:10}
{\kalas{04:29 05:16 09:19 08:30 10:08 16:39 10:57 13:24 15:50 17:28 19:04 21:14 22:42 01:38(+1)}}}
{\tnykdata{\anga{\tithi{4}{शुक्ल-चतुर्थी}}{\time{25-8}{16:07}}\hspace{1ex}}%
{\anga{कृत्तिका}{\time{10-35}{10:18}}\hspace{1ex}}{चन्द्रराशिः—\mbox{वृषभः}}%
{\anga{आयुष्मान्}{\time{32-3}{18:53}}\hspace{1ex}\uanga{सौभाग्यः}}%
{\anga{विष्टिः}{\time{25-8}{16:07}}\hspace{1ex}\anga{बवः}{\time{54-37}{03:54(+1)}}\hspace{1ex}\uanga{बालवः}}{}
}
{मुत्तुस्वामि-दीक्षित-जयन्ती~\#{२४५}\eventsep सुखा-अङ्गारक-चतुर्थी}
{Tue} 
\cfoot{\rygdata{15:14--16:45}{09:07--10:39}{12:10--13:42}}
\caldata{APRIL}{10}{\sunmonth{मीनः}{27}{}{चैत्रः}{वसन्तऋतुः}{बुधः}{विलम्बः}{उत्तरायणम्}{शिशिरऋतुः}}
{\sunmoonrsdata{06:03}{18:17}{09:37}{22:52}{12:10}
{\kalas{04:29 05:16 09:19 08:30 10:08 16:39 10:57 13:23 15:50 17:28 19:04 21:13 22:42 01:38(+1)}}}
{\tnykdata{\anga{\tithi{5}{शुक्ल-पञ्चमी}}{\time{23-52}{15:36}}\hspace{1ex}}%
{\anga{रोहिणी}{\time{11-12}{10:32}}\hspace{1ex}}{चन्द्रराशिः—\mbox{वृषभः\RIGHTarrow\textsf{22:31}}}%
{\anga{सौभाग्यः}{\time{28-14}{17:20}}\hspace{1ex}\uanga{शोभनः}}%
{\anga{बालवः}{\time{23-52}{15:36}}\hspace{1ex}\anga{कौलवः}{\time{52-52}{03:12(+1)}}\hspace{1ex}\uanga{तैतिलः}}{}
}
{हय-पूजा\eventsep कूर्म-कल्पादिः\eventsep लक्ष्मी-पञ्चमी\eventsep \tamil{நேச நாயனார் (58) குருபூஜை}\eventsep शालिहोत्र-व्रत-आरम्भः}
{Wed} 
\cfoot{\rygdata{12:10--13:42}{07:35--09:06}{10:38--12:10}}
\caldata{APRIL}{11}{\sunmonth{मीनः}{28}{}{चैत्रः}{वसन्तऋतुः}{गुरुः}{विलम्बः}{उत्तरायणम्}{शिशिरऋतुः}}
{\sunmoonrsdata{06:02}{18:17}{10:33}{23:49}{12:10}
{\kalas{04:28 05:15 09:18 08:29 10:07 16:39 10:56 13:23 15:50 17:28 19:04 21:13 22:41 01:37(+1)}}}
{\tnykdata{\anga{\tithi{6}{शुक्ल-षष्ठी}}{\time{21-38}{14:42}}\hspace{1ex}}%
{\anga{मृगशीर्षम्}{\time{10-54}{10:24}}\hspace{1ex}}{चन्द्रराशिः—\mbox{मिथुनम्}}%
{\anga{शोभनः}{\time{23-38}{15:30}}\hspace{1ex}\uanga{अतिगण्डः}}%
{\anga{तैतिलः}{\time{21-38}{14:42}}\hspace{1ex}\anga{गरः}{\time{50-8}{02:06(+1)}}\hspace{1ex}\uanga{वणिजः}}{}
}
{षष्ठी-व्रतम्\eventsep यमुना-जयन्ती}
{Thu} 
\cfoot{\rygdata{13:42--15:14}{06:02--07:34}{09:06--10:38}}
\caldata{APRIL}{12}{\sunmonth{मीनः}{29}{}{चैत्रः}{वसन्तऋतुः}{शुक्रः}{विलम्बः}{उत्तरायणम्}{शिशिरऋतुः}}
{\sunmoonrsdata{06:02}{18:18}{11:31}{00:45(+1)}{12:10}
{\kalas{04:28 05:15 09:18 08:29 10:07 16:39 10:56 13:23 15:50 17:28 19:04 21:13 22:41 01:37(+1)}}}
{\tnykdata{\anga{\tithi{7}{शुक्ल-सप्तमी}}{\time{18-24}{13:23}}\hspace{1ex}}%
{\anga{आर्द्रा}{\time{9-37}{09:52}}\hspace{1ex}}{चन्द्रराशिः—\mbox{मिथुनम्\RIGHTarrow\textsf{03:13(+1)}}}%
{\anga{अतिगण्डः}{\time{18-15}{13:20}}\hspace{1ex}\uanga{सुकर्म}}%
{\anga{वणिजः}{\time{18-24}{13:23}}\hspace{1ex}\anga{विष्टिः}{\time{46-24}{00:35(+1)}}\hspace{1ex}\uanga{बवः}}{}
}
{\tamil{கணநாத நாயனார் (37) குருபூஜை}}
{Fri} 
\cfoot{\rygdata{10:38--12:09}{15:13--16:45}{07:34--09:06}}
\caldata{APRIL}{13}{\sunmonth{मीनः}{30}{}{चैत्रः}{वसन्तऋतुः}{शनिः}{विलम्बः}{उत्तरायणम्}{शिशिरऋतुः}}
{\sunmoonrsdata{06:01}{18:18}{12:31}{01:38(+1)}{12:09}
{\kalas{04:27 05:14 09:17 08:28 10:06 16:39 10:56 13:23 15:50 17:28 19:04 21:13 22:41 01:37(+1)}}}
{\tnykdata{\anga{\tithi{8}{शुक्ल-अष्टमी}}{\time{14-10}{11:41}}\hspace{1ex}}%
{\anga{पुनर्वसुः}{\time{7-20}{08:57}}\hspace{1ex}}{चन्द्रराशिः—\mbox{कटकः}}%
{\anga{सुकर्म}{\time{12-4}{10:51}}\hspace{1ex}\uanga{धृतिः}}%
{\anga{बवः}{\time{14-10}{11:41}}\hspace{1ex}\anga{बालवः}{\time{41-40}{22:41}}\hspace{1ex}\uanga{कौलवः}}{}
}
{अशोकाष्टमी\eventsep भवान्युत्पत्तिः\eventsep काञ्ची ४३ जगद्गुरु श्री-आनन्दघनेन्द्र सरस्वती आराधना~\#{१००६}\eventsep महातारा-जयन्ती\eventsep श्रीरामनवमी}
{Sat} 
\cfoot{\rygdata{09:05--10:37}{13:41--15:13}{06:01--07:33}}
\caldata{APRIL}{14}{\sunmonth{मेषः}{1}{\mbox{मीनः{\tiny\RIGHTarrow}\textsf{13:54}}}{चैत्रः}{वसन्तऋतुः}{भानुः}{विकारी}{उत्तरायणम्}{वसन्तऋतुः}}
{\sunmoonrsdata{06:00}{18:18}{13:31}{02:29(+1)}{12:09}
{\kalas{04:27 05:13 09:17 08:28 10:06 16:39 10:55 13:23 15:50 17:29 19:04 21:13 22:41 01:36(+1)}}}
{\tnykdata{\anga{\tithi{9}{शुक्ल-नवमी}}{\time{8-57}{09:35}}\hspace{1ex}}%
{\anga{पुष्यः}{\time{4-4}{07:38}}\hspace{1ex}\anga{आश्रेषा}{\time{59-53}{05:58(+1)}}\hspace{1ex}}{चन्द्रराशिः—\mbox{कटकः\RIGHTarrow\textsf{05:58(+1)}}}%
{\anga{धृतिः}{\time{5-5}{08:02}}\hspace{1ex}\anga{शूलः}{\time{57-20}{04:56(+1)}}\hspace{1ex}\uanga{गण्डः}}%
{\anga{कौलवः}{\time{8-57}{09:35}}\hspace{1ex}\anga{तैतिलः}{\time{35-59}{20:24}}\hspace{1ex}\uanga{गरः}}{}
}
{मेष-सङ्क्रान्तिः (विकारी-संवत्सरः)\eventsep मेष-सङ्क्रमण-पुण्यकालः~\textsf{09:54}{\RIGHTarrow}\textsf{17:54}\eventsep पञ्चाङ्ग-पठनम्\eventsep रविपुष्ययोग-पुण्यकालः\eventsep वसन्तनवरात्र-समापनम्\eventsep \tamil{விஷுக்கனி}}
{Sun} 
\cfoot{\rygdata{16:45--18:18}{12:09--13:41}{15:13--16:45}}
\caldata{APRIL}{15}{\sunmonth{मेषः}{2}{}{चैत्रः}{वसन्तऋतुः}{सोमः}{विकारी}{उत्तरायणम्}{वसन्तऋतुः}}
{\sunmoonrsdata{06:00}{18:18}{14:30}{03:17(+1)}{12:09}
{\kalas{04:26 05:13 09:16 08:27 10:06 16:39 10:55 13:23 15:50 17:29 19:05 21:13 22:41 01:36(+1)}}}
{\tnykdata{\anga{\tithi{10}{शुक्ल-दशमी}}{\time{2-50}{07:08}}\hspace{1ex}\anga{\tithi{11}{शुक्ल-एकादशी}}{\time{55-57}{04:23(+1)}}\hspace{1ex}}%
{\anga{मघा}{\time{54-59}{03:59(+1)}}\hspace{1ex}}{चन्द्रराशिः—\mbox{सिंहः}}%
{\anga{गण्डः}{\time{49-0}{01:36(+1)}}\hspace{1ex}\uanga{वृद्धिः}}%
{\anga{गरः}{\time{2-50}{07:08}}\hspace{1ex}\anga{वणिजः}{\time{29-29}{17:47}}\hspace{1ex}\anga{विष्टिः}{\time{55-57}{04:23(+1)}}\hspace{1ex}\uanga{बवः}}{}
}
{धर्मराज-दशमी\eventsep समुद्र-मन्थनम्\eventsep स्मार्त-कामदा-एकादशी}
{Mon} 
\cfoot{\rygdata{07:32--09:04}{10:36--12:09}{13:41--15:13}}
\caldata{APRIL}{16}{\sunmonth{मेषः}{3}{}{चैत्रः}{वसन्तऋतुः}{मङ्गलः}{विकारी}{उत्तरायणम्}{वसन्तऋतुः}}
{\sunmoonrsdata{05:59}{18:18}{15:28}{04:03(+1)}{12:09}
{\kalas{04:26 05:12 09:16 08:27 10:05 16:39 10:55 13:22 15:50 17:29 19:05 21:13 22:41 01:36(+1)}}}
{\tnykdata{\anga{\tithi{12}{शुक्ल-द्वादशी}}{\time{48-36}{01:26(+1)}}\hspace{1ex}}%
{\anga{पूर्वफल्गुनी}{\time{49-34}{01:49(+1)}}\hspace{1ex}}{चन्द्रराशिः—\mbox{सिंहः}}%
{\anga{वृद्धिः}{\time{40-13}{22:04}}\hspace{1ex}\uanga{ध्रुवः}}%
{\anga{बवः}{\time{22-20}{14:55}}\hspace{1ex}\anga{बालवः}{\time{48-36}{01:26(+1)}}\hspace{1ex}\uanga{कौलवः}}{}
}
{भ्रातृप्राप्ति-व्रत-आरम्भः\eventsep दमनकारोपण-द्वादशी\eventsep हरिवासरः{\RIGHTarrow}\textsf{09:39}\eventsep तुलसी-जननं-क्षीरसागरतः\eventsep वेङ्कटाचले वसन्तोत्सव-आरम्भः\eventsep वैष्णव-कामदा-एकादशी}
{Tue} 
\cfoot{\rygdata{15:13--16:45}{09:04--10:36}{12:08--13:41}}
\caldata{APRIL}{17}{\sunmonth{मेषः}{4}{}{चैत्रः}{वसन्तऋतुः}{बुधः}{विकारी}{उत्तरायणम्}{वसन्तऋतुः}}
{\sunmoonrsdata{05:58}{18:18}{16:26}{04:50(+1)}{12:08}
{\kalas{04:25 05:12 09:16 08:26 10:05 16:39 10:54 13:22 15:50 17:29 19:05 21:13 22:40 01:35(+1)}}}
{\tnykdata{\anga{\tithi{13}{शुक्ल-त्रयोदशी}}{\time{41-3}{22:24}}\hspace{1ex}}%
{\anga{उत्तरफल्गुनी}{\time{43-59}{23:34}}\hspace{1ex}}{चन्द्रराशिः—\mbox{सिंहः\RIGHTarrow\textsf{07:15}}}%
{\anga{ध्रुवः}{\time{31-14}{18:28}}\hspace{1ex}\uanga{व्याघातः}}%
{\anga{कौलवः}{\time{14-51}{11:55}}\hspace{1ex}\anga{तैतिलः}{\time{41-3}{22:24}}\hspace{1ex}\uanga{गरः}}{}
}
{दमनक-चोरी-उत्सवः\eventsep मदन-त्रयोदशी\eventsep प्रदोष-व्रतम्\eventsep वेङ्कटाचले वसन्तोत्सवः}
{Wed} 
\cfoot{\rygdata{12:08--13:41}{07:31--09:03}{10:36--12:08}}
\caldata{APRIL}{18}{\sunmonth{मेषः}{5}{}{चैत्रः}{वसन्तऋतुः}{गुरुः}{विकारी}{उत्तरायणम्}{वसन्तऋतुः}}
{\sunmoonrsdata{05:58}{18:18}{17:24}{05:36(+1)}{12:08}
{\kalas{04:25 05:11 09:15 08:26 10:05 16:39 10:54 13:22 15:50 17:29 19:05 21:13 22:40 01:35(+1)}}}
{\tnykdata{\anga{\tithi{14}{शुक्ल-चतुर्दशी}}{\time{33-40}{19:26}}\hspace{1ex}}%
{\anga{हस्तः}{\time{38-34}{21:24}}\hspace{1ex}}{चन्द्रराशिः—\mbox{कन्या}}%
{\anga{व्याघातः}{\time{22-20}{14:54}}\hspace{1ex}\uanga{हर्षणः}}%
{\anga{गरः}{\time{7-19}{08:54}}\hspace{1ex}\anga{वणिजः}{\time{33-40}{19:26}}\hspace{1ex}\uanga{विष्टिः}}{}
}
{चित्रा-पूर्णिमा\eventsep दमनक-चतुर्दशी\eventsep मदन-चतुर्दशी\eventsep वेङ्कटाचले पूर्णिमा-गरुड-सेवा\eventsep वेङ्कटाचले वसन्तोत्सव-समापनम्}
{Thu} 
\cfoot{\rygdata{13:41--15:13}{05:58--07:31}{09:03--10:36}}
\caldata{APRIL}{19}{\sunmonth{मेषः}{6}{}{चैत्रः}{वसन्तऋतुः}{शुक्रः}{विकारी}{उत्तरायणम्}{वसन्तऋतुः}}
{\sunmoonrsdata{05:57}{18:18}{18:22}{---}{12:08}
{\kalas{04:24 05:11 09:15 08:26 10:04 16:39 10:54 13:22 15:50 17:29 19:05 21:13 22:40 01:35(+1)}}}
{\tnykdata{\anga{\tithi{15}{पौर्णमासी}}{\time{26-51}{16:42}}\hspace{1ex}}%
{\anga{चित्रा}{\time{33-46}{19:28}}\hspace{1ex}}{चन्द्रराशिः—\mbox{कन्या\RIGHTarrow\textsf{08:23}}}%
{\anga{हर्षणः}{\time{13-50}{11:30}}\hspace{1ex}\uanga{वज्रम्}}%
{\anga{विष्टिः}{\time{0-10}{06:02}}\hspace{1ex}\anga{बवः}{\time{26-51}{16:42}}\hspace{1ex}\anga{बालवः}{\time{53-45}{03:28(+1)}}\hspace{1ex}\uanga{कौलवः}}{}
}
{चैत्र-पूर्णिमा\eventsep चित्रगुप्त-व्रतम्\eventsep गजेन्द्र-मोक्षः\eventsep \tamil{இசைஞானியார் நாயனார் (62) குருபூஜை}\eventsep \tamil{மதுரகவி ஆழ்வார் திருநக்ஷத்திரம்}\eventsep मन्वादिः-(रौच्यः-[१३])\eventsep पार्वणव्रतम् पूर्णिमायाम्\eventsep पूर्णिमा-व्रतम्\eventsep श्री-हनूमत्-जयन्ती}
{Fri} 
\cfoot{\rygdata{10:35--12:08}{15:13--16:46}{07:30--09:03}}
\caldata{APRIL}{20}{\sunmonth{मेषः}{7}{}{चैत्रः}{वसन्तऋतुः}{शनिः}{विकारी}{उत्तरायणम्}{वसन्तऋतुः}}
{\sunmoonsrdata{05:57}{18:18}{19:21}{06:24}{12:08}
{\kalas{04:24 05:10 09:15 08:25 10:04 16:39 10:53 13:22 15:50 17:29 19:05 21:13 22:40 01:35(+1)}}}
{\tnykdata{\anga{\tithi{16}{कृष्ण-प्रथमा}}{\time{20-59}{14:21}}\hspace{1ex}}%
{\anga{स्वाती}{\time{29-59}{17:57}}\hspace{1ex}}{चन्द्रराशिः—\mbox{तुला}}%
{\anga{वज्रम्}{\time{6-3}{08:22}}\hspace{1ex}\anga{सिद्धिः}{\time{59-16}{05:39(+1)}}\hspace{1ex}\uanga{व्यतीपातः}}%
{\anga{कौलवः}{\time{20-59}{14:21}}\hspace{1ex}\anga{तैतिलः}{\time{48-32}{01:22(+1)}}\hspace{1ex}\uanga{गरः}}{}
}
{काञ्ची ६० जगद्गुरु श्री-अद्वैतात्मप्रकाशेन्द्र सरस्वती आराधना~\#{३१६}\eventsep मधु-मासः{\RIGHTarrow}\textsf{14:25}\eventsep पूर्र्णमासेष्टिः\eventsep स्थालीपाकः\eventsep \tamil{திருக்குறிப்புத் தொண்ட நாயனார் (18) குருபூஜை}\eventsep विष्णुपदी-पुण्यकालः~\textsf{08:01}{\RIGHTarrow}\textsf{20:49}}
{Sat} 
\cfoot{\rygdata{09:02--10:35}{13:40--15:13}{05:57--07:30}}
\caldata{APRIL}{21}{\sunmonth{मेषः}{8}{}{चैत्रः}{वसन्तऋतुः}{भानुः}{विकारी}{उत्तरायणम्}{वसन्तऋतुः}}
{\sunmoonsrdata{05:56}{18:19}{20:19}{07:13}{12:07}
{\kalas{04:23 05:10 09:14 08:25 10:04 16:39 10:53 13:22 15:50 17:29 19:05 21:13 22:40 01:34(+1)}}}
{\tnykdata{\anga{\tithi{17}{कृष्ण-द्वितीया}}{\time{16-30}{12:32}}\hspace{1ex}}%
{\anga{विशाखा}{\time{27-37}{16:59}}\hspace{1ex}}{चन्द्रराशिः—\mbox{तुला\RIGHTarrow\textsf{11:10}}}%
{\anga{व्यतीपातः}{\time{53-49}{03:28(+1)}}\hspace{1ex}\uanga{वरीयान्}}%
{\anga{गरः}{\time{16-30}{12:32}}\hspace{1ex}\anga{वणिजः}{\time{44-51}{23:53}}\hspace{1ex}\uanga{विष्टिः}}{}
}
{व्यतीपात-श्राद्धम्}
{Sun} 
\cfoot{\rygdata{16:46--18:19}{12:07--13:40}{15:13--16:46}}
\caldata{APRIL}{22}{\sunmonth{मेषः}{9}{}{चैत्रः}{वसन्तऋतुः}{सोमः}{विकारी}{उत्तरायणम्}{वसन्तऋतुः}}
{\sunmoonsrdata{05:56}{18:19}{21:16}{08:04}{12:07}
{\kalas{04:23 05:09 09:14 08:24 10:03 16:40 10:53 13:21 15:50 17:29 19:05 21:13 22:40 01:34(+1)}}}
{\tnykdata{\anga{\tithi{18}{कृष्ण-तृतीया}}{\time{13-42}{11:25}}\hspace{1ex}}%
{\anga{अनूराधा}{\time{26-59}{16:44}}\hspace{1ex}}{चन्द्रराशिः—\mbox{वृश्चिकः}}%
{\anga{वरीयान्}{\time{49-52}{01:53(+1)}}\hspace{1ex}\uanga{परिघः}}%
{\anga{विष्टिः}{\time{13-42}{11:25}}\hspace{1ex}\anga{बवः}{\time{43-0}{23:08}}\hspace{1ex}\uanga{बालवः}}{}
}
{विकट-महागणपति सङ्कटहर-चतुर्थी-व्रतम्}
{Mon} 
\cfoot{\rygdata{07:29--09:02}{10:34--12:07}{13:40--15:13}}
\caldata{APRIL}{23}{\sunmonth{मेषः}{10}{}{चैत्रः}{वसन्तऋतुः}{मङ्गलः}{विकारी}{उत्तरायणम्}{वसन्तऋतुः}}
{\sunmoonsrdata{05:55}{18:19}{22:12}{08:57}{12:07}
{\kalas{04:22 05:09 09:13 08:24 10:03 16:40 10:53 13:21 15:50 17:29 19:05 21:13 22:40 01:34(+1)}}}
{\tnykdata{\anga{\tithi{19}{कृष्ण-चतुर्थी}}{\time{12-51}{11:04}}\hspace{1ex}}%
{\anga{ज्येष्ठा}{\time{28-18}{17:15}}\hspace{1ex}}{चन्द्रराशिः—\mbox{वृश्चिकः\RIGHTarrow\textsf{17:15}}}%
{\anga{परिघः}{\time{47-30}{00:56(+1)}}\hspace{1ex}\uanga{शिवः}}%
{\anga{बालवः}{\time{12-51}{11:04}}\hspace{1ex}\anga{कौलवः}{\time{43-10}{23:12}}\hspace{1ex}\uanga{तैतिलः}}{}
}
{अङ्गारक-चतुर्थी\eventsep गुरु-सङ्क्रान्तिः~(धनुः\To{}वृश्चिकः)}
{Tue} 
\cfoot{\rygdata{15:13--16:46}{09:01--10:34}{12:07--13:40}}
\caldata{APRIL}{24}{\sunmonth{मेषः}{11}{}{चैत्रः}{वसन्तऋतुः}{बुधः}{विकारी}{उत्तरायणम्}{वसन्तऋतुः}}
{\sunmoonsrdata{05:55}{18:19}{23:04}{09:50}{12:07}
{\kalas{04:22 05:08 09:13 08:24 10:03 16:40 10:52 13:21 15:50 17:29 19:05 21:13 22:40 01:34(+1)}}}
{\tnykdata{\anga{\tithi{20}{कृष्ण-पञ्चमी}}{\time{14-2}{11:32}}\hspace{1ex}}%
{\anga{मूला}{\time{31-36}{18:33}}\hspace{1ex}}{चन्द्रराशिः—\mbox{धनुः}}%
{\anga{शिवः}{\time{46-44}{00:36(+1)}}\hspace{1ex}\uanga{सिद्धः}}%
{\anga{तैतिलः}{\time{14-2}{11:32}}\hspace{1ex}\anga{गरः}{\time{45-21}{00:04(+1)}}\hspace{1ex}\uanga{वणिजः}}{}
}
{लक्ष्मी-वराह-जयन्ती}
{Wed} 
\cfoot{\rygdata{12:07--13:40}{07:28--09:01}{10:34--12:07}}
\caldata{APRIL}{25}{\sunmonth{मेषः}{12}{}{चैत्रः}{वसन्तऋतुः}{गुरुः}{विकारी}{उत्तरायणम्}{वसन्तऋतुः}}
{\sunmoonsrdata{05:54}{18:19}{23:53}{10:43}{12:07}
{\kalas{04:22 05:08 09:13 08:23 10:02 16:40 10:52 13:21 15:50 17:29 19:05 21:13 22:40 01:33(+1)}}}
{\tnykdata{\anga{\tithi{21}{कृष्ण-षष्ठी}}{\time{17-9}{12:46}}\hspace{1ex}}%
{\anga{पूर्वाषाढा}{\time{36-43}{20:35}}\hspace{1ex}}{चन्द्रराशिः—\mbox{धनुः\RIGHTarrow\textsf{03:12(+1)}}}%
{\anga{सिद्धः}{\time{47-21}{00:51(+1)}}\hspace{1ex}\uanga{साध्यः}}%
{\anga{वणिजः}{\time{17-9}{12:46}}\hspace{1ex}\anga{विष्टिः}{\time{49-21}{01:39(+1)}}\hspace{1ex}\uanga{बवः}}{}
}
{}
{Thu} 
\cfoot{\rygdata{13:40--15:13}{05:54--07:27}{09:00--10:34}}
\caldata{APRIL}{26}{\sunmonth{मेषः}{13}{}{चैत्रः}{वसन्तऋतुः}{शुक्रः}{विकारी}{उत्तरायणम्}{वसन्तऋतुः}}
{\sunmoonsrdata{05:54}{18:19}{00:38(+1)}{11:34}{12:06}
{\kalas{04:21 05:07 09:12 08:23 10:02 16:40 10:52 13:21 15:50 17:29 19:05 21:13 22:39 01:33(+1)}}}
{\tnykdata{\anga{\tithi{22}{कृष्ण-सप्तमी}}{\time{21-55}{14:40}}\hspace{1ex}}%
{\anga{उत्तराषाढा}{\time{43-16}{23:12}}\hspace{1ex}}{चन्द्रराशिः—\mbox{मकरः}}%
{\anga{साध्यः}{\time{49-2}{01:31(+1)}}\hspace{1ex}\uanga{शुभः}}%
{\anga{बवः}{\time{21-55}{14:40}}\hspace{1ex}\anga{बालवः}{\time{54-45}{03:48(+1)}}\hspace{1ex}\uanga{कौलवः}}{}
}
{}
{Fri} 
\cfoot{\rygdata{10:33--12:06}{15:13--16:46}{07:27--09:00}}
\caldata{APRIL}{27}{\sunmonth{मेषः}{14}{}{चैत्रः}{वसन्तऋतुः}{शनिः}{विकारी}{उत्तरायणम्}{वसन्तऋतुः}}
{\sunmoonsrdata{05:53}{18:19}{01:19(+1)}{12:24}{12:06}
{\kalas{04:21 05:07 09:12 08:22 10:02 16:40 10:52 13:21 15:50 17:30 19:06 21:13 22:39 01:33(+1)}}}
{\tnykdata{\anga{\tithi{23}{कृष्ण-अष्टमी}}{\time{27-48}{17:01}}\hspace{1ex}}%
{\anga{श्रवणः}{\time{50-43}{02:11(+1)}}\hspace{1ex}}{चन्द्रराशिः—\mbox{मकरः}}%
{\anga{शुभः}{\time{51-24}{02:27(+1)}}\hspace{1ex}\uanga{शुक्लः}}%
{\anga{कौलवः}{\time{27-48}{17:01}}\hspace{1ex}\uanga{तैतिलः}}{}
}
{काञ्ची ५६ जगद्गुरु श्री-सर्वज्ञ सदाशिव बोधेन्द्र सरस्वती आराधना~\#{४८१}\eventsep \tamil{நடராஜர் சித்திரை ஓணம் மஹாபிஷேகம்}\eventsep श्रवण-व्रतम्}
{Sat} 
\cfoot{\rygdata{09:00--10:33}{13:40--15:13}{05:53--07:26}}
\caldata{APRIL}{28}{\sunmonth{मेषः}{15}{}{चैत्रः}{वसन्तऋतुः}{भानुः}{विकारी}{उत्तरायणम्}{वसन्तऋतुः}}
{\sunmoonsrdata{05:53}{18:20}{01:59(+1)}{13:12}{12:06}
{\kalas{04:20 05:07 09:12 08:22 10:02 16:40 10:51 13:21 15:50 17:30 19:06 21:13 22:39 01:33(+1)}}}
{\tnykdata{\anga{\tithi{24}{कृष्ण-नवमी}}{\time{34-13}{19:34}}\hspace{1ex}}%
{\anga{श्रविष्ठा}{\time{58-27}{05:16(+1)}}\hspace{1ex}}{चन्द्रराशिः—\mbox{मकरः\RIGHTarrow\textsf{15:43}}}%
{\anga{शुक्लः}{\time{54-0}{03:29(+1)}}\hspace{1ex}\uanga{ब्रह्म}}%
{\anga{तैतिलः}{\time{1-0}{06:17}}\hspace{1ex}\anga{गरः}{\time{34-13}{19:34}}\hspace{1ex}\uanga{वणिजः}}{}
}
{}
{Sun} 
\cfoot{\rygdata{16:46--18:20}{12:06--13:39}{15:13--16:46}}
\caldata{APRIL}{29}{\sunmonth{मेषः}{16}{}{चैत्रः}{वसन्तऋतुः}{सोमः}{विकारी}{उत्तरायणम्}{वसन्तऋतुः}}
{\sunmoonsrdata{05:52}{18:20}{02:36(+1)}{13:59}{12:06}
{\kalas{04:20 05:06 09:12 08:22 10:01 16:40 10:51 13:21 15:50 17:30 19:06 21:13 22:39 01:32(+1)}}}
{\tnykdata{\anga{\tithi{25}{कृष्ण-दशमी}}{\time{40-29}{22:04}}\hspace{1ex}}%
{\fullanga{शतभिषक्}}{चन्द्रराशिः—\mbox{कुम्भः}}%
{\anga{ब्रह्म}{\time{56-21}{04:25(+1)}}\hspace{1ex}\uanga{इन्द्रः}}%
{\anga{वणिजः}{\time{7-24}{08:50}}\hspace{1ex}\anga{विष्टिः}{\time{40-29}{22:04}}\hspace{1ex}\uanga{बवः}}{}
}
{\tamil{திருநாவுக்கரச நாயனார் (20) குருபூஜை}}
{Mon} 
\cfoot{\rygdata{07:26--08:59}{10:33--12:06}{13:39--15:13}}
\caldata{APRIL}{30}{\sunmonth{मेषः}{17}{}{चैत्रः}{वसन्तऋतुः}{मङ्गलः}{विकारी}{उत्तरायणम्}{वसन्तऋतुः}}
{\sunmoonsrdata{05:52}{18:20}{03:13(+1)}{14:45}{12:06}
{\kalas{04:20 05:06 09:11 08:21 10:01 16:40 10:51 13:21 15:50 17:30 19:06 21:13 22:39 01:32(+1)}}}
{\tnykdata{\anga{\tithi{26}{कृष्ण-एकादशी}}{\time{46-4}{00:18(+1)}}\hspace{1ex}}%
{\anga{शतभिषक्}{\time{5-52}{08:13}}\hspace{1ex}}{चन्द्रराशिः—\mbox{कुम्भः\RIGHTarrow\textsf{04:13(+1)}}}%
{\anga{इन्द्रः}{\time{58-6}{05:06(+1)}}\hspace{1ex}\uanga{वैधृतिः}}%
{\anga{बवः}{\time{13-24}{11:14}}\hspace{1ex}\anga{बालवः}{\time{46-4}{00:18(+1)}}\hspace{1ex}\uanga{कौलवः}}{}
}
{सर्व-वरूथिनी-एकादशी\eventsep वल्लभाचार्य-जयन्ती~\#{५४१}}
{Tue} 
\cfoot{\rygdata{15:13--16:46}{08:59--10:32}{12:06--13:39}}
\caldata{MAY}{1}{\sunmonth{मेषः}{18}{}{चैत्रः}{वसन्तऋतुः}{बुधः}{विकारी}{उत्तरायणम्}{वसन्तऋतुः}}
{\sunmoonsrdata{05:51}{18:20}{03:50(+1)}{15:31}{12:06}
{\kalas{04:19 05:05 09:11 08:21 10:01 16:40 10:51 13:21 15:50 17:30 19:06 21:13 22:39 01:32(+1)}}}
{\tnykdata{\anga{\tithi{27}{कृष्ण-द्वादशी}}{\time{50-33}{02:05(+1)}}\hspace{1ex}}%
{\anga{पूर्वप्रोष्ठपदा}{\time{12-27}{10:50}}\hspace{1ex}}{चन्द्रराशिः—\mbox{मीनः}}%
{\anga{वैधृतिः}{\time{58-59}{05:27(+1)}}\hspace{1ex}\uanga{विष्कम्भः}}%
{\anga{कौलवः}{\time{18-28}{13:15}}\hspace{1ex}\anga{तैतिलः}{\time{50-33}{02:05(+1)}}\hspace{1ex}\uanga{गरः}}{}
}
{हरिवासरः{\RIGHTarrow}\textsf{06:47}\eventsep वैधृति-श्राद्धम्}
{Wed} 
\cfoot{\rygdata{12:06--13:39}{07:25--08:59}{10:32--12:06}}
\caldata{MAY}{2}{\sunmonth{मेषः}{19}{}{चैत्रः}{वसन्तऋतुः}{गुरुः}{विकारी}{उत्तरायणम्}{वसन्तऋतुः}}
{\sunmoonsrdata{05:51}{18:20}{04:29(+1)}{16:19}{12:06}
{\kalas{04:19 05:05 09:11 08:21 10:01 16:40 10:51 13:21 15:50 17:30 19:06 21:13 22:39 01:32(+1)}}}
{\tnykdata{\anga{\tithi{28}{कृष्ण-त्रयोदशी}}{\time{53-44}{03:21(+1)}}\hspace{1ex}}%
{\anga{उत्तरप्रोष्ठपदा}{\time{17-52}{13:00}}\hspace{1ex}}{चन्द्रराशिः—\mbox{मीनः}}%
{\anga{विष्कम्भः}{\time{58-51}{05:24(+1)}}\hspace{1ex}\uanga{प्रीतिः}}%
{\anga{गरः}{\time{22-19}{14:47}}\hspace{1ex}\anga{वणिजः}{\time{53-44}{03:21(+1)}}\hspace{1ex}\uanga{विष्टिः}}{}
}
{मत्स्य-जयन्ती\eventsep प्रदोष-व्रतम्\eventsep रमण-महर्षि-आराधना~\#{६९}}
{Thu} 
\cfoot{\rygdata{13:39--15:13}{05:51--07:25}{08:58--10:32}}
\caldata{MAY}{3}{\sunmonth{मेषः}{20}{}{चैत्रः}{वसन्तऋतुः}{शुक्रः}{विकारी}{उत्तरायणम्}{वसन्तऋतुः}}
{\sunmoonsrdata{05:51}{18:20}{05:09(+1)}{17:08}{12:05}
{\kalas{04:19 05:05 09:11 08:21 10:01 16:40 10:51 13:20 15:50 17:30 19:06 21:13 22:39 01:32(+1)}}}
{\tnykdata{\anga{\tithi{29}{कृष्ण-चतुर्दशी}}{\time{55-32}{04:04(+1)}}\hspace{1ex}}%
{\anga{रेवती}{\time{21-59}{14:38}}\hspace{1ex}}{चन्द्रराशिः—\mbox{मीनः\RIGHTarrow\textsf{14:38}}}%
{\anga{प्रीतिः}{\time{57-39}{04:54(+1)}}\hspace{1ex}\uanga{आयुष्मान्}}%
{\anga{विष्टिः}{\time{24-48}{15:46}}\hspace{1ex}\anga{शकुनिः}{\time{55-32}{04:04(+1)}}\hspace{1ex}\uanga{चतुष्पात्}}{}
}
{भृगुरेवती-पुण्यकालः\eventsep गङ्गा-स्नानम्\eventsep मासशिवरात्रिः}
{Fri} 
\cfoot{\rygdata{10:32--12:06}{15:13--16:47}{07:24--08:58}}
\caldata{MAY}{4}{\sunmonth{मेषः}{21}{}{चैत्रः}{वसन्तऋतुः}{शनिः}{विकारी}{उत्तरायणम्}{वसन्तऋतुः}}
{\sunmoonsrdata{05:50}{18:21}{---}{17:59}{12:05}
{\kalas{04:18 05:04 09:10 08:20 10:00 16:40 10:50 13:20 15:50 17:31 19:06 21:13 22:39 01:31(+1)}}}
{\tnykdata{\anga{\tithi{30}{अमावास्या}}{\time{56-1}{04:15(+1)}}\hspace{1ex}}%
{\anga{अश्विनी}{\time{24-47}{15:45}}\hspace{1ex}}{चन्द्रराशिः—\mbox{मेषः}}%
{\anga{आयुष्मान्}{\time{55-26}{04:01(+1)}}\hspace{1ex}\uanga{सौभाग्यः}}%
{\anga{चतुष्पात्}{\time{25-57}{16:13}}\hspace{1ex}\anga{नाग}{\time{56-1}{04:15(+1)}}\hspace{1ex}\uanga{किंस्तुघ्नः}}{}
}
{भार्गव-राम-पूजा\eventsep चैत्र-अमावास्या\eventsep काञ्ची ४७ जगद्गुरु श्री-चन्द्रशेखरेन्द्र सरस्वती ३ आराधना~\#{८५४}\eventsep पार्वणव्रतम् अमावास्यायाम्\eventsep वह्नि-व्रतम्}
{Sat} 
\cfoot{\rygdata{08:58--10:32}{13:39--15:13}{05:50--07:24}}
\caldata{MAY}{5}{\sunmonth{मेषः}{22}{}{वैशाखः}{वसन्तऋतुः}{भानुः}{विकारी}{उत्तरायणम्}{वसन्तऋतुः}}
{\sunmoonrsdata{05:50}{18:21}{05:53}{18:52}{12:05}
{\kalas{04:18 05:04 09:10 08:20 10:00 16:41 10:50 13:20 15:51 17:31 19:07 21:13 22:39 01:31(+1)}}}
{\tnykdata{\anga{\tithi{1}{शुक्ल-प्रथमा}}{\time{55-21}{03:58(+1)}}\hspace{1ex}}%
{\anga{अपभरणी}{\time{26-22}{16:23}}\hspace{1ex}}{चन्द्रराशिः—\mbox{मेषः\RIGHTarrow\textsf{22:28}}}%
{\anga{सौभाग्यः}{\time{52-18}{02:45(+1)}}\hspace{1ex}\uanga{शोभनः}}%
{\anga{किंस्तुघ्नः}{\time{25-50}{16:10}}\hspace{1ex}\anga{बवः}{\time{55-21}{03:58(+1)}}\hspace{1ex}\uanga{बालवः}}{}
}
{अग्निनक्षत्र-आरम्भः\eventsep \tamil{சிறுத்தொண்ட நாயனார் (35) குருபூஜை}\eventsep दर्शेष्टिः\eventsep कृत्तिका-व्रतम्\eventsep पराशर-महर्षि-जयन्ती\eventsep स्थालीपाकः\eventsep वैशाख-मास-आरम्भः}
{Sun} 
\cfoot{\rygdata{16:47--18:21}{12:05--13:39}{15:13--16:47}}
\caldata{MAY}{6}{\sunmonth{मेषः}{23}{}{वैशाखः}{वसन्तऋतुः}{सोमः}{विकारी}{उत्तरायणम्}{वसन्तऋतुः}}
{\sunmoonrsdata{05:49}{18:21}{06:40}{19:49}{12:05}
{\kalas{04:18 05:03 09:10 08:20 10:00 16:41 10:50 13:20 15:51 17:31 19:07 21:13 22:39 01:31(+1)}}}
{\tnykdata{\anga{\tithi{2}{शुक्ल-द्वितीया}}{\time{53-40}{03:18(+1)}}\hspace{1ex}}%
{\anga{कृत्तिका}{\time{26-53}{16:35}}\hspace{1ex}}{चन्द्रराशिः—\mbox{वृषभः}}%
{\anga{शोभनः}{\time{48-22}{01:10(+1)}}\hspace{1ex}\uanga{अतिगण्डः}}%
{\anga{बालवः}{\time{24-38}{15:41}}\hspace{1ex}\anga{कौलवः}{\time{53-40}{03:18(+1)}}\hspace{1ex}\uanga{तैतिलः}}{}
}
{चन्द्र-दर्शनम्\eventsep श्यामा-शास्त्री-जयन्ती~\#{२५८}}
{Mon} 
\cfoot{\rygdata{07:23--08:57}{10:31--12:05}{13:39--15:13}}
\caldata{MAY}{7}{\sunmonth{मेषः}{24}{}{वैशाखः}{वसन्तऋतुः}{मङ्गलः}{विकारी}{उत्तरायणम्}{वसन्तऋतुः}}
{\sunmoonrsdata{05:49}{18:21}{07:32}{20:47}{12:05}
{\kalas{04:17 05:03 09:10 08:19 10:00 16:41 10:50 13:20 15:51 17:31 19:07 21:13 22:39 01:31(+1)}}}
{\tnykdata{\anga{\tithi{3}{शुक्ल-तृतीया}}{\time{51-9}{02:17(+1)}}\hspace{1ex}}%
{\anga{रोहिणी}{\time{26-30}{16:25}}\hspace{1ex}}{चन्द्रराशिः—\mbox{वृषभः\RIGHTarrow\textsf{04:14(+1)}}}%
{\anga{अतिगण्डः}{\time{43-45}{23:19}}\hspace{1ex}\uanga{सुकर्म}}%
{\anga{तैतिलः}{\time{22-30}{14:49}}\hspace{1ex}\anga{गरः}{\time{51-9}{02:17(+1)}}\hspace{1ex}\uanga{वणिजः}}{}
}
{अक्षय्य-तृतीया\eventsep बलराम-जयन्ती\eventsep कृतयुगादिः\eventsep \tamil{மங்கையர்க்கரசியார் நாயனார் (57) குருபூஜை}\eventsep पार्थिव-कल्पादिः\eventsep परशुराम-जयन्ती\eventsep राज-मातङ्गी-जयन्ती}
{Tue} 
\cfoot{\rygdata{15:13--16:47}{08:57--10:31}{12:05--13:39}}
\caldata{MAY}{8}{\sunmonth{मेषः}{25}{}{वैशाखः}{वसन्तऋतुः}{बुधः}{विकारी}{उत्तरायणम्}{वसन्तऋतुः}}
{\sunmoonrsdata{05:49}{18:21}{08:28}{21:44}{12:05}
{\kalas{04:17 05:03 09:09 08:19 10:00 16:41 10:50 13:20 15:51 17:31 19:07 21:13 22:39 01:31(+1)}}}
{\tnykdata{\anga{\tithi{4}{शुक्ल-चतुर्थी}}{\time{47-55}{00:59(+1)}}\hspace{1ex}}%
{\anga{मृगशीर्षम्}{\time{25-23}{15:58}}\hspace{1ex}}{चन्द्रराशिः—\mbox{मिथुनम्}}%
{\anga{सुकर्म}{\time{38-34}{21:14}}\hspace{1ex}\uanga{धृतिः}}%
{\anga{वणिजः}{\time{19-37}{13:40}}\hspace{1ex}\anga{विष्टिः}{\time{47-55}{00:59(+1)}}\hspace{1ex}\uanga{बवः}}{}
}
{बगलामुखी-जयन्ती\eventsep वार्ता-गौरी-व्रतम्}
{Wed} 
\cfoot{\rygdata{12:05--13:39}{07:23--08:57}{10:31--12:05}}
\caldata{MAY}{9}{\sunmonth{मेषः}{26}{}{वैशाखः}{वसन्तऋतुः}{गुरुः}{विकारी}{उत्तरायणम्}{वसन्तऋतुः}}
{\sunmoonrsdata{05:48}{18:22}{09:26}{22:41}{12:05}
{\kalas{04:17 05:03 09:09 08:19 10:00 16:41 10:50 13:20 15:51 17:31 19:07 21:13 22:39 01:31(+1)}}}
{\tnykdata{\anga{\tithi{5}{शुक्ल-पञ्चमी}}{\time{44-5}{23:27}}\hspace{1ex}}%
{\anga{आर्द्रा}{\time{23-37}{15:15}}\hspace{1ex}}{चन्द्रराशिः—\mbox{मिथुनम्}}%
{\anga{धृतिः}{\time{32-52}{18:57}}\hspace{1ex}\uanga{शूलः}}%
{\anga{बवः}{\time{16-4}{12:14}}\hspace{1ex}\anga{बालवः}{\time{44-5}{23:27}}\hspace{1ex}\uanga{कौलवः}}{}
}
{लावण्य-गौरी-व्रतम्\eventsep रामानुज-जन्म-नक्षत्रम्~\#{१००३}\eventsep सूरदास-जयन्ती~\#{५४२}\eventsep सर्प-पूजा\eventsep \tamil{விறன்மிண்ட நாயனார் (5) குருபூஜை}\eventsep शङ्कर-जयन्ती~\#{२५२८}}
{Thu} 
\cfoot{\rygdata{13:39--15:13}{05:48--07:23}{08:57--10:31}}
\caldata{MAY}{10}{\sunmonth{मेषः}{27}{}{वैशाखः}{वसन्तऋतुः}{शुक्रः}{विकारी}{उत्तरायणम्}{वसन्तऋतुः}}
{\sunmoonrsdata{05:48}{18:22}{10:26}{23:35}{12:05}
{\kalas{04:17 05:02 09:09 08:19 09:59 16:41 10:50 13:20 15:51 17:32 19:08 21:13 22:39 01:31(+1)}}}
{\tnykdata{\anga{\tithi{6}{शुक्ल-षष्ठी}}{\time{39-42}{21:41}}\hspace{1ex}}%
{\anga{पुनर्वसुः}{\time{21-18}{14:19}}\hspace{1ex}}{चन्द्रराशिः—\mbox{मिथुनम्\RIGHTarrow\textsf{08:35}}}%
{\anga{शूलः}{\time{26-44}{16:30}}\hspace{1ex}\uanga{गण्डः}}%
{\anga{कौलवः}{\time{11-58}{10:35}}\hspace{1ex}\anga{तैतिलः}{\time{39-42}{21:41}}\hspace{1ex}\uanga{गरः}}{}
}
{षष्ठी-व्रतम्\eventsep काञ्ची ४० जगद्गुरु श्री-महादेवेन्द्र सरस्वती २ आराधना~\#{११०५}\eventsep रामानुज-जयन्ती~\#{१००३}}
{Fri} 
\cfoot{\rygdata{10:31--12:05}{15:13--16:48}{07:22--08:57}}
\caldata{MAY}{11}{\sunmonth{मेषः}{28}{}{वैशाखः}{वसन्तऋतुः}{शनिः}{विकारी}{उत्तरायणम्}{वसन्तऋतुः}}
{\sunmoonrsdata{05:48}{18:22}{11:25}{00:26(+1)}{12:05}
{\kalas{04:16 05:02 09:09 08:19 09:59 16:41 10:50 13:20 15:51 17:32 19:08 21:13 22:39 01:31(+1)}}}
{\tnykdata{\anga{\tithi{7}{शुक्ल-सप्तमी}}{\time{34-50}{19:44}}\hspace{1ex}}%
{\anga{पुष्यः}{\time{18-29}{13:12}}\hspace{1ex}}{चन्द्रराशिः—\mbox{कटकः}}%
{\anga{गण्डः}{\time{20-12}{13:53}}\hspace{1ex}\uanga{वृद्धिः}}%
{\anga{गरः}{\time{7-20}{08:44}}\hspace{1ex}\anga{वणिजः}{\time{34-50}{19:44}}\hspace{1ex}\uanga{विष्टिः}}{}
}
{गङ्गा-सप्तमी\eventsep त्यागराज-जयन्ती~\#{२५३}\eventsep विद्यारण्य-स्वामि-जयन्ती\eventsep शर्करा-सप्तमी}
{Sat} 
\cfoot{\rygdata{08:56--10:31}{13:39--15:13}{05:48--07:22}}
\caldata{MAY}{12}{\sunmonth{मेषः}{29}{}{वैशाखः}{वसन्तऋतुः}{भानुः}{विकारी}{उत्तरायणम्}{वसन्तऋतुः}}
{\sunmoonrsdata{05:48}{18:22}{12:24}{01:13(+1)}{12:05}
{\kalas{04:16 05:02 09:09 08:18 09:59 16:42 10:49 13:20 15:51 17:32 19:08 21:14 22:39 01:30(+1)}}}
{\tnykdata{\anga{\tithi{8}{शुक्ल-अष्टमी}}{\time{29-33}{17:37}}\hspace{1ex}}%
{\anga{आश्रेषा}{\time{15-14}{11:53}}\hspace{1ex}}{चन्द्रराशिः—\mbox{कटकः\RIGHTarrow\textsf{11:53}}}%
{\anga{वृद्धिः}{\time{13-18}{11:07}}\hspace{1ex}\uanga{ध्रुवः}}%
{\anga{विष्टिः}{\time{2-15}{06:42}}\hspace{1ex}\anga{बवः}{\time{29-33}{17:37}}\hspace{1ex}\anga{बालवः}{\time{56-45}{04:30(+1)}}\hspace{1ex}\uanga{कौलवः}}{}
}
{काञ्ची २६ जगद्गुरु श्री-प्रज्ञाघनेन्द्र सरस्वती आराधना~\#{१४५६}}
{Sun} 
\cfoot{\rygdata{16:48--18:22}{12:05--13:39}{15:14--16:48}}
\caldata{MAY}{13}{\sunmonth{मेषः}{30}{}{वैशाखः}{वसन्तऋतुः}{सोमः}{विकारी}{उत्तरायणम्}{वसन्तऋतुः}}
{\sunmoonrsdata{05:47}{18:23}{13:21}{01:59(+1)}{12:05}
{\kalas{04:16 05:02 09:09 08:18 09:59 16:42 10:49 13:20 15:51 17:32 19:08 21:14 22:39 01:30(+1)}}}
{\tnykdata{\anga{\tithi{9}{शुक्ल-नवमी}}{\time{23-54}{15:21}}\hspace{1ex}}%
{\anga{मघा}{\time{11-35}{10:25}}\hspace{1ex}}{चन्द्रराशिः—\mbox{सिंहः}}%
{\anga{ध्रुवः}{\time{6-6}{08:14}}\hspace{1ex}\anga{व्याघातः}{\time{58-38}{05:15(+1)}}\hspace{1ex}\uanga{हर्षणः}}%
{\anga{कौलवः}{\time{23-54}{15:21}}\hspace{1ex}\anga{तैतिलः}{\time{50-58}{02:11(+1)}}\hspace{1ex}\uanga{गरः}}{}
}
{नॆरूर्-श्री-सदाशिव-ब्रह्मेन्द्र-आराधना~\#{१०५}\eventsep पुरी गोवर्धन-मठ-प्रतिष्ठापन-जयन्ती~\#{२५०४}\eventsep सीतानवमी\eventsep सिंहाचलं-चन्दन-महोत्सवः\eventsep वेङ्कटाचले पद्मावती-परिणयोत्सव-आरम्भः (गज-वाहनम्)\eventsep वसिष्ठ-महर्षि-जयन्ती}
{Mon} 
\cfoot{\rygdata{07:22--08:56}{10:31--12:05}{13:39--15:14}}
\caldata{MAY}{14}{\sunmonth{मेषः}{31}{}{वैशाखः}{वसन्तऋतुः}{मङ्गलः}{विकारी}{उत्तरायणम्}{वसन्तऋतुः}}
{\sunmoonrsdata{05:47}{18:23}{14:17}{02:44(+1)}{12:05}
{\kalas{04:16 05:01 09:09 08:18 09:59 16:42 10:49 13:20 15:52 17:32 19:08 21:14 22:39 01:30(+1)}}}
{\tnykdata{\anga{\tithi{10}{शुक्ल-दशमी}}{\time{18-0}{12:59}}\hspace{1ex}}%
{\anga{पूर्वफल्गुनी}{\time{7-40}{08:51}}\hspace{1ex}}{चन्द्रराशिः—\mbox{सिंहः\RIGHTarrow\textsf{14:27}}}%
{\anga{हर्षणः}{\time{51-5}{02:13(+1)}}\hspace{1ex}\uanga{वज्रम्}}%
{\anga{गरः}{\time{18-0}{12:59}}\hspace{1ex}\anga{वणिजः}{\time{45-0}{23:47}}\hspace{1ex}\uanga{विष्टिः}}{}
}
{\tamil{மீனாக்ஷீ திருக்கல்யாணம்}\eventsep निमिषाम्बा-जयन्ती\eventsep वेङ्कटाचले पद्मावती-परिणयम् (अश्व-वाहनम्)\eventsep वृषभ-रवि-सङ्क्रमण-विष्णुपदी-पुण्यकालः~\textsf{04:20(+1)}{\RIGHTarrow}\textsf{17:08(+1)}\eventsep श्री-वासवी-जयन्ती}
{Tue} 
\cfoot{\rygdata{15:14--16:48}{08:56--10:30}{12:05--13:39}}
\caldata{MAY}{15}{\sunmonth{वृषभः}{1}{\mbox{मेषः{\tiny\RIGHTarrow}\textsf{10:44}}}{वैशाखः}{वसन्तऋतुः}{बुधः}{विकारी}{उत्तरायणम्}{वसन्तऋतुः}}
{\sunmoonrsdata{05:47}{18:23}{15:13}{03:29(+1)}{12:05}
{\kalas{04:16 05:01 09:08 08:18 09:59 16:42 10:49 13:21 15:52 17:33 19:09 21:14 22:39 01:30(+1)}}}
{\tnykdata{\anga{\tithi{11}{शुक्ल-एकादशी}}{\time{12-1}{10:35}}\hspace{1ex}}%
{\anga{उत्तरफल्गुनी}{\time{3-39}{07:14}}\hspace{1ex}\anga{हस्तः}{\time{59-43}{05:40(+1)}}\hspace{1ex}}{चन्द्रराशिः—\mbox{कन्या}}%
{\anga{वज्रम्}{\time{43-34}{23:12}}\hspace{1ex}\uanga{सिद्धिः}}%
{\anga{विष्टिः}{\time{12-1}{10:35}}\hspace{1ex}\anga{बवः}{\time{39-4}{21:24}}\hspace{1ex}\uanga{बालवः}}{}
}
{बुध-जयन्ती\eventsep हरिवासरः{\RIGHTarrow}\textsf{16:00}\eventsep सर्व-मोहिनी-एकादशी\eventsep वेङ्कटाचले पद्मावती-परिणयोत्सव-समापनम् (गरुड-वाहनम्)}
{Wed} 
\cfoot{\rygdata{12:05--13:39}{07:21--08:56}{10:30--12:05}}
\caldata{MAY}{16}{\sunmonth{वृषभः}{2}{}{वैशाखः}{वसन्तऋतुः}{गुरुः}{विकारी}{उत्तरायणम्}{वसन्तऋतुः}}
{\sunmoonrsdata{05:47}{18:23}{16:09}{04:15(+1)}{12:05}
{\kalas{04:15 05:01 09:08 08:18 09:59 16:42 10:49 13:21 15:52 17:33 19:09 21:14 22:39 01:30(+1)}}}
{\tnykdata{\anga{\tithi{12}{शुक्ल-द्वादशी}}{\time{6-11}{08:15}}\hspace{1ex}}%
{\anga{चित्रा}{\time{56-10}{04:15(+1)}}\hspace{1ex}}{चन्द्रराशिः—\mbox{कन्या\RIGHTarrow\textsf{16:56}}}%
{\anga{सिद्धिः}{\time{36-17}{20:18}}\hspace{1ex}\uanga{व्यतीपातः}}%
{\anga{बालवः}{\time{6-11}{08:15}}\hspace{1ex}\anga{कौलवः}{\time{33-23}{19:08}}\hspace{1ex}\uanga{तैतिलः}}{}
}
{गिरिजा-कल्याणम्\eventsep परशुराम-द्वादशी\eventsep प्रदोष-व्रतम्\eventsep रुक्मिणी-द्वादशी}
{Thu} 
\cfoot{\rygdata{13:40--15:14}{05:47--07:21}{08:56--10:30}}
\caldata{MAY}{17}{\sunmonth{वृषभः}{3}{}{वैशाखः}{वसन्तऋतुः}{शुक्रः}{विकारी}{उत्तरायणम्}{वसन्तऋतुः}}
{\sunmoonrsdata{05:46}{18:24}{17:06}{05:02(+1)}{12:05}
{\kalas{04:15 05:01 09:08 08:18 09:59 16:43 10:49 13:21 15:52 17:33 19:09 21:14 22:40 01:30(+1)}}}
{\tnykdata{\anga{\tithi{13}{शुक्ल-त्रयोदशी}}{\time{0-45}{06:04}}\hspace{1ex}\anga{\tithi{14}{शुक्ल-चतुर्दशी}}{\time{56-0}{04:10(+1)}}\hspace{1ex}}%
{\anga{स्वाती}{\time{53-17}{03:05(+1)}}\hspace{1ex}}{चन्द्रराशिः—\mbox{तुला}}%
{\anga{व्यतीपातः}{\time{29-29}{17:34}}\hspace{1ex}\uanga{वरीयान्}}%
{\anga{तैतिलः}{\time{0-45}{06:04}}\hspace{1ex}\anga{गरः}{\time{28-16}{17:05}}\hspace{1ex}\anga{वणिजः}{\time{56-0}{04:10(+1)}}\hspace{1ex}\uanga{विष्टिः}}{}
}
{काञ्ची ३९ जगद्गुरु श्री-सच्चिद्विलासेन्द्र सरस्वती आराधना~\#{११४७}\eventsep नृसिंह-जयन्ती\eventsep व्यतीपात-श्राद्धम्}
{Fri} 
\cfoot{\rygdata{10:30--12:05}{15:14--16:49}{07:21--08:56}}
\caldata{MAY}{18}{\sunmonth{वृषभः}{4}{}{वैशाखः}{वसन्तऋतुः}{शनिः}{विकारी}{उत्तरायणम्}{वसन्तऋतुः}}
{\sunmoonrsdata{05:46}{18:24}{18:04}{---}{12:05}
{\kalas{04:15 05:01 09:08 08:18 09:59 16:43 10:49 13:21 15:52 17:33 19:09 21:14 22:40 01:30(+1)}}}
{\tnykdata{\anga{\tithi{15}{पौर्णमासी}}{\time{52-17}{02:41(+1)}}\hspace{1ex}}%
{\anga{विशाखा}{\time{51-24}{02:20(+1)}}\hspace{1ex}}{चन्द्रराशिः—\mbox{तुला\RIGHTarrow\textsf{20:29}}}%
{\anga{वरीयान्}{\time{23-22}{15:07}}\hspace{1ex}\uanga{परिघः}}%
{\anga{विष्टिः}{\time{24-0}{15:22}}\hspace{1ex}\anga{बवः}{\time{52-17}{02:41(+1)}}\hspace{1ex}\uanga{बालवः}}{}
}
{अन्नमाचार्य-जयन्ती\eventsep अर्धनारीश्वर-व्रतम्\eventsep छिन्नमस्ता-जयन्ती\eventsep काञ्ची कामकोटि-मठ-प्रतिष्ठापन-जयन्ती~\#{२५०१}\eventsep \tamil{நம்மாழ்வார் திருநக்ஷத்திரம்}\eventsep पार्वणव्रतम् पूर्णिमायाम्\eventsep पूर्णिमा-व्रतम्\eventsep सम्पत्-गौरी-व्रतम्\eventsep वेङ्कटाचले पूर्णिमा-गरुड-सेवा\eventsep \tamil{வைகாசி~விஶாகம்}\eventsep वैशाख-पूर्णिमा-स्नानम्\eventsep शरभ-जयन्ती}
{Sat} 
\cfoot{\rygdata{08:55--10:30}{13:40--15:14}{05:46--07:21}}
\caldata{MAY}{19}{\sunmonth{वृषभः}{5}{}{वैशाखः}{वसन्तऋतुः}{भानुः}{विकारी}{उत्तरायणम्}{वसन्तऋतुः}}
{\sunmoonsrdata{05:46}{18:24}{19:02}{05:52}{12:05}
{\kalas{04:15 05:00 09:08 08:17 09:59 16:43 10:49 13:21 15:52 17:34 19:10 21:14 22:40 01:30(+1)}}}
{\tnykdata{\anga{\tithi{16}{कृष्ण-प्रथमा}}{\time{49-51}{01:42(+1)}}\hspace{1ex}}%
{\anga{अनूराधा}{\time{50-49}{02:06(+1)}}\hspace{1ex}}{चन्द्रराशिः—\mbox{वृश्चिकः}}%
{\anga{परिघः}{\time{18-12}{13:03}}\hspace{1ex}\uanga{शिवः}}%
{\anga{बालवः}{\time{20-53}{14:07}}\hspace{1ex}\anga{कौलवः}{\time{49-51}{01:42(+1)}}\hspace{1ex}\uanga{तैतिलः}}{}
}
{काञ्ची जगद्गुरु श्री-चन्द्रशेखरेन्द्र सरस्वती ७ जयन्ती~\#{१२६}\eventsep पूर्र्णमासेष्टिः\eventsep स्थालीपाकः}
{Sun} 
\cfoot{\rygdata{16:49--18:24}{12:05--13:40}{15:15--16:49}}
\caldata{MAY}{20}{\sunmonth{वृषभः}{6}{}{वैशाखः}{वसन्तऋतुः}{सोमः}{विकारी}{उत्तरायणम्}{वसन्तऋतुः}}
{\sunmoonsrdata{05:46}{18:24}{19:58}{06:45}{12:05}
{\kalas{04:15 05:00 09:08 08:17 09:59 16:43 10:49 13:21 15:53 17:34 19:10 21:15 22:40 01:30(+1)}}}
{\tnykdata{\anga{\tithi{17}{कृष्ण-द्वितीया}}{\time{48-58}{01:21(+1)}}\hspace{1ex}}%
{\anga{ज्येष्ठा}{\time{51-44}{02:27(+1)}}\hspace{1ex}}{चन्द्रराशिः—\mbox{वृश्चिकः\RIGHTarrow\textsf{02:27(+1)}}}%
{\anga{शिवः}{\time{14-11}{11:26}}\hspace{1ex}\uanga{सिद्धः}}%
{\anga{तैतिलः}{\time{19-13}{13:27}}\hspace{1ex}\anga{गरः}{\time{48-58}{01:21(+1)}}\hspace{1ex}\uanga{वणिजः}}{}
}
{नारद-जयन्ती}
{Mon} 
\cfoot{\rygdata{07:21--08:55}{10:30--12:05}{13:40--15:15}}
\caldata{MAY}{21}{\sunmonth{वृषभः}{7}{}{वैशाखः}{वसन्तऋतुः}{मङ्गलः}{विकारी}{उत्तरायणम्}{वसन्तऋतुः}}
{\sunmoonsrdata{05:46}{18:25}{20:53}{07:38}{12:05}
{\kalas{04:15 05:00 09:08 08:17 09:59 16:43 10:49 13:21 15:53 17:34 19:10 21:15 22:40 01:30(+1)}}}
{\tnykdata{\anga{\tithi{18}{कृष्ण-तृतीया}}{\time{49-46}{01:40(+1)}}\hspace{1ex}}%
{\anga{मूला}{\time{54-19}{03:29(+1)}}\hspace{1ex}}{चन्द्रराशिः—\mbox{धनुः}}%
{\anga{सिद्धः}{\time{11-27}{10:20}}\hspace{1ex}\uanga{साध्यः}}%
{\anga{वणिजः}{\time{19-9}{13:25}}\hspace{1ex}\anga{विष्टिः}{\time{49-46}{01:40(+1)}}\hspace{1ex}\uanga{बवः}}{}
}
{षडशीति-पुण्यकालः~\textsf{13:29}{\RIGHTarrow}\textsf{13:29(+1)}\eventsep काञ्ची जगद्गुरु श्री-शङ्कर विजयेन्द्र सरस्वती आश्रम-स्वीकार-जयन्ती~\#{३७}\eventsep माधव-मासः/वसन्तऋतुः{\RIGHTarrow}\textsf{13:29}\eventsep \tamil{முருக நாயனார் (15) குருபூஜை}\eventsep \tamil{திருஞானஸம்பந்தமூர்த்தி நாயனார் (27) குருபூஜை}\eventsep \tamil{திருநீலகண்ட யாழ்ப்பாண நாயனார் (60) குருபூஜை}\eventsep \tamil{திருநீலநக்க நாயனார் (25) குருபூஜை}}
{Tue} 
\cfoot{\rygdata{15:15--16:50}{08:55--10:30}{12:05--13:40}}
\caldata{MAY}{22}{\sunmonth{वृषभः}{8}{}{वैशाखः}{वसन्तऋतुः}{बुधः}{विकारी}{उत्तरायणम्}{वसन्तऋतुः}}
{\sunmoonsrdata{05:45}{18:25}{21:44}{08:32}{12:05}
{\kalas{04:15 05:00 09:08 08:17 09:59 16:44 10:49 13:21 15:53 17:34 19:10 21:15 22:40 01:30(+1)}}}
{\tnykdata{\anga{\tithi{19}{कृष्ण-चतुर्थी}}{\time{52-18}{02:41(+1)}}\hspace{1ex}}%
{\anga{पूर्वाषाढा}{\time{58-34}{05:11(+1)}}\hspace{1ex}}{चन्द्रराशिः—\mbox{धनुः}}%
{\anga{साध्यः}{\time{10-4}{09:47}}\hspace{1ex}\uanga{शुभः}}%
{\anga{बवः}{\time{20-49}{14:05}}\hspace{1ex}\anga{बालवः}{\time{52-18}{02:41(+1)}}\hspace{1ex}\uanga{कौलवः}}{}
}
{एकदन्त-महागणपति सङ्कटहर-चतुर्थी-व्रतम्\eventsep काञ्ची ३० जगद्गुरु श्री-बोधेन्द्र सरस्वती २ आराधना~\#{१३६५}\eventsep \tamil{மயிலை~வெள்ளீஶ்வரர்~ப்ரஹ்மோத்ஸவம்}\eventsep सावित्री-व्रतम्\eventsep सती-अनसूया-जयन्ती}
{Wed} 
\cfoot{\rygdata{12:05--13:40}{07:20--08:55}{10:30--12:05}}
\caldata{MAY}{23}{\sunmonth{वृषभः}{9}{}{वैशाखः}{वसन्तऋतुः}{गुरुः}{विकारी}{उत्तरायणम्}{वसन्तऋतुः}}
{\sunmoonsrdata{05:45}{18:25}{22:31}{09:25}{12:05}
{\kalas{04:15 05:00 09:08 08:17 09:59 16:44 10:49 13:21 15:53 17:35 19:11 21:15 22:40 01:30(+1)}}}
{\tnykdata{\anga{\tithi{20}{कृष्ण-पञ्चमी}}{\time{56-22}{04:18(+1)}}\hspace{1ex}}%
{\fullanga{उत्तराषाढा}}{चन्द्रराशिः—\mbox{धनुः\RIGHTarrow\textsf{11:42}}}%
{\anga{शुभः}{\time{10-0}{09:45}}\hspace{1ex}\uanga{शुक्लः}}%
{\anga{कौलवः}{\time{24-9}{15:25}}\hspace{1ex}\anga{तैतिलः}{\time{56-22}{04:18(+1)}}\hspace{1ex}\uanga{गरः}}{}
}
{कश्यप-महर्षि-जयन्ती}
{Thu} 
\cfoot{\rygdata{13:40--15:15}{05:45--07:20}{08:55--10:30}}
\caldata{MAY}{24}{\sunmonth{वृषभः}{10}{}{वैशाखः}{वसन्तऋतुः}{शुक्रः}{विकारी}{उत्तरायणम्}{वसन्तऋतुः}}
{\sunmoonsrdata{05:45}{18:26}{23:15}{10:16}{12:05}
{\kalas{04:15 05:00 09:08 08:17 09:59 16:44 10:49 13:21 15:54 17:35 19:11 21:15 22:40 01:30(+1)}}}
{\tnykdata{\fulltithi{\tithi{21}{कृष्ण-षष्ठी}}}%
{\anga{उत्तराषाढा}{\time{4-18}{07:28}}\hspace{1ex}}{चन्द्रराशिः—\mbox{मकरः}}%
{\anga{शुक्लः}{\time{11-4}{10:11}}\hspace{1ex}\uanga{ब्रह्म}}%
{\anga{गरः}{\time{28-54}{17:19}}\hspace{1ex}\uanga{वणिजः}}{}
}
{श्रवण-व्रतम्}
{Fri} 
\cfoot{\rygdata{10:30--12:05}{15:15--16:51}{07:20--08:55}}
\caldata{MAY}{25}{\sunmonth{वृषभः}{11}{}{वैशाखः}{वसन्तऋतुः}{शनिः}{विकारी}{उत्तरायणम्}{वसन्तऋतुः}}
{\sunmoonsrdata{05:45}{18:26}{23:55}{11:05}{12:05}
{\kalas{04:14 05:00 09:08 08:17 09:59 16:44 10:49 13:21 15:54 17:35 19:11 21:16 22:40 01:30(+1)}}}
{\tnykdata{\anga{\tithi{21}{कृष्ण-षष्ठी}}{\time{1-40}{06:25}}\hspace{1ex}}%
{\anga{श्रवणः}{\time{11-9}{10:13}}\hspace{1ex}}{चन्द्रराशिः—\mbox{मकरः\RIGHTarrow\textsf{23:41}}}%
{\anga{ब्रह्म}{\time{13-0}{10:57}}\hspace{1ex}\uanga{इन्द्रः}}%
{\anga{वणिजः}{\time{1-40}{06:25}}\hspace{1ex}\anga{विष्टिः}{\time{34-37}{19:36}}\hspace{1ex}\uanga{बवः}}{}
}
{}
{Sat} 
\cfoot{\rygdata{08:55--10:30}{13:41--15:16}{05:45--07:20}}
\caldata{MAY}{26}{\sunmonth{वृषभः}{12}{}{वैशाखः}{वसन्तऋतुः}{भानुः}{विकारी}{उत्तरायणम्}{वसन्तऋतुः}}
{\sunmoonsrdata{05:45}{18:26}{00:33(+1)}{11:52}{12:06}
{\kalas{04:14 05:00 09:08 08:17 09:59 16:45 10:49 13:22 15:54 17:35 19:11 21:16 22:41 01:30(+1)}}}
{\tnykdata{\anga{\tithi{22}{कृष्ण-सप्तमी}}{\time{7-40}{08:49}}\hspace{1ex}}%
{\anga{श्रविष्ठा}{\time{18-36}{13:12}}\hspace{1ex}}{चन्द्रराशिः—\mbox{कुम्भः}}%
{\anga{इन्द्रः}{\time{15-24}{11:55}}\hspace{1ex}\uanga{वैधृतिः}}%
{\anga{बवः}{\time{7-40}{08:49}}\hspace{1ex}\anga{बालवः}{\time{40-45}{22:03}}\hspace{1ex}\uanga{कौलवः}}{}
}
{भानुसप्तमी\eventsep वैधृति-श्राद्धम्}
{Sun} 
\cfoot{\rygdata{16:51--18:26}{12:06--13:41}{15:16--16:51}}
\caldata{MAY}{27}{\sunmonth{वृषभः}{13}{}{वैशाखः}{वसन्तऋतुः}{सोमः}{विकारी}{उत्तरायणम्}{वसन्तऋतुः}}
{\sunmoonsrdata{05:45}{18:26}{01:10(+1)}{12:39}{12:06}
{\kalas{04:14 05:00 09:08 08:17 09:59 16:45 10:50 13:22 15:54 17:36 19:12 21:16 22:41 01:30(+1)}}}
{\tnykdata{\anga{\tithi{23}{कृष्ण-अष्टमी}}{\time{13-47}{11:16}}\hspace{1ex}}%
{\anga{शतभिषक्}{\time{26-5}{16:11}}\hspace{1ex}}{चन्द्रराशिः—\mbox{कुम्भः}}%
{\anga{वैधृतिः}{\time{17-51}{12:53}}\hspace{1ex}\uanga{विष्कम्भः}}%
{\anga{कौलवः}{\time{13-47}{11:16}}\hspace{1ex}\anga{तैतिलः}{\time{46-42}{00:26(+1)}}\hspace{1ex}\uanga{गरः}}{}
}
{काञ्ची ७ जगद्गुरु श्री-आनन्दज्ञानेन्द्र सरस्वती आराधना~\#{२०७४}}
{Mon} 
\cfoot{\rygdata{07:20--08:55}{10:30--12:06}{13:41--15:16}}
\caldata{MAY}{28}{\sunmonth{वृषभः}{14}{}{वैशाखः}{वसन्तऋतुः}{मङ्गलः}{विकारी}{उत्तरायणम्}{वसन्तऋतुः}}
{\sunmoonsrdata{05:45}{18:27}{01:47(+1)}{13:25}{12:06}
{\kalas{04:14 05:00 09:08 08:17 09:59 16:45 10:50 13:22 15:54 17:36 19:12 21:16 22:41 01:31(+1)}}}
{\tnykdata{\anga{\tithi{24}{कृष्ण-नवमी}}{\time{19-24}{13:31}}\hspace{1ex}}%
{\anga{पूर्वप्रोष्ठपदा}{\time{32-58}{18:56}}\hspace{1ex}}{चन्द्रराशिः—\mbox{कुम्भः\RIGHTarrow\textsf{12:17}}}%
{\anga{विष्कम्भः}{\time{19-53}{13:42}}\hspace{1ex}\uanga{प्रीतिः}}%
{\anga{गरः}{\time{19-24}{13:31}}\hspace{1ex}\anga{वणिजः}{\time{51-52}{02:30(+1)}}\hspace{1ex}\uanga{विष्टिः}}{}
}
{}
{Tue} 
\cfoot{\rygdata{15:16--16:52}{08:55--10:31}{12:06--13:41}}
\caldata{MAY}{29}{\sunmonth{वृषभः}{15}{}{वैशाखः}{वसन्तऋतुः}{बुधः}{विकारी}{उत्तरायणम्}{वसन्तऋतुः}}
{\sunmoonsrdata{05:45}{18:27}{02:25(+1)}{14:11}{12:06}
{\kalas{04:14 04:59 09:08 08:17 09:59 16:45 10:50 13:22 15:55 17:36 19:12 21:16 22:41 01:31(+1)}}}
{\tnykdata{\anga{\tithi{25}{कृष्ण-दशमी}}{\time{24-1}{15:21}}\hspace{1ex}}%
{\anga{उत्तरप्रोष्ठपदा}{\time{38-47}{21:16}}\hspace{1ex}}{चन्द्रराशिः—\mbox{मीनः}}%
{\anga{प्रीतिः}{\time{21-8}{14:12}}\hspace{1ex}\uanga{आयुष्मान्}}%
{\anga{विष्टिः}{\time{24-1}{15:21}}\hspace{1ex}\anga{बवः}{\time{55-48}{04:04(+1)}}\hspace{1ex}\uanga{बालवः}}{}
}
{आयुष्मद्-बव-सौम्य-संयॊगः\eventsep अग्निनक्षत्र-समापनम्}
{Wed} 
\cfoot{\rygdata{12:06--13:41}{07:20--08:55}{10:31--12:06}}
\caldata{MAY}{30}{\sunmonth{वृषभः}{16}{}{वैशाखः}{वसन्तऋतुः}{गुरुः}{विकारी}{उत्तरायणम्}{वसन्तऋतुः}}
{\sunmoonsrdata{05:45}{18:27}{03:04(+1)}{14:59}{12:06}
{\kalas{04:14 04:59 09:08 08:17 09:59 16:46 10:50 13:22 15:55 17:36 19:13 21:17 22:41 01:31(+1)}}}
{\tnykdata{\anga{\tithi{26}{कृष्ण-एकादशी}}{\time{27-12}{16:38}}\hspace{1ex}}%
{\anga{रेवती}{\time{43-12}{23:02}}\hspace{1ex}}{चन्द्रराशिः—\mbox{मीनः\RIGHTarrow\textsf{23:02}}}%
{\anga{आयुष्मान्}{\time{21-19}{14:16}}\hspace{1ex}\uanga{सौभाग्यः}}%
{\anga{बालवः}{\time{27-12}{16:38}}\hspace{1ex}\anga{कौलवः}{\time{58-13}{05:02(+1)}}\hspace{1ex}\uanga{तैतिलः}}{}
}
{भद्रकाळी-जयन्ती\eventsep हरिवासरः{\RIGHTarrow}\textsf{22:51}\eventsep सर्व-अपरा-एकादशी}
{Thu} 
\cfoot{\rygdata{13:41--15:17}{05:45--07:20}{08:55--10:31}}
\caldata{MAY}{31}{\sunmonth{वृषभः}{17}{}{वैशाखः}{वसन्तऋतुः}{शुक्रः}{विकारी}{उत्तरायणम्}{वसन्तऋतुः}}
{\sunmoonsrdata{05:45}{18:28}{03:46(+1)}{15:49}{12:06}
{\kalas{04:14 04:59 09:08 08:17 09:59 16:46 10:50 13:22 15:55 17:37 19:13 21:17 22:41 01:31(+1)}}}
{\tnykdata{\anga{\tithi{27}{कृष्ण-द्वादशी}}{\time{28-49}{17:16}}\hspace{1ex}}%
{\anga{अश्विनी}{\time{46-3}{00:10(+1)}}\hspace{1ex}}{चन्द्रराशिः—\mbox{मेषः}}%
{\anga{सौभाग्यः}{\time{20-17}{13:52}}\hspace{1ex}\uanga{शोभनः}}%
{\anga{तैतिलः}{\time{28-49}{17:16}}\hspace{1ex}\anga{गरः}{\time{59-1}{05:21(+1)}}\hspace{1ex}\uanga{वणिजः}}{}
}
{प्रदोष-व्रतम्}
{Fri} 
\cfoot{\rygdata{10:31--12:06}{15:17--16:52}{07:20--08:55}}
\caldata{JUNE}{1}{\sunmonth{वृषभः}{18}{}{वैशाखः}{वसन्तऋतुः}{शनिः}{विकारी}{उत्तरायणम्}{वसन्तऋतुः}}
{\sunmoonsrdata{05:45}{18:28}{04:32(+1)}{16:42}{12:06}
{\kalas{04:14 04:59 09:08 08:17 09:59 16:46 10:50 13:23 15:55 17:37 19:13 21:17 22:42 01:31(+1)}}}
{\tnykdata{\anga{\tithi{28}{कृष्ण-त्रयोदशी}}{\time{28-49}{17:16}}\hspace{1ex}}%
{\anga{अपभरणी}{\time{47-20}{00:41(+1)}}\hspace{1ex}}{चन्द्रराशिः—\mbox{मेषः}}%
{\anga{शोभनः}{\time{17-58}{12:56}}\hspace{1ex}\uanga{अतिगण्डः}}%
{\anga{वणिजः}{\time{28-49}{17:16}}\hspace{1ex}\anga{विष्टिः}{\time{58-14}{05:02(+1)}}\hspace{1ex}\uanga{शकुनिः}}{}
}
{\tamil{கழற்சிங்க நாயனார் (51) குருபூஜை}\eventsep मासशिवरात्रिः}
{Sat} 
\cfoot{\rygdata{08:55--10:31}{13:42--15:17}{05:45--07:20}}
\caldata{JUNE}{2}{\sunmonth{वृषभः}{19}{}{वैशाखः}{वसन्तऋतुः}{भानुः}{विकारी}{उत्तरायणम्}{वसन्तऋतुः}}
{\sunmoonsrdata{05:45}{18:28}{05:23(+1)}{17:37}{12:06}
{\kalas{04:14 05:00 09:08 08:17 09:59 16:46 10:50 13:23 15:56 17:37 19:13 21:17 22:42 01:31(+1)}}}
{\tnykdata{\anga{\tithi{29}{कृष्ण-चतुर्दशी}}{\time{27-17}{16:40}}\hspace{1ex}}%
{\anga{कृत्तिका}{\time{47-10}{00:37(+1)}}\hspace{1ex}}{चन्द्रराशिः—\mbox{मेषः\RIGHTarrow\textsf{06:43}}}%
{\anga{अतिगण्डः}{\time{14-25}{11:31}}\hspace{1ex}\uanga{सुकर्म}}%
{\anga{शकुनिः}{\time{27-17}{16:40}}\hspace{1ex}\anga{चतुष्पात्}{\time{56-1}{04:09(+1)}}\hspace{1ex}\uanga{नाग}}{}
}
{काञ्ची ३ जगद्गुरु श्री-सर्वज्ञात्मेन्द्र सरस्वती आराधना~\#{२३८३}\eventsep कृत्तिका-व्रतम्}
{Sun} 
\cfoot{\rygdata{16:53--18:28}{12:06--13:42}{15:17--16:53}}
\caldata{JUNE}{3}{\sunmonth{वृषभः}{20}{}{वैशाखः}{वसन्तऋतुः}{सोमः}{विकारी}{उत्तरायणम्}{वसन्तऋतुः}}
{\sunmoonsrdata{05:45}{18:28}{---}{18:36}{12:07}
{\kalas{04:15 05:00 09:08 08:17 09:59 16:47 10:50 13:23 15:56 17:38 19:14 21:18 22:42 01:31(+1)}}}
{\tnykdata{\anga{\tithi{30}{अमावास्या}}{\time{24-27}{15:31}}\hspace{1ex}}%
{\anga{रोहिणी}{\time{45-47}{00:04(+1)}}\hspace{1ex}}{चन्द्रराशिः—\mbox{वृषभः}}%
{\anga{सुकर्म}{\time{9-47}{09:40}}\hspace{1ex}\uanga{धृतिः}}%
{\anga{नाग}{\time{24-27}{15:31}}\hspace{1ex}\anga{किंस्तुघ्नः}{\time{52-36}{02:47(+1)}}\hspace{1ex}\uanga{बवः}}{}
}
{पार्वणव्रतम् अमावास्यायाम्\eventsep सोमवती अमावास्या\eventsep वैशाख-अमावास्या (अलभ्यम्–पुष्कला)\eventsep वैशाख-मास-समापनम्\eventsep वैशाख-स्नानपूर्तिः\eventsep शनि-जयन्ती\eventsep शुक-महर्षि-जयन्ती}
{Mon} 
\cfoot{\rygdata{07:20--08:56}{10:31--12:07}{13:42--15:18}}
\caldata{JUNE}{4}{\sunmonth{वृषभः}{21}{}{ज्यैष्ठः}{ग्रीष्मऋतुः}{मङ्गलः}{विकारी}{उत्तरायणम्}{वसन्तऋतुः}}
{\sunmoonrsdata{05:45}{18:29}{06:18}{19:35}{12:07}
{\kalas{04:15 05:00 09:08 08:18 09:59 16:47 10:50 13:23 15:56 17:38 19:14 21:18 22:42 01:31(+1)}}}
{\tnykdata{\anga{\tithi{1}{शुक्ल-प्रथमा}}{\time{20-31}{13:57}}\hspace{1ex}}%
{\anga{मृगशीर्षम्}{\time{43-24}{23:07}}\hspace{1ex}}{चन्द्रराशिः—\mbox{वृषभः\RIGHTarrow\textsf{11:38}}}%
{\anga{धृतिः}{\time{4-14}{07:26}}\hspace{1ex}\anga{शूलः}{\time{57-56}{04:55(+1)}}\hspace{1ex}\uanga{गण्डः}}%
{\anga{बवः}{\time{20-31}{13:57}}\hspace{1ex}\anga{बालवः}{\time{48-14}{01:02(+1)}}\hspace{1ex}\uanga{कौलवः}}{}
}
{भद्र-चतुष्टय-व्रतम्\eventsep चन्द्र-दर्शनम्\eventsep दर्शेष्टिः\eventsep करवीर-व्रतम्\eventsep पुन्नाग-गौरी-व्रतम्\eventsep स्थालीपाकः\eventsep शृङ्गेरी ३२ जगद्गुरु श्री-नृसिंह भारती आराधना}
{Tue} 
\cfoot{\rygdata{15:18--16:53}{08:56--10:31}{12:07--13:42}}
\caldata{JUNE}{5}{\sunmonth{वृषभः}{22}{}{ज्यैष्ठः}{ग्रीष्मऋतुः}{बुधः}{विकारी}{उत्तरायणम्}{वसन्तऋतुः}}
{\sunmoonrsdata{05:45}{18:29}{07:17}{20:34}{12:07}
{\kalas{04:15 05:00 09:09 08:18 10:00 16:47 10:51 13:23 15:56 17:38 19:14 21:18 22:42 01:31(+1)}}}
{\tnykdata{\anga{\tithi{2}{शुक्ल-द्वितीया}}{\time{15-45}{12:03}}\hspace{1ex}}%
{\anga{आर्द्रा}{\time{40-19}{21:52}}\hspace{1ex}}{चन्द्रराशिः—\mbox{मिथुनम्}}%
{\anga{गण्डः}{\time{51-6}{02:11(+1)}}\hspace{1ex}\uanga{वृद्धिः}}%
{\anga{कौलवः}{\time{15-45}{12:03}}\hspace{1ex}\anga{तैतिलः}{\time{43-8}{23:00}}\hspace{1ex}\uanga{गरः}}{}
}
{}
{Wed} 
\cfoot{\rygdata{12:07--13:42}{07:20--08:56}{10:31--12:07}}
\caldata{JUNE}{6}{\sunmonth{वृषभः}{23}{}{ज्यैष्ठः}{ग्रीष्मऋतुः}{गुरुः}{विकारी}{उत्तरायणम्}{वसन्तऋतुः}}
{\sunmoonrsdata{05:45}{18:29}{08:18}{21:30}{12:07}
{\kalas{04:15 05:00 09:09 08:18 10:00 16:48 10:51 13:24 15:57 17:38 19:14 21:18 22:43 01:32(+1)}}}
{\tnykdata{\anga{\tithi{3}{शुक्ल-तृतीया}}{\time{10-24}{09:55}}\hspace{1ex}}%
{\anga{पुनर्वसुः}{\time{36-44}{20:27}}\hspace{1ex}}{चन्द्रराशिः—\mbox{मिथुनम्\RIGHTarrow\textsf{14:49}}}%
{\anga{वृद्धिः}{\time{43-54}{23:18}}\hspace{1ex}\uanga{ध्रुवः}}%
{\anga{गरः}{\time{10-24}{09:55}}\hspace{1ex}\anga{वणिजः}{\time{37-35}{20:47}}\hspace{1ex}\uanga{विष्टिः}}{}
}
{रम्भा-तृतीया}
{Thu} 
\cfoot{\rygdata{13:43--15:18}{05:45--07:20}{08:56--10:32}}
\caldata{JUNE}{7}{\sunmonth{वृषभः}{24}{}{ज्यैष्ठः}{ग्रीष्मऋतुः}{शुक्रः}{विकारी}{उत्तरायणम्}{वसन्तऋतुः}}
{\sunmoonrsdata{05:45}{18:30}{09:19}{22:23}{12:07}
{\kalas{04:15 05:00 09:09 08:18 10:00 16:48 10:51 13:24 15:57 17:39 19:15 21:19 22:43 01:32(+1)}}}
{\tnykdata{\anga{\tithi{4}{शुक्ल-चतुर्थी}}{\time{4-41}{07:38}}\hspace{1ex}\anga{\tithi{5}{शुक्ल-पञ्चमी}}{\time{58-48}{05:16(+1)}}\hspace{1ex}}%
{\anga{पुष्यः}{\time{32-53}{18:54}}\hspace{1ex}}{चन्द्रराशिः—\mbox{कटकः}}%
{\anga{ध्रुवः}{\time{36-30}{20:21}}\hspace{1ex}\uanga{व्याघातः}}%
{\anga{विष्टिः}{\time{4-41}{07:38}}\hspace{1ex}\anga{बवः}{\time{31-45}{18:27}}\hspace{1ex}\anga{बालवः}{\time{58-48}{05:16(+1)}}\hspace{1ex}\uanga{कौलवः}}{}
}
{कदली-गौरी-व्रतम्/पूजा\eventsep \tamil{நமிநந்தியடிகள் நாயனார் (26) குருபூஜை}\eventsep उमा-अवतारः\eventsep श्रीनिवासमङ्गापुरे साक्षात्कार-वैभवोत्सव-आरम्भः}
{Fri} 
\cfoot{\rygdata{10:32--12:07}{15:19--16:54}{07:20--08:56}}
\caldata{JUNE}{8}{\sunmonth{वृषभः}{25}{}{ज्यैष्ठः}{ग्रीष्मऋतुः}{शनिः}{विकारी}{उत्तरायणम्}{वसन्तऋतुः}}
{\sunmoonrsdata{05:45}{18:30}{10:19}{23:12}{12:07}
{\kalas{04:15 05:00 09:09 08:18 10:00 16:48 10:51 13:24 15:57 17:39 19:15 21:19 22:43 01:32(+1)}}}
{\tnykdata{\anga{\tithi{6}{शुक्ल-षष्ठी}}{\time{52-54}{02:55(+1)}}\hspace{1ex}}%
{\anga{आश्रेषा}{\time{28-57}{17:20}}\hspace{1ex}}{चन्द्रराशिः—\mbox{कटकः\RIGHTarrow\textsf{17:20}}}%
{\anga{व्याघातः}{\time{29-2}{17:22}}\hspace{1ex}\uanga{हर्षणः}}%
{\anga{कौलवः}{\time{25-50}{16:05}}\hspace{1ex}\anga{तैतिलः}{\time{52-54}{02:55(+1)}}\hspace{1ex}\uanga{गरः}}{}
}
{आरण्य-गौरी-व्रतम्\eventsep षष्ठी-व्रतम्\eventsep काञ्ची ५० जगद्गुरु श्री-चन्द्रचूडेन्द्र सरस्वती २ आराधना~\#{७२३}\eventsep काञ्ची ६ जगद्गुरु श्री-शुद्धानन्देन्द्र सरस्वती आराधना~\#{२१४३}\eventsep \tamil{ஸோமாஸிமார நாயனார் (32) குருபூஜை}\eventsep विन्ध्यावासिनी-देवी-पूजा\eventsep श्रीनिवासमङ्गापुरे साक्षात्कार-वैभवोत्सवः}
{Sat} 
\cfoot{\rygdata{08:56--10:32}{13:43--15:19}{05:45--07:21}}
\caldata{JUNE}{9}{\sunmonth{वृषभः}{26}{}{ज्यैष्ठः}{ग्रीष्मऋतुः}{भानुः}{विकारी}{उत्तरायणम्}{वसन्तऋतुः}}
{\sunmoonrsdata{05:45}{18:30}{11:17}{23:58}{12:08}
{\kalas{04:15 05:00 09:09 08:18 10:00 16:48 10:51 13:24 15:57 17:39 19:15 21:19 22:43 01:32(+1)}}}
{\tnykdata{\anga{\tithi{7}{शुक्ल-सप्तमी}}{\time{47-7}{00:36(+1)}}\hspace{1ex}}%
{\anga{मघा}{\time{25-5}{15:47}}\hspace{1ex}}{चन्द्रराशिः—\mbox{सिंहः}}%
{\anga{हर्षणः}{\time{21-38}{14:24}}\hspace{1ex}\uanga{वज्रम्}}%
{\anga{गरः}{\time{19-59}{13:45}}\hspace{1ex}\anga{वणिजः}{\time{47-7}{00:36(+1)}}\hspace{1ex}\uanga{विष्टिः}}{}
}
{काञ्ची २३ जगद्गुरु श्री-सच्चित्सुखेन्द्र सरस्वती आराधना~\#{१५०८}\eventsep वरुण-पूजा\eventsep विजया-भानुसप्तमी\eventsep श्रीनिवासमङ्गापुरे साक्षात्कार-वैभवोत्सव-समापनम्}
{Sun} 
\cfoot{\rygdata{16:55--18:30}{12:08--13:43}{15:19--16:55}}
\caldata{JUNE}{10}{\sunmonth{वृषभः}{27}{}{ज्यैष्ठः}{ग्रीष्मऋतुः}{सोमः}{विकारी}{उत्तरायणम्}{वसन्तऋतुः}}
{\sunmoonrsdata{05:45}{18:31}{12:13}{00:43(+1)}{12:08}
{\kalas{04:15 05:00 09:09 08:18 10:00 16:49 10:51 13:24 15:58 17:39 19:16 21:19 22:44 01:32(+1)}}}
{\tnykdata{\anga{\tithi{8}{शुक्ल-अष्टमी}}{\time{41-35}{22:23}}\hspace{1ex}}%
{\anga{पूर्वफल्गुनी}{\time{21-25}{14:19}}\hspace{1ex}}{चन्द्रराशिः—\mbox{सिंहः\RIGHTarrow\textsf{19:58}}}%
{\anga{वज्रम्}{\time{14-23}{11:31}}\hspace{1ex}\uanga{सिद्धिः}}%
{\anga{विष्टिः}{\time{14-19}{11:29}}\hspace{1ex}\anga{बवः}{\time{41-35}{22:23}}\hspace{1ex}\uanga{बालवः}}{}
}
{धूमावती-जयन्ती\eventsep ज्येष्ठाष्टमी\eventsep काञ्ची १६ जगद्गुरु श्री-उज्ज्वल शङ्करेन्द्र सरस्वती आराधना~\#{१६५३}}
{Mon} 
\cfoot{\rygdata{07:21--08:56}{10:32--12:08}{13:44--15:19}}
\caldata{JUNE}{11}{\sunmonth{वृषभः}{28}{}{ज्यैष्ठः}{ग्रीष्मऋतुः}{मङ्गलः}{विकारी}{उत्तरायणम्}{वसन्तऋतुः}}
{\sunmoonrsdata{05:45}{18:31}{13:08}{01:27(+1)}{12:08}
{\kalas{04:15 05:00 09:09 08:18 10:00 16:49 10:52 13:25 15:58 17:40 19:16 21:20 22:44 01:32(+1)}}}
{\tnykdata{\anga{\tithi{9}{शुक्ल-नवमी}}{\time{36-25}{20:19}}\hspace{1ex}}%
{\anga{उत्तरफल्गुनी}{\time{18-3}{12:59}}\hspace{1ex}}{चन्द्रराशिः—\mbox{कन्या}}%
{\anga{सिद्धिः}{\time{7-23}{08:43}}\hspace{1ex}\uanga{व्यतीपातः}}%
{\anga{बालवः}{\time{8-57}{09:20}}\hspace{1ex}\anga{कौलवः}{\time{36-25}{20:19}}\hspace{1ex}\uanga{तैतिलः}}{}
}
{ब्रह्माणी-पूजा\eventsep काञ्ची ६१ जगद्गुरु श्री-महादेवेन्द्र सरस्वती ४ आराधना~\#{२७४}\eventsep महेश-नवमी\eventsep व्यतीपात-श्राद्धम्\eventsep शुक्ल-देवी-पूजा}
{Tue} 
\cfoot{\rygdata{15:20--16:55}{08:57--10:32}{12:08--13:44}}
\caldata{JUNE}{12}{\sunmonth{वृषभः}{29}{}{ज्यैष्ठः}{ग्रीष्मऋतुः}{बुधः}{विकारी}{उत्तरायणम्}{वसन्तऋतुः}}
{\sunmoonrsdata{05:45}{18:31}{14:03}{02:11(+1)}{12:08}
{\kalas{04:16 05:00 09:10 08:19 10:01 16:49 10:52 13:25 15:58 17:40 19:16 21:20 22:44 01:33(+1)}}}
{\tnykdata{\anga{\tithi{10}{शुक्ल-दशमी}}{\time{31-43}{18:27}}\hspace{1ex}}%
{\anga{हस्तः}{\time{15-9}{11:49}}\hspace{1ex}}{चन्द्रराशिः—\mbox{कन्या\RIGHTarrow\textsf{23:19}}}%
{\anga{व्यतीपातः}{\time{0-44}{06:03}}\hspace{1ex}\anga{वरीयान्}{\time{54-32}{03:34(+1)}}\hspace{1ex}\uanga{परिघः}}%
{\anga{तैतिलः}{\time{4-0}{07:21}}\hspace{1ex}\anga{गरः}{\time{31-43}{18:27}}\hspace{1ex}\anga{वणिजः}{\time{59-36}{05:36(+1)}}\hspace{1ex}\uanga{विष्टिः}}{}
}
{दशहरा/गङ्गावतरणम्/दशपापहरा-दशमी\eventsep काञ्ची १७ जगद्गुरु श्री-सदाशिवेन्द्र सरस्वती आराधना~\#{१६४५}\eventsep काञ्ची ५३ जगद्गुरु श्री-पूर्णानन्द सदाशिवेन्द्र सरस्वती आराधना~\#{५२२}}
{Wed} 
\cfoot{\rygdata{12:08--13:44}{07:21--08:57}{10:33--12:08}}
\caldata{JUNE}{13}{\sunmonth{वृषभः}{30}{}{ज्यैष्ठः}{ग्रीष्मऋतुः}{गुरुः}{विकारी}{उत्तरायणम्}{वसन्तऋतुः}}
{\sunmoonrsdata{05:46}{18:31}{14:58}{02:57(+1)}{12:09}
{\kalas{04:16 05:01 09:10 08:19 10:01 16:49 10:52 13:25 15:58 17:40 19:16 21:20 22:44 01:33(+1)}}}
{\tnykdata{\anga{\tithi{11}{शुक्ल-एकादशी}}{\time{27-38}{16:49}}\hspace{1ex}}%
{\anga{चित्रा}{\time{12-48}{10:53}}\hspace{1ex}}{चन्द्रराशिः—\mbox{तुला}}%
{\anga{परिघः}{\time{48-53}{01:19(+1)}}\hspace{1ex}\uanga{शिवः}}%
{\anga{विष्टिः}{\time{27-38}{16:49}}\hspace{1ex}\anga{बवः}{\time{55-53}{04:07(+1)}}\hspace{1ex}\uanga{बालवः}}{}
}
{अलर्मेल्मङ्गापुरे प्लवोत्सव-आरम्भः\eventsep हरिवासरः{\RIGHTarrow}\textsf{22:27}\eventsep काञ्ची १ जगद्गुरु श्री-आदि-शङ्कर भगवत्पाद आराधना~\#{२४९५}\eventsep सर्व-पाण्डव-निर्जला-एकादशी}
{Thu} 
\cfoot{\rygdata{13:44--15:20}{05:46--07:21}{08:57--10:33}}
\caldata{JUNE}{14}{\sunmonth{वृषभः}{31}{}{ज्यैष्ठः}{ग्रीष्मऋतुः}{शुक्रः}{विकारी}{उत्तरायणम्}{वसन्तऋतुः}}
{\sunmoonrsdata{05:46}{18:32}{15:55}{03:45(+1)}{12:09}
{\kalas{04:16 05:01 09:10 08:19 10:01 16:50 10:52 13:25 15:59 17:41 19:17 21:20 22:44 01:33(+1)}}}
{\tnykdata{\anga{\tithi{12}{शुक्ल-द्वादशी}}{\time{24-20}{15:30}}\hspace{1ex}}%
{\anga{स्वाती}{\time{11-12}{10:15}}\hspace{1ex}}{चन्द्रराशिः—\mbox{तुला\RIGHTarrow\textsf{03:59(+1)}}}%
{\anga{शिवः}{\time{43-54}{23:19}}\hspace{1ex}\uanga{सिद्धः}}%
{\anga{बालवः}{\time{24-20}{15:30}}\hspace{1ex}\anga{कौलवः}{\time{53-1}{02:58(+1)}}\hspace{1ex}\uanga{तैतिलः}}{}
}
{अलर्मेल्मङ्गापुरे प्लवोत्सवः\eventsep चम्पक-द्वादशी\eventsep गवामयन-द्वादशी\eventsep काञ्ची २ जगद्गुरु श्री-सुरेश्वराचार्य आराधना~\#{२४२५}\eventsep प्रदोष-व्रतम्\eventsep रामलक्ष्मण-द्वादशी}
{Fri} 
\cfoot{\rygdata{10:33--12:09}{15:20--16:56}{07:21--08:57}}
\caldata{JUNE}{15}{\sunmonth{मिथुनम्}{1}{\mbox{वृषभः{\tiny\RIGHTarrow}\textsf{17:18}}}{ज्यैष्ठः}{ग्रीष्मऋतुः}{शनिः}{विकारी}{उत्तरायणम्}{ग्रीष्मऋतुः}}
{\sunmoonrsdata{05:46}{18:32}{16:51}{04:36(+1)}{12:09}
{\kalas{04:16 05:01 09:10 08:19 10:01 16:50 10:52 13:25 15:59 17:41 19:17 21:20 22:45 01:33(+1)}}}
{\tnykdata{\anga{\tithi{13}{शुक्ल-त्रयोदशी}}{\time{21-57}{14:33}}\hspace{1ex}}%
{\anga{विशाखा}{\time{10-28}{09:57}}\hspace{1ex}}{चन्द्रराशिः—\mbox{वृश्चिकः}}%
{\anga{सिद्धः}{\time{39-42}{21:39}}\hspace{1ex}\uanga{साध्यः}}%
{\anga{तैतिलः}{\time{21-57}{14:33}}\hspace{1ex}\anga{गरः}{\time{51-9}{02:14(+1)}}\hspace{1ex}\uanga{वणिजः}}{}
}
{अलर्मेल्मङ्गापुरे प्लवोत्सवः\eventsep छत्रपति-शिवाजी-राज्याभिषेकः~\#{३४६}\eventsep दुर्गन्ध-दौर्भाग्य-नाशक-त्रयोदशी\eventsep मिथुन-रवि-सङ्क्रमण-षडशीति-पुण्यकालः~\textsf{17:18}{\RIGHTarrow}\textsf{17:18(+1)}\eventsep वेङ्कटाचले ज्येष्ठ-अभिद्येयकाभिषेकः (वज्र-कवचम्)\eventsep विद्यारण्य-स्वामि-आराधना~\#{६२८}}
{Sat} 
\cfoot{\rygdata{08:57--10:33}{13:45--15:20}{05:46--07:22}}
\caldata{JUNE}{16}{\sunmonth{मिथुनम्}{2}{}{ज्यैष्ठः}{ग्रीष्मऋतुः}{भानुः}{विकारी}{उत्तरायणम्}{ग्रीष्मऋतुः}}
{\sunmoonrsdata{05:46}{18:32}{17:48}{05:28(+1)}{12:09}
{\kalas{04:16 05:01 09:10 08:19 10:01 16:50 10:53 13:26 15:59 17:41 19:17 21:21 22:45 01:33(+1)}}}
{\tnykdata{\anga{\tithi{14}{शुक्ल-चतुर्दशी}}{\time{20-39}{14:02}}\hspace{1ex}}%
{\anga{अनूराधा}{\time{10-46}{10:05}}\hspace{1ex}}{चन्द्रराशिः—\mbox{वृश्चिकः}}%
{\anga{साध्यः}{\time{36-25}{20:20}}\hspace{1ex}\uanga{शुभः}}%
{\anga{वणिजः}{\time{20-39}{14:02}}\hspace{1ex}\anga{विष्टिः}{\time{50-27}{01:57(+1)}}\hspace{1ex}\uanga{बवः}}{}
}
{अलर्मेल्मङ्गापुरे प्लवोत्सवः\eventsep मन्वादिः-(भौत्यः-[१४])\eventsep वेङ्कटाचले ज्येष्ठ-अभिद्येयकाभिषेकः (मुत्यल-कवचम्)\eventsep वेङ्कटाचले पूर्णिमा-गरुड-सेवा}
{Sun} 
\cfoot{\rygdata{16:56--18:32}{12:09--13:45}{15:21--16:56}}
\caldata{JUNE}{17}{\sunmonth{मिथुनम्}{3}{}{ज्यैष्ठः}{ग्रीष्मऋतुः}{सोमः}{विकारी}{उत्तरायणम्}{ग्रीष्मऋतुः}}
{\sunmoonrsdata{05:46}{18:32}{18:43}{---}{12:09}
{\kalas{04:16 05:01 09:11 08:19 10:02 16:50 10:53 13:26 15:59 17:41 19:17 21:21 22:45 01:34(+1)}}}
{\tnykdata{\anga{\tithi{15}{पौर्णमासी}}{\time{20-34}{14:00}}\hspace{1ex}}%
{\anga{ज्येष्ठा}{\time{12-16}{10:41}}\hspace{1ex}}{चन्द्रराशिः—\mbox{वृश्चिकः\RIGHTarrow\textsf{10:41}}}%
{\anga{शुभः}{\time{34-8}{19:26}}\hspace{1ex}\uanga{शुक्लः}}%
{\anga{बवः}{\time{20-34}{14:00}}\hspace{1ex}\anga{बालवः}{\time{51-2}{02:11(+1)}}\hspace{1ex}\uanga{कौलवः}}{}
}
{अलर्मेल्मङ्गापुरे प्लवोत्सव-समापनम्\eventsep ऎरुवक-पूर्णिमा\eventsep कबीरदास-जयन्ती\eventsep पार्वणव्रतम् पूर्णिमायाम्\eventsep पूर्णिमा-व्रतम्\eventsep वेङ्कटाचले ज्येष्ठ-अभिद्येयकाभिषेकः (स्वर्ण-कवचम्)\eventsep वट-पूर्णिमा/वट-सावित्री-व्रतम्}
{Mon} 
\cfoot{\rygdata{07:22--08:58}{10:34--12:09}{13:45--15:21}}
\caldata{JUNE}{18}{\sunmonth{मिथुनम्}{4}{}{ज्यैष्ठः}{ग्रीष्मऋतुः}{मङ्गलः}{विकारी}{उत्तरायणम्}{ग्रीष्मऋतुः}}
{\sunmoonsrdata{05:46}{18:33}{19:35}{06:21}{12:10}
{\kalas{04:17 05:01 09:11 08:20 10:02 16:51 10:53 13:26 15:59 17:42 19:18 21:21 22:45 01:34(+1)}}}
{\tnykdata{\anga{\tithi{16}{कृष्ण-प्रथमा}}{\time{21-50}{14:31}}\hspace{1ex}}%
{\anga{मूला}{\time{15-4}{11:48}}\hspace{1ex}}{चन्द्रराशिः—\mbox{धनुः}}%
{\anga{शुक्लः}{\time{32-55}{18:57}}\hspace{1ex}\uanga{ब्रह्म}}%
{\anga{कौलवः}{\time{21-50}{14:31}}\hspace{1ex}\anga{तैतिलः}{\time{52-59}{02:58(+1)}}\hspace{1ex}\uanga{गरः}}{}
}
{पूर्र्णमासेष्टिः\eventsep स्थालीपाकः}
{Tue} 
\cfoot{\rygdata{15:21--16:57}{08:58--10:34}{12:10--13:45}}
\caldata{JUNE}{19}{\sunmonth{मिथुनम्}{5}{}{ज्यैष्ठः}{ग्रीष्मऋतुः}{बुधः}{विकारी}{उत्तरायणम्}{ग्रीष्मऋतुः}}
{\sunmoonsrdata{05:47}{18:33}{20:24}{07:15}{12:10}
{\kalas{04:17 05:02 09:11 08:20 10:02 16:51 10:53 13:26 16:00 17:42 19:18 21:21 22:46 01:34(+1)}}}
{\tnykdata{\anga{\tithi{17}{कृष्ण-द्वितीया}}{\time{24-28}{15:34}}\hspace{1ex}}%
{\anga{पूर्वाषाढा}{\time{19-12}{13:27}}\hspace{1ex}}{चन्द्रराशिः—\mbox{धनुः\RIGHTarrow\textsf{19:57}}}%
{\anga{ब्रह्म}{\time{32-45}{18:53}}\hspace{1ex}\uanga{इन्द्रः}}%
{\anga{गरः}{\time{24-28}{15:34}}\hspace{1ex}\anga{वणिजः}{\time{56-16}{04:17(+1)}}\hspace{1ex}\uanga{विष्टिः}}{}
}
{}
{Wed} 
\cfoot{\rygdata{12:10--13:45}{07:22--08:58}{10:34--12:10}}
\caldata{JUNE}{20}{\sunmonth{मिथुनम्}{6}{}{ज्यैष्ठः}{ग्रीष्मऋतुः}{गुरुः}{विकारी}{उत्तरायणम्}{ग्रीष्मऋतुः}}
{\sunmoonsrdata{05:47}{18:33}{21:09}{08:07}{12:10}
{\kalas{04:17 05:02 09:11 08:20 10:02 16:51 10:53 13:27 16:00 17:42 19:18 21:22 22:46 01:34(+1)}}}
{\tnykdata{\anga{\tithi{18}{कृष्ण-तृतीया}}{\time{28-22}{17:08}}\hspace{1ex}}%
{\anga{उत्तराषाढा}{\time{24-35}{15:37}}\hspace{1ex}}{चन्द्रराशिः—\mbox{मकरः}}%
{\anga{इन्द्रः}{\time{33-36}{19:13}}\hspace{1ex}\uanga{वैधृतिः}}%
{\anga{विष्टिः}{\time{28-22}{17:08}}\hspace{1ex}\uanga{बवः}}{}
}
{कृष्णपिङ्गल-महागणपति सङ्कटहर-चतुर्थी-व्रतम्}
{Thu} 
\cfoot{\rygdata{13:46--15:22}{05:47--07:23}{08:58--10:34}}
\caldata{JUNE}{21}{\sunmonth{मिथुनम्}{7}{}{ज्यैष्ठः}{ग्रीष्मऋतुः}{शुक्रः}{विकारी}{उत्तरायणम्}{ग्रीष्मऋतुः}}
{\sunmoonsrdata{05:47}{18:33}{21:51}{08:57}{12:10}
{\kalas{04:17 05:02 09:11 08:20 10:02 16:51 10:54 13:27 16:00 17:42 19:18 21:22 22:46 01:35(+1)}}}
{\tnykdata{\anga{\tithi{19}{कृष्ण-चतुर्थी}}{\time{33-23}{19:08}}\hspace{1ex}}%
{\anga{श्रवणः}{\time{31-3}{18:12}}\hspace{1ex}}{चन्द्रराशिः—\mbox{मकरः}}%
{\anga{वैधृतिः}{\time{35-15}{19:53}}\hspace{1ex}\uanga{विष्कम्भः}}%
{\anga{बवः}{\time{0-46}{06:05}}\hspace{1ex}\anga{बालवः}{\time{33-23}{19:08}}\hspace{1ex}\uanga{कौलवः}}{}
}
{दक्षिणायन-पुण्यकालः~\textsf{09:24}{\RIGHTarrow}\textsf{21:24}\eventsep वैधृति-श्राद्धम्\eventsep श्रवण-व्रतम्\eventsep शुक्र-मासः/उत्तरायणम्{\RIGHTarrow}\textsf{21:24}}
{Fri} 
\cfoot{\rygdata{10:34--12:10}{15:22--16:58}{07:23--08:59}}
\caldata{JUNE}{22}{\sunmonth{मिथुनम्}{8}{}{ज्यैष्ठः}{ग्रीष्मऋतुः}{शनिः}{विकारी}{उत्तरायणम्}{ग्रीष्मऋतुः}}
{\sunmoonsrdata{05:47}{18:34}{22:30}{09:46}{12:10}
{\kalas{04:17 05:02 09:12 08:20 10:03 16:51 10:54 13:27 16:00 17:43 19:19 21:22 22:46 01:35(+1)}}}
{\tnykdata{\anga{\tithi{20}{कृष्ण-पञ्चमी}}{\time{39-9}{21:27}}\hspace{1ex}}%
{\anga{श्रविष्ठा}{\time{38-15}{21:05}}\hspace{1ex}}{चन्द्रराशिः—\mbox{मकरः\RIGHTarrow\textsf{07:37}}}%
{\anga{विष्कम्भः}{\time{37-29}{20:47}}\hspace{1ex}\uanga{प्रीतिः}}%
{\anga{कौलवः}{\time{6-12}{08:16}}\hspace{1ex}\anga{तैतिलः}{\time{39-9}{21:27}}\hspace{1ex}\uanga{गरः}}{}
}
{दक्षिणायनारम्भः}
{Sat} 
\cfoot{\rygdata{08:59--10:35}{13:46--15:22}{05:47--07:23}}
\caldata{JUNE}{23}{\sunmonth{मिथुनम्}{9}{}{ज्यैष्ठः}{ग्रीष्मऋतुः}{भानुः}{विकारी}{उत्तरायणम्}{ग्रीष्मऋतुः}}
{\sunmoonsrdata{05:47}{18:34}{23:08}{10:32}{12:11}
{\kalas{04:18 05:02 09:12 08:21 10:03 16:52 10:54 13:27 16:01 17:43 19:19 21:22 22:46 01:35(+1)}}}
{\tnykdata{\anga{\tithi{21}{कृष्ण-षष्ठी}}{\time{45-12}{23:53}}\hspace{1ex}}%
{\anga{शतभिषक्}{\time{45-44}{00:05(+1)}}\hspace{1ex}}{चन्द्रराशिः—\mbox{कुम्भः}}%
{\anga{प्रीतिः}{\time{39-56}{21:46}}\hspace{1ex}\uanga{आयुष्मान्}}%
{\anga{गरः}{\time{12-10}{10:40}}\hspace{1ex}\anga{वणिजः}{\time{45-12}{23:53}}\hspace{1ex}\uanga{विष्टिः}}{}
}
{}
{Sun} 
\cfoot{\rygdata{16:58--18:34}{12:11--13:46}{15:22--16:58}}
\caldata{JUNE}{24}{\sunmonth{मिथुनम्}{10}{}{ज्यैष्ठः}{ग्रीष्मऋतुः}{सोमः}{विकारी}{उत्तरायणम्}{ग्रीष्मऋतुः}}
{\sunmoonsrdata{05:48}{18:34}{23:44}{11:18}{12:11}
{\kalas{04:18 05:03 09:12 08:21 10:03 16:52 10:54 13:27 16:01 17:43 19:19 21:22 22:47 01:35(+1)}}}
{\tnykdata{\anga{\tithi{22}{कृष्ण-सप्तमी}}{\time{51-1}{02:12(+1)}}\hspace{1ex}}%
{\anga{पूर्वप्रोष्ठपदा}{\time{52-59}{03:00(+1)}}\hspace{1ex}}{चन्द्रराशिः—\mbox{कुम्भः\RIGHTarrow\textsf{20:17}}}%
{\anga{आयुष्मान्}{\time{42-15}{22:42}}\hspace{1ex}\uanga{सौभाग्यः}}%
{\anga{विष्टिः}{\time{18-10}{13:04}}\hspace{1ex}\anga{बवः}{\time{51-1}{02:12(+1)}}\hspace{1ex}\uanga{बालवः}}{}
}
{}
{Mon} 
\cfoot{\rygdata{07:23--08:59}{10:35--12:11}{13:47--15:22}}
\caldata{JUNE}{25}{\sunmonth{मिथुनम्}{11}{}{ज्यैष्ठः}{ग्रीष्मऋतुः}{मङ्गलः}{विकारी}{उत्तरायणम्}{ग्रीष्मऋतुः}}
{\sunmoonsrdata{05:48}{18:34}{00:21(+1)}{12:04}{12:11}
{\kalas{04:18 05:03 09:12 08:21 10:03 16:52 10:54 13:28 16:01 17:43 19:19 21:23 22:47 01:35(+1)}}}
{\tnykdata{\anga{\tithi{23}{कृष्ण-अष्टमी}}{\time{56-3}{04:13(+1)}}\hspace{1ex}}%
{\anga{उत्तरप्रोष्ठपदा}{\time{59-28}{05:35(+1)}}\hspace{1ex}}{चन्द्रराशिः—\mbox{मीनः}}%
{\anga{सौभाग्यः}{\time{44-1}{23:24}}\hspace{1ex}\uanga{शोभनः}}%
{\anga{बालवः}{\time{23-40}{15:16}}\hspace{1ex}\anga{कौलवः}{\time{56-3}{04:13(+1)}}\hspace{1ex}\uanga{तैतिलः}}{}
}
{तिन्दुकाष्टमी\eventsep त्रिलोचनाष्टमी\eventsep विनायकाष्टमी\eventsep शीतलाष्टमी}
{Tue} 
\cfoot{\rygdata{15:23--16:58}{09:00--10:35}{12:11--13:47}}
\caldata{JUNE}{26}{\sunmonth{मिथुनम्}{12}{}{ज्यैष्ठः}{ग्रीष्मऋतुः}{बुधः}{विकारी}{उत्तरायणम्}{ग्रीष्मऋतुः}}
{\sunmoonsrdata{05:48}{18:34}{00:59(+1)}{12:50}{12:11}
{\kalas{04:18 05:03 09:12 08:21 10:04 16:52 10:55 13:28 16:01 17:43 19:19 21:23 22:47 01:36(+1)}}}
{\tnykdata{\anga{\tithi{24}{कृष्ण-नवमी}}{\time{59-49}{05:44(+1)}}\hspace{1ex}}%
{\fullanga{रेवती}}{चन्द्रराशिः—\mbox{मीनः}}%
{\anga{शोभनः}{\time{44-52}{23:45}}\hspace{1ex}\uanga{अतिगण्डः}}%
{\anga{तैतिलः}{\time{28-6}{17:03}}\hspace{1ex}\anga{गरः}{\time{59-49}{05:44(+1)}}\hspace{1ex}\uanga{वणिजः}}{}
}
{\tamil{ஏயர்கோன் கலிக்காம நாயனார் (28) குருபூஜை}\eventsep दुर्गा-स्वापनम्}
{Wed} 
\cfoot{\rygdata{12:11--13:47}{07:24--09:00}{10:35--12:11}}
\caldata{JUNE}{27}{\sunmonth{मिथुनम्}{13}{}{ज्यैष्ठः}{ग्रीष्मऋतुः}{गुरुः}{विकारी}{उत्तरायणम्}{ग्रीष्मऋतुः}}
{\sunmoonsrdata{05:48}{18:34}{01:39(+1)}{13:39}{12:11}
{\kalas{04:19 05:03 09:13 08:22 10:04 16:52 10:55 13:28 16:01 17:43 19:19 21:23 22:47 01:36(+1)}}}
{\tnykdata{\fulltithi{\tithi{25}{कृष्ण-दशमी}}}%
{\anga{रेवती}{\time{4-42}{07:41}}\hspace{1ex}}{चन्द्रराशिः—\mbox{मीनः\RIGHTarrow\textsf{07:41}}}%
{\anga{अतिगण्डः}{\time{44-32}{23:37}}\hspace{1ex}\uanga{सुकर्म}}%
{\anga{वणिजः}{\time{31-6}{18:15}}\hspace{1ex}\uanga{विष्टिः}}{}
}
{}
{Thu} 
\cfoot{\rygdata{13:47--15:23}{05:48--07:24}{09:00--10:36}}
\caldata{JUNE}{28}{\sunmonth{मिथुनम्}{14}{}{ज्यैष्ठः}{ग्रीष्मऋतुः}{शुक्रः}{विकारी}{उत्तरायणम्}{ग्रीष्मऋतुः}}
{\sunmoonsrdata{05:49}{18:35}{02:22(+1)}{14:29}{12:12}
{\kalas{04:19 05:04 09:13 08:22 10:04 16:53 10:55 13:28 16:01 17:44 19:20 21:23 22:48 01:36(+1)}}}
{\tnykdata{\anga{\tithi{25}{कृष्ण-दशमी}}{\time{1-58}{06:36}}\hspace{1ex}}%
{\anga{अश्विनी}{\time{8-22}{09:09}}\hspace{1ex}}{चन्द्रराशिः—\mbox{मेषः}}%
{\anga{सुकर्म}{\time{42-49}{22:56}}\hspace{1ex}\uanga{धृतिः}}%
{\anga{विष्टिः}{\time{1-58}{06:36}}\hspace{1ex}\anga{बवः}{\time{32-24}{18:46}}\hspace{1ex}\uanga{बालवः}}{}
}
{}
{Fri} 
\cfoot{\rygdata{10:36--12:12}{15:23--16:59}{07:24--09:00}}
\caldata{JUNE}{29}{\sunmonth{मिथुनम्}{15}{}{ज्यैष्ठः}{ग्रीष्मऋतुः}{शनिः}{विकारी}{उत्तरायणम्}{ग्रीष्मऋतुः}}
{\sunmoonsrdata{05:49}{18:35}{03:11(+1)}{15:23}{12:12}
{\kalas{04:19 05:04 09:13 08:22 10:04 16:53 10:55 13:28 16:02 17:44 19:20 21:23 22:48 01:36(+1)}}}
{\tnykdata{\anga{\tithi{26}{कृष्ण-एकादशी}}{\time{2-21}{06:45}}\hspace{1ex}}%
{\anga{अपभरणी}{\time{10-16}{09:56}}\hspace{1ex}}{चन्द्रराशिः—\mbox{मेषः\RIGHTarrow\textsf{16:00}}}%
{\anga{धृतिः}{\time{39-40}{21:41}}\hspace{1ex}\uanga{शूलः}}%
{\anga{बालवः}{\time{2-21}{06:45}}\hspace{1ex}\anga{कौलवः}{\time{31-51}{18:33}}\hspace{1ex}\uanga{तैतिलः}}{}
}
{चिदम्बरे ध्वजारोहणम्/पञ्चमूर्ति रथोत्सवः\eventsep हरिवासरः{\RIGHTarrow}\textsf{12:41}\eventsep कृत्तिका-व्रतम्\eventsep कूर्म-जयन्ती\eventsep सर्व-योगिनी-एकादशी}
{Sat} 
\cfoot{\rygdata{09:00--10:36}{13:48--15:23}{05:49--07:25}}
\caldata{JUNE}{30}{\sunmonth{मिथुनम्}{16}{}{ज्यैष्ठः}{ग्रीष्मऋतुः}{भानुः}{विकारी}{उत्तरायणम्}{ग्रीष्मऋतुः}}
{\sunmoonsrdata{05:49}{18:35}{04:04(+1)}{16:20}{12:12}
{\kalas{04:19 05:04 09:13 08:22 10:04 16:53 10:55 13:29 16:02 17:44 19:20 21:24 22:48 01:36(+1)}}}
{\tnykdata{\anga{\tithi{27}{कृष्ण-द्वादशी}}{\time{0-55}{06:11}}\hspace{1ex}\anga{\tithi{28}{कृष्ण-त्रयोदशी}}{\time{57-47}{04:56(+1)}}\hspace{1ex}}%
{\anga{कृत्तिका}{\time{10-24}{09:59}}\hspace{1ex}}{चन्द्रराशिः—\mbox{वृषभः}}%
{\anga{शूलः}{\time{35-6}{19:52}}\hspace{1ex}\uanga{गण्डः}}%
{\anga{तैतिलः}{\time{0-55}{06:11}}\hspace{1ex}\anga{गरः}{\time{29-33}{17:38}}\hspace{1ex}\anga{वणिजः}{\time{57-47}{04:56(+1)}}\hspace{1ex}\uanga{विष्टिः}}{}
}
{चिदम्बरे रजत चन्द्रप्रभ वाहनम्\eventsep प्रदोष-व्रतम्}
{Sun} 
\cfoot{\rygdata{16:59--18:35}{12:12--13:48}{15:24--16:59}}
\caldata{JULY}{1}{\sunmonth{मिथुनम्}{17}{}{ज्यैष्ठः}{ग्रीष्मऋतुः}{सोमः}{विकारी}{उत्तरायणम्}{ग्रीष्मऋतुः}}
{\sunmoonsrdata{05:49}{18:35}{05:02(+1)}{17:20}{12:12}
{\kalas{04:19 05:04 09:14 08:22 10:05 16:53 10:56 13:29 16:02 17:44 19:20 21:24 22:48 01:37(+1)}}}
{\tnykdata{\anga{\tithi{29}{कृष्ण-चतुर्दशी}}{\time{53-10}{03:06(+1)}}\hspace{1ex}}%
{\anga{रोहिणी}{\time{8-53}{09:23}}\hspace{1ex}}{चन्द्रराशिः—\mbox{वृषभः\RIGHTarrow\textsf{20:51}}}%
{\anga{गण्डः}{\time{29-15}{17:32}}\hspace{1ex}\uanga{वृद्धिः}}%
{\anga{विष्टिः}{\time{25-38}{16:05}}\hspace{1ex}\anga{शकुनिः}{\time{53-10}{03:06(+1)}}\hspace{1ex}\uanga{चतुष्पात्}}{}
}
{चिदम्बरे स्वर्ण-सूर्यप्रभ वाहनम्\eventsep मासशिवरात्रिः\eventsep सोममृगशीर्ष-पुण्यकालः}
{Mon} 
\cfoot{\rygdata{07:25--09:01}{10:37--12:12}{13:48--15:24}}
\caldata{JULY}{2}{\sunmonth{मिथुनम्}{18}{}{ज्यैष्ठः}{ग्रीष्मऋतुः}{मङ्गलः}{विकारी}{उत्तरायणम्}{ग्रीष्मऋतुः}}
{\sunmoonsrdata{05:50}{18:35}{---}{18:20}{12:12}
{\kalas{04:20 05:05 09:14 08:23 10:05 16:53 10:56 13:29 16:02 17:44 19:20 21:24 22:48 01:37(+1)}}}
{\tnykdata{\anga{\tithi{30}{अमावास्या}}{\time{47-20}{00:46(+1)}}\hspace{1ex}}%
{\anga{मृगशीर्षम्}{\time{5-56}{08:12}}\hspace{1ex}}{चन्द्रराशिः—\mbox{मिथुनम्}}%
{\anga{वृद्धिः}{\time{22-19}{14:45}}\hspace{1ex}\uanga{ध्रुवः}}%
{\anga{चतुष्पात्}{\time{20-22}{13:59}}\hspace{1ex}\anga{नाग}{\time{47-20}{00:46(+1)}}\hspace{1ex}\uanga{किंस्तुघ्नः}}{}
}
{भोगशायि-पूजा\eventsep भौमार्द्रा-पुण्यकालः\eventsep चिदम्बरे रजत भूत वाहनम्\eventsep ज्यैष्ठ-अमावास्या (अलभ्यम्–आर्द्रा, पुष्कला)\eventsep पार्वणव्रतम् अमावास्यायाम्}
{Tue} 
\cfoot{\rygdata{15:24--16:59}{09:01--10:37}{12:12--13:48}}
\caldata{JULY}{3}{\sunmonth{मिथुनम्}{19}{}{आषाढः}{ग्रीष्मऋतुः}{बुधः}{विकारी}{उत्तरायणम्}{ग्रीष्मऋतुः}}
{\sunmoonrsdata{05:50}{18:35}{06:03}{19:19}{12:13}
{\kalas{04:20 05:05 09:14 08:23 10:05 16:53 10:56 13:29 16:02 17:44 19:20 21:24 22:48 01:37(+1)}}}
{\tnykdata{\anga{\tithi{1}{शुक्ल-प्रथमा}}{\time{40-36}{22:05}}\hspace{1ex}}%
{\anga{आर्द्रा}{\time{1-50}{06:34}}\hspace{1ex}\anga{पुनर्वसुः}{\time{56-57}{04:37(+1)}}\hspace{1ex}}{चन्द्रराशिः—\mbox{मिथुनम्\RIGHTarrow\textsf{23:08}}}%
{\anga{ध्रुवः}{\time{14-31}{11:38}}\hspace{1ex}\uanga{व्याघातः}}%
{\anga{किंस्तुघ्नः}{\time{14-3}{11:27}}\hspace{1ex}\anga{बवः}{\time{40-36}{22:05}}\hspace{1ex}\uanga{बालवः}}{}
}
{चिदम्बरे रजत ऋषभ वाहनम्\eventsep दर्शेष्टिः\eventsep काञ्ची २५ जगद्गुरु श्री-सच्चिदानन्दघनेन्द्र सरस्वती आराधना~\#{१४७२}\eventsep स्थालीपाकः\eventsep वाराही-नवरात्र-आरम्भः}
{Wed} 
\cfoot{\rygdata{12:13--13:48}{07:26--09:01}{10:37--12:13}}
\caldata{JULY}{4}{\sunmonth{मिथुनम्}{20}{}{आषाढः}{ग्रीष्मऋतुः}{गुरुः}{विकारी}{उत्तरायणम्}{ग्रीष्मऋतुः}}
{\sunmoonrsdata{05:50}{18:35}{07:06}{20:15}{12:13}
{\kalas{04:20 05:05 09:14 08:23 10:05 16:53 10:56 13:29 16:02 17:44 19:20 21:24 22:49 01:37(+1)}}}
{\tnykdata{\anga{\tithi{2}{शुक्ल-द्वितीया}}{\time{33-18}{19:10}}\hspace{1ex}}%
{\anga{पुष्यः}{\time{51-34}{02:28(+1)}}\hspace{1ex}}{चन्द्रराशिः—\mbox{कटकः}}%
{\anga{व्याघातः}{\time{6-6}{08:17}}\hspace{1ex}\anga{हर्षणः}{\time{57-21}{04:47(+1)}}\hspace{1ex}\uanga{वज्रम्}}%
{\anga{बालवः}{\time{7-0}{08:38}}\hspace{1ex}\anga{कौलवः}{\time{33-18}{19:10}}\hspace{1ex}\anga{तैतिलः}{\time{59-33}{05:39(+1)}}\hspace{1ex}\uanga{गरः}}{}
}
{अमृतलक्ष्मी-व्रतम्\eventsep चन्द्र-दर्शनम्\eventsep चिदम्बरे रजत-गजवाहनम्\eventsep गुरुपुष्य-पुण्यकालः\eventsep जगन्नाथ-रथ-यात्रा}
{Thu} 
\cfoot{\rygdata{13:49--15:24}{05:50--07:26}{09:02--10:37}}
\caldata{JULY}{5}{\sunmonth{मिथुनम्}{21}{}{आषाढः}{ग्रीष्मऋतुः}{शुक्रः}{विकारी}{उत्तरायणम्}{ग्रीष्मऋतुः}}
{\sunmoonrsdata{05:50}{18:36}{08:08}{21:07}{12:13}
{\kalas{04:20 05:06 09:14 08:24 10:05 16:53 10:56 13:30 16:02 17:45 19:20 21:24 22:49 01:38(+1)}}}
{\tnykdata{\anga{\tithi{3}{शुक्ल-तृतीया}}{\time{25-45}{16:09}}\hspace{1ex}}%
{\anga{आश्रेषा}{\time{46-4}{00:16(+1)}}\hspace{1ex}}{चन्द्रराशिः—\mbox{कटकः\RIGHTarrow\textsf{00:16(+1)}}}%
{\anga{वज्रम्}{\time{48-32}{01:15(+1)}}\hspace{1ex}\uanga{सिद्धिः}}%
{\anga{गरः}{\time{25-45}{16:09}}\hspace{1ex}\anga{वणिजः}{\time{52-0}{02:39(+1)}}\hspace{1ex}\uanga{विष्टिः}}{}
}
{चिदम्बरे कैलास वाहनम्\eventsep \tamil{புகழ்த்துணை நாயனார் (54) குருபூஜை}}
{Fri} 
\cfoot{\rygdata{10:37--12:13}{15:24--17:00}{07:26--09:02}}
\caldata{JULY}{6}{\sunmonth{मिथुनम्}{22}{}{आषाढः}{ग्रीष्मऋतुः}{शनिः}{विकारी}{उत्तरायणम्}{ग्रीष्मऋतुः}}
{\sunmoonrsdata{05:51}{18:36}{09:09}{21:55}{12:13}
{\kalas{04:21 05:06 09:15 08:24 10:06 16:54 10:57 13:30 16:03 17:45 19:21 21:24 22:49 01:38(+1)}}}
{\tnykdata{\anga{\tithi{4}{शुक्ल-चतुर्थी}}{\time{18-16}{13:09}}\hspace{1ex}}%
{\anga{मघा}{\time{40-43}{22:08}}\hspace{1ex}}{चन्द्रराशिः—\mbox{सिंहः}}%
{\anga{सिद्धिः}{\time{39-51}{21:47}}\hspace{1ex}\uanga{व्यतीपातः}}%
{\anga{विष्टिः}{\time{18-16}{13:09}}\hspace{1ex}\anga{बवः}{\time{44-39}{23:43}}\hspace{1ex}\uanga{बालवः}}{}
}
{चिदम्बरे भिक्षाटन स्वर्णरथः\eventsep \tamil{மாணிக்கவாசகர் குருபூஜை}}
{Sat} 
\cfoot{\rygdata{09:02--10:38}{13:49--15:24}{05:51--07:26}}
\caldata{JULY}{7}{\sunmonth{मिथुनम्}{23}{}{आषाढः}{ग्रीष्मऋतुः}{भानुः}{विकारी}{उत्तरायणम्}{ग्रीष्मऋतुः}}
{\sunmoonrsdata{05:51}{18:36}{10:07}{22:41}{12:13}
{\kalas{04:21 05:06 09:15 08:24 10:06 16:54 10:57 13:30 16:03 17:45 19:21 21:25 22:49 01:38(+1)}}}
{\tnykdata{\anga{\tithi{5}{शुक्ल-पञ्चमी}}{\time{11-8}{10:19}}\hspace{1ex}}%
{\anga{पूर्वफल्गुनी}{\time{35-51}{20:12}}\hspace{1ex}}{चन्द्रराशिः—\mbox{सिंहः\RIGHTarrow\textsf{01:45(+1)}}}%
{\anga{व्यतीपातः}{\time{31-33}{18:28}}\hspace{1ex}\uanga{वरीयान्}}%
{\anga{बालवः}{\time{11-8}{10:19}}\hspace{1ex}\anga{कौलवः}{\time{37-47}{20:58}}\hspace{1ex}\uanga{तैतिलः}}{}
}
{\tamil{அமரநீதி நாயனார் (6) குருபூஜை}\eventsep चिदम्बरे रथोत्सवः\eventsep काञ्ची ३५ जगद्गुरु श्री-चित्सुखेन्द्र सरस्वती आराधना~\#{१२८३}\eventsep कुमार-षष्ठी-व्रतम्\eventsep स्कन्द-पञ्चमी\eventsep व्यतीपात-श्राद्धम्\eventsep शमी-गौरी-व्रतम्}
{Sun} 
\cfoot{\rygdata{17:00--18:36}{12:13--13:49}{15:24--17:00}}
\caldata{JULY}{8}{\sunmonth{मिथुनम्}{24}{}{आषाढः}{ग्रीष्मऋतुः}{सोमः}{विकारी}{उत्तरायणम्}{ग्रीष्मऋतुः}}
{\sunmoonrsdata{05:51}{18:36}{11:04}{23:26}{12:13}
{\kalas{04:21 05:06 09:15 08:24 10:06 16:54 10:57 13:30 16:03 17:45 19:21 21:25 22:49 01:38(+1)}}}
{\tnykdata{\anga{\tithi{6}{शुक्ल-षष्ठी}}{\time{4-36}{07:42}}\hspace{1ex}\anga{\tithi{7}{शुक्ल-सप्तमी}}{\time{58-53}{05:25(+1)}}\hspace{1ex}}%
{\anga{उत्तरफल्गुनी}{\time{31-40}{18:32}}\hspace{1ex}}{चन्द्रराशिः—\mbox{कन्या}}%
{\anga{वरीयान्}{\time{23-49}{15:23}}\hspace{1ex}\uanga{परिघः}}%
{\anga{तैतिलः}{\time{4-36}{07:42}}\hspace{1ex}\anga{गरः}{\time{31-38}{18:31}}\hspace{1ex}\anga{वणिजः}{\time{58-53}{05:25(+1)}}\hspace{1ex}\uanga{विष्टिः}}{}
}
{चिदम्बरे नटराजस्य राजसभायां महाभिषेकः\eventsep \tamil{நடராஜர் ஆனி திருமஞ்சனம்}\eventsep वैवस्वत-सप्तमी}
{Mon} 
\cfoot{\rygdata{07:27--09:02}{10:38--12:13}{13:49--15:25}}
\caldata{JULY}{9}{\sunmonth{मिथुनम्}{25}{}{आषाढः}{ग्रीष्मऋतुः}{मङ्गलः}{विकारी}{उत्तरायणम्}{ग्रीष्मऋतुः}}
{\sunmoonrsdata{05:52}{18:36}{11:59}{00:10(+1)}{12:14}
{\kalas{04:21 05:07 09:15 08:24 10:06 16:54 10:57 13:30 16:03 17:45 19:21 21:25 22:49 01:38(+1)}}}
{\tnykdata{\anga{\tithi{8}{शुक्ल-अष्टमी}}{\time{54-7}{03:30(+1)}}\hspace{1ex}}%
{\anga{हस्तः}{\time{28-23}{17:13}}\hspace{1ex}}{चन्द्रराशिः—\mbox{कन्या\RIGHTarrow\textsf{04:43(+1)}}}%
{\anga{परिघः}{\time{16-48}{12:35}}\hspace{1ex}\uanga{शिवः}}%
{\anga{विष्टिः}{\time{26-21}{16:24}}\hspace{1ex}\anga{बवः}{\time{54-7}{03:30(+1)}}\hspace{1ex}\uanga{बालवः}}{}
}
{चिदम्बरे मुत्तुप्पल्लक्कु\eventsep महिषघ्नी-पूजा}
{Tue} 
\cfoot{\rygdata{15:25--17:00}{09:03--10:38}{12:14--13:49}}
\caldata{JULY}{10}{\sunmonth{मिथुनम्}{26}{}{आषाढः}{ग्रीष्मऋतुः}{बुधः}{विकारी}{उत्तरायणम्}{ग्रीष्मऋतुः}}
{\sunmoonrsdata{05:52}{18:36}{12:54}{00:56(+1)}{12:14}
{\kalas{04:22 05:07 09:16 08:25 10:06 16:54 10:57 13:30 16:03 17:45 19:21 21:25 22:49 01:38(+1)}}}
{\tnykdata{\anga{\tithi{9}{शुक्ल-नवमी}}{\time{50-26}{02:02(+1)}}\hspace{1ex}}%
{\anga{चित्रा}{\time{26-9}{16:20}}\hspace{1ex}}{चन्द्रराशिः—\mbox{तुला}}%
{\anga{शिवः}{\time{10-36}{10:07}}\hspace{1ex}\uanga{सिद्धः}}%
{\anga{बालवः}{\time{22-7}{14:43}}\hspace{1ex}\anga{कौलवः}{\time{50-26}{02:02(+1)}}\hspace{1ex}\uanga{तैतिलः}}{}
}
{ऐन्द्री-दुर्गा-पूजा\eventsep काञ्ची १२ जगद्गुरु श्री-चन्द्रशेखरेन्द्र सरस्वती आराधना~\#{१७८५}\eventsep सुदर्शन-जयन्ती\eventsep उपेन्द्र-नवमी\eventsep वाराही-नवरात्र-समापनम्}
{Wed} 
\cfoot{\rygdata{12:14--13:49}{07:27--09:03}{10:38--12:14}}
\caldata{JULY}{11}{\sunmonth{मिथुनम्}{27}{}{आषाढः}{ग्रीष्मऋतुः}{गुरुः}{विकारी}{उत्तरायणम्}{ग्रीष्मऋतुः}}
{\sunmoonrsdata{05:52}{18:36}{13:50}{01:42(+1)}{12:14}
{\kalas{04:22 05:07 09:16 08:25 10:07 16:54 10:58 13:30 16:03 17:45 19:21 21:25 22:50 01:39(+1)}}}
{\tnykdata{\anga{\tithi{10}{शुक्ल-दशमी}}{\time{47-54}{01:02(+1)}}\hspace{1ex}}%
{\anga{स्वाती}{\time{25-2}{15:53}}\hspace{1ex}}{चन्द्रराशिः—\mbox{तुला}}%
{\anga{सिद्धः}{\time{5-20}{08:00}}\hspace{1ex}\uanga{साध्यः}}%
{\anga{तैतिलः}{\time{19-1}{13:29}}\hspace{1ex}\anga{गरः}{\time{47-54}{01:02(+1)}}\hspace{1ex}\uanga{वणिजः}}{}
}
{आशा-दशमी\eventsep चातुर्मास्यव्रत-आरम्भः\eventsep काञ्ची ४८ जगद्गुरु श्री-अद्वैतानन्दबोधेन्द्र सरस्वती आराधना~\#{८२०}\eventsep मन्वादिः-(वैवस्वतः-[७])\eventsep \tamil{பெரியாழ்வார் திருநக்ஷத்திரம்}}
{Thu} 
\cfoot{\rygdata{13:49--15:25}{05:52--07:28}{09:03--10:38}}
\caldata{JULY}{12}{\sunmonth{मिथुनम्}{28}{}{आषाढः}{ग्रीष्मऋतुः}{शुक्रः}{विकारी}{उत्तरायणम्}{ग्रीष्मऋतुः}}
{\sunmoonrsdata{05:52}{18:36}{14:45}{02:31(+1)}{12:14}
{\kalas{04:22 05:07 09:16 08:25 10:07 16:54 10:58 13:30 16:03 17:45 19:21 21:25 22:50 01:39(+1)}}}
{\tnykdata{\anga{\tithi{11}{शुक्ल-एकादशी}}{\time{46-35}{00:31(+1)}}\hspace{1ex}}%
{\anga{विशाखा}{\time{25-5}{15:55}}\hspace{1ex}}{चन्द्रराशिः—\mbox{तुला\RIGHTarrow\textsf{09:52}}}%
{\anga{साध्यः}{\time{1-0}{06:16}}\hspace{1ex}\anga{शुभः}{\time{57-38}{04:56(+1)}}\hspace{1ex}\uanga{शुक्लः}}%
{\anga{वणिजः}{\time{17-6}{12:43}}\hspace{1ex}\anga{विष्टिः}{\time{46-35}{00:31(+1)}}\hspace{1ex}\uanga{बवः}}{}
}
{गोपद्म-व्रत-आरम्भः\eventsep सर्व-शयन-एकादशी}
{Fri} 
\cfoot{\rygdata{10:39--12:14}{15:25--17:00}{07:28--09:03}}
\caldata{JULY}{13}{\sunmonth{मिथुनम्}{29}{}{आषाढः}{ग्रीष्मऋतुः}{शनिः}{विकारी}{उत्तरायणम्}{ग्रीष्मऋतुः}}
{\sunmoonrsdata{05:53}{18:36}{15:41}{03:22(+1)}{12:14}
{\kalas{04:22 05:08 09:16 08:25 10:07 16:54 10:58 13:31 16:03 17:45 19:21 21:25 22:50 01:39(+1)}}}
{\tnykdata{\anga{\tithi{12}{शुक्ल-द्वादशी}}{\time{46-28}{00:28(+1)}}\hspace{1ex}}%
{\anga{अनूराधा}{\time{26-20}{16:25}}\hspace{1ex}}{चन्द्रराशिः—\mbox{वृश्चिकः}}%
{\anga{शुक्लः}{\time{55-14}{03:59(+1)}}\hspace{1ex}\uanga{ब्रह्म}}%
{\anga{बवः}{\time{16-22}{12:26}}\hspace{1ex}\anga{बालवः}{\time{46-28}{00:28(+1)}}\hspace{1ex}\uanga{कौलवः}}{}
}
{हरिवासरः{\RIGHTarrow}\textsf{06:27}\eventsep काञ्ची ३१ जगद्गुरु श्री-ब्रह्मानन्दघनेन्द्र सरस्वती आराधना~\#{१३५२}\eventsep काञ्ची ६३ जगद्गुरु श्री-महादेवेन्द्र सरस्वती ५ आराधना~\#{२०६}\eventsep वासुदेव-द्वादशी\eventsep शाकव्रत-आरम्भः}
{Sat} 
\cfoot{\rygdata{09:03--10:39}{13:50--15:25}{05:53--07:28}}
\caldata{JULY}{14}{\sunmonth{मिथुनम्}{30}{}{आषाढः}{ग्रीष्मऋतुः}{भानुः}{विकारी}{उत्तरायणम्}{ग्रीष्मऋतुः}}
{\sunmoonrsdata{05:53}{18:36}{16:36}{04:15(+1)}{12:14}
{\kalas{04:23 05:08 09:16 08:26 10:07 16:54 10:58 13:31 16:03 17:45 19:21 21:25 22:50 01:39(+1)}}}
{\tnykdata{\anga{\tithi{13}{शुक्ल-त्रयोदशी}}{\time{47-33}{00:54(+1)}}\hspace{1ex}}%
{\anga{ज्येष्ठा}{\time{28-46}{17:24}}\hspace{1ex}}{चन्द्रराशिः—\mbox{वृश्चिकः\RIGHTarrow\textsf{17:24}}}%
{\anga{ब्रह्म}{\time{53-45}{03:23(+1)}}\hspace{1ex}\uanga{इन्द्रः}}%
{\anga{कौलवः}{\time{16-51}{12:38}}\hspace{1ex}\anga{तैतिलः}{\time{47-33}{00:54(+1)}}\hspace{1ex}\uanga{गरः}}{}
}
{\tamil{ஜ்யேஷ்டாபிஷேகம்}\eventsep प्रदोष-व्रतम्}
{Sun} 
\cfoot{\rygdata{17:00--18:36}{12:14--13:50}{15:25--17:00}}
\caldata{JULY}{15}{\sunmonth{मिथुनम्}{31}{}{आषाढः}{ग्रीष्मऋतुः}{सोमः}{विकारी}{उत्तरायणम्}{ग्रीष्मऋतुः}}
{\sunmoonrsdata{05:53}{18:36}{17:29}{05:08(+1)}{12:14}
{\kalas{04:23 05:08 09:16 08:26 10:07 16:54 10:58 13:31 16:03 17:45 19:21 21:25 22:50 01:39(+1)}}}
{\tnykdata{\anga{\tithi{14}{शुक्ल-चतुर्दशी}}{\time{49-46}{01:48(+1)}}\hspace{1ex}}%
{\anga{मूला}{\time{32-19}{18:49}}\hspace{1ex}}{चन्द्रराशिः—\mbox{धनुः}}%
{\anga{इन्द्रः}{\time{53-11}{03:10(+1)}}\hspace{1ex}\uanga{वैधृतिः}}%
{\anga{गरः}{\time{18-31}{13:18}}\hspace{1ex}\anga{वणिजः}{\time{49-46}{01:48(+1)}}\hspace{1ex}\uanga{विष्टिः}}{}
}
{पवित्र-चतुर्दशी}
{Mon} 
\cfoot{\rygdata{07:28--09:04}{10:39--12:14}{13:50--15:25}}
\caldata{JULY}{16}{\sunmonth{मिथुनम्}{32}{\mbox{मिथुनम्{\tiny\RIGHTarrow}\textsf{04:09(+1)}}}{आषाढः}{ग्रीष्मऋतुः}{मङ्गलः}{विकारी}{उत्तरायणम्}{ग्रीष्मऋतुः}}
{\sunmoonrsdata{05:54}{18:35}{18:19}{---}{12:14}
{\kalas{04:23 05:08 09:17 08:26 10:07 16:54 10:58 13:31 16:03 17:45 19:21 21:25 22:50 01:39(+1)}}}
{\tnykdata{\anga{\tithi{15}{पौर्णमासी}}{\time{53-5}{03:08(+1)}}\hspace{1ex}}%
{\anga{पूर्वाषाढा}{\time{36-57}{20:41}}\hspace{1ex}}{चन्द्रराशिः—\mbox{धनुः\RIGHTarrow\textsf{03:12(+1)}}}%
{\anga{वैधृतिः}{\time{53-27}{03:17(+1)}}\hspace{1ex}\uanga{विष्कम्भः}}%
{\anga{विष्टिः}{\time{21-17}{14:25}}\hspace{1ex}\anga{बवः}{\time{53-5}{03:08(+1)}}\hspace{1ex}\uanga{बालवः}}{}
}
{आषाढ-पूर्णिमा-स्नानम्\eventsep चन्द्र-ग्रहणम्-(केतुग्रस्त)~\textsf{01:31(+1)}{\RIGHTarrow}\textsf{04:29(+1)}\eventsep गुरु-पूर्णिमा/व्यास-पूजा\eventsep काञ्ची १० जगद्गुरु श्री-सुरेश्वरेन्द्र सरस्वती आराधना~\#{१८९३}\eventsep कोकिल-व्रतम्\eventsep कटक-सङ्क्रमण-पुण्यकालः~\textsf{16:09}{\RIGHTarrow}\textsf{04:09(+1)}\eventsep मन्वादिः-(ब्रह्मः-[१०])\eventsep पार्वणव्रतम् पूर्णिमायाम्\eventsep पूर्णिमा-व्रतम्\eventsep वेङ्कटाचले पूर्णिमा-गरुड-सेवा\eventsep वैधृति-श्राद्धम्\eventsep यतिचातुर्मास्यव्रत-आरम्भः\eventsep शिव-शयनोत्सवः}
{Tue} 
\cfoot{\rygdata{15:25--17:00}{09:04--10:39}{12:14--13:50}}
\caldata{JULY}{17}{\sunmonth{कटकः}{1}{}{आषाढः}{ग्रीष्मऋतुः}{बुधः}{विकारी}{दक्षिणायनम्}{ग्रीष्मऋतुः}}
{\sunmoonsrdata{05:54}{18:35}{19:05}{06:00}{12:15}
{\kalas{04:23 05:09 09:17 08:26 10:08 16:54 10:58 13:31 16:03 17:45 19:21 21:25 22:50 01:40(+1)}}}
{\tnykdata{\anga{\tithi{16}{कृष्ण-प्रथमा}}{\time{57-23}{04:51(+1)}}\hspace{1ex}}%
{\anga{उत्तराषाढा}{\time{42-35}{22:56}}\hspace{1ex}}{चन्द्रराशिः—\mbox{मकरः}}%
{\anga{विष्कम्भः}{\time{54-29}{03:42(+1)}}\hspace{1ex}\uanga{प्रीतिः}}%
{\anga{बालवः}{\time{25-7}{15:57}}\hspace{1ex}\anga{कौलवः}{\time{57-23}{04:51(+1)}}\hspace{1ex}\uanga{तैतिलः}}{}
}
{काञ्ची ५४ जगद्गुरु श्री-व्यासाचल महादेवेन्द्र सरस्वती आराधना~\#{५१३}\eventsep पूर्र्णमासेष्टिः\eventsep सर्वनदी-रजस्वला\eventsep स्थालीपाकः}
{Wed} 
\cfoot{\rygdata{12:15--13:50}{07:29--09:04}{10:39--12:15}}
\caldata{JULY}{18}{\sunmonth{कटकः}{2}{}{आषाढः}{ग्रीष्मऋतुः}{गुरुः}{विकारी}{दक्षिणायनम्}{ग्रीष्मऋतुः}}
{\sunmoonsrdata{05:54}{18:35}{19:48}{06:51}{12:15}
{\kalas{04:24 05:09 09:17 08:26 10:08 16:54 10:59 13:31 16:03 17:45 19:20 21:25 22:50 01:40(+1)}}}
{\tnykdata{\fulltithi{\tithi{17}{कृष्ण-द्वितीया}}}%
{\anga{श्रवणः}{\time{49-4}{01:32(+1)}}\hspace{1ex}}{चन्द्रराशिः—\mbox{मकरः}}%
{\anga{प्रीतिः}{\time{56-10}{04:22(+1)}}\hspace{1ex}\uanga{आयुष्मान्}}%
{\anga{तैतिलः}{\time{29-52}{17:51}}\hspace{1ex}\uanga{गरः}}{}
}
{अष्टनाग-पूजा\eventsep अशून्यशयन-व्रतम्\eventsep सर्वनदी-रजस्वला\eventsep श्रवण-व्रतम्}
{Thu} 
\cfoot{\rygdata{13:50--15:25}{05:54--07:29}{09:04--10:39}}
\caldata{JULY}{19}{\sunmonth{कटकः}{3}{}{आषाढः}{ग्रीष्मऋतुः}{शुक्रः}{विकारी}{दक्षिणायनम्}{ग्रीष्मऋतुः}}
{\sunmoonsrdata{05:54}{18:35}{20:28}{07:40}{12:15}
{\kalas{04:24 05:09 09:17 08:27 10:08 16:54 10:59 13:31 16:03 17:44 19:20 21:25 22:50 01:40(+1)}}}
{\tnykdata{\anga{\tithi{17}{कृष्ण-द्वितीया}}{\time{2-31}{06:55}}\hspace{1ex}}%
{\anga{श्रविष्ठा}{\time{56-10}{04:23(+1)}}\hspace{1ex}}{चन्द्रराशिः—\mbox{मकरः\RIGHTarrow\textsf{14:56}}}%
{\anga{आयुष्मान्}{\time{58-20}{05:15(+1)}}\hspace{1ex}\uanga{सौभाग्यः}}%
{\anga{गरः}{\time{2-31}{06:55}}\hspace{1ex}\anga{वणिजः}{\time{35-20}{20:03}}\hspace{1ex}\uanga{विष्टिः}}{}
}
{\tamil{ஆடி~வெள்ளிக்கிழமை}\eventsep सर्वनदी-रजस्वला}
{Fri} 
\cfoot{\rygdata{10:40--12:15}{15:25--17:00}{07:30--09:05}}
\caldata{JULY}{20}{\sunmonth{कटकः}{4}{}{आषाढः}{ग्रीष्मऋतुः}{शनिः}{विकारी}{दक्षिणायनम्}{ग्रीष्मऋतुः}}
{\sunmoonsrdata{05:55}{18:35}{21:06}{08:27}{12:15}
{\kalas{04:24 05:09 09:17 08:27 10:08 16:54 10:59 13:31 16:03 17:44 19:20 21:25 22:50 01:40(+1)}}}
{\tnykdata{\anga{\tithi{18}{कृष्ण-तृतीया}}{\time{8-16}{09:13}}\hspace{1ex}}%
{\fullanga{शतभिषक्}}{चन्द्रराशिः—\mbox{कुम्भः}}%
{\fullanga{सौभाग्यः}}%
{\anga{विष्टिः}{\time{8-16}{09:13}}\hspace{1ex}\anga{बवः}{\time{41-18}{22:26}}\hspace{1ex}\uanga{बालवः}}{}
}
{गजानन-महागणपति सङ्कटहर-चतुर्थी-व्रतम्}
{Sat} 
\cfoot{\rygdata{09:05--10:40}{13:50--15:25}{05:55--07:30}}
\caldata{JULY}{21}{\sunmonth{कटकः}{5}{}{आषाढः}{ग्रीष्मऋतुः}{भानुः}{विकारी}{दक्षिणायनम्}{ग्रीष्मऋतुः}}
{\sunmoonsrdata{05:55}{18:35}{21:43}{09:13}{12:15}
{\kalas{04:24 05:10 09:18 08:27 10:08 16:53 10:59 13:31 16:03 17:44 19:20 21:25 22:50 01:40(+1)}}}
{\tnykdata{\anga{\tithi{19}{कृष्ण-चतुर्थी}}{\time{14-21}{11:39}}\hspace{1ex}}%
{\anga{शतभिषक्}{\time{3-38}{07:22}}\hspace{1ex}}{चन्द्रराशिः—\mbox{कुम्भः\RIGHTarrow\textsf{03:38(+1)}}}%
{\anga{सौभाग्यः}{\time{0-46}{06:14}}\hspace{1ex}\uanga{शोभनः}}%
{\anga{बालवः}{\time{14-21}{11:39}}\hspace{1ex}\anga{कौलवः}{\time{47-23}{00:52(+1)}}\hspace{1ex}\uanga{तैतिलः}}{}
}
{}
{Sun} 
\cfoot{\rygdata{17:00--18:35}{12:15--13:50}{15:25--17:00}}
\caldata{JULY}{22}{\sunmonth{कटकः}{6}{}{आषाढः}{ग्रीष्मऋतुः}{सोमः}{विकारी}{दक्षिणायनम्}{ग्रीष्मऋतुः}}
{\sunmoonsrdata{05:55}{18:35}{22:19}{09:59}{12:15}
{\kalas{04:24 05:10 09:18 08:27 10:08 16:53 10:59 13:31 16:03 17:44 19:20 21:25 22:50 01:40(+1)}}}
{\tnykdata{\anga{\tithi{20}{कृष्ण-पञ्चमी}}{\time{20-21}{14:04}}\hspace{1ex}}%
{\anga{पूर्वप्रोष्ठपदा}{\time{11-7}{10:22}}\hspace{1ex}}{चन्द्रराशिः—\mbox{मीनः}}%
{\anga{शोभनः}{\time{3-14}{07:13}}\hspace{1ex}\uanga{अतिगण्डः}}%
{\anga{तैतिलः}{\time{20-21}{14:04}}\hspace{1ex}\anga{गरः}{\time{53-12}{03:12(+1)}}\hspace{1ex}\uanga{वणिजः}}{}
}
{विष्णुपदी-पुण्यकालः~\textsf{01:56(+1)}{\RIGHTarrow}\textsf{14:44(+1)}}
{Mon} 
\cfoot{\rygdata{07:30--09:05}{10:40--12:15}{13:50--15:25}}
\caldata{JULY}{23}{\sunmonth{कटकः}{7}{}{आषाढः}{ग्रीष्मऋतुः}{मङ्गलः}{विकारी}{दक्षिणायनम्}{ग्रीष्मऋतुः}}
{\sunmoonsrdata{05:55}{18:34}{22:56}{10:44}{12:15}
{\kalas{04:25 05:10 09:18 08:27 10:08 16:53 10:59 13:31 16:03 17:44 19:20 21:25 22:50 01:40(+1)}}}
{\tnykdata{\anga{\tithi{21}{कृष्ण-षष्ठी}}{\time{25-51}{16:16}}\hspace{1ex}}%
{\anga{उत्तरप्रोष्ठपदा}{\time{18-9}{13:11}}\hspace{1ex}}{चन्द्रराशिः—\mbox{मीनः}}%
{\anga{अतिगण्डः}{\time{5-24}{08:05}}\hspace{1ex}\uanga{सुकर्म}}%
{\anga{वणिजः}{\time{25-51}{16:16}}\hspace{1ex}\anga{विष्टिः}{\time{58-16}{05:14(+1)}}\hspace{1ex}\uanga{बवः}}{}
}
{शुचि-मासः/ग्रीष्मऋतुः{\RIGHTarrow}\textsf{08:20}}
{Tue} 
\cfoot{\rygdata{15:25--17:00}{09:05--10:40}{12:15--13:50}}
\caldata{JULY}{24}{\sunmonth{कटकः}{8}{}{आषाढः}{ग्रीष्मऋतुः}{बुधः}{विकारी}{दक्षिणायनम्}{ग्रीष्मऋतुः}}
{\sunmoonsrdata{05:56}{18:34}{23:34}{11:31}{12:15}
{\kalas{04:25 05:10 09:18 08:27 10:08 16:53 10:59 13:31 16:03 17:44 19:20 21:25 22:50 01:40(+1)}}}
{\tnykdata{\anga{\tithi{22}{कृष्ण-सप्तमी}}{\time{30-23}{18:05}}\hspace{1ex}}%
{\anga{रेवती}{\time{24-19}{15:40}}\hspace{1ex}}{चन्द्रराशिः—\mbox{मीनः\RIGHTarrow\textsf{15:40}}}%
{\anga{सुकर्म}{\time{6-56}{08:42}}\hspace{1ex}\uanga{धृतिः}}%
{\anga{बवः}{\time{30-23}{18:05}}\hspace{1ex}\uanga{बालवः}}{}
}
{चामुण्डेश्वरी-जयन्ती}
{Wed} 
\cfoot{\rygdata{12:15--13:50}{07:30--09:05}{10:40--12:15}}
\caldata{JULY}{25}{\sunmonth{कटकः}{9}{}{आषाढः}{ग्रीष्मऋतुः}{गुरुः}{विकारी}{दक्षिणायनम्}{ग्रीष्मऋतुः}}
{\sunmoonsrdata{05:56}{18:34}{00:15(+1)}{12:20}{12:15}
{\kalas{04:25 05:10 09:18 08:28 10:09 16:53 10:59 13:31 16:02 17:44 19:20 21:25 22:50 01:40(+1)}}}
{\tnykdata{\anga{\tithi{23}{कृष्ण-अष्टमी}}{\time{33-32}{19:21}}\hspace{1ex}}%
{\anga{अश्विनी}{\time{29-11}{17:36}}\hspace{1ex}}{चन्द्रराशिः—\mbox{मेषः}}%
{\anga{धृतिः}{\time{7-30}{08:56}}\hspace{1ex}\uanga{शूलः}}%
{\anga{बालवः}{\time{2-9}{06:48}}\hspace{1ex}\anga{कौलवः}{\time{33-32}{19:21}}\hspace{1ex}\uanga{तैतिलः}}{}
}
{}
{Thu} 
\cfoot{\rygdata{13:50--15:25}{05:56--07:31}{09:05--10:40}}
\caldata{JULY}{26}{\sunmonth{कटकः}{10}{}{आषाढः}{ग्रीष्मऋतुः}{शुक्रः}{विकारी}{दक्षिणायनम्}{ग्रीष्मऋतुः}}
{\sunmoonsrdata{05:56}{18:34}{01:00(+1)}{13:11}{12:15}
{\kalas{04:25 05:11 09:18 08:28 10:09 16:53 10:59 13:31 16:02 17:43 19:19 21:24 22:50 01:40(+1)}}}
{\tnykdata{\anga{\tithi{24}{कृष्ण-नवमी}}{\time{34-59}{19:56}}\hspace{1ex}}%
{\anga{अपभरणी}{\time{32-25}{18:54}}\hspace{1ex}}{चन्द्रराशिः—\mbox{मेषः\RIGHTarrow\textsf{01:07(+1)}}}%
{\anga{शूलः}{\time{6-51}{08:41}}\hspace{1ex}\uanga{गण्डः}}%
{\anga{तैतिलः}{\time{4-29}{07:44}}\hspace{1ex}\anga{गरः}{\time{34-59}{19:56}}\hspace{1ex}\uanga{वणिजः}}{}
}
{\tamil{ஆடி~வெள்ளிக்கிழமை}}
{Fri} 
\cfoot{\rygdata{10:40--12:15}{15:24--16:59}{07:31--09:06}}
\caldata{JULY}{27}{\sunmonth{कटकः}{11}{}{आषाढः}{ग्रीष्मऋतुः}{शनिः}{विकारी}{दक्षिणायनम्}{ग्रीष्मऋतुः}}
{\sunmoonsrdata{05:56}{18:34}{01:50(+1)}{14:06}{12:15}
{\kalas{04:25 05:11 09:18 08:28 10:09 16:53 10:59 13:31 16:02 17:43 19:19 21:24 22:50 01:40(+1)}}}
{\tnykdata{\anga{\tithi{25}{कृष्ण-दशमी}}{\time{34-33}{19:46}}\hspace{1ex}}%
{\anga{कृत्तिका}{\time{33-48}{19:28}}\hspace{1ex}}{चन्द्रराशिः—\mbox{वृषभः}}%
{\anga{गण्डः}{\time{4-46}{07:51}}\hspace{1ex}\uanga{वृद्धिः}}%
{\anga{वणिजः}{\time{5-0}{07:57}}\hspace{1ex}\anga{विष्टिः}{\time{34-33}{19:46}}\hspace{1ex}\uanga{बवः}}{}
}
{\tamil{ஆடிக்~கிருத்திகை}\eventsep कृत्तिका-व्रतम्\eventsep \tamil{மூர்த்தி நாயனார் (15) குருபூஜை}\eventsep \tamil{புகழ்ச்சோழ நாயனார் (39) குருபூஜை}}
{Sat} 
\cfoot{\rygdata{09:06--10:40}{13:50--15:24}{05:56--07:31}}
\caldata{JULY}{28}{\sunmonth{कटकः}{12}{}{आषाढः}{ग्रीष्मऋतुः}{भानुः}{विकारी}{दक्षिणायनम्}{ग्रीष्मऋतुः}}
{\sunmoonsrdata{05:56}{18:33}{02:45(+1)}{15:03}{12:15}
{\kalas{04:25 05:11 09:18 08:28 10:09 16:52 10:59 13:31 16:02 17:43 19:19 21:24 22:50 01:40(+1)}}}
{\tnykdata{\anga{\tithi{26}{कृष्ण-एकादशी}}{\time{32-12}{18:49}}\hspace{1ex}}%
{\anga{रोहिणी}{\time{33-17}{19:16}}\hspace{1ex}}{चन्द्रराशिः—\mbox{वृषभः}}%
{\anga{वृद्धिः}{\time{1-9}{06:24}}\hspace{1ex}\anga{ध्रुवः}{\time{56-0}{04:21(+1)}}\hspace{1ex}\uanga{व्याघातः}}%
{\anga{बवः}{\time{3-37}{07:23}}\hspace{1ex}\anga{बालवः}{\time{32-12}{18:49}}\hspace{1ex}\uanga{कौलवः}}{}
}
{सर्व-कामिका-एकादशी\eventsep \tamil{திருப்பாணாழ்வார் திருநக்ஷத்திரம்}}
{Sun} 
\cfoot{\rygdata{16:59--18:33}{12:15--13:50}{15:24--16:59}}
\caldata{JULY}{29}{\sunmonth{कटकः}{13}{}{आषाढः}{ग्रीष्मऋतुः}{सोमः}{विकारी}{दक्षिणायनम्}{ग्रीष्मऋतुः}}
{\sunmoonsrdata{05:57}{18:33}{03:44(+1)}{16:02}{12:15}
{\kalas{04:26 05:11 09:18 08:28 10:09 16:52 10:59 13:31 16:02 17:43 19:19 21:24 22:50 01:40(+1)}}}
{\tnykdata{\anga{\tithi{27}{कृष्ण-द्वादशी}}{\time{28-0}{17:09}}\hspace{1ex}}%
{\anga{मृगशीर्षम्}{\time{30-57}{18:20}}\hspace{1ex}}{चन्द्रराशिः—\mbox{वृषभः\RIGHTarrow\textsf{06:53}}}%
{\anga{व्याघातः}{\time{49-24}{01:43(+1)}}\hspace{1ex}\uanga{हर्षणः}}%
{\anga{कौलवः}{\time{0-19}{06:04}}\hspace{1ex}\anga{तैतिलः}{\time{28-0}{17:09}}\hspace{1ex}\anga{गरः}{\time{55-16}{04:03(+1)}}\hspace{1ex}\uanga{वणिजः}}{}
}
{हरिवासरः{\RIGHTarrow}\textsf{00:28}\eventsep सोम-प्रदोष-व्रतम्\eventsep सोममृगशीर्ष-पुण्यकालः}
{Mon} 
\cfoot{\rygdata{07:31--09:06}{10:40--12:15}{13:49--15:24}}
\caldata{JULY}{30}{\sunmonth{कटकः}{14}{}{आषाढः}{ग्रीष्मऋतुः}{मङ्गलः}{विकारी}{दक्षिणायनम्}{ग्रीष्मऋतुः}}
{\sunmoonsrdata{05:57}{18:33}{04:46(+1)}{17:01}{12:15}
{\kalas{04:26 05:11 09:19 08:28 10:09 16:52 10:59 13:31 16:02 17:42 19:18 21:24 22:49 01:41(+1)}}}
{\tnykdata{\anga{\tithi{28}{कृष्ण-त्रयोदशी}}{\time{22-10}{14:49}}\hspace{1ex}}%
{\anga{आर्द्रा}{\time{27-0}{16:45}}\hspace{1ex}}{चन्द्रराशिः—\mbox{मिथुनम्}}%
{\anga{हर्षणः}{\time{41-34}{22:35}}\hspace{1ex}\uanga{वज्रम्}}%
{\anga{वणिजः}{\time{22-10}{14:49}}\hspace{1ex}\anga{विष्टिः}{\time{48-44}{01:27(+1)}}\hspace{1ex}\uanga{शकुनिः}}{}
}
{भौमार्द्रा-पुण्यकालः\eventsep \tamil{கூற்றுவ நாயனார் (38) குருபூஜை}\eventsep मासशिवरात्रिः}
{Tue} 
\cfoot{\rygdata{15:24--16:58}{09:06--10:40}{12:15--13:49}}
\caldata{JULY}{31}{\sunmonth{कटकः}{15}{}{आषाढः}{ग्रीष्मऋतुः}{बुधः}{विकारी}{दक्षिणायनम्}{ग्रीष्मऋतुः}}
{\sunmoonsrdata{05:57}{18:32}{05:50(+1)}{18:00}{12:15}
{\kalas{04:26 05:12 09:19 08:28 10:09 16:52 10:59 13:30 16:01 17:42 19:18 21:24 22:49 01:41(+1)}}}
{\tnykdata{\anga{\tithi{29}{कृष्ण-चतुर्दशी}}{\time{15-0}{11:57}}\hspace{1ex}}%
{\anga{पुनर्वसुः}{\time{21-43}{14:39}}\hspace{1ex}}{चन्द्रराशिः—\mbox{मिथुनम्\RIGHTarrow\textsf{09:13}}}%
{\anga{वज्रम्}{\time{32-42}{19:02}}\hspace{1ex}\uanga{सिद्धिः}}%
{\anga{शकुनिः}{\time{15-0}{11:57}}\hspace{1ex}\anga{चतुष्पात्}{\time{41-1}{22:22}}\hspace{1ex}\uanga{नाग}}{}
}
{आषाढ (कटक) अमावास्या (अलभ्यम्–पुनर्वसुः)\eventsep काञ्ची ३८ जगद्गुरु श्री-अभिनवशङ्करेन्द्र सरस्वती आराधना~\#{११८०}\eventsep काञ्ची ४६ जगद्गुरु श्री-सान्द्रानन्दबोधेन्द्र सरस्वती आराधना~\#{९२२}\eventsep पार्वणव्रतम् अमावास्यायाम्}
{Wed} 
\cfoot{\rygdata{12:15--13:49}{07:32--09:06}{10:40--12:15}}
\caldata{AUGUST}{1}{\sunmonth{कटकः}{16}{}{आषाढः}{ग्रीष्मऋतुः}{गुरुः}{विकारी}{दक्षिणायनम्}{ग्रीष्मऋतुः}}
{\sunmoonsrdata{05:57}{18:32}{---}{18:55}{12:15}
{\kalas{04:26 05:12 09:19 08:28 10:09 16:52 10:59 13:30 16:01 17:42 19:18 21:24 22:49 01:41(+1)}}}
{\tnykdata{\anga{\tithi{30}{अमावास्या}}{\time{6-50}{08:42}}\hspace{1ex}\anga{\tithi{1}{शुक्ल-प्रथमा}}{\time{58-4}{05:11(+1)}}\hspace{1ex}}%
{\anga{पुष्यः}{\time{15-30}{12:09}}\hspace{1ex}}{चन्द्रराशिः—\mbox{कटकः}}%
{\anga{सिद्धिः}{\time{23-6}{15:12}}\hspace{1ex}\uanga{व्यतीपातः}}%
{\anga{नाग}{\time{6-50}{08:42}}\hspace{1ex}\anga{किंस्तुघ्नः}{\time{32-30}{18:58}}\hspace{1ex}\anga{बवः}{\time{58-4}{05:11(+1)}}\hspace{1ex}\uanga{बालवः}}{}
}
{आषाढ-स्नानपूर्तिः\eventsep दीप-पूजा\eventsep दर्शेष्टिः\eventsep गुरुपुष्य-पुण्यकालः\eventsep काञ्ची ४१ जगद्गुरु श्री-गङ्गाधरेन्द्र सरस्वती २ आराधना~\#{१०७०}\eventsep पति-सञ्जीवनी-व्रतम्\eventsep स्थालीपाकः\eventsep व्यतीपात-श्राद्धम्}
{Thu} 
\cfoot{\rygdata{13:49--15:23}{05:57--07:32}{09:06--10:40}}
\caldata{AUGUST}{2}{\sunmonth{कटकः}{17}{}{श्रावणः}{वर्षऋतुः}{शुक्रः}{विकारी}{दक्षिणायनम्}{ग्रीष्मऋतुः}}
{\sunmoonrsdata{05:58}{18:32}{06:53}{19:46}{12:15}
{\kalas{04:26 05:12 09:19 08:28 10:09 16:51 10:59 13:30 16:01 17:42 19:18 21:23 22:49 01:41(+1)}}}
{\tnykdata{\anga{\tithi{2}{शुक्ल-द्वितीया}}{\time{49-5}{01:36(+1)}}\hspace{1ex}}%
{\anga{आश्रेषा}{\time{8-43}{09:27}}\hspace{1ex}}{चन्द्रराशिः—\mbox{कटकः\RIGHTarrow\textsf{09:27}}}%
{\anga{व्यतीपातः}{\time{13-6}{11:12}}\hspace{1ex}\uanga{वरीयान्}}%
{\anga{बालवः}{\time{23-34}{15:24}}\hspace{1ex}\anga{कौलवः}{\time{49-5}{01:36(+1)}}\hspace{1ex}\uanga{तैतिलः}}{}
}
{\tamil{ஆடி~வெள்ளிக்கிழமை}\eventsep चन्द्र-दर्शनम्\eventsep मनोरथ-द्वितीया\eventsep सत्यनारायण-जयन्ती}
{Fri} 
\cfoot{\rygdata{10:40--12:15}{15:23--16:58}{07:32--09:06}}
\caldata{AUGUST}{3}{\sunmonth{कटकः}{18}{}{श्रावणः}{वर्षऋतुः}{शनिः}{विकारी}{दक्षिणायनम्}{ग्रीष्मऋतुः}}
{\sunmoonrsdata{05:58}{18:31}{07:54}{20:35}{12:15}
{\kalas{04:26 05:12 09:19 08:29 10:09 16:51 10:59 13:30 16:01 17:41 19:17 21:23 22:49 01:41(+1)}}}
{\tnykdata{\anga{\tithi{3}{शुक्ल-तृतीया}}{\time{40-18}{22:05}}\hspace{1ex}}%
{\anga{मघा}{\time{1-49}{06:42}}\hspace{1ex}\anga{पूर्वफल्गुनी}{\time{55-13}{04:03(+1)}}\hspace{1ex}}{चन्द्रराशिः—\mbox{सिंहः}}%
{\anga{वरीयान्}{\time{3-2}{07:11}}\hspace{1ex}\anga{परिघः}{\time{53-14}{03:15(+1)}}\hspace{1ex}\uanga{शिवः}}%
{\anga{तैतिलः}{\time{14-39}{11:49}}\hspace{1ex}\anga{गरः}{\time{40-18}{22:05}}\hspace{1ex}\uanga{वणिजः}}{}
}
{\tamil{ஆடிப்~பெருக்கு}\eventsep हरियाली-तृतीया\eventsep मधुश्रावणि-व्रतम्\eventsep पार्वती-पवित्रारोपणम्\eventsep स्वर्ण-गौरी-व्रतम्\eventsep \tamil{திருவாடிப்பூரம்}}
{Sat} 
\cfoot{\rygdata{09:06--10:40}{13:49--15:23}{05:58--07:32}}
\caldata{AUGUST}{4}{\sunmonth{कटकः}{19}{}{श्रावणः}{वर्षऋतुः}{भानुः}{विकारी}{दक्षिणायनम्}{ग्रीष्मऋतुः}}
{\sunmoonrsdata{05:58}{18:31}{08:54}{21:22}{12:15}
{\kalas{04:26 05:12 09:19 08:29 10:09 16:51 10:59 13:30 16:01 17:41 19:17 21:23 22:49 01:41(+1)}}}
{\tnykdata{\anga{\tithi{4}{शुक्ल-चतुर्थी}}{\time{32-6}{18:49}}\hspace{1ex}}%
{\anga{उत्तरफल्गुनी}{\time{49-19}{01:42(+1)}}\hspace{1ex}}{चन्द्रराशिः—\mbox{सिंहः\RIGHTarrow\textsf{09:26}}}%
{\anga{शिवः}{\time{43-59}{23:34}}\hspace{1ex}\uanga{सिद्धः}}%
{\anga{वणिजः}{\time{6-6}{08:25}}\hspace{1ex}\anga{विष्टिः}{\time{32-6}{18:49}}\hspace{1ex}\anga{बवः}{\time{58-20}{05:18(+1)}}\hspace{1ex}\uanga{बालवः}}{}
}
{दूर्वा-गणपति-व्रतम्}
{Sun} 
\cfoot{\rygdata{16:57--18:31}{12:15--13:49}{15:23--16:57}}
\caldata{AUGUST}{5}{\sunmonth{कटकः}{20}{}{श्रावणः}{वर्षऋतुः}{सोमः}{विकारी}{दक्षिणायनम्}{ग्रीष्मऋतुः}}
{\sunmoonrsdata{05:58}{18:31}{09:51}{22:07}{12:14}
{\kalas{04:27 05:12 09:19 08:29 10:09 16:50 10:59 13:30 16:00 17:41 19:17 21:23 22:49 01:40(+1)}}}
{\tnykdata{\anga{\tithi{5}{शुक्ल-पञ्चमी}}{\time{24-50}{15:54}}\hspace{1ex}}%
{\anga{हस्तः}{\time{44-27}{23:45}}\hspace{1ex}}{चन्द्रराशिः—\mbox{कन्या}}%
{\anga{सिद्धः}{\time{35-34}{20:12}}\hspace{1ex}\uanga{साध्यः}}%
{\anga{बालवः}{\time{24-50}{15:54}}\hspace{1ex}\anga{कौलवः}{\time{51-39}{02:38(+1)}}\hspace{1ex}\uanga{तैतिलः}}{}
}
{गरुड-पञ्चमी\eventsep कल्कि-जयन्ती\eventsep नाग-पञ्चमी}
{Mon} 
\cfoot{\rygdata{07:32--09:06}{10:40--12:14}{13:49--15:23}}
\caldata{AUGUST}{6}{\sunmonth{कटकः}{21}{}{श्रावणः}{वर्षऋतुः}{मङ्गलः}{विकारी}{दक्षिणायनम्}{ग्रीष्मऋतुः}}
{\sunmoonrsdata{05:58}{18:30}{10:48}{22:53}{12:14}
{\kalas{04:27 05:13 09:19 08:29 10:09 16:50 10:59 13:30 16:00 17:40 19:16 21:22 22:48 01:40(+1)}}}
{\tnykdata{\anga{\tithi{6}{शुक्ल-षष्ठी}}{\time{18-48}{13:30}}\hspace{1ex}}%
{\anga{चित्रा}{\time{40-55}{22:21}}\hspace{1ex}}{चन्द्रराशिः—\mbox{कन्या\RIGHTarrow\textsf{10:59}}}%
{\anga{साध्यः}{\time{28-13}{17:16}}\hspace{1ex}\uanga{शुभः}}%
{\anga{तैतिलः}{\time{18-48}{13:30}}\hspace{1ex}\anga{गरः}{\time{46-20}{00:31(+1)}}\hspace{1ex}\uanga{वणिजः}}{}
}
{षष्ठी-व्रतम्\eventsep \tamil{பெருமிழலைக் குறும்ப நாயனார் (22) குருபூஜை}\eventsep सूपौदन-व्रतम्}
{Tue} 
\cfoot{\rygdata{15:22--16:56}{09:06--10:40}{12:14--13:48}}
\caldata{AUGUST}{7}{\sunmonth{कटकः}{22}{}{श्रावणः}{वर्षऋतुः}{बुधः}{विकारी}{दक्षिणायनम्}{ग्रीष्मऋतुः}}
{\sunmoonrsdata{05:58}{18:30}{11:45}{23:40}{12:14}
{\kalas{04:27 05:13 09:19 08:29 10:09 16:50 10:59 13:29 16:00 17:40 19:16 21:22 22:48 01:40(+1)}}}
{\tnykdata{\anga{\tithi{7}{शुक्ल-सप्तमी}}{\time{14-15}{11:41}}\hspace{1ex}}%
{\anga{स्वाती}{\time{38-56}{21:33}}\hspace{1ex}}{चन्द्रराशिः—\mbox{तुला}}%
{\anga{शुभः}{\time{22-6}{14:49}}\hspace{1ex}\uanga{शुक्लः}}%
{\anga{वणिजः}{\time{14-15}{11:41}}\hspace{1ex}\anga{विष्टिः}{\time{42-34}{23:00}}\hspace{1ex}\uanga{बवः}}{}
}
{अव्यङ्ग-सप्तमी\eventsep \tamil{சுந்தரமூர்த்தி நாயனார் (63) குருபூஜை}\eventsep द्वादश-सप्तमी\eventsep \tamil{கழறிற்றறிவார்/சேரமான் பெருமாள் நாயனார் (36) குருபூஜை}\eventsep पापनाशनी-सप्तमी\eventsep तुलसीदास-जयन्ती\eventsep शीतला-सप्तमी}
{Wed} 
\cfoot{\rygdata{12:14--13:48}{07:32--09:06}{10:40--12:14}}
\caldata{AUGUST}{8}{\sunmonth{कटकः}{23}{}{श्रावणः}{वर्षऋतुः}{गुरुः}{विकारी}{दक्षिणायनम्}{ग्रीष्मऋतुः}}
{\sunmoonrsdata{05:59}{18:30}{12:41}{00:29(+1)}{12:14}
{\kalas{04:27 05:13 09:19 08:29 10:09 16:50 10:59 13:29 15:59 17:39 19:16 21:22 22:48 01:40(+1)}}}
{\tnykdata{\anga{\tithi{8}{शुक्ल-अष्टमी}}{\time{11-18}{10:30}}\hspace{1ex}}%
{\anga{विशाखा}{\time{38-35}{21:25}}\hspace{1ex}}{चन्द्रराशिः—\mbox{तुला\RIGHTarrow\textsf{15:23}}}%
{\anga{शुक्लः}{\time{17-19}{12:55}}\hspace{1ex}\uanga{ब्रह्म}}%
{\anga{बवः}{\time{11-18}{10:30}}\hspace{1ex}\anga{बालवः}{\time{40-28}{22:10}}\hspace{1ex}\uanga{कौलवः}}{}
}
{दुर्गा-व्रत-आरम्भः}
{Thu} 
\cfoot{\rygdata{13:48--15:22}{05:59--07:33}{09:06--10:40}}
\caldata{AUGUST}{9}{\sunmonth{कटकः}{24}{}{श्रावणः}{वर्षऋतुः}{शुक्रः}{विकारी}{दक्षिणायनम्}{ग्रीष्मऋतुः}}
{\sunmoonrsdata{05:59}{18:29}{13:37}{01:19(+1)}{12:14}
{\kalas{04:27 05:13 09:19 08:29 10:09 16:49 10:59 13:29 15:59 17:39 19:15 21:22 22:48 01:40(+1)}}}
{\tnykdata{\anga{\tithi{9}{शुक्ल-नवमी}}{\time{10-3}{10:00}}\hspace{1ex}}%
{\anga{अनूराधा}{\time{39-52}{21:56}}\hspace{1ex}}{चन्द्रराशिः—\mbox{वृश्चिकः}}%
{\anga{ब्रह्म}{\time{13-52}{11:32}}\hspace{1ex}\uanga{इन्द्रः}}%
{\anga{कौलवः}{\time{10-3}{10:00}}\hspace{1ex}\anga{तैतिलः}{\time{40-1}{21:59}}\hspace{1ex}\uanga{गरः}}{}
}
{\tamil{ஆடி~வெள்ளிக்கிழமை}\eventsep काञ्ची ५७ जगद्गुरु श्री-परमशिवेन्द्र सरस्वती २ आराधना~\#{४३४}\eventsep कौमारी-पूजा\eventsep सेङ्गालिपुरम् अनन्तराम-दीक्षित-जयन्ती~\#{११७}\eventsep वरलक्ष्मी-व्रतम्}
{Fri} 
\cfoot{\rygdata{10:40--12:14}{15:22--16:55}{07:33--09:06}}
\caldata{AUGUST}{10}{\sunmonth{कटकः}{25}{}{श्रावणः}{वर्षऋतुः}{शनिः}{विकारी}{दक्षिणायनम्}{ग्रीष्मऋतुः}}
{\sunmoonrsdata{05:59}{18:29}{14:32}{02:11(+1)}{12:14}
{\kalas{04:27 05:13 09:19 08:29 10:09 16:49 10:59 13:29 15:59 17:39 19:15 21:21 22:48 01:40(+1)}}}
{\tnykdata{\anga{\tithi{10}{शुक्ल-दशमी}}{\time{10-23}{10:08}}\hspace{1ex}}%
{\anga{ज्येष्ठा}{\time{42-40}{23:03}}\hspace{1ex}}{चन्द्रराशिः—\mbox{वृश्चिकः\RIGHTarrow\textsf{23:03}}}%
{\anga{इन्द्रः}{\time{11-41}{10:39}}\hspace{1ex}\uanga{वैधृतिः}}%
{\anga{गरः}{\time{10-23}{10:08}}\hspace{1ex}\anga{वणिजः}{\time{41-7}{22:26}}\hspace{1ex}\uanga{विष्टिः}}{}
}
{काञ्ची २९ जगद्गुरु श्री-पूर्णबोधेन्द्र सरस्वती आराधना~\#{१४०२}\eventsep \tamil{கோட்புலி நாயனார் (55) குருபூஜை}\eventsep \tamil{கலிய நாயனார் (43) குருபூஜை}\eventsep वेद/दधि-व्रत-आरम्भः\eventsep वैधृति-श्राद्धम्}
{Sat} 
\cfoot{\rygdata{09:06--10:40}{13:48--15:21}{05:59--07:33}}
\caldata{AUGUST}{11}{\sunmonth{कटकः}{26}{}{श्रावणः}{वर्षऋतुः}{भानुः}{विकारी}{दक्षिणायनम्}{ग्रीष्मऋतुः}}
{\sunmoonrsdata{05:59}{18:28}{15:25}{03:03(+1)}{12:14}
{\kalas{04:27 05:13 09:19 08:29 10:09 16:48 10:59 13:29 15:58 17:38 19:14 21:21 22:47 01:40(+1)}}}
{\tnykdata{\anga{\tithi{11}{शुक्ल-एकादशी}}{\time{12-12}{10:52}}\hspace{1ex}}%
{\anga{मूला}{\time{46-48}{00:42(+1)}}\hspace{1ex}}{चन्द्रराशिः—\mbox{धनुः}}%
{\anga{वैधृतिः}{\time{10-39}{10:15}}\hspace{1ex}\uanga{विष्कम्भः}}%
{\anga{विष्टिः}{\time{12-12}{10:52}}\hspace{1ex}\anga{बवः}{\time{43-36}{23:26}}\hspace{1ex}\uanga{बालवः}}{}
}
{हरिवासरः{\RIGHTarrow}\textsf{17:08}\eventsep सर्व-पवित्रोपान-एकादशी}
{Sun} 
\cfoot{\rygdata{16:55--18:28}{12:14--13:47}{15:21--16:55}}
\caldata{AUGUST}{12}{\sunmonth{कटकः}{27}{}{श्रावणः}{वर्षऋतुः}{सोमः}{विकारी}{दक्षिणायनम्}{ग्रीष्मऋतुः}}
{\sunmoonrsdata{05:59}{18:28}{16:15}{03:55(+1)}{12:14}
{\kalas{04:27 05:13 09:19 08:29 10:09 16:48 10:59 13:28 15:58 17:38 19:14 21:21 22:47 01:40(+1)}}}
{\tnykdata{\anga{\tithi{12}{शुक्ल-द्वादशी}}{\time{15-18}{12:06}}\hspace{1ex}}%
{\anga{पूर्वाषाढा}{\time{52-3}{02:49(+1)}}\hspace{1ex}}{चन्द्रराशिः—\mbox{धनुः}}%
{\anga{विष्कम्भः}{\time{10-37}{10:14}}\hspace{1ex}\uanga{प्रीतिः}}%
{\anga{बालवः}{\time{15-18}{12:06}}\hspace{1ex}\anga{कौलवः}{\time{47-15}{00:54(+1)}}\hspace{1ex}\uanga{तैतिलः}}{}
}
{दामोदर-द्वादशी\eventsep दधि-व्रत-आरम्भः\eventsep सोम-प्रदोष-व्रतम्\eventsep शाकव्रत-समापनम्}
{Mon} 
\cfoot{\rygdata{07:33--09:06}{10:40--12:13}{13:47--15:21}}
\caldata{AUGUST}{13}{\sunmonth{कटकः}{28}{}{श्रावणः}{वर्षऋतुः}{मङ्गलः}{विकारी}{दक्षिणायनम्}{ग्रीष्मऋतुः}}
{\sunmoonrsdata{05:59}{18:27}{17:02}{04:46(+1)}{12:13}
{\kalas{04:27 05:13 09:19 08:29 10:09 16:48 10:59 13:28 15:58 17:37 19:14 21:20 22:47 01:40(+1)}}}
{\tnykdata{\anga{\tithi{13}{शुक्ल-त्रयोदशी}}{\time{19-26}{13:46}}\hspace{1ex}}%
{\anga{उत्तराषाढा}{\time{58-11}{05:16(+1)}}\hspace{1ex}}{चन्द्रराशिः—\mbox{धनुः\RIGHTarrow\textsf{09:24}}}%
{\anga{प्रीतिः}{\time{11-25}{10:34}}\hspace{1ex}\uanga{आयुष्मान्}}%
{\anga{तैतिलः}{\time{19-26}{13:46}}\hspace{1ex}\anga{गरः}{\time{51-50}{02:44(+1)}}\hspace{1ex}\uanga{वणिजः}}{}
}
{अनङ्ग-त्रयोदशी}
{Tue} 
\cfoot{\rygdata{15:20--16:54}{09:06--10:40}{12:13--13:47}}
\caldata{AUGUST}{14}{\sunmonth{कटकः}{29}{}{श्रावणः}{वर्षऋतुः}{बुधः}{विकारी}{दक्षिणायनम्}{ग्रीष्मऋतुः}}
{\sunmoonrsdata{06:00}{18:27}{17:46}{05:36(+1)}{12:13}
{\kalas{04:27 05:13 09:19 08:29 10:09 16:47 10:58 13:28 15:57 17:37 19:13 21:20 22:47 01:40(+1)}}}
{\tnykdata{\anga{\tithi{14}{शुक्ल-चतुर्दशी}}{\time{24-24}{15:45}}\hspace{1ex}}%
{\fullanga{श्रवणः}}{चन्द्रराशिः—\mbox{मकरः}}%
{\anga{आयुष्मान्}{\time{12-51}{11:08}}\hspace{1ex}\uanga{सौभाग्यः}}%
{\anga{वणिजः}{\time{24-24}{15:45}}\hspace{1ex}\anga{विष्टिः}{\time{57-7}{04:51(+1)}}\hspace{1ex}\uanga{बवः}}{}
}
{वेङ्कटाचले पूर्णिमा-गरुड-सेवा\eventsep श्रवण-व्रतम्}
{Wed} 
\cfoot{\rygdata{12:13--13:47}{07:33--09:06}{10:40--12:13}}
\caldata{AUGUST}{15}{\sunmonth{कटकः}{30}{}{श्रावणः}{वर्षऋतुः}{गुरुः}{विकारी}{दक्षिणायनम्}{ग्रीष्मऋतुः}}
{\sunmoonrsdata{06:00}{18:26}{18:27}{---}{12:13}
{\kalas{04:27 05:13 09:19 08:29 10:08 16:47 10:58 13:28 15:57 17:37 19:13 21:20 22:46 01:40(+1)}}}
{\tnykdata{\anga{\tithi{15}{पौर्णमासी}}{\time{29-57}{17:59}}\hspace{1ex}}%
{\anga{श्रवणः}{\time{4-58}{07:59}}\hspace{1ex}}{चन्द्रराशिः—\mbox{मकरः\RIGHTarrow\textsf{21:25}}}%
{\anga{सौभाग्यः}{\time{14-46}{11:54}}\hspace{1ex}\uanga{शोभनः}}%
{\anga{बवः}{\time{29-57}{17:59}}\hspace{1ex}\uanga{बालवः}}{}
}
{ऋग्वेद-उपाकर्म\eventsep गायत्री-जयन्ती\eventsep हयग्रीव-जयन्ती\eventsep काञ्ची २० जगद्गुरु श्री-मूकशङ्करेन्द्र सरस्वती आराधना~\#{१५८३}\eventsep नारिकेल-पूर्णिमा\eventsep पार्वणव्रतम् पूर्णिमायाम्\eventsep पूर्णिमा-व्रतम्\eventsep रक्षाबन्धनम्\eventsep संस्कृत-दिवसः\eventsep सर्प-बलि-प्रारम्भः\eventsep वैखानस-महर्षि-जयन्ती\eventsep यजुर्वेद-उपाकर्म}
{Thu} 
\cfoot{\rygdata{13:46--15:20}{06:00--07:33}{09:06--10:40}}
\caldata{AUGUST}{16}{\sunmonth{कटकः}{31}{}{श्रावणः}{वर्षऋतुः}{शुक्रः}{विकारी}{दक्षिणायनम्}{ग्रीष्मऋतुः}}
{\sunmoonsrdata{06:00}{18:26}{19:06}{06:23}{12:13}
{\kalas{04:27 05:14 09:19 08:29 10:08 16:46 10:58 13:27 15:57 17:36 19:12 21:19 22:46 01:40(+1)}}}
{\tnykdata{\anga{\tithi{16}{कृष्ण-प्रथमा}}{\time{35-54}{20:21}}\hspace{1ex}}%
{\anga{श्रविष्ठा}{\time{12-12}{10:53}}\hspace{1ex}}{चन्द्रराशिः—\mbox{कुम्भः}}%
{\anga{शोभनः}{\time{17-1}{12:48}}\hspace{1ex}\uanga{अतिगण्डः}}%
{\anga{बालवः}{\time{2-53}{07:09}}\hspace{1ex}\anga{कौलवः}{\time{35-54}{20:21}}\hspace{1ex}\uanga{तैतिलः}}{}
}
{\tamil{ஆடி~வெள்ளிக்கிழமை}\eventsep गायत्री-जपः\eventsep काञ्ची जगद्गुरु श्री-जयेन्द्र सरस्वती जयन्ती~\#{८५}\eventsep पूर्र्णमासेष्टिः\eventsep स्थालीपाकः}
{Fri} 
\cfoot{\rygdata{10:40--12:13}{15:19--16:53}{07:33--09:06}}
\caldata{AUGUST}{17}{\sunmonth{सिंहः}{1}{\mbox{कटकः{\tiny\RIGHTarrow}\textsf{12:33}}}{श्रावणः}{वर्षऋतुः}{शनिः}{विकारी}{दक्षिणायनम्}{वर्षऋतुः}}
{\sunmoonsrdata{06:00}{18:25}{19:43}{07:10}{12:13}
{\kalas{04:27 05:14 09:19 08:29 10:08 16:46 10:58 13:27 15:56 17:36 19:12 21:19 22:46 01:39(+1)}}}
{\tnykdata{\anga{\tithi{17}{कृष्ण-द्वितीया}}{\time{42-1}{22:48}}\hspace{1ex}}%
{\anga{शतभिषक्}{\time{19-41}{13:52}}\hspace{1ex}}{चन्द्रराशिः—\mbox{कुम्भः}}%
{\anga{अतिगण्डः}{\time{19-26}{13:46}}\hspace{1ex}\uanga{सुकर्म}}%
{\anga{तैतिलः}{\time{8-57}{09:35}}\hspace{1ex}\anga{गरः}{\time{42-1}{22:48}}\hspace{1ex}\uanga{वणिजः}}{}
}
{अशून्यशयन-व्रतम्\eventsep बृहती-वृक्षक-पूजा\eventsep भीम-चण्डी-जयन्ती\eventsep सिंह-रवि-सङ्क्रमण-विष्णुपदी-पुण्यकालः~\textsf{06:09}{\RIGHTarrow}\textsf{18:57}\eventsep श्री-राघवेन्द्र-स्वामि-आराधना~\#{३४८}}
{Sat} 
\cfoot{\rygdata{09:06--10:39}{13:46--15:19}{06:00--07:33}}
\caldata{AUGUST}{18}{\sunmonth{सिंहः}{2}{}{श्रावणः}{वर्षऋतुः}{भानुः}{विकारी}{दक्षिणायनम्}{वर्षऋतुः}}
{\sunmoonsrdata{06:00}{18:25}{20:19}{07:56}{12:12}
{\kalas{04:27 05:14 09:19 08:29 10:08 16:45 10:58 13:27 15:56 17:35 19:11 21:19 22:45 01:39(+1)}}}
{\tnykdata{\anga{\tithi{18}{कृष्ण-तृतीया}}{\time{48-3}{01:13(+1)}}\hspace{1ex}}%
{\anga{पूर्वप्रोष्ठपदा}{\time{27-10}{16:52}}\hspace{1ex}}{चन्द्रराशिः—\mbox{कुम्भः\RIGHTarrow\textsf{10:07}}}%
{\anga{सुकर्म}{\time{21-51}{14:45}}\hspace{1ex}\uanga{धृतिः}}%
{\anga{वणिजः}{\time{15-3}{12:01}}\hspace{1ex}\anga{विष्टिः}{\time{48-3}{01:13(+1)}}\hspace{1ex}\uanga{बवः}}{}
}
{\tamil{ஆவணி~ஞாயிற்றுக்கிழமை}\eventsep कज्जली-तृतीया\eventsep तुष्टि-प्राप्ति-तृतीया}
{Sun} 
\cfoot{\rygdata{16:52--18:25}{12:12--13:45}{15:19--16:52}}
\caldata{AUGUST}{19}{\sunmonth{सिंहः}{3}{}{श्रावणः}{वर्षऋतुः}{सोमः}{विकारी}{दक्षिणायनम्}{वर्षऋतुः}}
{\sunmoonsrdata{06:00}{18:24}{20:55}{08:41}{12:12}
{\kalas{04:27 05:14 09:19 08:29 10:08 16:45 10:58 13:26 15:55 17:35 19:11 21:18 22:45 01:39(+1)}}}
{\tnykdata{\anga{\tithi{19}{कृष्ण-चतुर्थी}}{\time{53-44}{03:30(+1)}}\hspace{1ex}}%
{\anga{उत्तरप्रोष्ठपदा}{\time{34-23}{19:45}}\hspace{1ex}}{चन्द्रराशिः—\mbox{मीनः}}%
{\anga{धृतिः}{\time{24-6}{15:39}}\hspace{1ex}\uanga{शूलः}}%
{\anga{बवः}{\time{20-57}{14:23}}\hspace{1ex}\anga{बालवः}{\time{53-44}{03:30(+1)}}\hspace{1ex}\uanga{कौलवः}}{}
}
{बहुला-चतुर्थी\eventsep हेरम्ब-महागणपति महासङ्कटहर-चतुर्थी-व्रतम्\eventsep हेरम्ब-महागणपति सङ्कटहर-चतुर्थी-व्रतम्}
{Mon} 
\cfoot{\rygdata{07:33--09:06}{10:39--12:12}{13:45--15:18}}
\caldata{AUGUST}{20}{\sunmonth{सिंहः}{4}{}{श्रावणः}{वर्षऋतुः}{मङ्गलः}{विकारी}{दक्षिणायनम्}{वर्षऋतुः}}
{\sunmoonsrdata{06:00}{18:24}{21:33}{09:27}{12:12}
{\kalas{04:27 05:14 09:18 08:29 10:08 16:44 10:58 13:26 15:55 17:34 19:10 21:18 22:45 01:39(+1)}}}
{\tnykdata{\anga{\tithi{20}{कृष्ण-पञ्चमी}}{\time{58-45}{05:30(+1)}}\hspace{1ex}}%
{\anga{रेवती}{\time{41-4}{22:26}}\hspace{1ex}}{चन्द्रराशिः—\mbox{मीनः\RIGHTarrow\textsf{22:26}}}%
{\anga{शूलः}{\time{25-56}{16:23}}\hspace{1ex}\uanga{गण्डः}}%
{\anga{कौलवः}{\time{26-21}{16:33}}\hspace{1ex}\anga{तैतिलः}{\time{58-45}{05:30(+1)}}\hspace{1ex}\uanga{गरः}}{}
}
{रक्षा-पञ्चमी}
{Tue} 
\cfoot{\rygdata{15:18--16:51}{09:06--10:39}{12:12--13:45}}
\caldata{AUGUST}{21}{\sunmonth{सिंहः}{5}{}{श्रावणः}{वर्षऋतुः}{बुधः}{विकारी}{दक्षिणायनम्}{वर्षऋतुः}}
{\sunmoonsrdata{06:00}{18:23}{22:12}{10:14}{12:12}
{\kalas{04:27 05:14 09:18 08:29 10:08 16:44 10:57 13:26 15:55 17:34 19:10 21:17 22:44 01:39(+1)}}}
{\tnykdata{\fulltithi{\tithi{21}{कृष्ण-षष्ठी}}}%
{\anga{अश्विनी}{\time{46-50}{00:44(+1)}}\hspace{1ex}}{चन्द्रराशिः—\mbox{मेषः}}%
{\anga{गण्डः}{\time{27-7}{16:51}}\hspace{1ex}\uanga{वृद्धिः}}%
{\anga{गरः}{\time{30-53}{18:22}}\hspace{1ex}\uanga{वणिजः}}{}
}
{हल-षष्ठी}
{Wed} 
\cfoot{\rygdata{12:12--13:44}{07:33--09:06}{10:39--12:12}}
\caldata{AUGUST}{22}{\sunmonth{सिंहः}{6}{}{श्रावणः}{वर्षऋतुः}{गुरुः}{विकारी}{दक्षिणायनम्}{वर्षऋतुः}}
{\sunmoonsrdata{06:00}{18:23}{22:54}{11:04}{12:11}
{\kalas{04:27 05:14 09:18 08:29 10:08 16:44 10:57 13:26 15:54 17:33 19:09 21:17 22:44 01:39(+1)}}}
{\tnykdata{\anga{\tithi{21}{कृष्ण-षष्ठी}}{\time{2-44}{07:06}}\hspace{1ex}}%
{\anga{अपभरणी}{\time{51-22}{02:33(+1)}}\hspace{1ex}}{चन्द्रराशिः—\mbox{मेषः}}%
{\anga{वृद्धिः}{\time{27-24}{16:58}}\hspace{1ex}\uanga{ध्रुवः}}%
{\anga{वणिजः}{\time{2-44}{07:06}}\hspace{1ex}\anga{विष्टिः}{\time{34-14}{19:42}}\hspace{1ex}\uanga{बवः}}{}
}
{}
{Thu} 
\cfoot{\rygdata{13:44--15:17}{06:00--07:33}{09:06--10:39}}
\caldata{AUGUST}{23}{\sunmonth{सिंहः}{7}{}{श्रावणः}{वर्षऋतुः}{शुक्रः}{विकारी}{दक्षिणायनम्}{वर्षऋतुः}}
{\sunmoonsrdata{06:00}{18:22}{23:41}{11:55}{12:11}
{\kalas{04:27 05:14 09:18 08:29 10:07 16:43 10:57 13:25 15:54 17:32 19:08 21:16 22:44 01:38(+1)}}}
{\tnykdata{\anga{\tithi{22}{कृष्ण-सप्तमी}}{\time{5-21}{08:09}}\hspace{1ex}}%
{\anga{कृत्तिका}{\time{54-20}{03:45(+1)}}\hspace{1ex}}{चन्द्रराशिः—\mbox{मेषः\RIGHTarrow\textsf{08:55}}}%
{\anga{ध्रुवः}{\time{26-31}{16:37}}\hspace{1ex}\uanga{व्याघातः}}%
{\anga{बवः}{\time{5-21}{08:09}}\hspace{1ex}\anga{बालवः}{\time{36-3}{20:26}}\hspace{1ex}\uanga{कौलवः}}{}
}
{षडशीति-पुण्यकालः~\textsf{15:32}{\RIGHTarrow}\textsf{15:32(+1)}\eventsep काञ्ची २१ जगद्गुरु श्री-सार्वभौमगुरुः चन्द्रचूडेन्द्र सरस्वती आराधना~\#{१५७३}\eventsep कृत्तिका-व्रतम्\eventsep महाकाली-जयन्ती\eventsep मन्वादिः-(दक्षः-[९])\eventsep नभो-मासः{\RIGHTarrow}\textsf{15:32}\eventsep श्रीकृष्णजन्माष्टमी}
{Fri} 
\cfoot{\rygdata{10:38--12:11}{15:16--16:49}{07:33--09:06}}
\caldata{AUGUST}{24}{\sunmonth{सिंहः}{8}{}{श्रावणः}{वर्षऋतुः}{शनिः}{विकारी}{दक्षिणायनम्}{वर्षऋतुः}}
{\sunmoonsrdata{06:00}{18:21}{00:32(+1)}{12:50}{12:11}
{\kalas{04:27 05:14 09:18 08:29 10:07 16:43 10:57 13:25 15:53 17:32 19:08 21:16 22:43 01:38(+1)}}}
{\tnykdata{\anga{\tithi{23}{कृष्ण-अष्टमी}}{\time{6-18}{08:32}}\hspace{1ex}}%
{\anga{रोहिणी}{\time{55-32}{04:13(+1)}}\hspace{1ex}}{चन्द्रराशिः—\mbox{वृषभः}}%
{\anga{व्याघातः}{\time{24-16}{15:43}}\hspace{1ex}\uanga{हर्षणः}}%
{\anga{कौलवः}{\time{6-18}{08:32}}\hspace{1ex}\anga{तैतिलः}{\time{36-6}{20:27}}\hspace{1ex}\uanga{गरः}}{}
}
{काञ्ची २४ जगद्गुरु श्री-चित्सुखेन्द्र सरस्वती आराधना~\#{१४९३}\eventsep नन्दोत्सवः\eventsep \tamil{திருச்செந்தூர் முருகன் ஆவணித் திருவிழா தொடக்கம்/கொடியேற்றம்}\eventsep \tamil{வரகூர் உறியடி உத்ஸவம்}\eventsep शनिरोहिणी-पुण्यकालः\eventsep श्री-जयन्ती\eventsep श्रीकृष्णदेवराय-राज्याभिषेकः}
{Sat} 
\cfoot{\rygdata{09:06--10:38}{13:43--15:16}{06:00--07:33}}
\caldata{AUGUST}{25}{\sunmonth{सिंहः}{9}{}{श्रावणः}{वर्षऋतुः}{भानुः}{विकारी}{दक्षिणायनम्}{वर्षऋतुः}}
{\sunmoonsrdata{06:01}{18:21}{01:27(+1)}{13:47}{12:11}
{\kalas{04:27 05:14 09:18 08:29 10:07 16:42 10:57 13:25 15:53 17:31 19:07 21:16 22:43 01:38(+1)}}}
{\tnykdata{\anga{\tithi{24}{कृष्ण-नवमी}}{\time{5-24}{08:10}}\hspace{1ex}}%
{\anga{मृगशीर्षम्}{\time{54-49}{03:56(+1)}}\hspace{1ex}}{चन्द्रराशिः—\mbox{वृषभः\RIGHTarrow\textsf{16:10}}}%
{\anga{हर्षणः}{\time{20-30}{14:13}}\hspace{1ex}\uanga{वज्रम्}}%
{\anga{गरः}{\time{5-24}{08:10}}\hspace{1ex}\anga{वणिजः}{\time{34-14}{19:42}}\hspace{1ex}\uanga{विष्टिः}}{}
}
{\tamil{ஆவணி~ஞாயிற்றுக்கிழமை}\eventsep अरविन्द-जयन्ती\eventsep चण्डिका-पूजा\eventsep कौमार-पूजा\eventsep \tamil{திருச்செந்தூர் முருகன் ஆவணித் திருவிழா 2ம் நாள்}}
{Sun} 
\cfoot{\rygdata{16:48--18:21}{12:11--13:43}{15:16--16:48}}
\caldata{AUGUST}{26}{\sunmonth{सिंहः}{10}{}{श्रावणः}{वर्षऋतुः}{सोमः}{विकारी}{दक्षिणायनम्}{वर्षऋतुः}}
{\sunmoonsrdata{06:01}{18:20}{02:27(+1)}{14:45}{12:10}
{\kalas{04:27 05:14 09:18 08:28 10:07 16:41 10:56 13:24 15:52 17:31 19:07 21:15 22:43 01:38(+1)}}}
{\tnykdata{\anga{\tithi{25}{कृष्ण-दशमी}}{\time{2-34}{07:02}}\hspace{1ex}\anga{\tithi{26}{कृष्ण-एकादशी}}{\time{57-52}{05:10(+1)}}\hspace{1ex}}%
{\anga{आर्द्रा}{\time{52-13}{02:54(+1)}}\hspace{1ex}}{चन्द्रराशिः—\mbox{मिथुनम्}}%
{\anga{वज्रम्}{\time{15-12}{12:05}}\hspace{1ex}\uanga{सिद्धिः}}%
{\anga{विष्टिः}{\time{2-34}{07:02}}\hspace{1ex}\anga{बवः}{\time{30-27}{18:11}}\hspace{1ex}\anga{बालवः}{\time{57-52}{05:10(+1)}}\hspace{1ex}\uanga{कौलवः}}{}
}
{स्मार्त-अजा-एकादशी\eventsep \tamil{திருச்செந்தூர் முருகன் ஆவணித் திருவிழா 3ம் நாள்—முருகன் பவனி}}
{Mon} 
\cfoot{\rygdata{07:33--09:05}{10:38--12:10}{13:43--15:15}}
\caldata{AUGUST}{27}{\sunmonth{सिंहः}{11}{}{श्रावणः}{वर्षऋतुः}{मङ्गलः}{विकारी}{दक्षिणायनम्}{वर्षऋतुः}}
{\sunmoonsrdata{06:01}{18:20}{03:29(+1)}{15:42}{12:10}
{\kalas{04:27 05:14 09:18 08:28 10:07 16:41 10:56 13:24 15:52 17:30 19:06 21:15 22:42 01:38(+1)}}}
{\tnykdata{\anga{\tithi{27}{कृष्ण-द्वादशी}}{\time{51-28}{02:36(+1)}}\hspace{1ex}}%
{\anga{पुनर्वसुः}{\time{47-55}{01:11(+1)}}\hspace{1ex}}{चन्द्रराशिः—\mbox{मिथुनम्\RIGHTarrow\textsf{19:40}}}%
{\anga{सिद्धिः}{\time{8-23}{09:22}}\hspace{1ex}\uanga{व्यतीपातः}}%
{\anga{कौलवः}{\time{24-52}{15:57}}\hspace{1ex}\anga{तैतिलः}{\time{51-28}{02:36(+1)}}\hspace{1ex}\uanga{गरः}}{}
}
{हरिवासरः{\RIGHTarrow}\textsf{10:35}\eventsep जयन्ती-महाद्वादशी\eventsep रोहिणी-द्वादशी\eventsep \tamil{திருச்செந்தூர் முருகன் ஆவணித் திருவிழா 4ம் நாள்—யானை வாஹநத்தில் முருகன்-அம்பாள் பவனி}\eventsep वैष्णव-अजा-एकादशी\eventsep व्यतीपात-श्राद्धम्}
{Tue} 
\cfoot{\rygdata{15:15--16:47}{09:05--10:38}{12:10--13:42}}
\caldata{AUGUST}{28}{\sunmonth{सिंहः}{12}{}{श्रावणः}{वर्षऋतुः}{बुधः}{विकारी}{दक्षिणायनम्}{वर्षऋतुः}}
{\sunmoonsrdata{06:01}{18:19}{04:32(+1)}{16:38}{12:10}
{\kalas{04:27 05:14 09:18 08:28 10:07 16:40 10:56 13:24 15:51 17:30 19:06 21:14 22:42 01:38(+1)}}}
{\tnykdata{\anga{\tithi{28}{कृष्ण-त्रयोदशी}}{\time{43-39}{23:28}}\hspace{1ex}}%
{\anga{पुष्यः}{\time{42-9}{22:53}}\hspace{1ex}}{चन्द्रराशिः—\mbox{कटकः}}%
{\anga{व्यतीपातः}{\time{0-13}{06:06}}\hspace{1ex}\anga{वरीयान्}{\time{50-55}{02:23(+1)}}\hspace{1ex}\uanga{परिघः}}%
{\anga{गरः}{\time{17-42}{13:06}}\hspace{1ex}\anga{वणिजः}{\time{43-39}{23:28}}\hspace{1ex}\uanga{विष्टिः}}{}
}
{\tamil{செருத்துணை நாயனார் (53) குருபூஜை}\eventsep मासशिवरात्रिः\eventsep प्रदोष-व्रतम्\eventsep \tamil{திருச்செந்தூர் முருகன் ஆவணித் திருவிழா 5ம் நாள்}}
{Wed} 
\cfoot{\rygdata{12:10--13:42}{07:33--09:05}{10:37--12:10}}
\caldata{AUGUST}{29}{\sunmonth{सिंहः}{13}{}{श्रावणः}{वर्षऋतुः}{गुरुः}{विकारी}{दक्षिणायनम्}{वर्षऋतुः}}
{\sunmoonsrdata{06:01}{18:18}{05:35(+1)}{17:31}{12:09}
{\kalas{04:27 05:14 09:17 08:28 10:06 16:40 10:56 13:23 15:51 17:29 19:05 21:14 22:42 01:37(+1)}}}
{\tnykdata{\anga{\tithi{29}{कृष्ण-चतुर्दशी}}{\time{34-46}{19:55}}\hspace{1ex}}%
{\anga{आश्रेषा}{\time{35-19}{20:09}}\hspace{1ex}}{चन्द्रराशिः—\mbox{कटकः\RIGHTarrow\textsf{20:09}}}%
{\anga{परिघः}{\time{40-46}{22:19}}\hspace{1ex}\uanga{शिवः}}%
{\anga{विष्टिः}{\time{9-19}{09:44}}\hspace{1ex}\anga{शकुनिः}{\time{34-46}{19:55}}\hspace{1ex}\uanga{चतुष्पात्}}{}
}
{अघोर-चतुर्दशी\eventsep \tamil{அதிபத்த நாயனார் (41) குருபூஜை}\eventsep \tamil{திருச்செந்தூர் முருகன் ஆவணித் திருவிழா 6ம் நாள்—வெள்ளித் தேர் பவனி}}
{Thu} 
\cfoot{\rygdata{13:42--15:14}{06:01--07:33}{09:05--10:37}}
\caldata{AUGUST}{30}{\sunmonth{सिंहः}{14}{}{श्रावणः}{वर्षऋतुः}{शुक्रः}{विकारी}{दक्षिणायनम्}{वर्षऋतुः}}
{\sunmoonsrdata{06:01}{18:18}{---}{18:22}{12:09}
{\kalas{04:27 05:14 09:17 08:28 10:06 16:39 10:55 13:23 15:50 17:29 19:04 21:13 22:41 01:37(+1)}}}
{\tnykdata{\anga{\tithi{30}{अमावास्या}}{\time{25-15}{16:07}}\hspace{1ex}}%
{\anga{मघा}{\time{27-50}{17:09}}\hspace{1ex}}{चन्द्रराशिः—\mbox{सिंहः}}%
{\anga{शिवः}{\time{30-7}{18:04}}\hspace{1ex}\uanga{सिद्धः}}%
{\anga{चतुष्पात्}{\time{0-3}{06:02}}\hspace{1ex}\anga{नाग}{\time{25-15}{16:07}}\hspace{1ex}\anga{किंस्तुघ्नः}{\time{50-23}{02:10(+1)}}\hspace{1ex}\uanga{बवः}}{}
}
{६४ योगिनी-पूजा\eventsep दर्भ-सङ्ग्रहः\eventsep \tamil{இளையான்குடி மாற நாயனார் (3) குருபூஜை}\eventsep पार्वणव्रतम् अमावास्यायाम्\eventsep \tamil{திருச்செந்தூர் முருகன் ஆவணித் திருவிழா 7ம் நாள்—சிகப்பு சாத்தி அலங்காரம்}\eventsep वृषभ-पूजा\eventsep श्रावण-अमावास्या}
{Fri} 
\cfoot{\rygdata{10:37--12:09}{15:13--16:45}{07:33--09:05}}
\caldata{AUGUST}{31}{\sunmonth{सिंहः}{15}{}{भाद्रपदः}{वर्षऋतुः}{शनिः}{विकारी}{दक्षिणायनम्}{वर्षऋतुः}}
{\sunmoonrsdata{06:01}{18:17}{06:36}{19:11}{12:09}
{\kalas{04:27 05:14 09:17 08:28 10:06 16:39 10:55 13:22 15:50 17:28 19:04 21:13 22:41 01:37(+1)}}}
{\tnykdata{\anga{\tithi{1}{शुक्ल-प्रथमा}}{\time{15-31}{12:13}}\hspace{1ex}}%
{\anga{पूर्वफल्गुनी}{\time{20-10}{14:05}}\hspace{1ex}}{चन्द्रराशिः—\mbox{सिंहः\RIGHTarrow\textsf{19:20}}}%
{\anga{सिद्धः}{\time{19-20}{13:45}}\hspace{1ex}\uanga{साध्यः}}%
{\anga{बवः}{\time{15-31}{12:13}}\hspace{1ex}\anga{बालवः}{\time{40-43}{22:18}}\hspace{1ex}\uanga{कौलवः}}{}
}
{चन्द्र-दर्शनम्\eventsep दर्शेष्टिः\eventsep मृगशीर्ष-व्रतम्\eventsep स्थालीपाकः\eventsep \tamil{திருச்செந்தூர் முருகன் ஆவணித் திருவிழா 8ம் நாள்—பச்சை சாத்தி அலங்காரம்}}
{Sat} 
\cfoot{\rygdata{09:05--10:37}{13:41--15:13}{06:01--07:33}}
\caldata{SEPTEMBER}{1}{\sunmonth{सिंहः}{16}{}{भाद्रपदः}{वर्षऋतुः}{भानुः}{विकारी}{दक्षिणायनम्}{वर्षऋतुः}}
{\sunmoonrsdata{06:01}{18:16}{07:36}{19:59}{12:09}
{\kalas{04:27 05:14 09:17 08:28 10:06 16:38 10:55 13:22 15:49 17:27 19:03 21:12 22:40 01:37(+1)}}}
{\tnykdata{\anga{\tithi{2}{शुक्ल-द्वितीया}}{\time{6-3}{08:26}}\hspace{1ex}\anga{\tithi{3}{शुक्ल-तृतीया}}{\time{57-18}{04:56(+1)}}\hspace{1ex}}%
{\anga{उत्तरफल्गुनी}{\time{12-48}{11:08}}\hspace{1ex}}{चन्द्रराशिः—\mbox{कन्या}}%
{\anga{साध्यः}{\time{8-48}{09:32}}\hspace{1ex}\anga{शुभः}{\time{58-53}{05:34(+1)}}\hspace{1ex}\uanga{शुक्लः}}%
{\anga{कौलवः}{\time{6-3}{08:26}}\hspace{1ex}\anga{तैतिलः}{\time{31-33}{18:38}}\hspace{1ex}\anga{गरः}{\time{57-18}{04:56(+1)}}\hspace{1ex}\uanga{वणिजः}}{}
}
{आदित्यहस्त-पुण्यकालः\eventsep \tamil{ஆவணி~ஞாயிற்றுக்கிழமை}\eventsep अङ्गारक-जयन्ती\eventsep हरितालिका-व्रतम्\eventsep मन्वादिः-(तामसः-[४])\eventsep \tamil{திருச்செந்தூர் முருகன் ஆவணித் திருவிழா 9ம் நாள்}\eventsep वराह-जयन्ती\eventsep विपत्तार-गौरी-व्रतम्}
{Sun} 
\cfoot{\rygdata{16:44--18:16}{12:09--13:40}{15:12--16:44}}
\caldata{SEPTEMBER}{2}{\sunmonth{सिंहः}{17}{}{भाद्रपदः}{वर्षऋतुः}{सोमः}{विकारी}{दक्षिणायनम्}{वर्षऋतुः}}
{\sunmoonrsdata{06:01}{18:16}{08:35}{20:46}{12:08}
{\kalas{04:27 05:14 09:17 08:28 10:06 16:38 10:55 13:22 15:49 17:27 19:03 21:12 22:40 01:36(+1)}}}
{\tnykdata{\anga{\tithi{4}{शुक्ल-चतुर्थी}}{\time{49-41}{01:53(+1)}}\hspace{1ex}}%
{\anga{हस्तः}{\time{6-13}{08:30}}\hspace{1ex}}{चन्द्रराशिः—\mbox{कन्या\RIGHTarrow\textsf{19:22}}}%
{\anga{शुक्लः}{\time{49-55}{01:59(+1)}}\hspace{1ex}\uanga{ब्रह्म}}%
{\anga{वणिजः}{\time{23-20}{15:21}}\hspace{1ex}\anga{विष्टिः}{\time{49-41}{01:53(+1)}}\hspace{1ex}\uanga{बवः}}{}
}
{सामवेद-उपाकर्म\eventsep \tamil{திருச்செந்தூர் முருகன் ஆவணித் திருவிழா 10ம் நாள்—தேர்}\eventsep श्रीविनायक-चतुर्थी}
{Mon} 
\cfoot{\rygdata{07:33--09:05}{10:36--12:08}{13:40--15:12}}
\caldata{SEPTEMBER}{3}{\sunmonth{सिंहः}{18}{}{भाद्रपदः}{वर्षऋतुः}{मङ्गलः}{विकारी}{दक्षिणायनम्}{वर्षऋतुः}}
{\sunmoonrsdata{06:01}{18:15}{09:34}{21:34}{12:08}
{\kalas{04:27 05:14 09:17 08:28 10:06 16:37 10:55 13:21 15:48 17:26 19:02 21:11 22:40 01:36(+1)}}}
{\tnykdata{\anga{\tithi{5}{शुक्ल-पञ्चमी}}{\time{43-36}{23:27}}\hspace{1ex}}%
{\anga{चित्रा}{\time{0-51}{06:22}}\hspace{1ex}\anga{स्वाती}{\time{57-5}{04:51(+1)}}\hspace{1ex}}{चन्द्रराशिः—\mbox{तुला}}%
{\anga{ब्रह्म}{\time{42-13}{22:54}}\hspace{1ex}\uanga{इन्द्रः}}%
{\anga{बवः}{\time{16-26}{12:35}}\hspace{1ex}\anga{बालवः}{\time{43-36}{23:27}}\hspace{1ex}\uanga{कौलवः}}{}
}
{ऋषि-पञ्चमी-व्रतम्\eventsep \tamil{திருச்செந்தூர் முருகன் ஆவணித் திருவிழா 11ம் நாள்}}
{Tue} 
\cfoot{\rygdata{15:11--16:43}{09:04--10:36}{12:08--13:40}}
\caldata{SEPTEMBER}{4}{\sunmonth{सिंहः}{19}{}{भाद्रपदः}{वर्षऋतुः}{बुधः}{विकारी}{दक्षिणायनम्}{वर्षऋतुः}}
{\sunmoonrsdata{06:01}{18:14}{10:33}{22:24}{12:08}
{\kalas{04:27 05:14 09:16 08:28 10:05 16:37 10:54 13:21 15:48 17:25 19:01 21:11 22:39 01:36(+1)}}}
{\tnykdata{\anga{\tithi{6}{शुक्ल-षष्ठी}}{\time{39-18}{21:44}}\hspace{1ex}}%
{\anga{विशाखा}{\time{55-9}{04:05(+1)}}\hspace{1ex}}{चन्द्रराशिः—\mbox{तुला\RIGHTarrow\textsf{22:12}}}%
{\anga{इन्द्रः}{\time{35-58}{20:24}}\hspace{1ex}\uanga{वैधृतिः}}%
{\anga{कौलवः}{\time{11-12}{10:30}}\hspace{1ex}\anga{तैतिलः}{\time{39-18}{21:44}}\hspace{1ex}\uanga{गरः}}{}
}
{षष्ठीदेवी-षष्ठी-व्रतम्\eventsep कुमारिका-स्वपनम्\eventsep ललिता-षष्ठी\eventsep मन्थन-षष्ठी\eventsep सूर्य-षष्ठी\eventsep \tamil{திருச்செந்தூர் ஆவணித் திருவிழா நிறைவு}}
{Wed} 
\cfoot{\rygdata{12:08--13:39}{07:33--09:04}{10:36--12:08}}
\caldata{SEPTEMBER}{5}{\sunmonth{सिंहः}{20}{}{भाद्रपदः}{वर्षऋतुः}{गुरुः}{विकारी}{दक्षिणायनम्}{वर्षऋतुः}}
{\sunmoonrsdata{06:01}{18:14}{11:30}{23:15}{12:07}
{\kalas{04:27 05:14 09:16 08:27 10:05 16:36 10:54 13:21 15:47 17:25 19:01 21:10 22:39 01:36(+1)}}}
{\tnykdata{\anga{\tithi{7}{शुक्ल-सप्तमी}}{\time{37-0}{20:49}}\hspace{1ex}}%
{\anga{अनूराधा}{\time{55-13}{04:06(+1)}}\hspace{1ex}}{चन्द्रराशिः—\mbox{वृश्चिकः}}%
{\anga{वैधृतिः}{\time{31-20}{18:33}}\hspace{1ex}\uanga{विष्कम्भः}}%
{\anga{गरः}{\time{7-54}{09:11}}\hspace{1ex}\anga{वणिजः}{\time{37-0}{20:49}}\hspace{1ex}\uanga{विष्टिः}}{}
}
{अमुक्ताभरण-सप्तमी\eventsep अनन्तफल-सप्तमी\eventsep कुक्कुटी-व्रतम्\eventsep \tamil{குலச்சிரை நாயனார் (21) குருபூஜை}\eventsep वैधृति-श्राद्धम्}
{Thu} 
\cfoot{\rygdata{13:39--15:10}{06:01--07:33}{09:04--10:36}}
\caldata{SEPTEMBER}{6}{\sunmonth{सिंहः}{21}{}{भाद्रपदः}{वर्षऋतुः}{शुक्रः}{विकारी}{दक्षिणायनम्}{वर्षऋतुः}}
{\sunmoonrsdata{06:01}{18:13}{12:27}{00:07(+1)}{12:07}
{\kalas{04:27 05:14 09:16 08:27 10:05 16:35 10:54 13:20 15:46 17:24 19:00 21:10 22:38 01:35(+1)}}}
{\tnykdata{\anga{\tithi{8}{शुक्ल-अष्टमी}}{\time{36-44}{20:43}}\hspace{1ex}}%
{\anga{ज्येष्ठा}{\time{57-14}{04:55(+1)}}\hspace{1ex}}{चन्द्रराशिः—\mbox{वृश्चिकः\RIGHTarrow\textsf{04:55(+1)}}}%
{\anga{विष्कम्भः}{\time{28-18}{17:20}}\hspace{1ex}\uanga{प्रीतिः}}%
{\anga{विष्टिः}{\time{6-37}{08:40}}\hspace{1ex}\anga{बवः}{\time{36-44}{20:43}}\hspace{1ex}\uanga{बालवः}}{}
}
{दूर्वाष्टमी\eventsep दधीचि-महर्षि-जयन्ति\eventsep राधाष्टमी}
{Fri} 
\cfoot{\rygdata{10:35--12:07}{15:10--16:41}{07:32--09:04}}
\caldata{SEPTEMBER}{7}{\sunmonth{सिंहः}{22}{}{भाद्रपदः}{वर्षऋतुः}{शनिः}{विकारी}{दक्षिणायनम्}{वर्षऋतुः}}
{\sunmoonrsdata{06:01}{18:12}{13:21}{00:59(+1)}{12:07}
{\kalas{04:26 05:14 09:16 08:27 10:05 16:35 10:53 13:20 15:46 17:24 18:59 21:09 22:38 01:35(+1)}}}
{\tnykdata{\anga{\tithi{9}{शुक्ल-नवमी}}{\time{38-22}{21:22}}\hspace{1ex}}%
{\fullanga{मूला}}{चन्द्रराशिः—\mbox{धनुः}}%
{\anga{प्रीतिः}{\time{26-48}{16:44}}\hspace{1ex}\uanga{आयुष्मान्}}%
{\anga{बालवः}{\time{7-19}{08:57}}\hspace{1ex}\anga{कौलवः}{\time{38-22}{21:22}}\hspace{1ex}\uanga{तैतिलः}}{}
}
{अदुःखनवमी\eventsep गोधूमा-नवमी\eventsep गजेन्द्र-मोक्षः\eventsep \tamil{குங்கிலியக்கலய நாயனார் (10) குருபூஜை}\eventsep महालक्ष्मी-व्रत-आरम्भः\eventsep नन्दा-नवमी\eventsep \tamil{பிட்டுக்கு மண் சுமந்த லீலை}\eventsep तालनवमी}
{Sat} 
\cfoot{\rygdata{09:04--10:35}{13:38--15:09}{06:01--07:32}}
\caldata{SEPTEMBER}{8}{\sunmonth{सिंहः}{23}{}{भाद्रपदः}{वर्षऋतुः}{भानुः}{विकारी}{दक्षिणायनम्}{वर्षऋतुः}}
{\sunmoonrsdata{06:01}{18:12}{14:12}{01:52(+1)}{12:06}
{\kalas{04:26 05:14 09:16 08:27 10:04 16:34 10:53 13:19 15:45 17:23 18:59 21:09 22:38 01:35(+1)}}}
{\tnykdata{\anga{\tithi{10}{शुक्ल-दशमी}}{\time{41-39}{22:41}}\hspace{1ex}}%
{\anga{मूला}{\time{1-3}{06:26}}\hspace{1ex}}{चन्द्रराशिः—\mbox{धनुः}}%
{\anga{आयुष्मान्}{\time{26-36}{16:40}}\hspace{1ex}\uanga{सौभाग्यः}}%
{\anga{तैतिलः}{\time{9-49}{09:57}}\hspace{1ex}\anga{गरः}{\time{41-39}{22:41}}\hspace{1ex}\uanga{वणिजः}}{}
}
{\tamil{ஆவணி~ஞாயிற்றுக்கிழமை}\eventsep दशावतार-व्रतम्\eventsep वितस्तोत्सवः}
{Sun} 
\cfoot{\rygdata{16:40--18:12}{12:06--13:38}{15:09--16:40}}
\caldata{SEPTEMBER}{9}{\sunmonth{सिंहः}{24}{}{भाद्रपदः}{वर्षऋतुः}{सोमः}{विकारी}{दक्षिणायनम्}{वर्षऋतुः}}
{\sunmoonrsdata{06:01}{18:11}{15:00}{02:43(+1)}{12:06}
{\kalas{04:26 05:14 09:16 08:27 10:04 16:34 10:53 13:19 15:45 17:22 18:58 21:08 22:37 01:35(+1)}}}
{\tnykdata{\anga{\tithi{11}{शुक्ल-एकादशी}}{\time{46-14}{00:31(+1)}}\hspace{1ex}}%
{\anga{पूर्वाषाढा}{\time{6-20}{08:33}}\hspace{1ex}}{चन्द्रराशिः—\mbox{धनुः\RIGHTarrow\textsf{15:09}}}%
{\anga{सौभाग्यः}{\time{27-29}{17:01}}\hspace{1ex}\uanga{शोभनः}}%
{\anga{वणिजः}{\time{13-48}{11:32}}\hspace{1ex}\anga{विष्टिः}{\time{46-14}{00:31(+1)}}\hspace{1ex}\uanga{बवः}}{}
}
{कटदानोत्सवः\eventsep सर्व-परिवर्तिनी-एकादशी}
{Mon} 
\cfoot{\rygdata{07:32--09:03}{10:35--12:06}{13:37--15:08}}
\caldata{SEPTEMBER}{10}{\sunmonth{सिंहः}{25}{}{भाद्रपदः}{वर्षऋतुः}{मङ्गलः}{विकारी}{दक्षिणायनम्}{वर्षऋतुः}}
{\sunmoonrsdata{06:01}{18:10}{15:45}{03:32(+1)}{12:06}
{\kalas{04:26 05:14 09:15 08:27 10:04 16:33 10:53 13:18 15:44 17:21 18:58 21:08 22:37 01:34(+1)}}}
{\tnykdata{\anga{\tithi{12}{शुक्ल-द्वादशी}}{\time{51-43}{02:42(+1)}}\hspace{1ex}}%
{\anga{उत्तराषाढा}{\time{12-43}{11:06}}\hspace{1ex}}{चन्द्रराशिः—\mbox{मकरः}}%
{\anga{शोभनः}{\time{29-7}{17:40}}\hspace{1ex}\uanga{अतिगण्डः}}%
{\anga{बवः}{\time{18-53}{13:34}}\hspace{1ex}\anga{बालवः}{\time{51-43}{02:42(+1)}}\hspace{1ex}\uanga{कौलवः}}{}
}
{अनन्त-द्वादशी\eventsep भुवनेश्वरी-जयन्ती\eventsep दधि-व्रत-समापनम्\eventsep हरिवासरः{\RIGHTarrow}\textsf{07:02}\eventsep वामन-जयन्ती\eventsep विजया/श्रवण-महाद्वादशी}
{Tue} 
\cfoot{\rygdata{15:08--16:39}{09:03--10:34}{12:06--13:37}}
\caldata{SEPTEMBER}{11}{\sunmonth{सिंहः}{26}{}{भाद्रपदः}{वर्षऋतुः}{बुधः}{विकारी}{दक्षिणायनम्}{वर्षऋतुः}}
{\sunmoonrsdata{06:01}{18:09}{16:27}{04:21(+1)}{12:05}
{\kalas{04:26 05:14 09:15 08:27 10:04 16:32 10:52 13:18 15:44 17:21 18:57 21:07 22:36 01:34(+1)}}}
{\tnykdata{\anga{\tithi{13}{शुक्ल-त्रयोदशी}}{\time{57-43}{05:06(+1)}}\hspace{1ex}}%
{\anga{श्रवणः}{\time{19-48}{13:56}}\hspace{1ex}}{चन्द्रराशिः—\mbox{मकरः\RIGHTarrow\textsf{03:25(+1)}}}%
{\anga{अतिगण्डः}{\time{31-14}{18:31}}\hspace{1ex}\uanga{सुकर्म}}%
{\anga{कौलवः}{\time{24-40}{15:53}}\hspace{1ex}\anga{तैतिलः}{\time{57-43}{05:06(+1)}}\hspace{1ex}\uanga{गरः}}{}
}
{\tamil{ஓணம்}\eventsep दूर्व-त्रि-व्रतम्\eventsep गो-त्रिरात्रि-व्रतम्\eventsep प्रदोष-व्रतम्\eventsep श्रवण-व्रतम्}
{Wed} 
\cfoot{\rygdata{12:05--13:36}{07:32--09:03}{10:34--12:05}}
\caldata{SEPTEMBER}{12}{\sunmonth{सिंहः}{27}{}{भाद्रपदः}{वर्षऋतुः}{गुरुः}{विकारी}{दक्षिणायनम्}{वर्षऋतुः}}
{\sunmoonrsdata{06:01}{18:09}{17:06}{05:07(+1)}{12:05}
{\kalas{04:26 05:14 09:15 08:27 10:04 16:32 10:52 13:18 15:43 17:20 18:56 21:07 22:36 01:34(+1)}}}
{\tnykdata{\fulltithi{\tithi{14}{शुक्ल-चतुर्दशी}}}%
{\anga{श्रविष्ठा}{\time{27-15}{16:55}}\hspace{1ex}}{चन्द्रराशिः—\mbox{कुम्भः}}%
{\anga{सुकर्म}{\time{33-36}{19:27}}\hspace{1ex}\uanga{धृतिः}}%
{\anga{गरः}{\time{30-48}{18:20}}\hspace{1ex}\uanga{वणिजः}}{}
}
{अनन्त-चतुर्दशी\eventsep अनन्त-पद्मनाभ-व्रतम्\eventsep \tamil{நடராஜர் மஹாபிஷேகம்}}
{Thu} 
\cfoot{\rygdata{13:36--15:07}{06:01--07:32}{09:03--10:34}}
\caldata{SEPTEMBER}{13}{\sunmonth{सिंहः}{28}{}{भाद्रपदः}{वर्षऋतुः}{शुक्रः}{विकारी}{दक्षिणायनम्}{वर्षऋतुः}}
{\sunmoonrsdata{06:01}{18:08}{17:43}{05:53(+1)}{12:04}
{\kalas{04:26 05:13 09:15 08:26 10:03 16:31 10:52 13:17 15:43 17:19 18:56 21:06 22:35 01:34(+1)}}}
{\tnykdata{\anga{\tithi{14}{शुक्ल-चतुर्दशी}}{\time{3-55}{07:35}}\hspace{1ex}}%
{\anga{शतभिषक्}{\time{34-46}{19:56}}\hspace{1ex}}{चन्द्रराशिः—\mbox{कुम्भः}}%
{\anga{धृतिः}{\time{36-0}{20:25}}\hspace{1ex}\uanga{शूलः}}%
{\anga{वणिजः}{\time{3-55}{07:35}}\hspace{1ex}\anga{विष्टिः}{\time{37-0}{20:49}}\hspace{1ex}\uanga{बवः}}{}
}
{काञ्ची ५९ जगद्गुरु श्री-भगवन्नाम बोधेन्द्र सरस्वती आराधना~\#{३२८}\eventsep पार्वणव्रतम् पूर्णिमायाम्\eventsep वेङ्कटाचले पूर्णिमा-गरुड-सेवा}
{Fri} 
\cfoot{\rygdata{10:34--12:04}{15:06--16:37}{07:32--09:03}}
\caldata{SEPTEMBER}{14}{\sunmonth{सिंहः}{29}{}{भाद्रपदः}{वर्षऋतुः}{शनिः}{विकारी}{दक्षिणायनम्}{वर्षऋतुः}}
{\sunmoonrsdata{06:01}{18:07}{18:20}{---}{12:04}
{\kalas{04:26 05:13 09:15 08:26 10:03 16:30 10:52 13:17 15:42 17:19 18:55 21:06 22:35 01:33(+1)}}}
{\tnykdata{\anga{\tithi{15}{पौर्णमासी}}{\time{10-3}{10:02}}\hspace{1ex}}%
{\anga{पूर्वप्रोष्ठपदा}{\time{42-8}{22:52}}\hspace{1ex}}{चन्द्रराशिः—\mbox{कुम्भः\RIGHTarrow\textsf{16:09}}}%
{\anga{शूलः}{\time{38-16}{21:19}}\hspace{1ex}\uanga{गण्डः}}%
{\anga{बवः}{\time{10-3}{10:02}}\hspace{1ex}\anga{बालवः}{\time{43-2}{23:14}}\hspace{1ex}\uanga{कौलवः}}{}
}
{दिक्पाल-पूजा\eventsep महालय-पक्ष-आरम्भः\eventsep पूर्र्णमासेष्टिः\eventsep पूर्णिमा-व्रतम्\eventsep स्थालीपाकः\eventsep उमा-महेश्वर-व्रतम्\eventsep उपाङ्ग-ललिता-गौरी-व्रतम्\eventsep विश्वरूप-यात्रा\eventsep यतिचातुर्मास्यव्रत-समापनम्}
{Sat} 
\cfoot{\rygdata{09:03--10:33}{13:35--15:06}{06:01--07:32}}
\caldata{SEPTEMBER}{15}{\sunmonth{सिंहः}{30}{}{भाद्रपदः}{वर्षऋतुः}{भानुः}{विकारी}{दक्षिणायनम्}{वर्षऋतुः}}
{\sunmoonsrdata{06:01}{18:07}{18:56}{06:39}{12:04}
{\kalas{04:26 05:13 09:14 08:26 10:03 16:30 10:51 13:16 15:41 17:18 18:54 21:05 22:34 01:33(+1)}}}
{\tnykdata{\anga{\tithi{16}{कृष्ण-प्रथमा}}{\time{15-56}{12:24}}\hspace{1ex}}%
{\anga{उत्तरप्रोष्ठपदा}{\time{49-11}{01:41(+1)}}\hspace{1ex}}{चन्द्रराशिः—\mbox{मीनः}}%
{\anga{गण्डः}{\time{40-17}{22:08}}\hspace{1ex}\uanga{वृद्धिः}}%
{\anga{कौलवः}{\time{15-56}{12:24}}\hspace{1ex}\anga{तैतिलः}{\time{48-45}{01:31(+1)}}\hspace{1ex}\uanga{गरः}}{}
}
{\tamil{ஆவணி~ஞாயிற்றுக்கிழமை}\eventsep अशून्यशयन-व्रतम्}
{Sun} 
\cfoot{\rygdata{16:36--18:07}{12:04--13:34}{15:05--16:36}}
\caldata{SEPTEMBER}{16}{\sunmonth{सिंहः}{31}{}{भाद्रपदः}{वर्षऋतुः}{सोमः}{विकारी}{दक्षिणायनम्}{वर्षऋतुः}}
{\sunmoonsrdata{06:01}{18:06}{19:33}{07:25}{12:03}
{\kalas{04:26 05:13 09:14 08:26 10:03 16:29 10:51 13:16 15:41 17:17 18:54 21:05 22:34 01:33(+1)}}}
{\tnykdata{\anga{\tithi{17}{कृष्ण-द्वितीया}}{\time{21-25}{14:35}}\hspace{1ex}}%
{\anga{रेवती}{\time{55-45}{04:19(+1)}}\hspace{1ex}}{चन्द्रराशिः—\mbox{मीनः\RIGHTarrow\textsf{04:19(+1)}}}%
{\anga{वृद्धिः}{\time{41-56}{22:48}}\hspace{1ex}\uanga{ध्रुवः}}%
{\anga{गरः}{\time{21-25}{14:35}}\hspace{1ex}\anga{वणिजः}{\time{53-57}{03:36(+1)}}\hspace{1ex}\uanga{विष्टिः}}{}
}
{}
{Mon} 
\cfoot{\rygdata{07:32--09:02}{10:33--12:03}{13:34--15:05}}
\caldata{SEPTEMBER}{17}{\sunmonth{कन्या}{1}{\mbox{सिंहः{\tiny\RIGHTarrow}\textsf{12:31}}}{भाद्रपदः}{वर्षऋतुः}{मङ्गलः}{विकारी}{दक्षिणायनम्}{वर्षऋतुः}}
{\sunmoonsrdata{06:01}{18:05}{20:12}{08:12}{12:03}
{\kalas{04:26 05:13 09:14 08:26 10:02 16:29 10:51 13:15 15:40 17:17 18:53 21:04 22:34 01:33(+1)}}}
{\tnykdata{\anga{\tithi{18}{कृष्ण-तृतीया}}{\time{26-18}{16:33}}\hspace{1ex}}%
{\fullanga{अश्विनी}}{चन्द्रराशिः—\mbox{मेषः}}%
{\anga{ध्रुवः}{\time{43-6}{23:16}}\hspace{1ex}\uanga{व्याघातः}}%
{\anga{विष्टिः}{\time{26-18}{16:33}}\hspace{1ex}\anga{बवः}{\time{58-28}{05:25(+1)}}\hspace{1ex}\uanga{बालवः}}{}
}
{अङ्गारकी विघ्नराज-महागणपति सङ्कटहर-चतुर्थी-व्रतम्\eventsep भौमाश्विनी-पुण्यकालः\eventsep गौरी-व्रतम्\eventsep कजरी-तृतीया\eventsep कन्या-रवि-सङ्क्रमण-षडशीति-पुण्यकालः~\textsf{12:31}{\RIGHTarrow}\textsf{12:31(+1)}\eventsep विश्वकर्मा-जयन्ती}
{Tue} 
\cfoot{\rygdata{15:04--16:35}{09:02--10:33}{12:03--13:34}}
\caldata{SEPTEMBER}{18}{\sunmonth{कन्या}{2}{}{भाद्रपदः}{वर्षऋतुः}{बुधः}{विकारी}{दक्षिणायनम्}{वर्षऋतुः}}
{\sunmoonsrdata{06:01}{18:04}{20:53}{09:00}{12:03}
{\kalas{04:25 05:13 09:14 08:26 10:02 16:28 10:50 13:15 15:40 17:16 18:52 21:04 22:33 01:32(+1)}}}
{\tnykdata{\anga{\tithi{19}{कृष्ण-चतुर्थी}}{\time{30-25}{18:11}}\hspace{1ex}}%
{\anga{अश्विनी}{\time{1-40}{06:41}}\hspace{1ex}}{चन्द्रराशिः—\mbox{मेषः}}%
{\anga{व्याघातः}{\time{43-38}{23:28}}\hspace{1ex}\uanga{हर्षणः}}%
{\anga{बालवः}{\time{30-25}{18:11}}\hspace{1ex}\uanga{कौलवः}}{}
}
{दिक्पाल-पूजा\eventsep महाभरणी}
{Wed} 
\cfoot{\rygdata{12:03--13:33}{07:31--09:02}{10:32--12:03}}
\caldata{SEPTEMBER}{19}{\sunmonth{कन्या}{3}{}{भाद्रपदः}{वर्षऋतुः}{गुरुः}{विकारी}{दक्षिणायनम्}{वर्षऋतुः}}
{\sunmoonsrdata{06:01}{18:04}{21:37}{09:50}{12:02}
{\kalas{04:25 05:13 09:14 08:26 10:02 16:27 10:50 13:15 15:39 17:16 18:51 21:03 22:33 01:32(+1)}}}
{\tnykdata{\anga{\tithi{20}{कृष्ण-पञ्चमी}}{\time{33-33}{19:26}}\hspace{1ex}}%
{\anga{अपभरणी}{\time{6-42}{08:42}}\hspace{1ex}}{चन्द्रराशिः—\mbox{मेषः\RIGHTarrow\textsf{15:08}}}%
{\anga{हर्षणः}{\time{43-21}{23:21}}\hspace{1ex}\uanga{वज्रम्}}%
{\anga{कौलवः}{\time{2-8}{06:52}}\hspace{1ex}\anga{तैतिलः}{\time{33-33}{19:26}}\hspace{1ex}\uanga{गरः}}{}
}
{चन्द्र-षष्ठी\eventsep कृत्तिका-व्रतम्\eventsep नाग-पूजा\eventsep सप्तर्षि-पूजा/अर्घ्यम्}
{Thu} 
\cfoot{\rygdata{13:33--15:03}{06:01--07:31}{09:02--10:32}}
\caldata{SEPTEMBER}{20}{\sunmonth{कन्या}{4}{}{भाद्रपदः}{वर्षऋतुः}{शुक्रः}{विकारी}{दक्षिणायनम्}{वर्षऋतुः}}
{\sunmoonsrdata{06:01}{18:03}{22:25}{10:43}{12:02}
{\kalas{04:25 05:13 09:13 08:25 10:02 16:27 10:50 13:14 15:39 17:15 18:51 21:03 22:32 01:32(+1)}}}
{\tnykdata{\anga{\tithi{21}{कृष्ण-षष्ठी}}{\time{35-25}{20:11}}\hspace{1ex}}%
{\anga{कृत्तिका}{\time{10-39}{10:17}}\hspace{1ex}}{चन्द्रराशिः—\mbox{वृषभः}}%
{\anga{वज्रम्}{\time{42-3}{22:51}}\hspace{1ex}\uanga{सिद्धिः}}%
{\anga{गरः}{\time{4-39}{07:53}}\hspace{1ex}\anga{वणिजः}{\time{35-25}{20:11}}\hspace{1ex}\uanga{विष्टिः}}{}
}
{काञ्ची ३३ जगद्गुरु श्री-सच्चिदानन्दघनेन्द्र सरस्वती २ आराधना~\#{१३२८}\eventsep कपिल-षष्ठी}
{Fri} 
\cfoot{\rygdata{10:32--12:02}{15:03--16:33}{07:31--09:02}}
\caldata{SEPTEMBER}{21}{\sunmonth{कन्या}{5}{}{भाद्रपदः}{वर्षऋतुः}{शनिः}{विकारी}{दक्षिणायनम्}{वर्षऋतुः}}
{\sunmoonsrdata{06:01}{18:02}{23:18}{11:38}{12:02}
{\kalas{04:25 05:13 09:13 08:25 10:01 16:26 10:50 13:14 15:38 17:14 18:50 21:02 22:32 01:32(+1)}}}
{\tnykdata{\anga{\tithi{22}{कृष्ण-सप्तमी}}{\time{35-48}{20:20}}\hspace{1ex}}%
{\anga{रोहिणी}{\time{13-14}{11:19}}\hspace{1ex}}{चन्द्रराशिः—\mbox{वृषभः\RIGHTarrow\textsf{23:36}}}%
{\anga{सिद्धिः}{\time{39-36}{21:51}}\hspace{1ex}\uanga{व्यतीपातः}}%
{\anga{विष्टिः}{\time{5-49}{08:21}}\hspace{1ex}\anga{बवः}{\time{35-48}{20:20}}\hspace{1ex}\uanga{बालवः}}{}
}
{महालक्ष्मी-व्रत-समापनम्\eventsep \tamil{புரட்டாசி~சனிக்கிழமை}\eventsep \tamil{திருநாளைப்போவார் நாயனார் (17) குருபூஜை}\eventsep शृङ्गेरी ३५ जगद्गुरु श्री-अभिनव विद्यातीर्थ महास्वामी आराधना\eventsep शनिरोहिणी-पुण्यकालः}
{Sat} 
\cfoot{\rygdata{09:01--10:31}{13:32--15:02}{06:01--07:31}}
\caldata{SEPTEMBER}{22}{\sunmonth{कन्या}{6}{}{भाद्रपदः}{वर्षऋतुः}{भानुः}{विकारी}{दक्षिणायनम्}{वर्षऋतुः}}
{\sunmoonsrdata{06:01}{18:02}{00:14(+1)}{12:34}{12:01}
{\kalas{04:25 05:13 09:13 08:25 10:01 16:25 10:49 13:13 15:37 17:14 18:49 21:01 22:31 01:31(+1)}}}
{\tnykdata{\anga{\tithi{23}{कृष्ण-अष्टमी}}{\time{34-32}{19:50}}\hspace{1ex}}%
{\anga{मृगशीर्षम्}{\time{14-16}{11:43}}\hspace{1ex}}{चन्द्रराशिः—\mbox{मिथुनम्}}%
{\anga{व्यतीपातः}{\time{35-48}{20:20}}\hspace{1ex}\uanga{वरीयान्}}%
{\anga{बालवः}{\time{5-23}{08:10}}\hspace{1ex}\anga{कौलवः}{\time{34-32}{19:50}}\hspace{1ex}\uanga{तैतिलः}}{}
}
{अशोकाष्टमी-व्रत-आरम्भः\eventsep जीमूतवाहन-पूजा\eventsep मध्याष्टमी\eventsep महाव्यतीपात-श्राद्धम्}
{Sun} 
\cfoot{\rygdata{16:32--18:02}{12:01--13:31}{15:01--16:32}}
\caldata{SEPTEMBER}{23}{\sunmonth{कन्या}{7}{}{भाद्रपदः}{वर्षऋतुः}{सोमः}{विकारी}{दक्षिणायनम्}{वर्षऋतुः}}
{\sunmoonsrdata{06:01}{18:01}{01:13(+1)}{13:29}{12:01}
{\kalas{04:25 05:13 09:13 08:25 10:01 16:25 10:49 13:13 15:37 17:13 18:49 21:01 22:31 01:31(+1)}}}
{\tnykdata{\anga{\tithi{24}{कृष्ण-नवमी}}{\time{31-30}{18:37}}\hspace{1ex}}%
{\anga{आर्द्रा}{\time{13-34}{11:27}}\hspace{1ex}}{चन्द्रराशिः—\mbox{मिथुनम्\RIGHTarrow\textsf{04:47(+1)}}}%
{\anga{वरीयान्}{\time{30-37}{18:16}}\hspace{1ex}\uanga{परिघः}}%
{\anga{तैतिलः}{\time{3-14}{07:19}}\hspace{1ex}\anga{गरः}{\time{31-30}{18:37}}\hspace{1ex}\anga{वणिजः}{\time{59-19}{05:45(+1)}}\hspace{1ex}\uanga{विष्टिः}}{}
}
{दक्षिण-विषुव-दिनम्\eventsep दुर्गा/गौरी-पूजा\eventsep नभस्य-मासः/वर्षऋतुः{\RIGHTarrow}\textsf{13:20}\eventsep सुमङ्गला-नवमी\eventsep तुला-विषु-पुण्यकालः~\textsf{09:20}{\RIGHTarrow}\textsf{17:20}}
{Mon} 
\cfoot{\rygdata{07:31--09:01}{10:31--12:01}{13:31--15:01}}
\caldata{SEPTEMBER}{24}{\sunmonth{कन्या}{8}{}{भाद्रपदः}{वर्षऋतुः}{मङ्गलः}{विकारी}{दक्षिणायनम्}{वर्षऋतुः}}
{\sunmoonsrdata{06:01}{18:00}{02:14(+1)}{14:24}{12:01}
{\kalas{04:25 05:13 09:13 08:25 10:01 16:24 10:49 13:13 15:36 17:12 18:48 21:00 22:31 01:31(+1)}}}
{\tnykdata{\anga{\tithi{25}{कृष्ण-दशमी}}{\time{26-43}{16:42}}\hspace{1ex}}%
{\anga{पुनर्वसुः}{\time{11-8}{10:28}}\hspace{1ex}}{चन्द्रराशिः—\mbox{कटकः}}%
{\anga{परिघः}{\time{24-0}{15:37}}\hspace{1ex}\uanga{शिवः}}%
{\anga{विष्टिः}{\time{26-43}{16:42}}\hspace{1ex}\anga{बवः}{\time{53-42}{03:30(+1)}}\hspace{1ex}\uanga{बालवः}}{}
}
{}
{Tue} 
\cfoot{\rygdata{15:00--16:30}{09:01--10:31}{12:01--13:31}}
\caldata{SEPTEMBER}{25}{\sunmonth{कन्या}{9}{}{भाद्रपदः}{वर्षऋतुः}{बुधः}{विकारी}{दक्षिणायनम्}{वर्षऋतुः}}
{\sunmoonsrdata{06:01}{17:59}{03:15(+1)}{15:17}{12:00}
{\kalas{04:25 05:13 09:13 08:25 10:00 16:24 10:48 13:12 15:36 17:12 18:47 21:00 22:30 01:30(+1)}}}
{\tnykdata{\anga{\tithi{26}{कृष्ण-एकादशी}}{\time{20-19}{14:09}}\hspace{1ex}}%
{\anga{पुष्यः}{\time{7-3}{08:50}}\hspace{1ex}}{चन्द्रराशिः—\mbox{कटकः}}%
{\anga{शिवः}{\time{16-4}{12:27}}\hspace{1ex}\uanga{सिद्धः}}%
{\anga{बालवः}{\time{20-19}{14:09}}\hspace{1ex}\anga{कौलवः}{\time{46-35}{00:39(+1)}}\hspace{1ex}\uanga{तैतिलः}}{}
}
{हरिवासरः{\RIGHTarrow}\textsf{19:25}\eventsep सर्व-इन्दिरा-एकादशी\eventsep यति-महालयम्}
{Wed} 
\cfoot{\rygdata{12:00--13:30}{07:31--09:01}{10:30--12:00}}
\caldata{SEPTEMBER}{26}{\sunmonth{कन्या}{10}{}{भाद्रपदः}{वर्षऋतुः}{गुरुः}{विकारी}{दक्षिणायनम्}{वर्षऋतुः}}
{\sunmoonsrdata{06:01}{17:59}{04:16(+1)}{16:08}{12:00}
{\kalas{04:25 05:13 09:12 08:25 10:00 16:23 10:48 13:12 15:35 17:11 18:47 20:59 22:30 01:30(+1)}}}
{\tnykdata{\anga{\tithi{27}{कृष्ण-द्वादशी}}{\time{12-33}{11:02}}\hspace{1ex}}%
{\anga{आश्रेषा}{\time{1-31}{06:38}}\hspace{1ex}\anga{मघा}{\time{54-53}{03:58(+1)}}\hspace{1ex}}{चन्द्रराशिः—\mbox{कटकः\RIGHTarrow\textsf{06:38}}}%
{\anga{सिद्धः}{\time{6-59}{08:49}}\hspace{1ex}\anga{साध्यः}{\time{57-0}{04:49(+1)}}\hspace{1ex}\uanga{शुभः}}%
{\anga{तैतिलः}{\time{12-33}{11:02}}\hspace{1ex}\anga{गरः}{\time{38-16}{21:20}}\hspace{1ex}\uanga{वणिजः}}{}
}
{द्वापरयुगादिः\eventsep काञ्ची ४४ जगद्गुरु श्री-पूर्णबोधेन्द्र सरस्वती २ आराधना~\#{९८०}\eventsep प्रदोष-व्रतम्}
{Thu} 
\cfoot{\rygdata{13:30--14:59}{06:01--07:31}{09:00--10:30}}
\caldata{SEPTEMBER}{27}{\sunmonth{कन्या}{11}{}{भाद्रपदः}{वर्षऋतुः}{शुक्रः}{विकारी}{दक्षिणायनम्}{वर्षऋतुः}}
{\sunmoonsrdata{06:01}{17:58}{05:16(+1)}{16:57}{12:00}
{\kalas{04:25 05:13 09:12 08:24 10:00 16:22 10:48 13:11 15:35 17:10 18:46 20:59 22:29 01:30(+1)}}}
{\tnykdata{\anga{\tithi{28}{कृष्ण-त्रयोदशी}}{\time{3-46}{07:32}}\hspace{1ex}\anga{\tithi{29}{कृष्ण-चतुर्दशी}}{\time{54-21}{03:46(+1)}}\hspace{1ex}}%
{\anga{पूर्वफल्गुनी}{\time{47-32}{01:02(+1)}}\hspace{1ex}}{चन्द्रराशिः—\mbox{सिंहः}}%
{\anga{शुभः}{\time{46-27}{00:36(+1)}}\hspace{1ex}\uanga{शुक्लः}}%
{\anga{वणिजः}{\time{3-46}{07:32}}\hspace{1ex}\anga{विष्टिः}{\time{29-7}{17:40}}\hspace{1ex}\anga{शकुनिः}{\time{54-21}{03:46(+1)}}\hspace{1ex}\uanga{चतुष्पात्}}{}
}
{कात्यायनी-जयन्ती\eventsep मासशिवरात्रिः\eventsep शस्त्रहतचतुर्दशी}
{Fri} 
\cfoot{\rygdata{10:30--12:00}{14:59--16:28}{07:31--09:00}}
\caldata{SEPTEMBER}{28}{\sunmonth{कन्या}{12}{}{भाद्रपदः}{वर्षऋतुः}{शनिः}{विकारी}{दक्षिणायनम्}{वर्षऋतुः}}
{\sunmoonsrdata{06:01}{17:57}{---}{17:45}{11:59}
{\kalas{04:25 05:13 09:12 08:24 10:00 16:22 10:48 13:11 15:34 17:10 18:46 20:58 22:29 01:30(+1)}}}
{\tnykdata{\anga{\tithi{30}{अमावास्या}}{\time{44-47}{23:56}}\hspace{1ex}}%
{\anga{उत्तरफल्गुनी}{\time{39-57}{22:00}}\hspace{1ex}}{चन्द्रराशिः—\mbox{सिंहः\RIGHTarrow\textsf{06:17}}}%
{\anga{शुक्लः}{\time{35-42}{20:18}}\hspace{1ex}\uanga{ब्रह्म}}%
{\anga{चतुष्पात्}{\time{19-33}{13:51}}\hspace{1ex}\anga{नाग}{\time{44-47}{23:56}}\hspace{1ex}\uanga{किंस्तुघ्नः}}{}
}
{(भाद्रपद) महालय अमावास्या\eventsep अश्वशिरो-देव-पूजा\eventsep महालय-पक्ष-समापनम्\eventsep पार्वणव्रतम् अमावास्यायाम्\eventsep \tamil{புரட்டாசி~சனிக்கிழமை}\eventsep सुजन्मप्राप्ति-व्रतम्\eventsep शृङ्गेरी ३४ जगद्गुरु श्री-चन्द्रशेखर भारती आराधना}
{Sat} 
\cfoot{\rygdata{09:00--10:30}{13:29--14:58}{06:01--07:31}}
\caldata{SEPTEMBER}{29}{\sunmonth{कन्या}{13}{}{आश्वयुजः}{शरदृतुः}{भानुः}{विकारी}{दक्षिणायनम्}{वर्षऋतुः}}
{\sunmoonrsdata{06:01}{17:57}{06:16}{18:33}{11:59}
{\kalas{04:25 05:13 09:12 08:24 10:00 16:21 10:47 13:10 15:33 17:09 18:45 20:58 22:28 01:29(+1)}}}
{\tnykdata{\anga{\tithi{1}{शुक्ल-प्रथमा}}{\time{35-30}{20:13}}\hspace{1ex}}%
{\anga{हस्तः}{\time{32-37}{19:04}}\hspace{1ex}}{चन्द्रराशिः—\mbox{कन्या\RIGHTarrow\textsf{05:42(+1)}}}%
{\anga{ब्रह्म}{\time{25-8}{16:04}}\hspace{1ex}\uanga{इन्द्रः}}%
{\anga{किंस्तुघ्नः}{\time{10-4}{10:03}}\hspace{1ex}\anga{बवः}{\time{35-30}{20:13}}\hspace{1ex}\uanga{बालवः}}{}
}
{आदित्यहस्त-पुण्यकालः\eventsep अग्रसेन-महाराज-जयन्ती\eventsep दर्शेष्टिः\eventsep दौहित्र-प्रतिपत्\eventsep गृहदेवी-पूजा\eventsep स्तनवृद्धि-गौरी-व्रतम्\eventsep स्थालीपाकः\eventsep शरन्नवरात्र-आरम्भः}
{Sun} 
\cfoot{\rygdata{16:27--17:57}{11:59--13:28}{14:58--16:27}}
\caldata{SEPTEMBER}{30}{\sunmonth{कन्या}{14}{}{आश्वयुजः}{शरदृतुः}{सोमः}{विकारी}{दक्षिणायनम्}{वर्षऋतुः}}
{\sunmoonrsdata{06:01}{17:56}{07:16}{19:22}{11:58}
{\kalas{04:24 05:13 09:12 08:24 09:59 16:21 10:47 13:10 15:33 17:08 18:44 20:57 22:28 01:29(+1)}}}
{\tnykdata{\anga{\tithi{2}{शुक्ल-द्वितीया}}{\time{27-0}{16:49}}\hspace{1ex}}%
{\anga{चित्रा}{\time{26-3}{16:26}}\hspace{1ex}}{चन्द्रराशिः—\mbox{तुला}}%
{\anga{इन्द्रः}{\time{15-7}{12:04}}\hspace{1ex}\uanga{वैधृतिः}}%
{\anga{बालवः}{\time{1-7}{06:28}}\hspace{1ex}\anga{कौलवः}{\time{27-0}{16:49}}\hspace{1ex}\anga{तैतिलः}{\time{53-11}{03:18(+1)}}\hspace{1ex}\uanga{गरः}}{}
}
{चन्द्र-दर्शनम्\eventsep वैधृति-श्राद्धम्}
{Mon} 
\cfoot{\rygdata{07:31--09:00}{10:29--11:58}{13:28--14:57}}
\caldata{OCTOBER}{1}{\sunmonth{कन्या}{15}{}{आश्वयुजः}{शरदृतुः}{मङ्गलः}{विकारी}{दक्षिणायनम्}{वर्षऋतुः}}
{\sunmoonrsdata{06:01}{17:55}{08:17}{20:13}{11:58}
{\kalas{04:24 05:13 09:12 08:24 09:59 16:20 10:47 13:10 15:32 17:08 18:44 20:57 22:27 01:29(+1)}}}
{\tnykdata{\anga{\tithi{3}{शुक्ल-तृतीया}}{\time{19-44}{13:55}}\hspace{1ex}}%
{\anga{स्वाती}{\time{20-42}{14:18}}\hspace{1ex}}{चन्द्रराशिः—\mbox{तुला}}%
{\anga{वैधृतिः}{\time{6-2}{08:26}}\hspace{1ex}\anga{विष्कम्भः}{\time{58-14}{05:19(+1)}}\hspace{1ex}\uanga{प्रीतिः}}%
{\anga{गरः}{\time{19-44}{13:55}}\hspace{1ex}\anga{वणिजः}{\time{46-41}{00:42(+1)}}\hspace{1ex}\uanga{विष्टिः}}{}
}
{मेघपालीय-तृतीया}
{Tue} 
\cfoot{\rygdata{14:57--16:26}{09:00--10:29}{11:58--13:27}}
\caldata{OCTOBER}{2}{\sunmonth{कन्या}{16}{}{आश्वयुजः}{शरदृतुः}{बुधः}{विकारी}{दक्षिणायनम्}{वर्षऋतुः}}
{\sunmoonrsdata{06:01}{17:55}{09:17}{21:05}{11:58}
{\kalas{04:24 05:13 09:11 08:24 09:59 16:19 10:46 13:09 15:32 17:07 18:43 20:56 22:27 01:29(+1)}}}
{\tnykdata{\anga{\tithi{4}{शुक्ल-चतुर्थी}}{\time{14-5}{11:39}}\hspace{1ex}}%
{\anga{विशाखा}{\time{16-59}{12:49}}\hspace{1ex}}{चन्द्रराशिः—\mbox{तुला\RIGHTarrow\textsf{07:07}}}%
{\anga{प्रीतिः}{\time{51-56}{02:48(+1)}}\hspace{1ex}\uanga{आयुष्मान्}}%
{\anga{विष्टिः}{\time{14-5}{11:39}}\hspace{1ex}\anga{बवः}{\time{42-0}{22:49}}\hspace{1ex}\uanga{बालवः}}{}
}
{बुधानुराधा-पुण्यकालः\eventsep देवता-सुवासिनी-पूजा\eventsep ललिता-पञ्चमी}
{Wed} 
\cfoot{\rygdata{11:58--13:27}{07:30--09:00}{10:29--11:58}}
\caldata{OCTOBER}{3}{\sunmonth{कन्या}{17}{}{आश्वयुजः}{शरदृतुः}{गुरुः}{विकारी}{दक्षिणायनम्}{वर्षऋतुः}}
{\sunmoonrsdata{06:01}{17:54}{10:16}{21:58}{11:58}
{\kalas{04:24 05:13 09:11 08:24 09:59 16:19 10:46 13:09 15:31 17:06 18:42 20:56 22:27 01:28(+1)}}}
{\tnykdata{\anga{\tithi{5}{शुक्ल-पञ्चमी}}{\time{10-25}{10:12}}\hspace{1ex}}%
{\anga{अनूराधा}{\time{15-15}{12:07}}\hspace{1ex}}{चन्द्रराशिः—\mbox{वृश्चिकः}}%
{\anga{आयुष्मान्}{\time{47-20}{00:58(+1)}}\hspace{1ex}\uanga{सौभाग्यः}}%
{\anga{बालवः}{\time{10-25}{10:12}}\hspace{1ex}\anga{कौलवः}{\time{39-23}{21:47}}\hspace{1ex}\uanga{तैतिलः}}{}
}
{षष्ठी-व्रतम्\eventsep उपाङ्ग-ललिता-व्रतम्\eventsep शान्ति-पञ्चमी-व्रतम्}
{Thu} 
\cfoot{\rygdata{13:27--14:56}{06:01--07:30}{08:59--10:28}}
\caldata{OCTOBER}{4}{\sunmonth{कन्या}{18}{}{आश्वयुजः}{शरदृतुः}{शुक्रः}{विकारी}{दक्षिणायनम्}{वर्षऋतुः}}
{\sunmoonrsdata{06:01}{17:53}{11:13}{22:52}{11:57}
{\kalas{04:24 05:13 09:11 08:24 09:59 16:18 10:46 13:08 15:31 17:06 18:42 20:55 22:26 01:28(+1)}}}
{\tnykdata{\anga{\tithi{6}{शुक्ल-षष्ठी}}{\time{8-54}{09:35}}\hspace{1ex}}%
{\anga{ज्येष्ठा}{\time{15-37}{12:16}}\hspace{1ex}}{चन्द्रराशिः—\mbox{वृश्चिकः\RIGHTarrow\textsf{12:16}}}%
{\anga{सौभाग्यः}{\time{44-27}{23:48}}\hspace{1ex}\uanga{शोभनः}}%
{\anga{तैतिलः}{\time{8-54}{09:35}}\hspace{1ex}\anga{गरः}{\time{38-57}{21:36}}\hspace{1ex}\uanga{वणिजः}}{}
}
{काञ्ची ४५ जगद्गुरु श्री-परमशिवेन्द्र सरस्वती १ आराधना~\#{९५९}\eventsep सरस्वती-आवाहनम्}
{Fri} 
\cfoot{\rygdata{10:28--11:57}{14:55--16:24}{07:30--08:59}}
\caldata{OCTOBER}{5}{\sunmonth{कन्या}{19}{}{आश्वयुजः}{शरदृतुः}{शनिः}{विकारी}{दक्षिणायनम्}{वर्षऋतुः}}
{\sunmoonrsdata{06:01}{17:53}{12:07}{23:46}{11:57}
{\kalas{04:24 05:13 09:11 08:24 09:58 16:18 10:46 13:08 15:30 17:05 18:41 20:55 22:26 01:28(+1)}}}
{\tnykdata{\anga{\tithi{7}{शुक्ल-सप्तमी}}{\time{9-32}{09:50}}\hspace{1ex}}%
{\anga{मूला}{\time{18-5}{13:15}}\hspace{1ex}}{चन्द्रराशिः—\mbox{धनुः}}%
{\anga{शोभनः}{\time{43-12}{23:18}}\hspace{1ex}\uanga{अतिगण्डः}}%
{\anga{वणिजः}{\time{9-32}{09:50}}\hspace{1ex}\anga{विष्टिः}{\time{40-37}{22:17}}\hspace{1ex}\uanga{बवः}}{}
}
{पत्रिका-प्रवेश-पूजा\eventsep \tamil{புரட்டாசி~சனிக்கிழமை}\eventsep शुभ-सप्तमी}
{Sat} 
\cfoot{\rygdata{08:59--10:28}{13:26--14:55}{06:01--07:30}}
\caldata{OCTOBER}{6}{\sunmonth{कन्या}{20}{}{आश्वयुजः}{शरदृतुः}{भानुः}{विकारी}{दक्षिणायनम्}{वर्षऋतुः}}
{\sunmoonrsdata{06:01}{17:52}{12:57}{00:38(+1)}{11:57}
{\kalas{04:24 05:13 09:11 08:24 09:58 16:17 10:46 13:08 15:30 17:04 18:41 20:54 22:25 01:28(+1)}}}
{\tnykdata{\anga{\tithi{8}{शुक्ल-अष्टमी}}{\time{12-11}{10:54}}\hspace{1ex}}%
{\anga{पूर्वाषाढा}{\time{22-27}{15:00}}\hspace{1ex}}{चन्द्रराशिः—\mbox{धनुः\RIGHTarrow\textsf{21:33}}}%
{\anga{अतिगण्डः}{\time{43-21}{23:22}}\hspace{1ex}\uanga{सुकर्म}}%
{\anga{बवः}{\time{12-11}{10:54}}\hspace{1ex}\anga{बालवः}{\time{44-10}{23:42}}\hspace{1ex}\uanga{कौलवः}}{}
}
{भद्रकाळी-पूजा\eventsep दुर्गाष्टमी\eventsep काञ्ची १९ जगद्गुरु श्री-मार्ताण्ड विद्याघनेन्द्र सरस्वती आराधना~\#{१६२२}\eventsep काल-त्रिरात्रि-व्रतम्\eventsep मन्वादिः-(स्वायम्भुवः-[१])}
{Sun} 
\cfoot{\rygdata{16:23--17:52}{11:57--13:25}{14:54--16:23}}
\caldata{OCTOBER}{7}{\sunmonth{कन्या}{21}{}{आश्वयुजः}{शरदृतुः}{सोमः}{विकारी}{दक्षिणायनम्}{वर्षऋतुः}}
{\sunmoonrsdata{06:02}{17:51}{13:43}{01:28(+1)}{11:56}
{\kalas{04:24 05:13 09:11 08:23 09:58 16:17 10:45 13:07 15:29 17:04 18:40 20:54 22:25 01:28(+1)}}}
{\tnykdata{\anga{\tithi{9}{शुक्ल-नवमी}}{\time{16-30}{12:38}}\hspace{1ex}}%
{\anga{उत्तराषाढा}{\time{28-21}{17:22}}\hspace{1ex}}{चन्द्रराशिः—\mbox{मकरः}}%
{\anga{सुकर्म}{\time{44-36}{23:52}}\hspace{1ex}\uanga{धृतिः}}%
{\anga{कौलवः}{\time{16-30}{12:38}}\hspace{1ex}\anga{तैतिलः}{\time{49-8}{01:41(+1)}}\hspace{1ex}\uanga{गरः}}{}
}
{\tamil{ஏனாதிநாத நாயனார் (8) குருபூஜை}\eventsep भद्रकाळी-व्रतम्\eventsep बुद्ध-जयन्ती\eventsep महानवमी/सरस्वती-पूजा\eventsep सोमश्रावणी-पुण्यकालः\eventsep शरन्नवरात्र-समापनम्}
{Mon} 
\cfoot{\rygdata{07:30--08:59}{10:28--11:56}{13:25--14:54}}
\caldata{OCTOBER}{8}{\sunmonth{कन्या}{22}{}{आश्वयुजः}{शरदृतुः}{मङ्गलः}{विकारी}{दक्षिणायनम्}{वर्षऋतुः}}
{\sunmoonrsdata{06:02}{17:51}{14:26}{02:17(+1)}{11:56}
{\kalas{04:24 05:13 09:11 08:23 09:58 16:16 10:45 13:07 15:29 17:03 18:39 20:53 22:25 01:27(+1)}}}
{\tnykdata{\anga{\tithi{10}{शुक्ल-दशमी}}{\time{22-0}{14:50}}\hspace{1ex}}%
{\anga{श्रवणः}{\time{35-18}{20:09}}\hspace{1ex}}{चन्द्रराशिः—\mbox{मकरः}}%
{\anga{धृतिः}{\time{46-34}{00:39(+1)}}\hspace{1ex}\uanga{शूलः}}%
{\anga{गरः}{\time{22-0}{14:50}}\hspace{1ex}\anga{वणिजः}{\time{55-3}{04:03(+1)}}\hspace{1ex}\uanga{विष्टिः}}{}
}
{आयुध-पूजा\eventsep दशहरा\eventsep दुर्गा-पूजा\eventsep गङ्गावतरणम्\eventsep कूष्माण्ड-दशमी\eventsep मध्वाचार्य-जयन्ती~\#{७८२}\eventsep विजयदशमी\eventsep युद्धदेवता-आराधना\eventsep शमी-पूजा\eventsep श्रवण-व्रतम्}
{Tue} 
\cfoot{\rygdata{14:53--16:22}{08:59--10:27}{11:56--13:25}}
\caldata{OCTOBER}{9}{\sunmonth{कन्या}{23}{}{आश्वयुजः}{शरदृतुः}{बुधः}{विकारी}{दक्षिणायनम्}{वर्षऋतुः}}
{\sunmoonrsdata{06:02}{17:50}{15:06}{03:04(+1)}{11:56}
{\kalas{04:24 05:13 09:10 08:23 09:58 16:16 10:45 13:07 15:28 17:03 18:39 20:53 22:24 01:27(+1)}}}
{\tnykdata{\anga{\tithi{11}{शुक्ल-एकादशी}}{\time{28-12}{17:18}}\hspace{1ex}}%
{\anga{श्रविष्ठा}{\time{42-47}{23:09}}\hspace{1ex}}{चन्द्रराशिः—\mbox{मकरः\RIGHTarrow\textsf{09:38}}}%
{\anga{शूलः}{\time{48-54}{01:36(+1)}}\hspace{1ex}\uanga{गण्डः}}%
{\anga{विष्टिः}{\time{28-12}{17:18}}\hspace{1ex}\uanga{बवः}}{}
}
{हरिवासरः{\RIGHTarrow}\textsf{23:57}\eventsep प्रदोष-व्रतम्\eventsep सर्व-पापाङ्कुशा-एकादशी}
{Wed} 
\cfoot{\rygdata{11:56--13:24}{07:30--08:59}{10:27--11:56}}
\caldata{OCTOBER}{10}{\sunmonth{कन्या}{24}{}{आश्वयुजः}{शरदृतुः}{गुरुः}{विकारी}{दक्षिणायनम्}{वर्षऋतुः}}
{\sunmoonrsdata{06:02}{17:49}{15:43}{03:50(+1)}{11:55}
{\kalas{04:24 05:13 09:10 08:23 09:58 16:15 10:45 13:06 15:28 17:02 18:38 20:52 22:24 01:27(+1)}}}
{\tnykdata{\anga{\tithi{12}{शुक्ल-द्वादशी}}{\time{34-34}{19:52}}\hspace{1ex}}%
{\anga{शतभिषक्}{\time{50-22}{02:11(+1)}}\hspace{1ex}}{चन्द्रराशिः—\mbox{कुम्भः}}%
{\anga{गण्डः}{\time{51-17}{02:33(+1)}}\hspace{1ex}\uanga{वृद्धिः}}%
{\anga{बवः}{\time{1-23}{06:35}}\hspace{1ex}\anga{बालवः}{\time{34-34}{19:52}}\hspace{1ex}\uanga{कौलवः}}{}
}
{द्विदल-व्रत-आरम्भः\eventsep \tamil{நரசிங்கமுனையரைய நாயனார் (40) குருபூஜை}}
{Thu} 
\cfoot{\rygdata{13:24--14:52}{06:02--07:30}{08:59--10:27}}
\caldata{OCTOBER}{11}{\sunmonth{कन्या}{25}{}{आश्वयुजः}{शरदृतुः}{शुक्रः}{विकारी}{दक्षिणायनम्}{वर्षऋतुः}}
{\sunmoonrsdata{06:02}{17:49}{16:20}{04:36(+1)}{11:55}
{\kalas{04:24 05:13 09:10 08:23 09:57 16:14 10:44 13:06 15:27 17:02 18:38 20:52 22:24 01:27(+1)}}}
{\tnykdata{\anga{\tithi{13}{शुक्ल-त्रयोदशी}}{\time{40-44}{22:20}}\hspace{1ex}}%
{\anga{पूर्वप्रोष्ठपदा}{\time{57-41}{05:06(+1)}}\hspace{1ex}}{चन्द्रराशिः—\mbox{कुम्भः\RIGHTarrow\textsf{22:23}}}%
{\anga{वृद्धिः}{\time{53-26}{03:24(+1)}}\hspace{1ex}\uanga{ध्रुवः}}%
{\anga{कौलवः}{\time{7-42}{09:07}}\hspace{1ex}\anga{तैतिलः}{\time{40-44}{22:20}}\hspace{1ex}\uanga{गरः}}{}
}
{प्रदोष-व्रतम्}
{Fri} 
\cfoot{\rygdata{10:27--11:55}{14:52--16:20}{07:30--08:58}}
\caldata{OCTOBER}{12}{\sunmonth{कन्या}{26}{}{आश्वयुजः}{शरदृतुः}{शनिः}{विकारी}{दक्षिणायनम्}{वर्षऋतुः}}
{\sunmoonrsdata{06:02}{17:48}{16:56}{05:22(+1)}{11:55}
{\kalas{04:24 05:13 09:10 08:23 09:57 16:14 10:44 13:06 15:27 17:01 18:37 20:52 22:23 01:27(+1)}}}
{\tnykdata{\anga{\tithi{14}{शुक्ल-चतुर्दशी}}{\time{46-26}{00:36(+1)}}\hspace{1ex}}%
{\fullanga{उत्तरप्रोष्ठपदा}}{चन्द्रराशिः—\mbox{मीनः}}%
{\anga{ध्रुवः}{\time{55-12}{04:07(+1)}}\hspace{1ex}\uanga{व्याघातः}}%
{\anga{गरः}{\time{13-40}{11:30}}\hspace{1ex}\anga{वणिजः}{\time{46-26}{00:36(+1)}}\hspace{1ex}\uanga{विष्टिः}}{}
}
{\tamil{நடராஜர் மஹாபிஷேகம்}\eventsep \tamil{புரட்டாசி~சனிக்கிழமை}}
{Sat} 
\cfoot{\rygdata{08:58--10:27}{13:23--14:52}{06:02--07:30}}
\caldata{OCTOBER}{13}{\sunmonth{कन्या}{27}{}{आश्वयुजः}{शरदृतुः}{भानुः}{विकारी}{दक्षिणायनम्}{वर्षऋतुः}}
{\sunmoonrsdata{06:02}{17:48}{17:33}{---}{11:55}
{\kalas{04:24 05:13 09:10 08:23 09:57 16:13 10:44 13:05 15:26 17:00 18:37 20:51 22:23 01:27(+1)}}}
{\tnykdata{\anga{\tithi{15}{पौर्णमासी}}{\time{51-28}{02:37(+1)}}\hspace{1ex}}%
{\anga{उत्तरप्रोष्ठपदा}{\time{4-29}{07:50}}\hspace{1ex}}{चन्द्रराशिः—\mbox{मीनः}}%
{\anga{व्याघातः}{\time{56-26}{04:37(+1)}}\hspace{1ex}\uanga{हर्षणः}}%
{\anga{विष्टिः}{\time{19-2}{13:39}}\hspace{1ex}\anga{बवः}{\time{51-28}{02:37(+1)}}\hspace{1ex}\uanga{बालवः}}{}
}
{काञ्ची ३६ जगद्गुरु श्री-चित्सुखानन्देन्द्र सरस्वती आराधना~\#{१२६२}\eventsep को-जागर्ति-व्रतम्\eventsep कौमुदी-उत्सवः\eventsep कुमार-पूर्णिमा/महा-अश्विनी\eventsep कुन्ती-(पार्वती)-व्रतम्\eventsep लक्ष्मी-इन्द्र-कुबेर-पूजा\eventsep मीराबाई-जयन्ती~\#{५२२}\eventsep पार्वणव्रतम् पूर्णिमायाम्\eventsep पूर्णिमा-व्रतम्\eventsep वाल्मीकि-महर्षि-जयन्ती\eventsep वेङ्कटाचले पूर्णिमा-गरुड-सेवा\eventsep शरद्-पूर्णिमा}
{Sun} 
\cfoot{\rygdata{16:19--17:48}{11:55--13:23}{14:51--16:19}}
\caldata{OCTOBER}{14}{\sunmonth{कन्या}{28}{}{आश्वयुजः}{शरदृतुः}{सोमः}{विकारी}{दक्षिणायनम्}{वर्षऋतुः}}
{\sunmoonsrdata{06:02}{17:47}{18:12}{06:09}{11:54}
{\kalas{04:24 05:13 09:10 08:23 09:57 16:13 10:44 13:05 15:26 17:00 18:36 20:51 22:23 01:26(+1)}}}
{\tnykdata{\anga{\tithi{16}{कृष्ण-प्रथमा}}{\time{55-46}{04:21(+1)}}\hspace{1ex}}%
{\anga{रेवती}{\time{10-37}{10:17}}\hspace{1ex}}{चन्द्रराशिः—\mbox{मीनः\RIGHTarrow\textsf{10:17}}}%
{\anga{हर्षणः}{\time{57-7}{04:53(+1)}}\hspace{1ex}\uanga{वज्रम्}}%
{\anga{बालवः}{\time{23-43}{15:31}}\hspace{1ex}\anga{कौलवः}{\time{55-46}{04:21(+1)}}\hspace{1ex}\uanga{तैतिलः}}{}
}
{अप्पय्य-दीक्षित-जयन्ती~\#{५०१}\eventsep जयावाप्ति-व्रतम्\eventsep पूर्र्णमासेष्टिः\eventsep सप्तम-अपरपक्ष-आरम्भः\eventsep स्थालीपाकः}
{Mon} 
\cfoot{\rygdata{07:30--08:58}{10:26--11:54}{13:23--14:51}}
\caldata{OCTOBER}{15}{\sunmonth{कन्या}{29}{}{आश्वयुजः}{शरदृतुः}{मङ्गलः}{विकारी}{दक्षिणायनम्}{वर्षऋतुः}}
{\sunmoonsrdata{06:02}{17:46}{18:52}{06:57}{11:54}
{\kalas{04:24 05:13 09:10 08:23 09:57 16:12 10:44 13:05 15:25 16:59 18:35 20:50 22:22 01:26(+1)}}}
{\tnykdata{\anga{\tithi{17}{कृष्ण-द्वितीया}}{\time{59-16}{05:45(+1)}}\hspace{1ex}}%
{\anga{अश्विनी}{\time{16-2}{12:27}}\hspace{1ex}}{चन्द्रराशिः—\mbox{मेषः}}%
{\anga{वज्रम्}{\time{57-10}{04:55(+1)}}\hspace{1ex}\uanga{सिद्धिः}}%
{\anga{तैतिलः}{\time{27-37}{17:05}}\hspace{1ex}\anga{गरः}{\time{59-16}{05:45(+1)}}\hspace{1ex}\uanga{वणिजः}}{}
}
{अशून्यशयन-व्रतम्\eventsep भौमाश्विनी-पुण्यकालः\eventsep \tamil{ருத்ர~பஶுபதி நாயனார் (16) குருபூஜை}}
{Tue} 
\cfoot{\rygdata{14:50--16:18}{08:58--10:26}{11:54--13:22}}
\caldata{OCTOBER}{16}{\sunmonth{कन्या}{30}{}{आश्वयुजः}{शरदृतुः}{बुधः}{विकारी}{दक्षिणायनम्}{वर्षऋतुः}}
{\sunmoonsrdata{06:02}{17:46}{19:36}{07:47}{11:54}
{\kalas{04:24 05:13 09:10 08:23 09:57 16:12 10:44 13:04 15:25 16:59 18:35 20:50 22:22 01:26(+1)}}}
{\tnykdata{\fulltithi{\tithi{18}{कृष्ण-तृतीया}}}%
{\anga{अपभरणी}{\time{20-39}{14:18}}\hspace{1ex}}{चन्द्रराशिः—\mbox{मेषः\RIGHTarrow\textsf{20:43}}}%
{\anga{सिद्धिः}{\time{56-34}{04:40(+1)}}\hspace{1ex}\uanga{व्यतीपातः}}%
{\anga{वणिजः}{\time{30-41}{18:19}}\hspace{1ex}\uanga{विष्टिः}}{}
}
{चन्द्रोदय-गौरी-व्रतम्\eventsep कृत्तिका-व्रतम्\eventsep कनक-गणेश-व्रतम्\eventsep ललिता-गौरी-व्रतम्}
{Wed} 
\cfoot{\rygdata{11:54--13:22}{07:30--08:58}{10:26--11:54}}
\caldata{OCTOBER}{17}{\sunmonth{कन्या}{31}{\mbox{कन्या{\tiny\RIGHTarrow}\textsf{00:31(+1)}}}{आश्वयुजः}{शरदृतुः}{गुरुः}{विकारी}{दक्षिणायनम्}{वर्षऋतुः}}
{\sunmoonsrdata{06:03}{17:45}{20:23}{08:39}{11:54}
{\kalas{04:24 05:13 09:10 08:23 09:57 16:12 10:44 13:04 15:25 16:58 18:34 20:50 22:22 01:26(+1)}}}
{\tnykdata{\anga{\tithi{18}{कृष्ण-तृतीया}}{\time{1-53}{06:48}}\hspace{1ex}}%
{\anga{कृत्तिका}{\time{24-25}{15:49}}\hspace{1ex}}{चन्द्रराशिः—\mbox{वृषभः}}%
{\anga{व्यतीपातः}{\time{55-14}{04:08(+1)}}\hspace{1ex}\uanga{वरीयान्}}%
{\anga{विष्टिः}{\time{1-53}{06:48}}\hspace{1ex}\anga{बवः}{\time{32-51}{19:11}}\hspace{1ex}\uanga{बालवः}}{}
}
{करक-चतुर्थी\eventsep महाव्यतीपात-श्राद्धम्\eventsep तुला-सङ्क्रमण-पुण्यकालः~\textsf{20:31}{\RIGHTarrow}\textsf{04:31(+1)}\eventsep वक्रतुण्ड-महागणपति सङ्कटहर-चतुर्थी-व्रतम्}
{Thu} 
\cfoot{\rygdata{13:22--14:50}{06:03--07:30}{08:58--10:26}}
\caldata{OCTOBER}{18}{\sunmonth{तुला}{1}{}{आश्वयुजः}{शरदृतुः}{शुक्रः}{विकारी}{दक्षिणायनम्}{शरदृतुः}}
{\sunmoonsrdata{06:03}{17:45}{21:14}{09:33}{11:54}
{\kalas{04:24 05:13 09:10 08:23 09:57 16:11 10:43 13:04 15:24 16:58 18:34 20:49 22:21 01:26(+1)}}}
{\tnykdata{\anga{\tithi{19}{कृष्ण-चतुर्थी}}{\time{3-34}{07:28}}\hspace{1ex}}%
{\anga{रोहिणी}{\time{27-13}{16:56}}\hspace{1ex}}{चन्द्रराशिः—\mbox{वृषभः\RIGHTarrow\textsf{05:20(+1)}}}%
{\anga{वरीयान्}{\time{53-6}{03:17(+1)}}\hspace{1ex}\uanga{परिघः}}%
{\anga{बालवः}{\time{3-34}{07:28}}\hspace{1ex}\anga{कौलवः}{\time{34-1}{19:39}}\hspace{1ex}\uanga{तैतिलः}}{}
}
{आकाशदीप-आरम्भः\eventsep भृगुवार-सुब्रह्मण्य-व्रतम्\eventsep तुला-कावेरी-स्नान-आरम्भः}
{Fri} 
\cfoot{\rygdata{10:26--11:54}{14:49--16:17}{07:30--08:58}}
\caldata{OCTOBER}{19}{\sunmonth{तुला}{2}{}{आश्वयुजः}{शरदृतुः}{शनिः}{विकारी}{दक्षिणायनम्}{शरदृतुः}}
{\sunmoonsrdata{06:03}{17:44}{22:08}{10:28}{11:53}
{\kalas{04:24 05:14 09:10 08:23 09:57 16:11 10:43 13:04 15:24 16:57 18:33 20:49 22:21 01:26(+1)}}}
{\tnykdata{\anga{\tithi{20}{कृष्ण-पञ्चमी}}{\time{4-11}{07:43}}\hspace{1ex}}%
{\anga{मृगशीर्षम्}{\time{28-56}{17:37}}\hspace{1ex}}{चन्द्रराशिः—\mbox{मिथुनम्}}%
{\anga{परिघः}{\time{50-2}{02:04(+1)}}\hspace{1ex}\uanga{शिवः}}%
{\anga{तैतिलः}{\time{4-11}{07:43}}\hspace{1ex}\anga{गरः}{\time{34-4}{19:40}}\hspace{1ex}\uanga{वणिजः}}{}
}
{घोटक-पञ्चमी\eventsep सेङ्गालिपुरम् अनन्तराम-दीक्षित-आराधना~\#{५०}}
{Sat} 
\cfoot{\rygdata{08:58--10:26}{13:21--14:49}{06:03--07:30}}
\caldata{OCTOBER}{20}{\sunmonth{तुला}{3}{}{आश्वयुजः}{शरदृतुः}{भानुः}{विकारी}{दक्षिणायनम्}{शरदृतुः}}
{\sunmoonsrdata{06:03}{17:44}{23:05}{11:23}{11:53}
{\kalas{04:24 05:14 09:10 08:23 09:57 16:10 10:43 13:03 15:23 16:57 18:33 20:48 22:21 01:26(+1)}}}
{\tnykdata{\anga{\tithi{21}{कृष्ण-षष्ठी}}{\time{3-36}{07:30}}\hspace{1ex}}%
{\anga{आर्द्रा}{\time{29-26}{17:49}}\hspace{1ex}}{चन्द्रराशिः—\mbox{मिथुनम्}}%
{\anga{शिवः}{\time{45-57}{00:26(+1)}}\hspace{1ex}\uanga{सिद्धः}}%
{\anga{वणिजः}{\time{3-36}{07:30}}\hspace{1ex}\anga{विष्टिः}{\time{32-50}{19:11}}\hspace{1ex}\uanga{बवः}}{}
}
{}
{Sun} 
\cfoot{\rygdata{16:16--17:44}{11:53--13:21}{14:48--16:16}}
\caldata{OCTOBER}{21}{\sunmonth{तुला}{4}{}{आश्वयुजः}{शरदृतुः}{सोमः}{विकारी}{दक्षिणायनम्}{शरदृतुः}}
{\sunmoonsrdata{06:03}{17:43}{00:03(+1)}{12:17}{11:53}
{\kalas{04:24 05:14 09:10 08:23 09:56 16:10 10:43 13:03 15:23 16:56 18:32 20:48 22:21 01:26(+1)}}}
{\tnykdata{\anga{\tithi{22}{कृष्ण-सप्तमी}}{\time{1-42}{06:44}}\hspace{1ex}\anga{\tithi{23}{कृष्ण-अष्टमी}}{\time{58-24}{05:25(+1)}}\hspace{1ex}}%
{\anga{पुनर्वसुः}{\time{28-35}{17:29}}\hspace{1ex}}{चन्द्रराशिः—\mbox{मिथुनम्\RIGHTarrow\textsf{11:37}}}%
{\anga{सिद्धः}{\time{40-45}{22:21}}\hspace{1ex}\uanga{साध्यः}}%
{\anga{बवः}{\time{1-42}{06:44}}\hspace{1ex}\anga{बालवः}{\time{30-14}{18:09}}\hspace{1ex}\anga{कौलवः}{\time{58-24}{05:25(+1)}}\hspace{1ex}\uanga{तैतिलः}}{}
}
{जीमूतवाहन-पूजा\eventsep जीवपुत्रिकाष्टमी\eventsep कालाष्टमी\eventsep मङ्गल-व्रतम्\eventsep महालक्ष्मी-व्रतम्}
{Mon} 
\cfoot{\rygdata{07:31--08:58}{10:26--11:53}{13:21--14:48}}
\caldata{OCTOBER}{22}{\sunmonth{तुला}{5}{}{आश्वयुजः}{शरदृतुः}{मङ्गलः}{विकारी}{दक्षिणायनम्}{शरदृतुः}}
{\sunmoonsrdata{06:03}{17:43}{01:02(+1)}{13:09}{11:53}
{\kalas{04:25 05:14 09:10 08:23 09:56 16:09 10:43 13:03 15:23 16:56 18:32 20:48 22:20 01:26(+1)}}}
{\tnykdata{\anga{\tithi{24}{कृष्ण-नवमी}}{\time{53-43}{03:32(+1)}}\hspace{1ex}}%
{\anga{पुष्यः}{\time{26-21}{16:36}}\hspace{1ex}}{चन्द्रराशिः—\mbox{कटकः}}%
{\anga{साध्यः}{\time{34-27}{19:50}}\hspace{1ex}\uanga{शुभः}}%
{\anga{तैतिलः}{\time{26-13}{16:33}}\hspace{1ex}\anga{गरः}{\time{53-43}{03:32(+1)}}\hspace{1ex}\uanga{वणिजः}}{}
}
{भीमसेन-जयन्ती}
{Tue} 
\cfoot{\rygdata{14:48--16:15}{08:58--10:25}{11:53--13:20}}
\caldata{OCTOBER}{23}{\sunmonth{तुला}{6}{}{आश्वयुजः}{शरदृतुः}{बुधः}{विकारी}{दक्षिणायनम्}{शरदृतुः}}
{\sunmoonsrdata{06:03}{17:42}{02:01(+1)}{13:58}{11:53}
{\kalas{04:25 05:14 09:10 08:23 09:56 16:09 10:43 13:03 15:22 16:55 18:31 20:47 22:20 01:26(+1)}}}
{\tnykdata{\anga{\tithi{25}{कृष्ण-दशमी}}{\time{47-43}{01:09(+1)}}\hspace{1ex}}%
{\anga{आश्रेषा}{\time{22-46}{15:10}}\hspace{1ex}}{चन्द्रराशिः—\mbox{कटकः\RIGHTarrow\textsf{15:10}}}%
{\anga{शुभः}{\time{27-2}{16:52}}\hspace{1ex}\uanga{शुक्लः}}%
{\anga{वणिजः}{\time{20-52}{14:24}}\hspace{1ex}\anga{विष्टिः}{\time{47-43}{01:09(+1)}}\hspace{1ex}\uanga{बवः}}{}
}
{इष-मासः{\RIGHTarrow}\textsf{22:49}\eventsep विष्णुपदी-पुण्यकालः~\textsf{16:25}{\RIGHTarrow}\textsf{05:13(+1)}}
{Wed} 
\cfoot{\rygdata{11:53--13:20}{07:31--08:58}{10:25--11:53}}
\caldata{OCTOBER}{24}{\sunmonth{तुला}{7}{}{आश्वयुजः}{शरदृतुः}{गुरुः}{विकारी}{दक्षिणायनम्}{शरदृतुः}}
{\sunmoonsrdata{06:04}{17:42}{02:59(+1)}{14:46}{11:53}
{\kalas{04:25 05:14 09:10 08:23 09:56 16:08 10:43 13:02 15:22 16:55 18:31 20:47 22:20 01:25(+1)}}}
{\tnykdata{\anga{\tithi{26}{कृष्ण-एकादशी}}{\time{40-36}{22:18}}\hspace{1ex}}%
{\anga{मघा}{\time{17-59}{13:15}}\hspace{1ex}}{चन्द्रराशिः—\mbox{सिंहः}}%
{\anga{शुक्लः}{\time{18-39}{13:32}}\hspace{1ex}\uanga{ब्रह्म}}%
{\anga{बवः}{\time{14-16}{11:46}}\hspace{1ex}\anga{बालवः}{\time{40-36}{22:18}}\hspace{1ex}\uanga{कौलवः}}{}
}
{सर्व-रमा-एकादशी\eventsep ताम्रपर्णी-अन्त्य-पुष्कर-आरम्भः\eventsep शृङ्गेरी ३४ जगद्गुरु श्री-चन्द्रशेखर भारती-३ जयन्ती}
{Thu} 
\cfoot{\rygdata{13:20--14:47}{06:04--07:31}{08:58--10:25}}
\caldata{OCTOBER}{25}{\sunmonth{तुला}{8}{}{आश्वयुजः}{शरदृतुः}{शुक्रः}{विकारी}{दक्षिणायनम्}{शरदृतुः}}
{\sunmoonsrdata{06:04}{17:41}{03:58(+1)}{15:33}{11:52}
{\kalas{04:25 05:14 09:10 08:23 09:56 16:08 10:43 13:02 15:22 16:55 18:31 20:47 22:20 01:25(+1)}}}
{\tnykdata{\anga{\tithi{27}{कृष्ण-द्वादशी}}{\time{32-40}{19:08}}\hspace{1ex}}%
{\anga{पूर्वफल्गुनी}{\time{12-14}{10:58}}\hspace{1ex}}{चन्द्रराशिः—\mbox{सिंहः\RIGHTarrow\textsf{16:20}}}%
{\anga{ब्रह्म}{\time{9-29}{09:51}}\hspace{1ex}\anga{इन्द्रः}{\time{59-47}{05:59(+1)}}\hspace{1ex}\uanga{वैधृतिः}}%
{\anga{कौलवः}{\time{6-43}{08:45}}\hspace{1ex}\anga{तैतिलः}{\time{32-40}{19:08}}\hspace{1ex}\anga{गरः}{\time{58-30}{05:28(+1)}}\hspace{1ex}\uanga{वणिजः}}{}
}
{\tamil{சத்தி நாயனார் (44) குருபூஜை}\eventsep गोवत्स-द्वादशी\eventsep हरिवासरः{\RIGHTarrow}\textsf{03:32}\eventsep प्रदोष-व्रतम्\eventsep वसुदेव-पूजा\eventsep व्याघ्र-द्वादशी}
{Fri} 
\cfoot{\rygdata{10:25--11:52}{14:47--16:14}{07:31--08:58}}
\caldata{OCTOBER}{26}{\sunmonth{तुला}{9}{}{आश्वयुजः}{शरदृतुः}{शनिः}{विकारी}{दक्षिणायनम्}{शरदृतुः}}
{\sunmoonsrdata{06:04}{17:41}{04:57(+1)}{16:20}{11:52}
{\kalas{04:25 05:15 09:10 08:23 09:56 16:08 10:43 13:02 15:21 16:54 18:30 20:47 22:20 01:25(+1)}}}
{\tnykdata{\anga{\tithi{28}{कृष्ण-त्रयोदशी}}{\time{24-15}{15:46}}\hspace{1ex}}%
{\anga{उत्तरफल्गुनी}{\time{5-51}{08:25}}\hspace{1ex}\anga{हस्तः}{\time{59-16}{05:47(+1)}}\hspace{1ex}}{चन्द्रराशिः—\mbox{कन्या}}%
{\anga{वैधृतिः}{\time{49-51}{02:01(+1)}}\hspace{1ex}\uanga{विष्कम्भः}}%
{\anga{वणिजः}{\time{24-15}{15:46}}\hspace{1ex}\anga{विष्टिः}{\time{49-59}{02:04(+1)}}\hspace{1ex}\uanga{शकुनिः}}{}
}
{(यम)-दीप-त्रयोदशी\eventsep धन-त्रयोदशी\eventsep धन्वन्तरि-जयन्ती\eventsep गो-त्रिरात्रि-व्रतम्\eventsep मासशिवरात्रिः\eventsep वैधृति-श्राद्धम्}
{Sat} 
\cfoot{\rygdata{08:58--10:25}{13:19--14:47}{06:04--07:31}}
\caldata{OCTOBER}{27}{\sunmonth{तुला}{10}{}{आश्वयुजः}{शरदृतुः}{भानुः}{विकारी}{दक्षिणायनम्}{शरदृतुः}}
{\sunmoonsrdata{06:04}{17:40}{05:57(+1)}{17:08}{11:52}
{\kalas{04:25 05:15 09:10 08:24 09:56 16:07 10:43 13:02 15:21 16:54 18:30 20:46 22:19 01:25(+1)}}}
{\tnykdata{\anga{\tithi{29}{कृष्ण-चतुर्दशी}}{\time{15-46}{12:23}}\hspace{1ex}}%
{\anga{चित्रा}{\time{52-54}{03:14(+1)}}\hspace{1ex}}{चन्द्रराशिः—\mbox{कन्या\RIGHTarrow\textsf{16:29}}}%
{\anga{विष्कम्भः}{\time{40-3}{22:06}}\hspace{1ex}\uanga{प्रीतिः}}%
{\anga{शकुनिः}{\time{15-46}{12:23}}\hspace{1ex}\anga{चतुष्पात्}{\time{41-38}{22:44}}\hspace{1ex}\uanga{नाग}}{}
}
{आश्वयुज-अमावास्या\eventsep दीपावली/लक्ष्मी-कुबेर-पूजा\eventsep दीपोत्सव-चतुर्दशी/यम-तर्पणम्\eventsep नरक-चतुर्दशी\eventsep पार्वणव्रतम् अमावास्यायाम्\eventsep प्रेत-चतुर्दशी\eventsep सप्तम-अपरपक्ष-समापनम्\eventsep शृङ्गेरी ३५ जगद्गुरु श्री-अभिनव विद्यातीर्थ महास्वामी जयन्ती}
{Sun} 
\cfoot{\rygdata{16:13--17:40}{11:52--13:19}{14:46--16:13}}
\caldata{OCTOBER}{28}{\sunmonth{तुला}{11}{}{आश्वयुजः}{शरदृतुः}{सोमः}{विकारी}{दक्षिणायनम्}{शरदृतुः}}
{\sunmoonsrdata{06:05}{17:40}{---}{17:58}{11:52}
{\kalas{04:25 05:15 09:10 08:24 09:56 16:07 10:43 13:02 15:21 16:54 18:29 20:46 22:19 01:25(+1)}}}
{\tnykdata{\anga{\tithi{30}{अमावास्या}}{\time{7-38}{09:08}}\hspace{1ex}}%
{\anga{स्वाती}{\time{47-12}{00:58(+1)}}\hspace{1ex}}{चन्द्रराशिः—\mbox{तुला}}%
{\anga{प्रीतिः}{\time{30-42}{18:22}}\hspace{1ex}\uanga{आयुष्मान्}}%
{\anga{नाग}{\time{7-38}{09:08}}\hspace{1ex}\anga{किंस्तुघ्नः}{\time{33-52}{19:37}}\hspace{1ex}\uanga{बवः}}{}
}
{आग्रयण-होमः द्राविडेषु\eventsep दर्शेष्टिः\eventsep गोवर्धन-पूजा\eventsep केदार-गौरी-व्रतम्\eventsep कमला-जयन्ती\eventsep सोमवती अमावास्या\eventsep स्थालीपाकः\eventsep विक्रमादित्य-पट्टाभिषेकः\eventsep श्रीराम-पट्टाभिषेकः}
{Mon} 
\cfoot{\rygdata{07:32--08:58}{10:25--11:52}{13:19--14:46}}
\caldata{OCTOBER}{29}{\sunmonth{तुला}{12}{}{कार्त्तिकः}{शरदृतुः}{मङ्गलः}{विकारी}{दक्षिणायनम्}{शरदृतुः}}
{\sunmoonrsdata{06:05}{17:39}{06:58}{18:50}{11:52}
{\kalas{04:25 05:15 09:10 08:24 09:56 16:07 10:43 13:02 15:21 16:53 18:29 20:46 22:19 01:25(+1)}}}
{\tnykdata{\anga{\tithi{1}{शुक्ल-प्रथमा}}{\time{0-20}{06:13}}\hspace{1ex}\anga{\tithi{2}{शुक्ल-द्वितीया}}{\time{54-16}{03:47(+1)}}\hspace{1ex}}%
{\anga{विशाखा}{\time{42-39}{23:09}}\hspace{1ex}}{चन्द्रराशिः—\mbox{तुला\RIGHTarrow\textsf{17:33}}}%
{\anga{आयुष्मान्}{\time{22-11}{14:57}}\hspace{1ex}\uanga{सौभाग्यः}}%
{\anga{बवः}{\time{0-20}{06:13}}\hspace{1ex}\anga{बालवः}{\time{27-7}{16:56}}\hspace{1ex}\anga{कौलवः}{\time{54-16}{03:47(+1)}}\hspace{1ex}\uanga{तैतिलः}}{}
}
{चन्द्र-दर्शनम्\eventsep यम/भ्रातृ-द्वितीया}
{Tue} 
\cfoot{\rygdata{14:46--16:13}{08:58--10:25}{11:52--13:19}}
\caldata{OCTOBER}{30}{\sunmonth{तुला}{13}{}{कार्त्तिकः}{शरदृतुः}{बुधः}{विकारी}{दक्षिणायनम्}{शरदृतुः}}
{\sunmoonrsdata{06:05}{17:39}{07:58}{19:44}{11:52}
{\kalas{04:26 05:15 09:10 08:24 09:56 16:06 10:43 13:01 15:20 16:53 18:29 20:46 22:19 01:25(+1)}}}
{\tnykdata{\anga{\tithi{3}{शुक्ल-तृतीया}}{\time{49-50}{02:01(+1)}}\hspace{1ex}}%
{\anga{अनूराधा}{\time{39-37}{21:56}}\hspace{1ex}}{चन्द्रराशिः—\mbox{वृश्चिकः}}%
{\anga{सौभाग्यः}{\time{14-48}{12:00}}\hspace{1ex}\uanga{शोभनः}}%
{\anga{तैतिलः}{\time{21-49}{14:49}}\hspace{1ex}\anga{गरः}{\time{49-50}{02:01(+1)}}\hspace{1ex}\uanga{वणिजः}}{}
}
{बुधानुराधा-पुण्यकालः\eventsep \tamil{பூசலார் நாயனார் (56) குருபூஜை}}
{Wed} 
\cfoot{\rygdata{11:52--13:19}{07:32--08:58}{10:25--11:52}}
\caldata{OCTOBER}{31}{\sunmonth{तुला}{14}{}{कार्त्तिकः}{शरदृतुः}{गुरुः}{विकारी}{दक्षिणायनम्}{शरदृतुः}}
{\sunmoonrsdata{06:05}{17:39}{08:58}{20:39}{11:52}
{\kalas{04:26 05:16 09:10 08:24 09:57 16:06 10:43 13:01 15:20 16:52 18:28 20:45 22:19 01:25(+1)}}}
{\tnykdata{\anga{\tithi{4}{शुक्ल-चतुर्थी}}{\time{47-18}{01:01(+1)}}\hspace{1ex}}%
{\anga{ज्येष्ठा}{\time{38-27}{21:28}}\hspace{1ex}}{चन्द्रराशिः—\mbox{वृश्चिकः\RIGHTarrow\textsf{21:28}}}%
{\anga{शोभनः}{\time{8-49}{09:37}}\hspace{1ex}\uanga{अतिगण्डः}}%
{\anga{वणिजः}{\time{18-18}{13:25}}\hspace{1ex}\anga{विष्टिः}{\time{47-18}{01:01(+1)}}\hspace{1ex}\uanga{बवः}}{}
}
{}
{Thu} 
\cfoot{\rygdata{13:19--14:45}{06:05--07:32}{08:59--10:25}}
\caldata{NOVEMBER}{1}{\sunmonth{तुला}{15}{}{कार्त्तिकः}{शरदृतुः}{शुक्रः}{विकारी}{दक्षिणायनम्}{शरदृतुः}}
{\sunmoonrsdata{06:06}{17:38}{09:55}{21:35}{11:52}
{\kalas{04:26 05:16 09:10 08:24 09:57 16:06 10:43 13:01 15:20 16:52 18:28 20:45 22:19 01:26(+1)}}}
{\tnykdata{\anga{\tithi{5}{शुक्ल-पञ्चमी}}{\time{46-53}{00:51(+1)}}\hspace{1ex}}%
{\anga{मूला}{\time{39-18}{21:49}}\hspace{1ex}}{चन्द्रराशिः—\mbox{धनुः}}%
{\anga{अतिगण्डः}{\time{4-25}{07:52}}\hspace{1ex}\uanga{सुकर्म}}%
{\anga{बवः}{\time{16-49}{12:49}}\hspace{1ex}\anga{बालवः}{\time{46-53}{00:51(+1)}}\hspace{1ex}\uanga{कौलवः}}{}
}
{\tamil{ஐயடிகள் காடவர்கோன் நாயனார் (45) குருபூஜை}\eventsep देवसेना-पञ्चमी\eventsep पाण्डव-(लाभ)-पञ्चमी\eventsep सर्प-पूजा}
{Fri} 
\cfoot{\rygdata{10:25--11:52}{14:45--16:12}{07:32--08:59}}
\caldata{NOVEMBER}{2}{\sunmonth{तुला}{16}{}{कार्त्तिकः}{शरदृतुः}{शनिः}{विकारी}{दक्षिणायनम्}{शरदृतुः}}
{\sunmoonrsdata{06:06}{17:38}{10:48}{22:29}{11:52}
{\kalas{04:26 05:16 09:10 08:24 09:57 16:06 10:43 13:01 15:20 16:52 18:28 20:45 22:19 01:26(+1)}}}
{\tnykdata{\anga{\tithi{6}{शुक्ल-षष्ठी}}{\time{48-31}{01:31(+1)}}\hspace{1ex}}%
{\anga{पूर्वाषाढा}{\time{42-10}{22:58}}\hspace{1ex}}{चन्द्रराशिः—\mbox{धनुः\RIGHTarrow\textsf{05:23(+1)}}}%
{\anga{सुकर्म}{\time{1-39}{06:46}}\hspace{1ex}\uanga{धृतिः}}%
{\anga{कौलवः}{\time{17-26}{13:05}}\hspace{1ex}\anga{तैतिलः}{\time{48-31}{01:31(+1)}}\hspace{1ex}\uanga{गरः}}{}
}
{स्कन्दषष्ठी-व्रतम्}
{Sat} 
\cfoot{\rygdata{08:59--10:25}{13:18--14:45}{06:06--07:32}}
\caldata{NOVEMBER}{3}{\sunmonth{तुला}{17}{}{कार्त्तिकः}{शरदृतुः}{भानुः}{विकारी}{दक्षिणायनम्}{शरदृतुः}}
{\sunmoonrsdata{06:06}{17:38}{11:37}{23:21}{11:52}
{\kalas{04:26 05:16 09:11 08:25 09:57 16:05 10:43 13:01 15:19 16:52 18:28 20:45 22:19 01:26(+1)}}}
{\tnykdata{\anga{\tithi{7}{शुक्ल-सप्तमी}}{\time{52-3}{02:56(+1)}}\hspace{1ex}}%
{\anga{उत्तराषाढा}{\time{46-53}{00:52(+1)}}\hspace{1ex}}{चन्द्रराशिः—\mbox{मकरः}}%
{\anga{धृतिः}{\time{0-29}{06:18}}\hspace{1ex}\uanga{शूलः}}%
{\anga{गरः}{\time{20-4}{14:08}}\hspace{1ex}\anga{वणिजः}{\time{52-3}{02:56(+1)}}\hspace{1ex}\uanga{विष्टिः}}{}
}
{सावित्र्य-कल्पादिः\eventsep विजया-भानुसप्तमी}
{Sun} 
\cfoot{\rygdata{16:11--17:38}{11:52--13:18}{14:45--16:11}}
\caldata{NOVEMBER}{4}{\sunmonth{तुला}{18}{}{कार्त्तिकः}{शरदृतुः}{सोमः}{विकारी}{दक्षिणायनम्}{शरदृतुः}}
{\sunmoonrsdata{06:07}{17:37}{12:22}{00:12(+1)}{11:52}
{\kalas{04:27 05:17 09:11 08:25 09:57 16:05 10:43 13:01 15:19 16:51 18:27 20:45 22:18 01:26(+1)}}}
{\tnykdata{\anga{\tithi{8}{शुक्ल-अष्टमी}}{\time{57-5}{04:57(+1)}}\hspace{1ex}}%
{\anga{श्रवणः}{\time{53-4}{03:20(+1)}}\hspace{1ex}}{चन्द्रराशिः—\mbox{मकरः}}%
{\anga{शूलः}{\time{0-41}{06:23}}\hspace{1ex}\uanga{गण्डः}}%
{\anga{विष्टिः}{\time{24-24}{15:52}}\hspace{1ex}\anga{बवः}{\time{57-5}{04:57(+1)}}\hspace{1ex}\uanga{बालवः}}{}
}
{गोपाष्टमी\eventsep कार्तवीर्यार्जुन-जयन्ती\eventsep कृत्तिका-सोमवासरः\eventsep \tamil{பொய்கையாழ்வார் திருநக்ஷத்திரம்}\eventsep सोमश्रावणी-पुण्यकालः\eventsep ताम्रपर्णी-अन्त्य-पुष्कर-समापनम्\eventsep श्रवण-व्रतम्}
{Mon} 
\cfoot{\rygdata{07:33--08:59}{10:26--11:52}{13:18--14:45}}
\caldata{NOVEMBER}{5}{\sunmonth{तुला}{19}{}{कार्त्तिकः}{शरदृतुः}{मङ्गलः}{विकारी}{दक्षिणायनम्}{शरदृतुः}}
{\sunmoonrsdata{06:07}{17:37}{13:03}{01:00(+1)}{11:52}
{\kalas{04:27 05:17 09:11 08:25 09:57 16:05 10:43 13:01 15:19 16:51 18:27 20:45 22:18 01:26(+1)}}}
{\tnykdata{\fulltithi{\tithi{9}{शुक्ल-नवमी}}}%
{\fullanga{श्रविष्ठा}}{चन्द्रराशिः—\mbox{मकरः\RIGHTarrow\textsf{16:44}}}%
{\anga{गण्डः}{\time{1-59}{06:55}}\hspace{1ex}\uanga{वृद्धिः}}%
{\anga{बालवः}{\time{30-0}{18:07}}\hspace{1ex}\uanga{कौलवः}}{}
}
{अक्षया-नवमी\eventsep \tamil{பூதத்தாழ்வார் திருநக்ஷத்திரம்}\eventsep गुरु-सङ्क्रान्तिः~(वृश्चिकः\To{}धनुः)\eventsep जगद्धात्री-पूजा\eventsep काञ्ची २२ जगद्गुरु श्री-परिपूर्णबोधेन्द्र सरस्वती आराधना~\#{१५३९}\eventsep सिन्धु-आदि-पुष्कर-आरम्भः\eventsep त्रेतायुगादिः}
{Tue} 
\cfoot{\rygdata{14:45--16:11}{08:59--10:26}{11:52--13:18}}
\caldata{NOVEMBER}{6}{\sunmonth{तुला}{20}{}{कार्त्तिकः}{शरदृतुः}{बुधः}{विकारी}{दक्षिणायनम्}{शरदृतुः}}
{\sunmoonrsdata{06:07}{17:37}{13:42}{01:46(+1)}{11:52}
{\kalas{04:27 05:17 09:11 08:25 09:57 16:05 10:43 13:01 15:19 16:51 18:27 20:45 22:18 01:26(+1)}}}
{\tnykdata{\anga{\tithi{9}{शुक्ल-नवमी}}{\time{3-5}{07:21}}\hspace{1ex}}%
{\anga{श्रविष्ठा}{\time{0-11}{06:12}}\hspace{1ex}}{चन्द्रराशिः—\mbox{कुम्भः}}%
{\anga{वृद्धिः}{\time{3-57}{07:42}}\hspace{1ex}\uanga{ध्रुवः}}%
{\anga{कौलवः}{\time{3-5}{07:21}}\hspace{1ex}\anga{तैतिलः}{\time{36-16}{20:38}}\hspace{1ex}\uanga{गरः}}{}
}
{}
{Wed} 
\cfoot{\rygdata{11:52--13:18}{07:33--09:00}{10:26--11:52}}
\caldata{NOVEMBER}{7}{\sunmonth{तुला}{21}{}{कार्त्तिकः}{शरदृतुः}{गुरुः}{विकारी}{दक्षिणायनम्}{शरदृतुः}}
{\sunmoonrsdata{06:08}{17:37}{14:19}{02:32(+1)}{11:52}
{\kalas{04:27 05:17 09:11 08:25 09:57 16:05 10:43 13:01 15:19 16:51 18:27 20:44 22:18 01:26(+1)}}}
{\tnykdata{\anga{\tithi{10}{शुक्ल-दशमी}}{\time{9-27}{09:55}}\hspace{1ex}}%
{\anga{शतभिषक्}{\time{7-41}{09:12}}\hspace{1ex}}{चन्द्रराशिः—\mbox{कुम्भः\RIGHTarrow\textsf{05:26(+1)}}}%
{\anga{ध्रुवः}{\time{6-11}{08:36}}\hspace{1ex}\uanga{व्याघातः}}%
{\anga{गरः}{\time{9-27}{09:55}}\hspace{1ex}\anga{वणिजः}{\time{42-37}{23:11}}\hspace{1ex}\uanga{विष्टिः}}{}
}
{कंस-वधः\eventsep \tamil{பேயாழ்வார் திருநக்ஷத்திரம்}}
{Thu} 
\cfoot{\rygdata{13:18--14:44}{06:08--07:34}{09:00--10:26}}
\caldata{NOVEMBER}{8}{\sunmonth{तुला}{22}{}{कार्त्तिकः}{शरदृतुः}{शुक्रः}{विकारी}{दक्षिणायनम्}{शरदृतुः}}
{\sunmoonrsdata{06:08}{17:36}{14:55}{03:17(+1)}{11:52}
{\kalas{04:28 05:18 09:11 08:26 09:57 16:05 10:43 13:01 15:19 16:50 18:26 20:44 22:18 01:26(+1)}}}
{\tnykdata{\anga{\tithi{11}{शुक्ल-एकादशी}}{\time{15-40}{12:24}}\hspace{1ex}}%
{\anga{पूर्वप्रोष्ठपदा}{\time{15-3}{12:09}}\hspace{1ex}}{चन्द्रराशिः—\mbox{मीनः}}%
{\anga{व्याघातः}{\time{8-20}{09:28}}\hspace{1ex}\uanga{हर्षणः}}%
{\anga{विष्टिः}{\time{15-40}{12:24}}\hspace{1ex}\anga{बवः}{\time{48-35}{01:34(+1)}}\hspace{1ex}\uanga{बालवः}}{}
}
{आदि-शङ्कर मानसिक-सन्न्यास-दिनम्\eventsep भीष्म-पञ्चक-व्रत-आरम्भः\eventsep हरिवासरः{\RIGHTarrow}\textsf{19:00}\eventsep मन्वादिः-(स्वारोचिषः-[२])\eventsep सर्व-उत्थान-एकादशी\eventsep तुलसी-विवाहः}
{Fri} 
\cfoot{\rygdata{10:26--11:52}{14:44--16:10}{07:34--09:00}}
\caldata{NOVEMBER}{9}{\sunmonth{तुला}{23}{}{कार्त्तिकः}{शरदृतुः}{शनिः}{विकारी}{दक्षिणायनम्}{शरदृतुः}}
{\sunmoonrsdata{06:08}{17:36}{15:32}{04:04(+1)}{11:52}
{\kalas{04:28 05:18 09:12 08:26 09:58 16:04 10:43 13:01 15:19 16:50 18:26 20:44 22:18 01:26(+1)}}}
{\tnykdata{\anga{\tithi{12}{शुक्ल-द्वादशी}}{\time{21-17}{14:39}}\hspace{1ex}}%
{\anga{उत्तरप्रोष्ठपदा}{\time{21-50}{14:53}}\hspace{1ex}}{चन्द्रराशिः—\mbox{मीनः}}%
{\anga{हर्षणः}{\time{10-2}{10:09}}\hspace{1ex}\uanga{वज्रम्}}%
{\anga{बालवः}{\time{21-17}{14:39}}\hspace{1ex}\anga{कौलवः}{\time{53-46}{03:39(+1)}}\hspace{1ex}\uanga{तैतिलः}}{}
}
{बृन्दावन-द्वादशी\eventsep द्विदल-व्रत-समापनम्\eventsep गोपद्म-व्रत-समापनम्\eventsep प्रबोधोत्सवः\eventsep याज्ञवल्क्य-जयन्ती\eventsep शनि-प्रदोष-व्रतम्}
{Sat} 
\cfoot{\rygdata{09:00--10:26}{13:18--14:44}{06:08--07:34}}
\caldata{NOVEMBER}{10}{\sunmonth{तुला}{24}{}{कार्त्तिकः}{शरदृतुः}{भानुः}{विकारी}{दक्षिणायनम्}{शरदृतुः}}
{\sunmoonrsdata{06:09}{17:36}{16:10}{04:52(+1)}{11:52}
{\kalas{04:28 05:18 09:12 08:26 09:58 16:04 10:43 13:01 15:18 16:50 18:26 20:44 22:18 01:27(+1)}}}
{\tnykdata{\anga{\tithi{13}{शुक्ल-त्रयोदशी}}{\time{26-0}{16:33}}\hspace{1ex}}%
{\anga{रेवती}{\time{27-47}{17:15}}\hspace{1ex}}{चन्द्रराशिः—\mbox{मीनः\RIGHTarrow\textsf{17:15}}}%
{\anga{वज्रम्}{\time{11-7}{10:35}}\hspace{1ex}\uanga{सिद्धिः}}%
{\anga{तैतिलः}{\time{26-0}{16:33}}\hspace{1ex}\anga{गरः}{\time{57-59}{05:20(+1)}}\hspace{1ex}\uanga{वणिजः}}{}
}
{कार्त्तिक-मास-अन्तिमत्रयतिथि-व्रत-आरम्भः}
{Sun} 
\cfoot{\rygdata{16:10--17:36}{11:52--13:18}{14:44--16:10}}
\caldata{NOVEMBER}{11}{\sunmonth{तुला}{25}{}{कार्त्तिकः}{शरदृतुः}{सोमः}{विकारी}{दक्षिणायनम्}{शरदृतुः}}
{\sunmoonrsdata{06:09}{17:36}{16:50}{05:42(+1)}{11:52}
{\kalas{04:29 05:19 09:12 08:26 09:58 16:04 10:44 13:01 15:18 16:50 18:26 20:44 22:18 01:27(+1)}}}
{\tnykdata{\anga{\tithi{14}{शुक्ल-चतुर्दशी}}{\time{29-40}{18:01}}\hspace{1ex}}%
{\anga{अश्विनी}{\time{32-43}{19:14}}\hspace{1ex}}{चन्द्रराशिः—\mbox{मेषः}}%
{\anga{सिद्धिः}{\time{11-24}{10:43}}\hspace{1ex}\uanga{व्यतीपातः}}%
{\anga{वणिजः}{\time{29-40}{18:01}}\hspace{1ex}\uanga{विष्टिः}}{}
}
{कृत्तिका-सोमवासरः\eventsep मणिकर्णिका-स्नानम्/वैकुण्ठ-चतुर्दशी\eventsep \tamil{திருமூல நாயனார் (29) குருபூஜை}\eventsep त्रिपुरोत्सवः\eventsep व्यतीपात-श्राद्धम्}
{Mon} 
\cfoot{\rygdata{07:35--09:01}{10:26--11:52}{13:18--14:44}}
\caldata{NOVEMBER}{12}{\sunmonth{तुला}{26}{}{कार्त्तिकः}{शरदृतुः}{मङ्गलः}{विकारी}{दक्षिणायनम्}{शरदृतुः}}
{\sunmoonrsdata{06:09}{17:36}{17:33}{---}{11:53}
{\kalas{04:29 05:19 09:12 08:27 09:58 16:04 10:44 13:01 15:18 16:50 18:26 20:44 22:18 01:27(+1)}}}
{\tnykdata{\anga{\tithi{15}{पौर्णमासी}}{\time{32-16}{19:04}}\hspace{1ex}}%
{\anga{अपभरणी}{\time{36-37}{20:48}}\hspace{1ex}}{चन्द्रराशिः—\mbox{मेषः\RIGHTarrow\textsf{03:08(+1)}}}%
{\anga{व्यतीपातः}{\time{10-53}{10:31}}\hspace{1ex}\uanga{वरीयान्}}%
{\anga{विष्टिः}{\time{1-6}{06:36}}\hspace{1ex}\anga{बवः}{\time{32-16}{19:04}}\hspace{1ex}\uanga{बालवः}}{}
}
{आग्रयण-होमः द्राविडेषु\eventsep भीष्म-पञ्चक-व्रत-समापनम्\eventsep चातुर्मास्यव्रत-समापनम्\eventsep कार्त्तिक-मास-अन्तिमत्रयतिथि-व्रत-समापनम्\eventsep कार्त्तिक-पूर्णिमा-स्नानम्\eventsep महा-अन्नाभिषेकः\eventsep मन्वादिः-(धर्मः-[११])\eventsep \tamil{நின்றசீர் நெடுமாற நாயனார் (48) குருபூஜை}\eventsep पार्वणव्रतम् पूर्णिमायाम्\eventsep पूर्णिमा-व्रतम्\eventsep वेङ्कटाचले पूर्णिमा-गरुड-सेवा\eventsep श्री-गोविन्द भगवत्पाद आराधना}
{Tue} 
\cfoot{\rygdata{14:44--16:10}{09:01--10:27}{11:52--13:18}}
\caldata{NOVEMBER}{13}{\sunmonth{तुला}{27}{}{कार्त्तिकः}{शरदृतुः}{बुधः}{विकारी}{दक्षिणायनम्}{शरदृतुः}}
{\sunmoonsrdata{06:10}{17:35}{18:19}{06:34}{11:53}
{\kalas{04:29 05:19 09:13 08:27 09:58 16:04 10:44 13:01 15:18 16:50 18:26 20:44 22:18 01:27(+1)}}}
{\tnykdata{\anga{\tithi{16}{कृष्ण-प्रथमा}}{\time{33-48}{19:41}}\hspace{1ex}}%
{\anga{कृत्तिका}{\time{39-30}{21:58}}\hspace{1ex}}{चन्द्रराशिः—\mbox{वृषभः}}%
{\anga{वरीयान्}{\time{9-32}{09:59}}\hspace{1ex}\uanga{परिघः}}%
{\anga{बालवः}{\time{3-9}{07:26}}\hspace{1ex}\anga{कौलवः}{\time{33-48}{19:41}}\hspace{1ex}\uanga{तैतिलः}}{}
}
{\tamil{இடங்கழி நாயனார் (52) குருபூஜை}\eventsep काञ्ची ६४ जगद्गुरु श्री-चन्द्रशेखरेन्द्र सरस्वती ५ आराधना~\#{१६९}\eventsep कृत्तिका-व्रतम्\eventsep नवम-अपरपक्ष-आरम्भः\eventsep पूर्र्णमासेष्टिः\eventsep स्थालीपाकः}
{Wed} 
\cfoot{\rygdata{11:53--13:18}{07:35--09:01}{10:27--11:53}}
\caldata{NOVEMBER}{14}{\sunmonth{तुला}{28}{}{कार्त्तिकः}{शरदृतुः}{गुरुः}{विकारी}{दक्षिणायनम्}{शरदृतुः}}
{\sunmoonsrdata{06:10}{17:35}{19:10}{07:28}{11:53}
{\kalas{04:30 05:20 09:13 08:27 09:59 16:04 10:44 13:01 15:18 16:50 18:26 20:44 22:19 01:27(+1)}}}
{\tnykdata{\anga{\tithi{17}{कृष्ण-द्वितीया}}{\time{34-21}{19:55}}\hspace{1ex}}%
{\anga{रोहिणी}{\time{41-25}{22:44}}\hspace{1ex}}{चन्द्रराशिः—\mbox{वृषभः}}%
{\anga{परिघः}{\time{7-24}{09:08}}\hspace{1ex}\uanga{शिवः}}%
{\anga{तैतिलः}{\time{4-11}{07:51}}\hspace{1ex}\anga{गरः}{\time{34-21}{19:55}}\hspace{1ex}\uanga{वणिजः}}{}
}
{}
{Thu} 
\cfoot{\rygdata{13:18--14:44}{06:10--07:36}{09:02--10:27}}
\caldata{NOVEMBER}{15}{\sunmonth{तुला}{29}{}{कार्त्तिकः}{शरदृतुः}{शुक्रः}{विकारी}{दक्षिणायनम्}{शरदृतुः}}
{\sunmoonsrdata{06:11}{17:35}{20:04}{08:24}{11:53}
{\kalas{04:30 05:20 09:13 08:28 09:59 16:04 10:44 13:01 15:18 16:50 18:26 20:44 22:19 01:28(+1)}}}
{\tnykdata{\anga{\tithi{18}{कृष्ण-तृतीया}}{\time{33-57}{19:45}}\hspace{1ex}}%
{\anga{मृगशीर्षम्}{\time{42-26}{23:09}}\hspace{1ex}}{चन्द्रराशिः—\mbox{वृषभः\RIGHTarrow\textsf{10:59}}}%
{\anga{शिवः}{\time{4-31}{07:59}}\hspace{1ex}\uanga{सिद्धः}}%
{\anga{वणिजः}{\time{4-15}{07:53}}\hspace{1ex}\anga{विष्टिः}{\time{33-57}{19:45}}\hspace{1ex}\uanga{बवः}}{}
}
{गणाधिप-महागणपति सङ्कटहर-चतुर्थी-व्रतम्\eventsep काञ्ची ९ जगद्गुरु श्री-कृपाशङ्करेन्द्र सरस्वती आराधना~\#{१९५१}\eventsep सौभाग्य-सुन्दरी-व्रतम्}
{Fri} 
\cfoot{\rygdata{10:27--11:53}{14:44--16:10}{07:36--09:02}}
\caldata{NOVEMBER}{16}{\sunmonth{तुला}{30}{\mbox{तुला{\tiny\RIGHTarrow}\textsf{00:21(+1)}}}{कार्त्तिकः}{शरदृतुः}{शनिः}{विकारी}{दक्षिणायनम्}{शरदृतुः}}
{\sunmoonsrdata{06:11}{17:35}{21:00}{09:20}{11:53}
{\kalas{04:30 05:21 09:13 08:28 09:59 16:04 10:45 13:01 15:18 16:50 18:26 20:44 22:19 01:28(+1)}}}
{\tnykdata{\anga{\tithi{19}{कृष्ण-चतुर्थी}}{\time{32-39}{19:15}}\hspace{1ex}}%
{\anga{आर्द्रा}{\time{42-34}{23:13}}\hspace{1ex}}{चन्द्रराशिः—\mbox{मिथुनम्}}%
{\anga{सिद्धः}{\time{0-55}{06:33}}\hspace{1ex}\anga{साध्यः}{\time{56-38}{04:51(+1)}}\hspace{1ex}\uanga{शुभः}}%
{\anga{बवः}{\time{3-24}{07:33}}\hspace{1ex}\anga{बालवः}{\time{32-39}{19:15}}\hspace{1ex}\uanga{कौलवः}}{}
}
{आकाशदीप-समापनम्\eventsep सिन्धु-आदि-पुष्कर-समापनम्\eventsep तुला-कावेरी-स्नान-समापनम्\eventsep वृश्चिक-रवि-सङ्क्रमण-विष्णुपदी-पुण्यकालः~\textsf{17:57}{\RIGHTarrow}\textsf{06:45(+1)}}
{Sat} 
\cfoot{\rygdata{09:02--10:27}{13:19--14:44}{06:11--07:37}}
\caldata{NOVEMBER}{17}{\sunmonth{वृश्चिकः}{1}{}{कार्त्तिकः}{शरदृतुः}{भानुः}{विकारी}{दक्षिणायनम्}{शरदृतुः}}
{\sunmoonsrdata{06:11}{17:35}{21:58}{10:14}{11:53}
{\kalas{04:31 05:21 09:14 08:28 09:59 16:04 10:45 13:02 15:18 16:49 18:25 20:44 22:19 01:28(+1)}}}
{\tnykdata{\anga{\tithi{20}{कृष्ण-पञ्चमी}}{\time{30-28}{18:23}}\hspace{1ex}}%
{\anga{पुनर्वसुः}{\time{41-51}{22:56}}\hspace{1ex}}{चन्द्रराशिः—\mbox{मिथुनम्\RIGHTarrow\textsf{17:02}}}%
{\anga{शुभः}{\time{51-38}{02:51(+1)}}\hspace{1ex}\uanga{शुक्लः}}%
{\anga{कौलवः}{\time{1-39}{06:51}}\hspace{1ex}\anga{तैतिलः}{\time{30-28}{18:23}}\hspace{1ex}\anga{गरः}{\time{59-3}{05:49(+1)}}\hspace{1ex}\uanga{वणिजः}}{}
}
{\tamil{கார்த்திகை~ஞாயிற்றுக்கிழமை}\eventsep कृत्तिका-मण्डल-पारायणम्\eventsep \tamil{முடவன் முழுக்கு}\eventsep तिरुविशलूर् गङ्गाकर्षण-महोत्सव-आरम्भः}
{Sun} 
\cfoot{\rygdata{16:10--17:35}{11:53--13:19}{14:44--16:10}}
\caldata{NOVEMBER}{18}{\sunmonth{वृश्चिकः}{2}{}{कार्त्तिकः}{शरदृतुः}{सोमः}{विकारी}{दक्षिणायनम्}{शरदृतुः}}
{\sunmoonsrdata{06:12}{17:35}{22:56}{11:06}{11:53}
{\kalas{04:31 05:21 09:14 08:29 10:00 16:04 10:45 13:02 15:18 16:49 18:25 20:44 22:19 01:28(+1)}}}
{\tnykdata{\anga{\tithi{21}{कृष्ण-षष्ठी}}{\time{27-23}{17:09}}\hspace{1ex}}%
{\anga{पुष्यः}{\time{40-15}{22:18}}\hspace{1ex}}{चन्द्रराशिः—\mbox{कटकः}}%
{\anga{शुक्लः}{\time{45-56}{00:35(+1)}}\hspace{1ex}\uanga{ब्रह्म}}%
{\anga{वणिजः}{\time{27-23}{17:09}}\hspace{1ex}\anga{विष्टिः}{\time{55-32}{04:25(+1)}}\hspace{1ex}\uanga{बवः}}{}
}
{कृत्तिका-सोमवासरः}
{Mon} 
\cfoot{\rygdata{07:37--09:03}{10:28--11:53}{13:19--14:44}}
\caldata{NOVEMBER}{19}{\sunmonth{वृश्चिकः}{3}{}{कार्त्तिकः}{शरदृतुः}{मङ्गलः}{विकारी}{दक्षिणायनम्}{शरदृतुः}}
{\sunmoonsrdata{06:12}{17:35}{23:54}{11:55}{11:54}
{\kalas{04:31 05:22 09:14 08:29 10:00 16:04 10:45 13:02 15:18 16:49 18:25 20:44 22:19 01:29(+1)}}}
{\tnykdata{\anga{\tithi{22}{कृष्ण-सप्तमी}}{\time{23-27}{15:35}}\hspace{1ex}}%
{\anga{आश्रेषा}{\time{37-48}{21:20}}\hspace{1ex}}{चन्द्रराशिः—\mbox{कटकः\RIGHTarrow\textsf{21:20}}}%
{\anga{ब्रह्म}{\time{39-33}{22:02}}\hspace{1ex}\uanga{इन्द्रः}}%
{\anga{बवः}{\time{23-27}{15:35}}\hspace{1ex}\anga{बालवः}{\time{51-10}{02:40(+1)}}\hspace{1ex}\uanga{कौलवः}}{}
}
{कालभैरवाष्टमी}
{Tue} 
\cfoot{\rygdata{14:44--16:10}{09:03--10:28}{11:54--13:19}}
\caldata{NOVEMBER}{20}{\sunmonth{वृश्चिकः}{4}{}{कार्त्तिकः}{शरदृतुः}{बुधः}{विकारी}{दक्षिणायनम्}{शरदृतुः}}
{\sunmoonsrdata{06:13}{17:35}{00:50(+1)}{12:42}{11:54}
{\kalas{04:32 05:22 09:15 08:29 10:00 16:04 10:46 13:02 15:19 16:49 18:25 20:45 22:19 01:29(+1)}}}
{\tnykdata{\anga{\tithi{23}{कृष्ण-अष्टमी}}{\time{18-39}{13:41}}\hspace{1ex}}%
{\anga{मघा}{\time{34-32}{20:02}}\hspace{1ex}}{चन्द्रराशिः—\mbox{सिंहः}}%
{\anga{इन्द्रः}{\time{32-29}{19:13}}\hspace{1ex}\uanga{वैधृतिः}}%
{\anga{कौलवः}{\time{18-39}{13:41}}\hspace{1ex}\anga{तैतिलः}{\time{45-59}{00:37(+1)}}\hspace{1ex}\uanga{गरः}}{}
}
{बुधाष्टमी\eventsep काञ्ची ४२ जगद्गुरु श्री-ब्रह्मानन्दघनेन्द्र सरस्वती २ आराधना~\#{१०४२}\eventsep काञ्ची ४९ जगद्गुरु श्री-महादेवेन्द्र सरस्वती ३ आराधना~\#{७७३}\eventsep काञ्ची ५८ जगद्गुरु श्री-आत्मबोधेन्द्र सरस्वती आराधना~\#{३८२}\eventsep महादेवाष्टमी}
{Wed} 
\cfoot{\rygdata{11:54--13:19}{07:38--09:03}{10:29--11:54}}
\caldata{NOVEMBER}{21}{\sunmonth{वृश्चिकः}{5}{}{कार्त्तिकः}{शरदृतुः}{गुरुः}{विकारी}{दक्षिणायनम्}{शरदृतुः}}
{\sunmoonsrdata{06:13}{17:35}{01:47(+1)}{13:28}{11:54}
{\kalas{04:32 05:23 09:15 08:30 10:01 16:04 10:46 13:02 15:19 16:50 18:26 20:45 22:20 01:29(+1)}}}
{\tnykdata{\anga{\tithi{24}{कृष्ण-नवमी}}{\time{13-7}{11:28}}\hspace{1ex}}%
{\anga{पूर्वफल्गुनी}{\time{30-33}{18:27}}\hspace{1ex}}{चन्द्रराशिः—\mbox{सिंहः\RIGHTarrow\textsf{00:01(+1)}}}%
{\anga{वैधृतिः}{\time{24-51}{16:10}}\hspace{1ex}\uanga{विष्कम्भः}}%
{\anga{गरः}{\time{13-7}{11:28}}\hspace{1ex}\anga{वणिजः}{\time{40-7}{22:16}}\hspace{1ex}\uanga{विष्टिः}}{}
}
{काञ्ची २८ जगद्गुरु श्री-महादेवेन्द्र सरस्वती १ आराधना~\#{१४१९}\eventsep वैधृति-श्राद्धम्}
{Thu} 
\cfoot{\rygdata{13:19--14:45}{06:13--07:39}{09:04--10:29}}
\caldata{NOVEMBER}{22}{\sunmonth{वृश्चिकः}{6}{}{कार्त्तिकः}{शरदृतुः}{शुक्रः}{विकारी}{दक्षिणायनम्}{शरदृतुः}}
{\sunmoonsrdata{06:14}{17:35}{02:44(+1)}{14:13}{11:54}
{\kalas{04:33 05:23 09:15 08:30 10:01 16:04 10:46 13:02 15:19 16:50 18:26 20:45 22:20 01:30(+1)}}}
{\tnykdata{\anga{\tithi{25}{कृष्ण-दशमी}}{\time{6-57}{09:01}}\hspace{1ex}}%
{\anga{उत्तरफल्गुनी}{\time{26-1}{16:38}}\hspace{1ex}}{चन्द्रराशिः—\mbox{कन्या}}%
{\anga{विष्कम्भः}{\time{16-44}{12:55}}\hspace{1ex}\uanga{प्रीतिः}}%
{\anga{विष्टिः}{\time{6-57}{09:01}}\hspace{1ex}\anga{बवः}{\time{33-43}{19:43}}\hspace{1ex}\uanga{बालवः}}{}
}
{षडशीति-पुण्यकालः~\textsf{20:28}{\RIGHTarrow}\textsf{20:28(+1)}\eventsep ऊर्ज-मासः/शरदृतुः{\RIGHTarrow}\textsf{20:28}\eventsep \tamil{மெய்ப்பொருள் நாயனார் (4) குருபூஜை}\eventsep स्मार्त-उत्पन्ना-एकादशी (गृहस्थ)}
{Fri} 
\cfoot{\rygdata{10:29--11:54}{14:45--16:10}{07:39--09:04}}
\caldata{NOVEMBER}{23}{\sunmonth{वृश्चिकः}{7}{}{कार्त्तिकः}{शरदृतुः}{शनिः}{विकारी}{दक्षिणायनम्}{शरदृतुः}}
{\sunmoonsrdata{06:14}{17:35}{03:41(+1)}{14:59}{11:55}
{\kalas{04:33 05:24 09:16 08:30 10:01 16:04 10:47 13:03 15:19 16:50 18:26 20:45 22:20 01:30(+1)}}}
{\tnykdata{\anga{\tithi{26}{कृष्ण-एकादशी}}{\time{0-23}{06:24}}\hspace{1ex}\anga{\tithi{27}{कृष्ण-द्वादशी}}{\time{53-41}{03:43(+1)}}\hspace{1ex}}%
{\anga{हस्तः}{\time{21-9}{14:42}}\hspace{1ex}}{चन्द्रराशिः—\mbox{कन्या\RIGHTarrow\textsf{01:43(+1)}}}%
{\anga{प्रीतिः}{\time{8-18}{09:34}}\hspace{1ex}\anga{आयुष्मान्}{\time{59-49}{06:10(+1)}}\hspace{1ex}\uanga{सौभाग्यः}}%
{\anga{बालवः}{\time{0-23}{06:24}}\hspace{1ex}\anga{कौलवः}{\time{27-2}{17:03}}\hspace{1ex}\anga{तैतिलः}{\time{53-41}{03:43(+1)}}\hspace{1ex}\uanga{गरः}}{}
}
{\tamil{ஆனாய நாயனார் (13) குருபூஜை}\eventsep हरिवासरः{\RIGHTarrow}\textsf{11:44}\eventsep स्मार्त-उत्पन्ना-एकादशी (सन्न्यस्थ)\eventsep त्रिस्पर्शा-महाद्वादशी\eventsep वैष्णव-उत्पन्ना-एकादशी}
{Sat} 
\cfoot{\rygdata{09:04--10:30}{13:20--14:45}{06:14--07:39}}
\caldata{NOVEMBER}{24}{\sunmonth{वृश्चिकः}{8}{}{कार्त्तिकः}{शरदृतुः}{भानुः}{विकारी}{दक्षिणायनम्}{शरदृतुः}}
{\sunmoonsrdata{06:15}{17:35}{04:40(+1)}{15:46}{11:55}
{\kalas{04:34 05:24 09:16 08:31 10:02 16:04 10:47 13:03 15:19 16:50 18:26 20:45 22:20 01:30(+1)}}}
{\tnykdata{\anga{\tithi{28}{कृष्ण-त्रयोदशी}}{\time{47-6}{01:05(+1)}}\hspace{1ex}}%
{\anga{चित्रा}{\time{16-15}{12:45}}\hspace{1ex}}{चन्द्रराशिः—\mbox{तुला}}%
{\anga{सौभाग्यः}{\time{51-27}{02:50(+1)}}\hspace{1ex}\uanga{शोभनः}}%
{\anga{गरः}{\time{20-21}{14:23}}\hspace{1ex}\anga{वणिजः}{\time{47-6}{01:05(+1)}}\hspace{1ex}\uanga{विष्टिः}}{}
}
{\tamil{கார்த்திகை~ஞாயிற்றுக்கிழமை}\eventsep मासशिवरात्रिः\eventsep प्रदोष-व्रतम्}
{Sun} 
\cfoot{\rygdata{16:10--17:35}{11:55--13:20}{14:45--16:10}}
\caldata{NOVEMBER}{25}{\sunmonth{वृश्चिकः}{9}{}{कार्त्तिकः}{शरदृतुः}{सोमः}{विकारी}{दक्षिणायनम्}{शरदृतुः}}
{\sunmoonsrdata{06:15}{17:35}{05:40(+1)}{16:36}{11:55}
{\kalas{04:34 05:25 09:17 08:31 10:02 16:04 10:47 13:03 15:19 16:50 18:26 20:45 22:20 01:31(+1)}}}
{\tnykdata{\anga{\tithi{29}{कृष्ण-चतुर्दशी}}{\time{41-2}{22:40}}\hspace{1ex}}%
{\anga{स्वाती}{\time{11-38}{10:55}}\hspace{1ex}}{चन्द्रराशिः—\mbox{तुला\RIGHTarrow\textsf{03:42(+1)}}}%
{\anga{शोभनः}{\time{43-31}{23:40}}\hspace{1ex}\uanga{अतिगण्डः}}%
{\anga{विष्टिः}{\time{13-58}{11:51}}\hspace{1ex}\anga{शकुनिः}{\time{41-2}{22:40}}\hspace{1ex}\uanga{चतुष्पात्}}{}
}
{कृत्तिका-सोमवासरः}
{Mon} 
\cfoot{\rygdata{07:40--09:05}{10:30--11:55}{13:20--14:45}}
\caldata{NOVEMBER}{26}{\sunmonth{वृश्चिकः}{10}{}{कार्त्तिकः}{शरदृतुः}{मङ्गलः}{विकारी}{दक्षिणायनम्}{शरदृतुः}}
{\sunmoonsrdata{06:16}{17:35}{---}{17:29}{11:56}
{\kalas{04:34 05:25 09:17 08:32 10:02 16:05 10:48 13:03 15:19 16:50 18:26 20:46 22:21 01:31(+1)}}}
{\tnykdata{\anga{\tithi{30}{अमावास्या}}{\time{35-48}{20:35}}\hspace{1ex}}%
{\anga{विशाखा}{\time{7-40}{09:20}}\hspace{1ex}}{चन्द्रराशिः—\mbox{वृश्चिकः}}%
{\anga{अतिगण्डः}{\time{36-16}{20:47}}\hspace{1ex}\uanga{सुकर्म}}%
{\anga{चतुष्पात्}{\time{8-17}{09:35}}\hspace{1ex}\anga{नाग}{\time{35-48}{20:35}}\hspace{1ex}\uanga{किंस्तुघ्नः}}{}
}
{आग्रयण-होमः द्राविडेषु\eventsep कार्त्तिक-अमावास्या (अलभ्यम्–अनूराधा, पुष्कला)\eventsep कार्त्तिक-स्नानपूर्तिः\eventsep नवम-अपरपक्ष-समापनम्\eventsep पार्वणव्रतम् अमावास्यायाम्\eventsep तिरुविशलूर् गङ्गाकर्षण-महोत्सव-समापनम्}
{Tue} 
\cfoot{\rygdata{14:45--16:10}{09:06--10:31}{11:55--13:20}}
\caldata{NOVEMBER}{27}{\sunmonth{वृश्चिकः}{11}{}{मार्गशीर्षः}{हेमन्तऋतुः}{बुधः}{विकारी}{दक्षिणायनम्}{शरदृतुः}}
{\sunmoonrsdata{06:16}{17:35}{06:40}{18:24}{11:56}
{\kalas{04:35 05:26 09:17 08:32 10:03 16:05 10:48 13:04 15:20 16:50 18:26 20:46 22:21 01:31(+1)}}}
{\tnykdata{\anga{\tithi{1}{शुक्ल-प्रथमा}}{\time{31-46}{18:59}}\hspace{1ex}}%
{\anga{अनूराधा}{\time{4-42}{08:10}}\hspace{1ex}}{चन्द्रराशिः—\mbox{वृश्चिकः}}%
{\anga{सुकर्म}{\time{29-59}{18:16}}\hspace{1ex}\uanga{धृतिः}}%
{\anga{किंस्तुघ्नः}{\time{3-36}{07:43}}\hspace{1ex}\anga{बवः}{\time{31-46}{18:59}}\hspace{1ex}\uanga{बालवः}}{}
}
{बुधानुराधा-पुण्यकालः\eventsep दर्शेष्टिः\eventsep काञ्ची १८ जगद्गुरु श्री-योगतिलक सुरेन्द्र सरस्वती आराधना~\#{१६३५}\eventsep स्थालीपाकः\eventsep वनदुर्गानवरात्र-आरम्भः}
{Wed} 
\cfoot{\rygdata{11:56--13:21}{07:41--09:06}{10:31--11:56}}
\caldata{NOVEMBER}{28}{\sunmonth{वृश्चिकः}{12}{}{मार्गशीर्षः}{हेमन्तऋतुः}{गुरुः}{विकारी}{दक्षिणायनम्}{शरदृतुः}}
{\sunmoonrsdata{06:17}{17:35}{07:39}{19:20}{11:56}
{\kalas{04:35 05:26 09:18 08:33 10:03 16:05 10:48 13:04 15:20 16:50 18:26 20:46 22:21 01:32(+1)}}}
{\tnykdata{\anga{\tithi{2}{शुक्ल-द्वितीया}}{\time{29-13}{17:58}}\hspace{1ex}}%
{\anga{ज्येष्ठा}{\time{3-5}{07:31}}\hspace{1ex}}{चन्द्रराशिः—\mbox{वृश्चिकः\RIGHTarrow\textsf{07:31}}}%
{\anga{धृतिः}{\time{24-52}{16:14}}\hspace{1ex}\uanga{शूलः}}%
{\anga{बालवः}{\time{0-16}{06:24}}\hspace{1ex}\anga{कौलवः}{\time{29-13}{17:58}}\hspace{1ex}\anga{तैतिलः}{\time{58-36}{05:43(+1)}}\hspace{1ex}\uanga{गरः}}{}
}
{चन्द्र-दर्शनम्\eventsep \tamil{மூர்க்க நாயனார் (31) குருபூஜை}\eventsep तिन्त्रिणी-गौरी-व्रतम्}
{Thu} 
\cfoot{\rygdata{13:21--14:46}{06:17--07:42}{09:07--10:31}}
\caldata{NOVEMBER}{29}{\sunmonth{वृश्चिकः}{13}{}{मार्गशीर्षः}{हेमन्तऋतुः}{शुक्रः}{विकारी}{दक्षिणायनम्}{शरदृतुः}}
{\sunmoonrsdata{06:17}{17:36}{08:35}{20:16}{11:56}
{\kalas{04:36 05:27 09:18 08:33 10:03 16:05 10:49 13:04 15:20 16:50 18:26 20:46 22:21 01:32(+1)}}}
{\tnykdata{\anga{\tithi{3}{शुक्ल-तृतीया}}{\time{28-24}{17:39}}\hspace{1ex}}%
{\anga{मूला}{\time{3-3}{07:31}}\hspace{1ex}}{चन्द्रराशिः—\mbox{धनुः}}%
{\anga{शूलः}{\time{21-6}{14:44}}\hspace{1ex}\uanga{गण्डः}}%
{\anga{गरः}{\time{28-24}{17:39}}\hspace{1ex}\anga{वणिजः}{\time{58-42}{05:46(+1)}}\hspace{1ex}\uanga{विष्टिः}}{}
}
{\tamil{சிறப்புலி நாயனார் (34) குருபூஜை}}
{Fri} 
\cfoot{\rygdata{10:32--11:56}{14:46--16:11}{07:42--09:07}}
\caldata{NOVEMBER}{30}{\sunmonth{वृश्चिकः}{14}{}{मार्गशीर्षः}{हेमन्तऋतुः}{शनिः}{विकारी}{दक्षिणायनम्}{शरदृतुः}}
{\sunmoonrsdata{06:18}{17:36}{09:27}{21:10}{11:57}
{\kalas{04:36 05:27 09:19 08:33 10:04 16:05 10:49 13:05 15:20 16:51 18:27 20:47 22:22 01:33(+1)}}}
{\tnykdata{\anga{\tithi{4}{शुक्ल-चतुर्थी}}{\time{29-26}{18:05}}\hspace{1ex}}%
{\anga{पूर्वाषाढा}{\time{4-47}{08:13}}\hspace{1ex}}{चन्द्रराशिः—\mbox{धनुः\RIGHTarrow\textsf{14:30}}}%
{\anga{गण्डः}{\time{18-44}{13:48}}\hspace{1ex}\uanga{वृद्धिः}}%
{\anga{विष्टिः}{\time{29-26}{18:05}}\hspace{1ex}\uanga{बवः}}{}
}
{बदरी-गौरी-व्रतम्}
{Sat} 
\cfoot{\rygdata{09:07--10:32}{13:22--14:46}{06:18--07:43}}
\caldata{DECEMBER}{1}{\sunmonth{वृश्चिकः}{15}{}{मार्गशीर्षः}{हेमन्तऋतुः}{भानुः}{विकारी}{दक्षिणायनम्}{शरदृतुः}}
{\sunmoonrsdata{06:19}{17:36}{10:15}{22:02}{11:57}
{\kalas{04:37 05:28 09:19 08:34 10:04 16:06 10:50 13:05 15:21 16:51 18:27 20:47 22:22 01:33(+1)}}}
{\tnykdata{\anga{\tithi{5}{शुक्ल-पञ्चमी}}{\time{32-15}{19:13}}\hspace{1ex}}%
{\anga{उत्तराषाढा}{\time{8-16}{09:37}}\hspace{1ex}}{चन्द्रराशिः—\mbox{मकरः}}%
{\anga{वृद्धिः}{\time{17-46}{13:25}}\hspace{1ex}\uanga{ध्रुवः}}%
{\anga{बवः}{\time{0-37}{06:33}}\hspace{1ex}\anga{बालवः}{\time{32-15}{19:13}}\hspace{1ex}\uanga{कौलवः}}{}
}
{\tamil{கார்த்திகை~ஞாயிற்றுக்கிழமை}\eventsep श्रवण-व्रतम्}
{Sun} 
\cfoot{\rygdata{16:11--17:36}{11:57--13:22}{14:47--16:11}}
\caldata{DECEMBER}{2}{\sunmonth{वृश्चिकः}{16}{}{मार्गशीर्षः}{हेमन्तऋतुः}{सोमः}{विकारी}{दक्षिणायनम्}{शरदृतुः}}
{\sunmoonrsdata{06:19}{17:36}{10:59}{22:52}{11:58}
{\kalas{04:37 05:28 09:20 08:34 10:05 16:06 10:50 13:05 15:21 16:51 18:27 20:47 22:22 01:33(+1)}}}
{\tnykdata{\anga{\tithi{6}{शुक्ल-षष्ठी}}{\time{36-40}{20:59}}\hspace{1ex}}%
{\anga{श्रवणः}{\time{13-22}{11:40}}\hspace{1ex}}{चन्द्रराशिः—\mbox{मकरः\RIGHTarrow\textsf{00:54(+1)}}}%
{\anga{ध्रुवः}{\time{18-3}{13:32}}\hspace{1ex}\uanga{व्याघातः}}%
{\anga{कौलवः}{\time{4-17}{08:02}}\hspace{1ex}\anga{तैतिलः}{\time{36-40}{20:59}}\hspace{1ex}\uanga{गरः}}{}
}
{काञ्ची ३२ जगद्गुरु श्री-चिदानन्दघनेन्द्र सरस्वती आराधना~\#{१३४८}\eventsep मार्गशीर्ष-शिवलिङ्ग-षष्ठी\eventsep सोमश्रावणी-पुण्यकालः\eventsep सुब्रह्मण्य-षष्ठी-व्रतम्}
{Mon} 
\cfoot{\rygdata{07:44--09:08}{10:33--11:58}{13:22--14:47}}
\caldata{DECEMBER}{3}{\sunmonth{वृश्चिकः}{17}{}{मार्गशीर्षः}{हेमन्तऋतुः}{मङ्गलः}{विकारी}{दक्षिणायनम्}{शरदृतुः}}
{\sunmoonrsdata{06:20}{17:36}{11:39}{23:39}{11:58}
{\kalas{04:38 05:29 09:20 08:35 10:05 16:06 10:50 13:06 15:21 16:51 18:27 20:47 22:23 01:34(+1)}}}
{\tnykdata{\anga{\tithi{7}{शुक्ल-सप्तमी}}{\time{42-15}{23:14}}\hspace{1ex}}%
{\anga{श्रविष्ठा}{\time{19-45}{14:14}}\hspace{1ex}}{चन्द्रराशिः—\mbox{कुम्भः}}%
{\anga{व्याघातः}{\time{19-18}{14:03}}\hspace{1ex}\uanga{हर्षणः}}%
{\anga{गरः}{\time{9-20}{10:04}}\hspace{1ex}\anga{वणिजः}{\time{42-15}{23:14}}\hspace{1ex}\uanga{विष्टिः}}{}
}
{काञ्ची ५ जगद्गुरु श्री-ज्ञानानन्देन्द्र सरस्वती आराधना~\#{२२२४}\eventsep मित्र-सप्तमी\eventsep नन्दा-सप्तमी}
{Tue} 
\cfoot{\rygdata{14:47--16:12}{09:09--10:33}{11:58--13:23}}
\caldata{DECEMBER}{4}{\sunmonth{वृश्चिकः}{18}{}{मार्गशीर्षः}{हेमन्तऋतुः}{बुधः}{विकारी}{दक्षिणायनम्}{शरदृतुः}}
{\sunmoonrsdata{06:20}{17:37}{12:16}{00:26(+1)}{11:58}
{\kalas{04:38 05:29 09:21 08:35 10:06 16:06 10:51 13:06 15:21 16:52 18:28 20:48 22:23 01:34(+1)}}}
{\tnykdata{\anga{\tithi{8}{शुक्ल-अष्टमी}}{\time{48-29}{01:44(+1)}}\hspace{1ex}}%
{\anga{शतभिषक्}{\time{26-55}{17:06}}\hspace{1ex}}{चन्द्रराशिः—\mbox{कुम्भः}}%
{\anga{हर्षणः}{\time{21-10}{14:48}}\hspace{1ex}\uanga{वज्रम्}}%
{\anga{विष्टिः}{\time{15-19}{12:28}}\hspace{1ex}\anga{बवः}{\time{48-29}{01:44(+1)}}\hspace{1ex}\uanga{बालवः}}{}
}
{बुधाष्टमी}
{Wed} 
\cfoot{\rygdata{11:58--13:23}{07:45--09:09}{10:34--11:58}}
\caldata{DECEMBER}{5}{\sunmonth{वृश्चिकः}{19}{}{मार्गशीर्षः}{हेमन्तऋतुः}{गुरुः}{विकारी}{दक्षिणायनम्}{शरदृतुः}}
{\sunmoonrsdata{06:21}{17:37}{12:53}{01:11(+1)}{11:59}
{\kalas{04:39 05:30 09:21 08:36 10:06 16:07 10:51 13:06 15:22 16:52 18:28 20:48 22:24 01:35(+1)}}}
{\tnykdata{\anga{\tithi{9}{शुक्ल-नवमी}}{\time{54-46}{04:15(+1)}}\hspace{1ex}}%
{\anga{पूर्वप्रोष्ठपदा}{\time{34-19}{20:04}}\hspace{1ex}}{चन्द्रराशिः—\mbox{कुम्भः\RIGHTarrow\textsf{13:20}}}%
{\anga{वज्रम्}{\time{23-16}{15:39}}\hspace{1ex}\uanga{सिद्धिः}}%
{\anga{बालवः}{\time{21-38}{15:00}}\hspace{1ex}\anga{कौलवः}{\time{54-46}{04:15(+1)}}\hspace{1ex}\uanga{तैतिलः}}{}
}
{प्रलय-कल्पादिः\eventsep वनदुर्गानवरात्र-समापनम्}
{Thu} 
\cfoot{\rygdata{13:23--14:48}{06:21--07:45}{09:10--10:34}}
\caldata{DECEMBER}{6}{\sunmonth{वृश्चिकः}{20}{}{मार्गशीर्षः}{हेमन्तऋतुः}{शुक्रः}{विकारी}{दक्षिणायनम्}{शरदृतुः}}
{\sunmoonrsdata{06:21}{17:37}{13:29}{01:57(+1)}{11:59}
{\kalas{04:39 05:30 09:22 08:36 10:07 16:07 10:52 13:07 15:22 16:52 18:28 20:48 22:24 01:35(+1)}}}
{\tnykdata{\fulltithi{\tithi{10}{शुक्ल-दशमी}}}%
{\anga{उत्तरप्रोष्ठपदा}{\time{41-22}{22:54}}\hspace{1ex}}{चन्द्रराशिः—\mbox{मीनः}}%
{\anga{सिद्धिः}{\time{25-10}{16:25}}\hspace{1ex}\uanga{व्यतीपातः}}%
{\anga{तैतिलः}{\time{27-43}{17:27}}\hspace{1ex}\uanga{गरः}}{}
}
{}
{Fri} 
\cfoot{\rygdata{10:35--11:59}{14:48--16:13}{07:46--09:10}}
\caldata{DECEMBER}{7}{\sunmonth{वृश्चिकः}{21}{}{मार्गशीर्षः}{हेमन्तऋतुः}{शनिः}{विकारी}{दक्षिणायनम्}{शरदृतुः}}
{\sunmoonrsdata{06:22}{17:38}{14:06}{02:44(+1)}{12:00}
{\kalas{04:40 05:31 09:22 08:37 10:07 16:07 10:52 13:07 15:22 16:53 18:29 20:49 22:24 01:36(+1)}}}
{\tnykdata{\anga{\tithi{10}{शुक्ल-दशमी}}{\time{0-30}{06:34}}\hspace{1ex}}%
{\anga{रेवती}{\time{47-38}{01:25(+1)}}\hspace{1ex}}{चन्द्रराशिः—\mbox{मीनः\RIGHTarrow\textsf{01:25(+1)}}}%
{\anga{व्यतीपातः}{\time{26-30}{16:58}}\hspace{1ex}\uanga{वरीयान्}}%
{\anga{गरः}{\time{0-30}{06:34}}\hspace{1ex}\anga{वणिजः}{\time{33-3}{19:35}}\hspace{1ex}\uanga{विष्टिः}}{}
}
{व्यतीपात-श्राद्धम्}
{Sat} 
\cfoot{\rygdata{09:11--10:35}{13:24--14:49}{06:22--07:46}}
\caldata{DECEMBER}{8}{\sunmonth{वृश्चिकः}{22}{}{मार्गशीर्षः}{हेमन्तऋतुः}{भानुः}{विकारी}{दक्षिणायनम्}{शरदृतुः}}
{\sunmoonrsdata{06:22}{17:38}{14:45}{03:33(+1)}{12:00}
{\kalas{04:40 05:31 09:22 08:37 10:07 16:08 10:53 13:08 15:23 16:53 18:29 20:49 22:25 01:36(+1)}}}
{\tnykdata{\anga{\tithi{11}{शुक्ल-एकादशी}}{\time{5-17}{08:29}}\hspace{1ex}}%
{\anga{अश्विनी}{\time{52-42}{03:27(+1)}}\hspace{1ex}}{चन्द्रराशिः—\mbox{मेषः}}%
{\anga{वरीयान्}{\time{27-0}{17:11}}\hspace{1ex}\uanga{परिघः}}%
{\anga{विष्टिः}{\time{5-17}{08:29}}\hspace{1ex}\anga{बवः}{\time{37-13}{21:15}}\hspace{1ex}\uanga{बालवः}}{}
}
{गीता-जयन्ती\eventsep गुरुवायुपुर-एकादशी\eventsep हरिवासरः{\RIGHTarrow}\textsf{14:53}\eventsep \tamil{கார்த்திகை~ஞாயிற்றுக்கிழமை}\eventsep कैशिक-एकादशी\eventsep सर्व-मोक्षदा-एकादशी}
{Sun} 
\cfoot{\rygdata{16:13--17:38}{12:00--13:25}{14:49--16:13}}
\caldata{DECEMBER}{9}{\sunmonth{वृश्चिकः}{23}{}{मार्गशीर्षः}{हेमन्तऋतुः}{सोमः}{विकारी}{दक्षिणायनम्}{शरदृतुः}}
{\sunmoonrsdata{06:23}{17:38}{15:27}{04:24(+1)}{12:01}
{\kalas{04:41 05:32 09:23 08:38 10:08 16:08 10:53 13:08 15:23 16:53 18:29 20:50 22:25 01:36(+1)}}}
{\tnykdata{\anga{\tithi{12}{शुक्ल-द्वादशी}}{\time{8-46}{09:54}}\hspace{1ex}}%
{\anga{अपभरणी}{\time{56-27}{04:58(+1)}}\hspace{1ex}}{चन्द्रराशिः—\mbox{मेषः}}%
{\anga{परिघः}{\time{26-29}{16:59}}\hspace{1ex}\uanga{शिवः}}%
{\anga{बालवः}{\time{8-46}{09:54}}\hspace{1ex}\anga{कौलवः}{\time{40-0}{22:23}}\hspace{1ex}\uanga{तैतिलः}}{}
}
{\tamil{பரணீ~தீபம்}\eventsep \tamil{கார்த்திகை}\eventsep कैशिक-द्वादशी\eventsep सोम-प्रदोष-व्रतम्\eventsep \tamil{திருவண்ணாமலை~தீபம்}}
{Mon} 
\cfoot{\rygdata{07:47--09:12}{10:36--12:01}{13:25--14:49}}
\caldata{DECEMBER}{10}{\sunmonth{वृश्चिकः}{24}{}{मार्गशीर्षः}{हेमन्तऋतुः}{मङ्गलः}{विकारी}{दक्षिणायनम्}{शरदृतुः}}
{\sunmoonrsdata{06:23}{17:39}{16:12}{05:18(+1)}{12:01}
{\kalas{04:41 05:32 09:23 08:38 10:08 16:08 10:53 13:08 15:24 16:54 18:30 20:50 22:26 01:37(+1)}}}
{\tnykdata{\anga{\tithi{13}{शुक्ल-त्रयोदशी}}{\time{10-50}{10:43}}\hspace{1ex}}%
{\anga{कृत्तिका}{\time{58-47}{05:54(+1)}}\hspace{1ex}}{चन्द्रराशिः—\mbox{मेषः\RIGHTarrow\textsf{11:15}}}%
{\anga{शिवः}{\time{24-51}{16:20}}\hspace{1ex}\uanga{सिद्धः}}%
{\anga{तैतिलः}{\time{10-50}{10:43}}\hspace{1ex}\anga{गरः}{\time{41-20}{22:55}}\hspace{1ex}\uanga{वणिजः}}{}
}
{कृत्तिका-व्रतम्\eventsep \tamil{கணம்புல்ல நாயனார் (46) குருபூஜை}\eventsep \tamil{திருமங்கையாழ்வார் திருநக்ஷத்திரம்}}
{Tue} 
\cfoot{\rygdata{14:50--16:14}{09:12--10:37}{12:01--13:25}}
\caldata{DECEMBER}{11}{\sunmonth{वृश्चिकः}{25}{}{मार्गशीर्षः}{हेमन्तऋतुः}{बुधः}{विकारी}{दक्षिणायनम्}{शरदृतुः}}
{\sunmoonrsdata{06:24}{17:39}{17:01}{06:14(+1)}{12:01}
{\kalas{04:42 05:33 09:24 08:39 10:09 16:09 10:54 13:09 15:24 16:54 18:30 20:50 22:26 01:37(+1)}}}
{\tnykdata{\anga{\tithi{14}{शुक्ल-चतुर्दशी}}{\time{11-26}{10:59}}\hspace{1ex}}%
{\anga{रोहिणी}{\time{59-48}{06:19(+1)}}\hspace{1ex}}{चन्द्रराशिः—\mbox{वृषभः}}%
{\anga{सिद्धः}{\time{22-8}{15:15}}\hspace{1ex}\uanga{साध्यः}}%
{\anga{वणिजः}{\time{11-26}{10:59}}\hspace{1ex}\anga{विष्टिः}{\time{41-15}{22:54}}\hspace{1ex}\uanga{बवः}}{}
}
{दत्तात्रेय-जयन्ती\eventsep पार्वणव्रतम् पूर्णिमायाम्\eventsep \tamil{ஸர்வாலய~தீபம்}\eventsep वेङ्कटाचले पूर्णिमा-गरुड-सेवा}
{Wed} 
\cfoot{\rygdata{12:01--13:26}{07:48--09:13}{10:37--12:01}}
\caldata{DECEMBER}{12}{\sunmonth{वृश्चिकः}{26}{}{मार्गशीर्षः}{हेमन्तऋतुः}{गुरुः}{विकारी}{दक्षिणायनम्}{शरदृतुः}}
{\sunmoonrsdata{06:25}{17:39}{17:55}{---}{12:02}
{\kalas{04:42 05:33 09:24 08:39 10:09 16:09 10:54 13:09 15:24 16:54 18:30 20:51 22:26 01:38(+1)}}}
{\tnykdata{\anga{\tithi{15}{पौर्णमासी}}{\time{10-42}{10:42}}\hspace{1ex}}%
{\anga{मृगशीर्षम्}{\time{59-38}{06:16(+1)}}\hspace{1ex}}{चन्द्रराशिः—\mbox{वृषभः\RIGHTarrow\textsf{18:21}}}%
{\anga{साध्यः}{\time{18-23}{13:46}}\hspace{1ex}\uanga{शुभः}}%
{\anga{बवः}{\time{10-42}{10:42}}\hspace{1ex}\anga{बालवः}{\time{39-54}{22:22}}\hspace{1ex}\uanga{कौलवः}}{}
}
{आग्रयण-होमः द्राविडेषु\eventsep अन्नपूर्णा-जयन्ती\eventsep काञ्ची १३ जगद्गुरु श्री-सच्चिद्घनेन्द्र सरस्वती आराधना~\#{१७४८}\eventsep मार्गशीर्ष-पूर्णिमा\eventsep पूर्र्णमासेष्टिः\eventsep पूर्णिमा-व्रतम्\eventsep सर्प-बल्युत्सर्जनम्\eventsep स्थालीपाकः\eventsep त्रिपुर-भैरवी-जयन्ती}
{Thu} 
\cfoot{\rygdata{13:26--14:51}{06:25--07:49}{09:13--10:38}}
\caldata{DECEMBER}{13}{\sunmonth{वृश्चिकः}{27}{}{मार्गशीर्षः}{हेमन्तऋतुः}{शुक्रः}{विकारी}{दक्षिणायनम्}{शरदृतुः}}
{\sunmoonsrdata{06:25}{17:40}{18:52}{07:12}{12:02}
{\kalas{04:43 05:34 09:25 08:40 10:10 16:10 10:55 13:10 15:25 16:55 18:31 20:51 22:27 01:38(+1)}}}
{\tnykdata{\anga{\tithi{16}{कृष्ण-प्रथमा}}{\time{8-48}{09:56}}\hspace{1ex}}%
{\anga{आर्द्रा}{\time{58-27}{05:48(+1)}}\hspace{1ex}}{चन्द्रराशिः—\mbox{मिथुनम्}}%
{\anga{शुभः}{\time{13-44}{11:55}}\hspace{1ex}\uanga{शुक्लः}}%
{\anga{कौलवः}{\time{8-48}{09:56}}\hspace{1ex}\anga{तैतिलः}{\time{37-28}{21:24}}\hspace{1ex}\uanga{गरः}}{}
}
{}
{Fri} 
\cfoot{\rygdata{10:38--12:02}{14:51--16:15}{07:49--09:14}}
\caldata{DECEMBER}{14}{\sunmonth{वृश्चिकः}{28}{}{मार्गशीर्षः}{हेमन्तऋतुः}{शनिः}{विकारी}{दक्षिणायनम्}{शरदृतुः}}
{\sunmoonsrdata{06:26}{17:40}{19:51}{08:08}{12:03}
{\kalas{04:43 05:35 09:25 08:40 10:10 16:10 10:55 13:10 15:25 16:55 18:31 20:52 22:27 01:39(+1)}}}
{\tnykdata{\anga{\tithi{17}{कृष्ण-द्वितीया}}{\time{5-53}{08:47}}\hspace{1ex}}%
{\anga{पुनर्वसुः}{\time{56-27}{05:01(+1)}}\hspace{1ex}}{चन्द्रराशिः—\mbox{मिथुनम्\RIGHTarrow\textsf{23:14}}}%
{\anga{शुक्लः}{\time{8-18}{09:45}}\hspace{1ex}\uanga{ब्रह्म}}%
{\anga{गरः}{\time{5-53}{08:47}}\hspace{1ex}\anga{वणिजः}{\time{34-7}{20:04}}\hspace{1ex}\uanga{विष्टिः}}{}
}
{नारायणीयं-जयन्ती~\#{४३४}}
{Sat} 
\cfoot{\rygdata{09:14--10:39}{13:27--14:51}{06:26--07:50}}
\caldata{DECEMBER}{15}{\sunmonth{वृश्चिकः}{29}{}{मार्गशीर्षः}{हेमन्तऋतुः}{भानुः}{विकारी}{दक्षिणायनम्}{शरदृतुः}}
{\sunmoonsrdata{06:26}{17:40}{20:51}{09:02}{12:03}
{\kalas{04:44 05:35 09:26 08:41 10:11 16:11 10:56 13:11 15:26 16:56 18:32 20:52 22:28 01:39(+1)}}}
{\tnykdata{\anga{\tithi{18}{कृष्ण-तृतीया}}{\time{2-9}{07:18}}\hspace{1ex}\anga{\tithi{19}{कृष्ण-चतुर्थी}}{\time{57-50}{05:34(+1)}}\hspace{1ex}}%
{\anga{पुष्यः}{\time{53-49}{03:58(+1)}}\hspace{1ex}}{चन्द्रराशिः—\mbox{कटकः}}%
{\anga{ब्रह्म}{\time{2-14}{07:20}}\hspace{1ex}\anga{इन्द्रः}{\time{55-42}{04:43(+1)}}\hspace{1ex}\uanga{वैधृतिः}}%
{\anga{विष्टिः}{\time{2-9}{07:18}}\hspace{1ex}\anga{बवः}{\time{30-3}{18:28}}\hspace{1ex}\anga{बालवः}{\time{57-50}{05:34(+1)}}\hspace{1ex}\uanga{कौलवः}}{}
}
{आखुरथ-महागणपति सङ्कटहर-चतुर्थी-व्रतम्\eventsep \tamil{கார்த்திகை~ஞாயிற்றுக்கிழமை}\eventsep रविपुष्ययोग-पुण्यकालः}
{Sun} 
\cfoot{\rygdata{16:16--17:40}{12:03--13:28}{14:52--16:16}}
\caldata{DECEMBER}{16}{\sunmonth{धनुः}{1}{\mbox{वृश्चिकः{\tiny\RIGHTarrow}\textsf{15:01}}}{मार्गशीर्षः}{हेमन्तऋतुः}{सोमः}{विकारी}{दक्षिणायनम्}{हेमन्तऋतुः}}
{\sunmoonsrdata{06:27}{17:41}{21:49}{09:53}{12:04}
{\kalas{04:44 05:36 09:26 08:42 10:11 16:11 10:56 13:11 15:26 16:56 18:32 20:52 22:28 01:40(+1)}}}
{\tnykdata{\anga{\tithi{20}{कृष्ण-पञ्चमी}}{\time{53-1}{03:39(+1)}}\hspace{1ex}}%
{\anga{आश्रेषा}{\time{50-44}{02:45(+1)}}\hspace{1ex}}{चन्द्रराशिः—\mbox{कटकः\RIGHTarrow\textsf{02:45(+1)}}}%
{\anga{वैधृतिः}{\time{48-46}{01:57(+1)}}\hspace{1ex}\uanga{विष्कम्भः}}%
{\anga{कौलवः}{\time{25-27}{16:38}}\hspace{1ex}\anga{तैतिलः}{\time{53-1}{03:39(+1)}}\hspace{1ex}\uanga{गरः}}{}
}
{धनूरवि-सङ्क्रमण-षडशीति-पुण्यकालः~\textsf{15:01}{\RIGHTarrow}\textsf{15:01(+1)}\eventsep वैधृति-श्राद्धम्}
{Mon} 
\cfoot{\rygdata{07:51--09:15}{10:40--12:04}{13:28--14:52}}
\caldata{DECEMBER}{17}{\sunmonth{धनुः}{2}{}{मार्गशीर्षः}{हेमन्तऋतुः}{मङ्गलः}{विकारी}{दक्षिणायनम्}{हेमन्तऋतुः}}
{\sunmoonsrdata{06:27}{17:41}{22:46}{10:41}{12:04}
{\kalas{04:45 05:36 09:27 08:42 10:12 16:12 10:57 13:12 15:27 16:56 18:32 20:53 22:29 01:40(+1)}}}
{\tnykdata{\anga{\tithi{21}{कृष्ण-षष्ठी}}{\time{47-54}{01:37(+1)}}\hspace{1ex}}%
{\anga{मघा}{\time{47-20}{01:23(+1)}}\hspace{1ex}}{चन्द्रराशिः—\mbox{सिंहः}}%
{\anga{विष्कम्भः}{\time{41-35}{23:05}}\hspace{1ex}\uanga{प्रीतिः}}%
{\anga{गरः}{\time{20-29}{14:39}}\hspace{1ex}\anga{वणिजः}{\time{47-54}{01:37(+1)}}\hspace{1ex}\uanga{विष्टिः}}{}
}
{}
{Tue} 
\cfoot{\rygdata{14:53--16:17}{09:16--10:40}{12:04--13:28}}
\caldata{DECEMBER}{18}{\sunmonth{धनुः}{3}{}{मार्गशीर्षः}{हेमन्तऋतुः}{बुधः}{विकारी}{दक्षिणायनम्}{हेमन्तऋतुः}}
{\sunmoonsrdata{06:28}{17:42}{23:42}{11:27}{12:05}
{\kalas{04:45 05:37 09:27 08:43 10:12 16:12 10:57 13:12 15:27 16:57 18:33 20:53 22:29 01:41(+1)}}}
{\tnykdata{\anga{\tithi{22}{कृष्ण-सप्तमी}}{\time{42-37}{23:31}}\hspace{1ex}}%
{\anga{पूर्वफल्गुनी}{\time{43-45}{23:58}}\hspace{1ex}}{चन्द्रराशिः—\mbox{सिंहः\RIGHTarrow\textsf{05:36(+1)}}}%
{\anga{प्रीतिः}{\time{34-14}{20:09}}\hspace{1ex}\uanga{आयुष्मान्}}%
{\anga{विष्टिः}{\time{15-16}{12:34}}\hspace{1ex}\anga{बवः}{\time{42-37}{23:31}}\hspace{1ex}\uanga{बालवः}}{}
}
{कुचेल-दिनम्\eventsep मार्गशीर्ष-अष्टका-पूर्वेद्युः}
{Wed} 
\cfoot{\rygdata{12:05--13:29}{07:52--09:16}{10:40--12:05}}
\caldata{DECEMBER}{19}{\sunmonth{धनुः}{4}{}{मार्गशीर्षः}{हेमन्तऋतुः}{गुरुः}{विकारी}{दक्षिणायनम्}{हेमन्तऋतुः}}
{\sunmoonsrdata{06:28}{17:42}{00:38(+1)}{12:12}{12:05}
{\kalas{04:46 05:37 09:28 08:43 10:13 16:12 10:58 13:13 15:27 16:57 18:33 20:54 22:30 01:41(+1)}}}
{\tnykdata{\anga{\tithi{23}{कृष्ण-अष्टमी}}{\time{37-15}{21:23}}\hspace{1ex}}%
{\anga{उत्तरफल्गुनी}{\time{40-7}{22:31}}\hspace{1ex}}{चन्द्रराशिः—\mbox{कन्या}}%
{\anga{आयुष्मान्}{\time{26-49}{17:12}}\hspace{1ex}\uanga{सौभाग्यः}}%
{\anga{बालवः}{\time{9-55}{10:27}}\hspace{1ex}\anga{कौलवः}{\time{37-15}{21:23}}\hspace{1ex}\uanga{तैतिलः}}{}
}
{\tamil{இயற்பகை நாயனார் (2) குருபூஜை}\eventsep काञ्ची ४ जगद्गुरु श्री-सत्यबोधेन्द्र सरस्वती आराधना~\#{२२८७}\eventsep मार्गशीर्ष-अष्टका-श्राद्धम्}
{Thu} 
\cfoot{\rygdata{13:30--14:54}{06:28--07:53}{09:17--10:41}}
\caldata{DECEMBER}{20}{\sunmonth{धनुः}{5}{}{मार्गशीर्षः}{हेमन्तऋतुः}{शुक्रः}{विकारी}{दक्षिणायनम्}{हेमन्तऋतुः}}
{\sunmoonsrdata{06:29}{17:43}{01:34(+1)}{12:56}{12:06}
{\kalas{04:47 05:38 09:28 08:44 10:13 16:13 10:58 13:13 15:28 16:58 18:34 20:54 22:30 01:42(+1)}}}
{\tnykdata{\anga{\tithi{24}{कृष्ण-नवमी}}{\time{31-59}{19:16}}\hspace{1ex}}%
{\anga{हस्तः}{\time{36-34}{21:06}}\hspace{1ex}}{चन्द्रराशिः—\mbox{कन्या}}%
{\anga{सौभाग्यः}{\time{19-27}{14:16}}\hspace{1ex}\uanga{शोभनः}}%
{\anga{तैतिलः}{\time{4-36}{08:19}}\hspace{1ex}\anga{गरः}{\time{31-59}{19:16}}\hspace{1ex}\anga{वणिजः}{\time{59-25}{06:15(+1)}}\hspace{1ex}\uanga{विष्टिः}}{}
}
{मार्गशीर्ष-अन्वष्टका-श्राद्धम्\eventsep श्री-शेषाद्रि-स्वामि-आराधना~\#{९१}}
{Fri} 
\cfoot{\rygdata{10:41--12:06}{14:54--16:19}{07:53--09:17}}
\caldata{DECEMBER}{21}{\sunmonth{धनुः}{6}{}{मार्गशीर्षः}{हेमन्तऋतुः}{शनिः}{विकारी}{दक्षिणायनम्}{हेमन्तऋतुः}}
{\sunmoonsrdata{06:29}{17:43}{02:30(+1)}{13:41}{12:06}
{\kalas{04:47 05:38 09:29 08:44 10:14 16:13 10:59 13:14 15:28 16:58 18:34 20:55 22:31 01:42(+1)}}}
{\tnykdata{\anga{\tithi{25}{कृष्ण-दशमी}}{\time{26-53}{17:15}}\hspace{1ex}}%
{\anga{चित्रा}{\time{33-12}{19:46}}\hspace{1ex}}{चन्द्रराशिः—\mbox{कन्या\RIGHTarrow\textsf{08:26}}}%
{\anga{शोभनः}{\time{12-12}{11:22}}\hspace{1ex}\uanga{अतिगण्डः}}%
{\anga{विष्टिः}{\time{26-53}{17:15}}\hspace{1ex}\anga{बवः}{\time{54-29}{04:17(+1)}}\hspace{1ex}\uanga{बालवः}}{}
}
{}
{Sat} 
\cfoot{\rygdata{09:18--10:42}{13:31--14:55}{06:29--07:54}}
\caldata{DECEMBER}{22}{\sunmonth{धनुः}{7}{}{मार्गशीर्षः}{हेमन्तऋतुः}{भानुः}{विकारी}{दक्षिणायनम्}{हेमन्तऋतुः}}
{\sunmoonsrdata{06:30}{17:44}{03:28(+1)}{14:28}{12:07}
{\kalas{04:48 05:39 09:29 08:45 10:14 16:14 10:59 13:14 15:29 16:59 18:35 20:55 22:31 01:43(+1)}}}
{\tnykdata{\anga{\tithi{26}{कृष्ण-एकादशी}}{\time{22-10}{15:22}}\hspace{1ex}}%
{\anga{स्वाती}{\time{30-14}{18:35}}\hspace{1ex}}{चन्द्रराशिः—\mbox{तुला}}%
{\anga{अतिगण्डः}{\time{5-13}{08:35}}\hspace{1ex}\anga{सुकर्म}{\time{58-37}{05:57(+1)}}\hspace{1ex}\uanga{धृतिः}}%
{\anga{बालवः}{\time{22-10}{15:22}}\hspace{1ex}\anga{कौलवः}{\time{50-0}{02:30(+1)}}\hspace{1ex}\uanga{तैतिलः}}{}
}
{गणितज्ञ-रामानुज-जन्म~\#{१३२}\eventsep हरिवासरः{\RIGHTarrow}\textsf{20:55}\eventsep \tamil{மானக்கஞ்சாற நாயனார் (11) குருபூஜை}\eventsep सहो-मासः/दक्षिणायनम्{\RIGHTarrow}\textsf{09:49}\eventsep सर्व-सफला-एकादशी\eventsep उत्तरायणारम्भः\eventsep उत्तरायण-पुण्यकालः~\textsf{09:49}{\RIGHTarrow}\textsf{17:49}}
{Sun} 
\cfoot{\rygdata{16:20--17:44}{12:07--13:31}{14:55--16:20}}
\caldata{DECEMBER}{23}{\sunmonth{धनुः}{8}{}{मार्गशीर्षः}{हेमन्तऋतुः}{सोमः}{विकारी}{दक्षिणायनम्}{हेमन्तऋतुः}}
{\sunmoonsrdata{06:30}{17:44}{04:27(+1)}{15:18}{12:07}
{\kalas{04:48 05:39 09:30 08:45 10:15 16:14 11:00 13:15 15:29 16:59 18:35 20:56 22:32 01:43(+1)}}}
{\tnykdata{\anga{\tithi{27}{कृष्ण-द्वादशी}}{\time{17-58}{13:42}}\hspace{1ex}}%
{\anga{विशाखा}{\time{27-47}{17:37}}\hspace{1ex}}{चन्द्रराशिः—\mbox{तुला\RIGHTarrow\textsf{11:50}}}%
{\anga{धृतिः}{\time{52-30}{03:31(+1)}}\hspace{1ex}\uanga{शूलः}}%
{\anga{तैतिलः}{\time{17-58}{13:42}}\hspace{1ex}\anga{गरः}{\time{46-8}{00:58(+1)}}\hspace{1ex}\uanga{वणिजः}}{}
}
{काञ्ची ६८ जगद्गुरु श्री-चन्द्रशेखरेन्द्र सरस्वती ७ आराधना~\#{२६}\eventsep सोम-प्रदोष-व्रतम्}
{Mon} 
\cfoot{\rygdata{07:55--09:19}{10:43--12:07}{13:32--14:56}}
\caldata{DECEMBER}{24}{\sunmonth{धनुः}{9}{}{मार्गशीर्षः}{हेमन्तऋतुः}{मङ्गलः}{विकारी}{दक्षिणायनम्}{हेमन्तऋतुः}}
{\sunmoonsrdata{06:31}{17:45}{05:25(+1)}{16:11}{12:08}
{\kalas{04:49 05:40 09:30 08:46 10:15 16:15 11:00 13:15 15:30 17:00 18:36 20:56 22:32 01:44(+1)}}}
{\tnykdata{\anga{\tithi{28}{कृष्ण-त्रयोदशी}}{\time{14-29}{12:18}}\hspace{1ex}}%
{\anga{अनूराधा}{\time{26-4}{16:57}}\hspace{1ex}}{चन्द्रराशिः—\mbox{वृश्चिकः}}%
{\anga{शूलः}{\time{47-4}{01:20(+1)}}\hspace{1ex}\uanga{गण्डः}}%
{\anga{वणिजः}{\time{14-29}{12:18}}\hspace{1ex}\anga{विष्टिः}{\time{43-4}{23:45}}\hspace{1ex}\uanga{शकुनिः}}{}
}
{मासशिवरात्रिः}
{Tue} 
\cfoot{\rygdata{14:56--16:21}{09:19--10:43}{12:08--13:32}}
\caldata{DECEMBER}{25}{\sunmonth{धनुः}{10}{}{मार्गशीर्षः}{हेमन्तऋतुः}{बुधः}{विकारी}{दक्षिणायनम्}{हेमन्तऋतुः}}
{\sunmoonsrdata{06:31}{17:45}{06:22(+1)}{17:06}{12:08}
{\kalas{04:49 05:40 09:31 08:46 10:16 16:15 11:01 13:16 15:30 17:00 18:36 20:57 22:33 01:44(+1)}}}
{\tnykdata{\anga{\tithi{29}{कृष्ण-चतुर्दशी}}{\time{11-54}{11:17}}\hspace{1ex}}%
{\anga{ज्येष्ठा}{\time{25-18}{16:39}}\hspace{1ex}}{चन्द्रराशिः—\mbox{वृश्चिकः\RIGHTarrow\textsf{16:39}}}%
{\anga{गण्डः}{\time{42-25}{23:30}}\hspace{1ex}\uanga{वृद्धिः}}%
{\anga{शकुनिः}{\time{11-54}{11:17}}\hspace{1ex}\anga{चतुष्पात्}{\time{41-2}{22:56}}\hspace{1ex}\uanga{नाग}}{}
}
{काञ्ची १४ जगद्गुरु श्री-विद्याघनेन्द्र सरस्वती आराधना~\#{१७०३}\eventsep काञ्ची ३४ जगद्गुरु श्री-चन्द्रशेखरेन्द्र सरस्वती २ आराधना~\#{१३१०}\eventsep मार्गशीर्ष-अमावास्या\eventsep पार्वणव्रतम् अमावास्यायाम्\eventsep \tamil{தொண்டரடிப்பொடியாழ்வார் திருநக்ஷத்திரம்}}
{Wed} 
\cfoot{\rygdata{12:08--13:33}{07:56--09:20}{10:44--12:08}}
\caldata{DECEMBER}{26}{\sunmonth{धनुः}{11}{}{मार्गशीर्षः}{हेमन्तऋतुः}{गुरुः}{विकारी}{दक्षिणायनम्}{हेमन्तऋतुः}}
{\sunmoonsrdata{06:32}{17:46}{---}{18:02}{12:09}
{\kalas{04:50 05:41 09:31 08:47 10:16 16:16 11:01 13:16 15:31 17:01 18:37 20:57 22:33 01:45(+1)}}}
{\tnykdata{\anga{\tithi{30}{अमावास्या}}{\time{10-27}{10:43}}\hspace{1ex}}%
{\anga{मूला}{\time{25-39}{16:48}}\hspace{1ex}}{चन्द्रराशिः—\mbox{धनुः}}%
{\anga{वृद्धिः}{\time{38-44}{22:01}}\hspace{1ex}\uanga{ध्रुवः}}%
{\anga{नाग}{\time{10-27}{10:43}}\hspace{1ex}\anga{किंस्तुघ्नः}{\time{40-12}{22:37}}\hspace{1ex}\uanga{बवः}}{}
}
{दर्शेष्टिः\eventsep सूर्य-ग्रहणं (केतुग्रस्त)~\textsf{08:08}{\RIGHTarrow}\textsf{11:19}\eventsep स्थालीपाकः\eventsep श्री-हनूमत्-जयन्ती}
{Thu} 
\cfoot{\rygdata{13:33--14:57}{06:32--07:56}{09:20--10:44}}
\caldata{DECEMBER}{27}{\sunmonth{धनुः}{12}{}{पौषः}{हेमन्तऋतुः}{शुक्रः}{विकारी}{दक्षिणायनम्}{हेमन्तऋतुः}}
{\sunmoonrsdata{06:32}{17:46}{07:16}{18:58}{12:09}
{\kalas{04:50 05:41 09:32 08:47 10:17 16:16 11:02 13:17 15:31 17:01 18:37 20:58 22:34 01:45(+1)}}}
{\tnykdata{\anga{\tithi{1}{शुक्ल-प्रथमा}}{\time{10-17}{10:39}}\hspace{1ex}}%
{\anga{पूर्वाषाढा}{\time{27-18}{17:27}}\hspace{1ex}}{चन्द्रराशिः—\mbox{धनुः\RIGHTarrow\textsf{23:43}}}%
{\anga{ध्रुवः}{\time{36-5}{20:58}}\hspace{1ex}\uanga{व्याघातः}}%
{\anga{बवः}{\time{10-17}{10:39}}\hspace{1ex}\anga{बालवः}{\time{40-44}{22:50}}\hspace{1ex}\uanga{कौलवः}}{}
}
{\tamil{சாக்கிய நாயனார் (33) குருபூஜை}\eventsep चन्द्र-दर्शनम्}
{Fri} 
\cfoot{\rygdata{10:45--12:09}{14:58--16:22}{07:56--09:21}}
\caldata{DECEMBER}{28}{\sunmonth{धनुः}{13}{}{पौषः}{हेमन्तऋतुः}{शनिः}{विकारी}{दक्षिणायनम्}{हेमन्तऋतुः}}
{\sunmoonrsdata{06:33}{17:47}{08:06}{19:51}{12:10}
{\kalas{04:51 05:42 09:32 08:48 10:17 16:17 11:02 13:17 15:32 17:02 18:38 20:58 22:34 01:46(+1)}}}
{\tnykdata{\anga{\tithi{2}{शुक्ल-द्वितीया}}{\time{11-32}{11:10}}\hspace{1ex}}%
{\anga{उत्तराषाढा}{\time{30-19}{18:41}}\hspace{1ex}}{चन्द्रराशिः—\mbox{मकरः}}%
{\anga{व्याघातः}{\time{34-33}{20:22}}\hspace{1ex}\uanga{हर्षणः}}%
{\anga{कौलवः}{\time{11-32}{11:10}}\hspace{1ex}\anga{तैतिलः}{\time{42-43}{23:38}}\hspace{1ex}\uanga{गरः}}{}
}
{}
{Sat} 
\cfoot{\rygdata{09:21--10:45}{13:34--14:58}{06:33--07:57}}
\caldata{DECEMBER}{29}{\sunmonth{धनुः}{14}{}{पौषः}{हेमन्तऋतुः}{भानुः}{विकारी}{दक्षिणायनम्}{हेमन्तऋतुः}}
{\sunmoonrsdata{06:33}{17:47}{08:52}{20:43}{12:10}
{\kalas{04:51 05:42 09:33 08:48 10:18 16:18 11:03 13:18 15:33 17:02 18:39 20:59 22:35 01:46(+1)}}}
{\tnykdata{\anga{\tithi{3}{शुक्ल-तृतीया}}{\time{14-15}{12:15}}\hspace{1ex}}%
{\anga{श्रवणः}{\time{34-45}{20:27}}\hspace{1ex}}{चन्द्रराशिः—\mbox{मकरः}}%
{\anga{हर्षणः}{\time{34-7}{20:12}}\hspace{1ex}\uanga{वज्रम्}}%
{\anga{गरः}{\time{14-15}{12:15}}\hspace{1ex}\anga{वणिजः}{\time{46-9}{01:01(+1)}}\hspace{1ex}\uanga{विष्टिः}}{}
}
{श्रवण-व्रतम्}
{Sun} 
\cfoot{\rygdata{16:23--17:47}{12:10--13:35}{14:59--16:23}}
\caldata{DECEMBER}{30}{\sunmonth{धनुः}{15}{}{पौषः}{हेमन्तऋतुः}{सोमः}{विकारी}{दक्षिणायनम्}{हेमन्तऋतुः}}
{\sunmoonrsdata{06:33}{17:48}{09:34}{21:31}{12:11}
{\kalas{04:51 05:42 09:33 08:48 10:18 16:18 11:03 13:18 15:33 17:03 18:39 21:00 22:35 01:47(+1)}}}
{\tnykdata{\anga{\tithi{4}{शुक्ल-चतुर्थी}}{\time{18-22}{13:54}}\hspace{1ex}}%
{\anga{श्रविष्ठा}{\time{40-26}{22:44}}\hspace{1ex}}{चन्द्रराशिः—\mbox{मकरः\RIGHTarrow\textsf{09:32}}}%
{\anga{वज्रम्}{\time{34-41}{20:26}}\hspace{1ex}\uanga{सिद्धिः}}%
{\anga{विष्टिः}{\time{18-22}{13:54}}\hspace{1ex}\anga{बवः}{\time{50-53}{02:55(+1)}}\hspace{1ex}\uanga{बालवः}}{}
}
{}
{Mon} 
\cfoot{\rygdata{07:58--09:22}{10:46--12:11}{13:35--14:59}}
\caldata{DECEMBER}{31}{\sunmonth{धनुः}{16}{}{पौषः}{हेमन्तऋतुः}{मङ्गलः}{विकारी}{दक्षिणायनम्}{हेमन्तऋतुः}}
{\sunmoonrsdata{06:34}{17:49}{10:13}{22:19}{12:11}
{\kalas{04:52 05:43 09:34 08:49 10:19 16:19 11:04 13:19 15:34 17:04 18:40 21:00 22:36 01:47(+1)}}}
{\tnykdata{\anga{\tithi{5}{शुक्ल-पञ्चमी}}{\time{23-38}{16:01}}\hspace{1ex}}%
{\anga{शतभिषक्}{\time{47-7}{01:25(+1)}}\hspace{1ex}}{चन्द्रराशिः—\mbox{कुम्भः}}%
{\anga{सिद्धिः}{\time{36-3}{21:00}}\hspace{1ex}\uanga{व्यतीपातः}}%
{\anga{बालवः}{\time{23-38}{16:01}}\hspace{1ex}\anga{कौलवः}{\time{56-36}{05:13(+1)}}\hspace{1ex}\uanga{तैतिलः}}{}
}
{}
{Tue} 
\cfoot{\rygdata{15:00--16:24}{09:23--10:47}{12:11--13:36}}
\end{document}
