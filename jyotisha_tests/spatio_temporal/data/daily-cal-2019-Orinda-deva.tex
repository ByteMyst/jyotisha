% !Tex program = xelatex
\documentclass[12pt]{article}
\usepackage[dvipsnames]{xcolor} 
\usepackage[paperwidth=135mm,paperheight=180mm,left=4mm,right=4mm,top=8mm,bottom=8mm]{geometry}
\usepackage[xetex]{graphicx}
\usepackage{array}
\usepackage{setspace}
\usepackage{multirow}
\usepackage{pxfonts}
\usepackage{bbding}
\usepackage{wasysym} 
\usepackage{fontspec}
\usepackage{multicol}
\usepackage{supertabular}
\usepackage{fancyhdr}
\pagestyle{fancy}
\fancyhf{}
\rhead{}
\lhead{}
\cfoot{}
\usepackage{../templates/listofitems}
\newcommand{\yearname}{2015}
%%%%%%%%%%%%%%%%%%%%%%%%%%%%%%%%%%%%%%%%%%%%%%%%%%%%%%%%%%%%%%%%%%%%%%%%%%%%%%%
%% MOONPHASE CODE 
%%%%%%%%%%%%%%%%%%%%%%%%%%%%%%%%%%%%%%%%%%%%%%%%%%%%%%%%%%%%%%%%%%%%%%%%%%%%%%%
%Credits: http://tex.stackexchange.com/questions/34785/how-to-typeset-moon-phase-symbols (Jake!)
\usepackage{tikz}
\usetikzlibrary{calendar,fpu}

\tikzset{
    moon colour/.style={
        moon fill/.style={
            fill=#1
        }
    },
    sky colour/.style={
        sky draw/.style={
            draw=#1
        },
        sky fill/.style={
            fill=#1
        }
    },
    southern hemisphere/.style={
        rotate=180
    }
}

\makeatletter
\pgfcalendardatetojulian{2010-01-15}{\c@pgf@counta} % 2010-01-15 07:11 UTC -- http://aa.usno.navy.mil/cgi-bin/aa_moonphases.pl?year=2010&ZZZ=END
\def\synodicmonth{29.530588853}
\newcommand{\moon}[2][]{%
    \edef\checkfordate{\noexpand\in@{-}{#2}}%
    \checkfordate%
    \ifin@%
        \pgfcalendardatetojulian{#2}{\c@pgf@countb}%
        \pgfkeys{/pgf/fpu=true,/pgf/fpu/output format=fixed}%
        \pgfmathsetmacro\dayssincenewmoon{\the\c@pgf@countb-\the\c@pgf@counta-(7/24+11/(24*60))}%
        \pgfmathsetmacro\lunarage{mod(\dayssincenewmoon,\synodicmonth)}
        \pgfkeys{/pgf/fpu=false}%%
    \else%
        \def\lunarage{#2}%
    \fi%
    \pgfmathsetmacro\leftside{ifthenelse(\lunarage<=\synodicmonth/2,cos(360*(\lunarage/\synodicmonth)),1)}%
    \pgfmathsetmacro\rightside{ifthenelse(\lunarage<=\synodicmonth/2,-1,-cos(360*(\lunarage/\synodicmonth))}%
    \tikz [moon colour=white,sky colour=black,#1]{
        \draw [moon fill, sky draw] (0,0) circle [radius=1ex];
        \draw [sky draw, sky fill] (0,1ex)
            arc (90:-90:\rightside ex and 1ex)
            arc (-90:90:\leftside ex and 1ex)
            -- cycle;
    }%
}
%%%%%%%%%%%%%%%%%%%%%%%%%%%%%%%%%%%%%%%%%%%%%%%%%%%%%%%%%%%%%%%%%%%%%%%%%%%%%%%
%% END MOONPHASE CODE
%%%%%%%%%%%%%%%%%%%%%%%%%%%%%%%%%%%%%%%%%%%%%%%%%%%%%%%%%%%%%%%%%%%%%%%%%%%%%%%
% \setlength{\footskip}{2mm}
% PDF SETUP
% ---- FILL IN HERE THE DOC TITLE AND AUTHOR
\defaultfontfeatures{Scale=MatchLowercase,Mapping=tex-text}
\setmainfont{siddhanta.ttf}[Path=../fonts/,Script=Devanagari] 
\setsansfont[Path=../fonts/,Scale=0.95,Numbers=Lining]{AlegreyaSans-Regular.ttf}
% \newfontfamily\noto[Path=../fonts/, Ligatures=TeX]{NotoSansUI-Regular}
%%%%%%% Numbers and counters %%%%%%%
\newcount\num
\newcount\tempone \newcount\temptwo
\newcommand{\devanumber}[1]{%
\num=#1\devanumberrecurse}
\setlength{\textheight}{150mm}
\newcommand{\devanumberrecurse}{%
{\tempone=\num
%  \showthe\tempone\ %
\ifnum\num > 0 
    \divide \num by 10%
    \temptwo=\num \multiply\temptwo by -10%
    \devanumberrecurse%
%   \\stage 2\ %
%   \showthe\temptwo\ %
%   temp1=\number\tempone\ %
%   num=\number\num\ %
    \advance\tempone by \temptwo%
    \devadigit
\fi
}}
\newcommand{\devadigit}{%
\ifcase\tempone०\or१\or२\or३\or४\or५\or६\or७\or८\or९\fi%\number\tempone%
}
\newcommand{\eventsep}{~\raisebox{1pt}{\scriptsize$\Diamondblack$} }
\newcommand{\TO}{\hspace{1pt}\raisebox{1pt}{\footnotesize\RIGHTarrow}\hspace{1pt}}
\newcommand{\To}{\hspace{1pt}\raisebox{1pt}{\footnotesize\RIGHTarrow}\hspace{1pt}}
\newcommand{\Too}{\hspace{1pt}\raisebox{1pt}{\footnotesize\RIGHTarrow\hspace{-5pt}\RIGHTarrow}\hspace{1pt}}
%%%%%%%% Calendar display stuff %%%%%%%%%%%
\newcommand{\samvatsaraName}{}
\newcommand{\solarMonthName}{}
\newcommand{\solarMonthEndTime}{}
\newcommand{\lunarMonthName}{}
\newcommand{\lunarRtu}{}
\newcommand{\solarMonthDate}{}
\newcommand{\vaaraName}{}
\newcommand{\rtuName}{}
\newcommand{\ayanamName}{}

\newcommand{\sunmonth}[9]{%
\renewcommand{\solarMonthName}{#1}
\renewcommand{\solarMonthDate}{#2}
\renewcommand{\solarMonthEndTime}{#3}
\renewcommand{\lunarMonthName}{#4}
\renewcommand{\lunarRtu}{#5}
\renewcommand{\vaaraName}{#6}
\renewcommand{\samvatsaraName}{#7}
\renewcommand{\ayanamName}{#8}
\renewcommand{\rtuName}{#9}
}
\newcommand{\tamil}[1]{%
{\fontspec[Scale=0.8,FakeStretch=0.9,Path=../fonts/]{NotoSansTamil-Regular.ttf} \footnotesize #1}%
}
\newcommand{\kalas}[1]{%
\setsepchar{ }
\readlist\arg{#1}
{\small{\mbox{ब्राह्म\,\textsf{\arg[1]}{\scriptsize\RIGHTarrow}\,सङ्गव\,\textsf{\arg[4]}{\scriptsize\RIGHTarrow}\,मध्याह्न\,\textsf{\arg[7]}{\scriptsize\RIGHTarrow}\,अपराह्ण\,\textsf{\arg[8]}{\scriptsize\RIGHTarrow}\,सायाह्न\,\textsf{\arg[9]}{\scriptsize\RIGHTarrow}}\hfill {दिनान्तः{\scriptsize\RIGHTarrow}\textsf{\arg[14]}}}}\\[-1ex]
{\small{\mbox{प्रातः सन्ध्या \textsf{\arg[2]}{\scriptsize\RIGHTarrow}\textsf{\arg[3]} माध्याह्निक \textsf{\arg[5]}{\scriptsize\RIGHTarrow}\textsf{\arg[6]} 
सायं \textsf{\arg[10]}{\scriptsize\RIGHTarrow}\textsf{\arg[11]}}\hfill शयन \textsf{\arg[12]}{\scriptsize\RIGHTarrow}\textsf{\arg[13]}}\\[-4.5ex]}
}
\newcommand{\sunmoonrsdata}[5]{%
\mbox{\large\sun{\small\UParrow}\textsf{#1}~\sun{\small\DOWNarrow}\textsf{#2}}\hfill
\mbox{\large\rightmoon{\small\UParrow}\textsf{#3}~\rightmoon{\small\DOWNarrow}\textsf{#4}}\\
#5
 }
\newcommand{\sunmoonsrdata}[5]{%
\mbox{\large\sun{\small\UParrow}\textsf{#1}~\sun{\small\DOWNarrow}\textsf{#2}}\hfill
\mbox{\large\rightmoon{\small\DOWNarrow}\textsf{#4}~\rightmoon{\small\UParrow}\textsf{#3}}\\
#5
 }
\newcommand{\ahorAtram}{अहोरात्रम्}
\newcommand{\tithi}[2]{\raisebox{-1pt}{\moon[scale=0.8]{#1}}\hspace{2pt}#2}
\newcommand{\tnykdata}[6]{\large%\fontsize{13pt}{16pt}\selectfont
{#1}\\%Tithi
{नक्षत्रम्–#2 (#3)}\\%Nakshatram and Rashi
{\setstretch{0.55}
\begin{tabular}{@{}r@{}p{108mm}@{}}
योगः–&#4\\[2pt]%Yogam
करणम्–&#5\\%Karanam
\end{tabular}}\mbox{}\\[3pt]
\parbox[c][2ex][c]{0.9\linewidth}{\footnotesize #6}%Lagna, if required
}
\newcommand{\avamA}{
    \raisebox{1.5pt}{\fcolorbox{white}{gray!40}{\scriptsize अवमा}}
}
\newcommand{\tridina}{
    \raisebox{1.5pt}{\fcolorbox{white}{gray!40}{\scriptsize त्रिदिनस्पृक्}}
}
\renewcommand{\time}[2]{#1 (#2)}
\newcommand{\anga}[2]{\mbox{#1\To{}\textsf{#2}}}
\newcommand{\fullanga}[1]{\mbox{#1\To{}\ahorAtram}}
\newcommand{\fulltithi}[1]{\mbox{#1\To{}\ahorAtram\tridina}}
\newcommand{\lagna}[2]{\mbox{#1\RIGHTarrow\textsf{#2}}}
\newcommand{\uanga}[1]{\mbox{#1\Too}}
\newcommand{\rygdata}[3]{%
\begin{minipage}{\linewidth}
\centering
\rule[-1ex]{0.7\textwidth}{.4pt}
\small राहु॰~\textsf{#1}~~यम॰~\textsf{#2}~~गुलिक॰~\textsf{#3}%Rahu Yama Gulika
\end{minipage}
}
\newcommand{\caldata}[7]{%
\clearpage
\begin{minipage}{\linewidth}
#3% Calls \sunmonth
\large% Fixes font size
{\centering\begin{tabular}{c|c}
\large \textsf{\yearname} & {\large\samvatsaraName}\\[-1ex]%YYYY
& {\footnotesize \ayanamName \hspace{6pt} \rtuName}\\[0.2ex]
\mbox{\sffamily\fontsize{20}{25}\selectfont {\uppercase{#1}}} & \parbox[c][14pt][c]{3cm}{\centering\LARGE\solarMonthName}\\[-4pt]%mmm
& {\mbox{\small \solarMonthEndTime}}\\[-2pt]
& {\parbox[c][15pt]{52mm}{\centering\lunarMonthName}}\\[-6pt]
& {\parbox[c][10pt]{52mm}{\centering\scriptsize(\lunarRtu)}}\\[-5pt]
\hspace{0.465\linewidth} & \hspace{0.465\linewidth} \\[-6pt]
\mbox{\sffamily\fontsize{96}{115}\selectfont #2} & \mbox{\fontsize{90}{24}\selectfont \devanumber{\solarMonthDate}}\\[1.6ex]%DD
\mbox{\sffamily\fontsize{24}{28}\selectfont\uppercase{#7}} & \parbox[c][24pt][t]{1cm}{\centering\LARGE\vaaraName}\\[1.2ex]%Day of the week
\hline
\end{tabular}
}\mbox{}\\[-4pt]
#4\\[0.5em]%Sun rise, kalas etc
#5\mbox{}\\[1em]%Tithi, Nakshatram, Varam, Yogam
% \vspace{\fill}
{\parbox[b]{0.95\linewidth}{\centering\normalsize\textcolor{RoyalBlue}{#6}}}%Festivals
\end{minipage}
}

\addtolength{\headsep}{-3ex}
\setlength\parindent{0pt}
\pagestyle{empty}
\begin{document}
\mbox{}
\renewcommand{\yearname}{2019}
\begin{center}
{\sffamily \fontsize{80}{80}\selectfont  2019\\[0.5cm]}
\mbox{\fontsize{48}{48}\selectfont विलम्बः–विकारी}\\
\mbox{\fontsize{32}{32}\selectfont कलि } %
{\sffamily \fontsize{43}{43}\selectfont  5119–5120\\[0.5cm]}
\hrule
\vspace{0.2cm}
{\sffamily \fontsize{50}{50}\selectfont  \uppercase{Orinda}\\[0.2cm]}
{\sffamily \fontsize{23}{23}\selectfont  {37.861°N, 122.183°W}\\[0.2cm]}
\hrule
\end{center}
\clearpage\pagestyle{fancy}
\caldata{JANUARY}{1}{\sunmonth{धनुः}{17}{}{मार्गशीर्षः}{हेमन्तऋतुः}{मङ्गलः}{विलम्बः}{दक्षिणायनम्}{हेमन्तऋतुः}}
{\sunmoonsrdata{07:29}{16:55}{04:27(+1)}{13:49}{12:12}
{\kalas{05:32 06:30 10:00 09:22 10:38 15:40 11:15 13:09 15:02 16:17 17:53 20:34 22:23 02:01(+1)}}}
{\tnykdata{\anga{\tithi{26}{कृष्ण-एकादशी}}{\time{11-13}{11:58}}\hspace{1ex}}%
{\anga{विशाखा}{\time{31-35}{20:07}}\hspace{1ex}}{चन्द्रराशिः—\mbox{तुला\RIGHTarrow{13:50}}}%
{\anga{धृतिः}{\time{14-5}{13:07}}\hspace{1ex}\uanga{शूलः}}%
{\anga{बालवः}{\time{11-13}{11:58}}\hspace{1ex}\anga{कौलवः}{\time{41-57}{00:16(+1)}}\hspace{1ex}\uanga{तैतिलः}}{}
}
{हरिवासरः\RIGHTarrow{}18:06\eventsep काञ्ची ६८ जगद्गुरु श्री-चन्द्रशेखरेन्द्र सरस्वती ७ आराधना~\#{२५}\eventsep सर्व-सफला-एकादशी}
{Tue} 
\cfoot{\rygdata{14:34--15:44}{09:50--11:01}{12:12--13:23}}
\caldata{JANUARY}{2}{\sunmonth{धनुः}{18}{}{मार्गशीर्षः}{हेमन्तऋतुः}{बुधः}{विलम्बः}{दक्षिणायनम्}{हेमन्तऋतुः}}
{\sunmoonsrdata{07:29}{16:56}{05:27(+1)}{14:29}{12:12}
{\kalas{05:32 06:31 10:00 09:22 10:38 15:40 11:16 13:09 15:03 16:18 17:54 20:34 22:23 02:02(+1)}}}
{\tnykdata{\anga{\tithi{27}{कृष्ण-द्वादशी}}{\time{12-58}{12:40}}\hspace{1ex}}%
{\anga{अनूराधा}{\time{35-4}{21:31}}\hspace{1ex}}{चन्द्रराशिः—\mbox{वृश्चिकः}}%
{\anga{शूलः}{\time{12-30}{12:29}}\hspace{1ex}\uanga{गण्डः}}%
{\anga{तैतिलः}{\time{12-58}{12:40}}\hspace{1ex}\anga{गरः}{\time{44-17}{01:12(+1)}}\hspace{1ex}\uanga{वणिजः}}{}
}
{बुधानुराधा-पुण्यकालः\eventsep प्रदोष-व्रतम्~16:56\RIGHTarrow{}17:54\eventsep \tamil{உந்து~மதக்களிற்றன்}}
{Wed} 
\cfoot{\rygdata{12:12--13:23}{08:40--09:51}{11:02--12:12}}
\caldata{JANUARY}{3}{\sunmonth{धनुः}{19}{}{मार्गशीर्षः}{हेमन्तऋतुः}{गुरुः}{विलम्बः}{दक्षिणायनम्}{हेमन्तऋतुः}}
{\sunmoonsrdata{07:29}{16:57}{06:21(+1)}{15:14}{12:13}
{\kalas{05:33 06:31 10:00 09:23 10:38 15:41 11:16 13:10 15:03 16:19 17:55 20:35 22:24 02:02(+1)}}}
{\tnykdata{\anga{\tithi{28}{कृष्ण-त्रयोदशी}}{\time{15-54}{13:51}}\hspace{1ex}}%
{\anga{ज्येष्ठा}{\time{39-39}{23:21}}\hspace{1ex}}{चन्द्रराशिः—\mbox{वृश्चिकः\RIGHTarrow{23:21}}}%
{\anga{गण्डः}{\time{11-48}{12:13}}\hspace{1ex}\uanga{वृद्धिः}}%
{\anga{वणिजः}{\time{15-54}{13:51}}\hspace{1ex}\anga{विष्टिः}{\time{47-47}{02:36(+1)}}\hspace{1ex}\uanga{शकुनिः}}{}
}
{मासशिवरात्रिः\eventsep पञ्च-पर्व-पूजा (चतुर्दशी)\eventsep \tamil{தொண்டரடிப்பொடியாழ்வார் திருநக்ஷத்திரம்}}
{Thu} 
\cfoot{\rygdata{13:24--14:35}{07:29--08:40}{09:51--11:02}}
\caldata{JANUARY}{4}{\sunmonth{धनुः}{20}{}{मार्गशीर्षः}{हेमन्तऋतुः}{शुक्रः}{विलम्बः}{दक्षिणायनम्}{हेमन्तऋतुः}}
{\sunmoonsrdata{07:29}{16:58}{07:10(+1)}{16:04}{12:13}
{\kalas{05:33 06:31 10:01 09:23 10:39 15:42 11:16 13:10 15:04 16:20 17:56 20:35 22:24 02:02(+1)}}}
{\tnykdata{\anga{\tithi{29}{कृष्ण-चतुर्दशी}}{\time{19-55}{15:27}}\hspace{1ex}}%
{\anga{मूला}{\time{45-14}{01:35(+1)}}\hspace{1ex}}{चन्द्रराशिः—\mbox{धनुः}}%
{\anga{वृद्धिः}{\time{11-57}{12:16}}\hspace{1ex}\uanga{ध्रुवः}}%
{\anga{शकुनिः}{\time{19-55}{15:27}}\hspace{1ex}\anga{चतुष्पात्}{\time{52-19}{04:25(+1)}}\hspace{1ex}\uanga{नाग}}{}
}
{पञ्च-पर्व-पूजा (अमावास्या)}
{Fri} 
\cfoot{\rygdata{11:02--12:13}{14:35--15:47}{08:40--09:51}}
\caldata{JANUARY}{5}{\sunmonth{धनुः}{21}{}{मार्गशीर्षः}{हेमन्तऋतुः}{शनिः}{विलम्बः}{दक्षिणायनम्}{हेमन्तऋतुः}}
{\sunmoonsrdata{07:29}{16:58}{---}{16:57}{12:14}
{\kalas{05:33 06:31 10:01 09:23 10:39 15:43 11:17 13:11 15:05 16:21 17:56 20:36 22:25 02:03(+1)}}}
{\tnykdata{\anga{\tithi{30}{अमावास्या}}{\time{24-56}{17:28}}\hspace{1ex}}%
{\anga{पूर्वाषाढा}{\time{51-43}{04:10(+1)}}\hspace{1ex}}{चन्द्रराशिः—\mbox{धनुः}}%
{\anga{ध्रुवः}{\time{12-51}{12:38}}\hspace{1ex}\uanga{व्याघातः}}%
{\anga{नाग}{\time{24-56}{17:28}}\hspace{1ex}\anga{किंस्तुघ्नः}{\time{57-46}{06:36(+1)}}\hspace{1ex}\uanga{बवः}}{}
}
{\tamil{சாக்கிய நாயனார் (33) குருபூஜை}\eventsep काञ्ची १४ जगद्गुरु श्री-विद्याघनेन्द्र सरस्वती आराधना~\#{१७०२}\eventsep काञ्ची ३४ जगद्गुरु श्री-चन्द्रशेखरेन्द्र सरस्वती २ आराधना~\#{१३०९}\eventsep पार्वणव्रतम् अमावास्यायाम्\eventsep सर्व-मार्गशीर्ष-अमावास्या\eventsep श्री-हनूमत्-जयन्ती}
{Sat} 
\cfoot{\rygdata{09:51--11:03}{13:25--14:36}{07:29--08:40}}
\caldata{JANUARY}{6}{\sunmonth{धनुः}{22}{}{पौषः}{हेमन्तऋतुः}{भानुः}{विलम्बः}{दक्षिणायनम्}{हेमन्तऋतुः}}
{\sunmoonrsdata{07:29}{16:59}{07:54}{17:53}{12:14}
{\kalas{05:33 06:31 10:01 09:23 10:39 15:43 11:17 13:11 15:05 16:21 17:57 20:37 22:25 02:03(+1)}}}
{\tnykdata{\anga{\tithi{1}{शुक्ल-प्रथमा}}{\time{30-47}{19:48}}\hspace{1ex}}%
{\anga{उत्तराषाढा}{\time{58-55}{07:03(+1)}}\hspace{1ex}}{चन्द्रराशिः—\mbox{धनुः\RIGHTarrow{10:52}}}%
{\anga{व्याघातः}{\time{14-23}{13:14}}\hspace{1ex}\uanga{हर्षणः}}%
{\anga{बवः}{\time{30-47}{19:48}}\hspace{1ex}\uanga{बालवः}}{}
}
{दर्शेष्टिः\eventsep पार्वण-प्रायश्चित्तावकाशः पौर्णमास्याम्\eventsep स्थालीपाकः}
{Sun} 
\cfoot{\rygdata{15:48--16:59}{12:14--13:25}{14:37--15:48}}
\caldata{JANUARY}{7}{\sunmonth{धनुः}{23}{}{पौषः}{हेमन्तऋतुः}{सोमः}{विलम्बः}{दक्षिणायनम्}{हेमन्तऋतुः}}
{\sunmoonrsdata{07:29}{17:00}{08:31}{18:51}{12:15}
{\kalas{05:33 06:31 10:01 09:23 10:39 15:44 11:18 13:12 15:06 16:22 17:58 20:37 22:26 02:03(+1)}}}
{\tnykdata{\anga{\tithi{2}{शुक्ल-द्वितीया}}{\time{37-17}{22:24}}\hspace{1ex}}%
{\fullanga{श्रवणः}}{चन्द्रराशिः—\mbox{मकरः}}%
{\anga{हर्षणः}{\time{16-26}{14:03}}\hspace{1ex}\uanga{वज्रम्}}%
{\anga{बालवः}{\time{3-58}{09:05}}\hspace{1ex}\anga{कौलवः}{\time{37-17}{22:24}}\hspace{1ex}\uanga{तैतिलः}}{}
}
{चन्द्र-दर्शनम्~17:00\RIGHTarrow{}17:58\eventsep सोमश्रावणी-पुण्यकालः\eventsep श्रवण-व्रतम्}
{Mon} 
\cfoot{\rygdata{08:40--09:52}{11:03--12:15}{13:26--14:37}}
\caldata{JANUARY}{8}{\sunmonth{धनुः}{24}{}{पौषः}{हेमन्तऋतुः}{मङ्गलः}{विलम्बः}{दक्षिणायनम्}{हेमन्तऋतुः}}
{\sunmoonrsdata{07:29}{17:01}{09:05}{19:48}{12:15}
{\kalas{05:33 06:31 10:02 09:23 10:40 15:45 11:18 13:12 15:07 16:23 17:59 20:38 22:27 02:03(+1)}}}
{\tnykdata{\anga{\tithi{3}{शुक्ल-तृतीया}}{\time{44-7}{01:08(+1)}}\hspace{1ex}}%
{\anga{श्रवणः}{\time{6-37}{10:08}}\hspace{1ex}}{चन्द्रराशिः—\mbox{मकरः\RIGHTarrow{23:43}}}%
{\anga{वज्रम्}{\time{18-49}{15:01}}\hspace{1ex}\uanga{सिद्धिः}}%
{\anga{तैतिलः}{\time{10-41}{11:45}}\hspace{1ex}\anga{गरः}{\time{44-7}{01:08(+1)}}\hspace{1ex}\uanga{वणिजः}}{}
}
{}
{Tue} 
\cfoot{\rygdata{14:38--15:50}{09:52--11:04}{12:15--13:27}}
\caldata{JANUARY}{9}{\sunmonth{धनुः}{25}{}{पौषः}{हेमन्तऋतुः}{बुधः}{विलम्बः}{दक्षिणायनम्}{हेमन्तऋतुः}}
{\sunmoonrsdata{07:29}{17:02}{09:34}{20:45}{12:16}
{\kalas{05:33 06:31 10:02 09:24 10:40 15:46 11:18 13:13 15:08 16:24 18:00 20:39 22:27 02:04(+1)}}}
{\tnykdata{\anga{\tithi{4}{शुक्ल-चतुर्थी}}{\time{50-56}{03:52(+1)}}\hspace{1ex}}%
{\anga{श्रविष्ठा}{\time{14-30}{13:17}}\hspace{1ex}}{चन्द्रराशिः—\mbox{कुम्भः}}%
{\anga{सिद्धिः}{\time{21-18}{16:00}}\hspace{1ex}\uanga{व्यतीपातः}}%
{\anga{वणिजः}{\time{17-34}{14:31}}\hspace{1ex}\anga{विष्टिः}{\time{50-56}{03:52(+1)}}\hspace{1ex}\uanga{बवः}}{}
}
{प्रवासि-भारतीय-दिवसम्~\#{१६}}
{Wed} 
\cfoot{\rygdata{12:15--13:27}{08:40--09:52}{11:04--12:15}}
\caldata{JANUARY}{10}{\sunmonth{धनुः}{26}{}{पौषः}{हेमन्तऋतुः}{गुरुः}{विलम्बः}{दक्षिणायनम्}{हेमन्तऋतुः}}
{\sunmoonrsdata{07:29}{17:03}{10:02}{21:41}{12:16}
{\kalas{05:33 06:31 10:02 09:24 10:40 15:47 11:18 13:13 15:08 16:25 18:01 20:39 22:28 02:04(+1)}}}
{\tnykdata{\anga{\tithi{5}{शुक्ल-पञ्चमी}}{\time{57-19}{06:24(+1)}}\hspace{1ex}}%
{\anga{शतभिषक्}{\time{22-12}{16:22}}\hspace{1ex}}{चन्द्रराशिः—\mbox{कुम्भः}}%
{\anga{व्यतीपातः}{\time{23-36}{16:55}}\hspace{1ex}\uanga{वरीयान्}}%
{\anga{बवः}{\time{24-13}{17:10}}\hspace{1ex}\anga{बालवः}{\time{57-19}{06:24(+1)}}\hspace{1ex}\uanga{कौलवः}}{}
}
{महाधनुर्व्यतीपात-स्नानम्\eventsep महाधनुर्व्यतीपात-श्राद्धम्}
{Thu} 
\cfoot{\rygdata{13:28--14:40}{07:29--08:40}{09:52--11:04}}
\caldata{JANUARY}{11}{\sunmonth{धनुः}{27}{}{पौषः}{हेमन्तऋतुः}{शुक्रः}{विलम्बः}{दक्षिणायनम्}{हेमन्तऋतुः}}
{\sunmoonrsdata{07:28}{17:04}{10:27}{22:38}{12:16}
{\kalas{05:33 06:31 10:02 09:24 10:40 15:47 11:19 13:14 15:09 16:26 18:02 20:40 22:28 02:04(+1)}}}
{\tnykdata{\fulltithi{\tithi{6}{शुक्ल-षष्ठी}}}%
{\anga{पूर्वप्रोष्ठपदा}{\time{29-15}{19:11}}\hspace{1ex}}{चन्द्रराशिः—\mbox{कुम्भः\RIGHTarrow{12:30}}}%
{\anga{वरीयान्}{\time{25-23}{17:38}}\hspace{1ex}\uanga{परिघः}}%
{\anga{कौलवः}{\time{30-11}{19:33}}\hspace{1ex}\uanga{तैतिलः}}{}
}
{षष्ठी-व्रतम्\eventsep \tamil{கூடாரவல்லீ}}
{Fri} 
\cfoot{\rygdata{11:04--12:16}{14:40--15:52}{08:40--09:52}}
\caldata{JANUARY}{12}{\sunmonth{धनुः}{28}{}{पौषः}{हेमन्तऋतुः}{शनिः}{विलम्बः}{दक्षिणायनम्}{हेमन्तऋतुः}}
{\sunmoonrsdata{07:28}{17:05}{10:53}{23:36}{12:17}
{\kalas{05:33 06:31 10:02 09:24 10:41 15:48 11:19 13:14 15:10 16:27 18:03 20:41 22:29 02:04(+1)}}}
{\tnykdata{\anga{\tithi{6}{शुक्ल-षष्ठी}}{\time{2-45}{08:35}}\hspace{1ex}}%
{\anga{उत्तरप्रोष्ठपदा}{\time{35-12}{21:33}}\hspace{1ex}}{चन्द्रराशिः—\mbox{मीनः}}%
{\anga{परिघः}{\time{26-21}{18:01}}\hspace{1ex}\uanga{शिवः}}%
{\anga{तैतिलः}{\time{2-45}{08:35}}\hspace{1ex}\anga{गरः}{\time{34-59}{21:28}}\hspace{1ex}\uanga{वणिजः}}{}
}
{\tamil{கறவைகள் பின்சென்று}}
{Sat} 
\cfoot{\rygdata{09:52--11:05}{13:29--14:41}{07:28--08:40}}
\caldata{JANUARY}{13}{\sunmonth{धनुः}{29}{\mbox{धनुः{\tiny\RIGHTarrow}{05:58(+1)}}}{पौषः}{हेमन्तऋतुः}{भानुः}{विलम्बः}{दक्षिणायनम्}{हेमन्तऋतुः}}
{\sunmoonrsdata{07:28}{17:06}{11:20}{00:36(+1)}{12:17}
{\kalas{05:33 06:31 10:02 09:24 10:41 15:49 11:19 13:15 15:11 16:28 18:04 20:42 22:29 02:05(+1)}}}
{\tnykdata{\anga{\tithi{7}{शुक्ल-सप्तमी}}{\time{6-49}{10:12}}\hspace{1ex}}%
{\anga{रेवती}{\time{39-40}{23:20}}\hspace{1ex}}{चन्द्रराशिः—\mbox{मीनः\RIGHTarrow{23:20}}}%
{\anga{शिवः}{\time{26-10}{17:56}}\hspace{1ex}\uanga{सिद्धः}}%
{\anga{वणिजः}{\time{6-49}{10:12}}\hspace{1ex}\anga{विष्टिः}{\time{38-12}{22:45}}\hspace{1ex}\uanga{बवः}}{}
}
{\tamil{போகி}\eventsep महाजया-सप्तमी\eventsep मकर-सङ्क्रमण-पुण्यकालः~05:58(+1)\RIGHTarrow{}13:58(+1)\eventsep \tamil{வாயிலார் நாயனார் (49) குருபூஜை}\eventsep विजया-भानुसप्तमी★}
{Sun} 
\cfoot{\rygdata{15:54--17:06}{12:17--13:29}{14:42--15:54}}
\caldata{JANUARY}{14}{\sunmonth{मकरः}{1}{}{पौषः}{हेमन्तऋतुः}{सोमः}{विलम्बः}{उत्तरायणम्}{हेमन्तऋतुः}}
{\sunmoonrsdata{07:28}{17:07}{11:50}{01:38(+1)}{12:18}
{\kalas{05:33 06:30 10:02 09:24 10:41 15:50 11:20 13:15 15:11 16:29 18:05 20:42 22:30 02:05(+1)}}}
{\tnykdata{\anga{\tithi{8}{शुक्ल-अष्टमी}}{\time{9-8}{11:07}}\hspace{1ex}}%
{\anga{अश्विनी}{\time{42-19}{00:24(+1)}}\hspace{1ex}}{चन्द्रराशिः—\mbox{मेषः}}%
{\anga{सिद्धः}{\time{24-36}{17:18}}\hspace{1ex}\uanga{साध्यः}}%
{\anga{बवः}{\time{9-8}{11:07}}\hspace{1ex}\anga{बालवः}{\time{39-33}{23:17}}\hspace{1ex}\uanga{कौलवः}}{}
}
{\tamil{மதுரை மீனாக்ஷீ கோயிலில் கல் யானைக்கு கரும்பு கோடுத்த லீலை}\eventsep मकर-ज्योतिः\eventsep मकर-सङ्क्रान्तिः}
{Mon} 
\cfoot{\rygdata{08:40--09:53}{11:05--12:18}{13:30--14:42}}
\caldata{JANUARY}{15}{\sunmonth{मकरः}{2}{}{पौषः}{हेमन्तऋतुः}{मङ्गलः}{विलम्बः}{उत्तरायणम्}{हेमन्तऋतुः}}
{\sunmoonrsdata{07:27}{17:08}{12:23}{02:43(+1)}{12:18}
{\kalas{05:33 06:30 10:02 09:24 10:41 15:51 11:20 13:16 15:12 16:30 18:06 20:43 22:30 02:05(+1)}}}
{\tnykdata{\anga{\tithi{9}{शुक्ल-नवमी}}{\time{9-28}{11:15}}\hspace{1ex}}%
{\anga{अपभरणी}{\time{43-0}{00:40(+1)}}\hspace{1ex}}{चन्द्रराशिः—\mbox{मेषः\RIGHTarrow{06:36(+1)}}}%
{\anga{साध्यः}{\time{21-29}{16:03}}\hspace{1ex}\uanga{शुभः}}%
{\anga{कौलवः}{\time{9-28}{11:15}}\hspace{1ex}\anga{तैतिलः}{\time{38-51}{23:00}}\hspace{1ex}\uanga{गरः}}{}
}
{इन्द्र-पूजा/गो-पूजा\eventsep \tamil{கனுப்~பொங்கல்}}
{Tue} 
\cfoot{\rygdata{14:43--15:56}{09:53--11:05}{12:18--13:31}}
\caldata{JANUARY}{16}{\sunmonth{मकरः}{3}{}{पौषः}{हेमन्तऋतुः}{बुधः}{विलम्बः}{उत्तरायणम्}{हेमन्तऋतुः}}
{\sunmoonrsdata{07:27}{17:09}{13:02}{03:50(+1)}{12:18}
{\kalas{05:33 06:30 10:02 09:24 10:41 15:52 11:20 13:16 15:13 16:31 18:07 20:44 22:31 02:05(+1)}}}
{\tnykdata{\anga{\tithi{10}{शुक्ल-दशमी}}{\time{7-45}{10:33}}\hspace{1ex}}%
{\anga{कृत्तिका}{\time{41-43}{00:08(+1)}}\hspace{1ex}}{चन्द्रराशिः—\mbox{वृषभः}}%
{\anga{शुभः}{\time{16-47}{14:10}}\hspace{1ex}\uanga{शुक्लः}}%
{\anga{गरः}{\time{7-45}{10:33}}\hspace{1ex}\anga{वणिजः}{\time{36-8}{21:54}}\hspace{1ex}\uanga{विष्टिः}}{}
}
{काञ्ची ५५ जगद्गुरु श्री-चन्द्रचूडेन्द्र सरस्वती ३ आराधना~\#{४९५}\eventsep कृत्तिका-व्रतम्\eventsep मन्वादिः-(चाक्षुषः-[६])\eventsep स्मार्त-पुत्रदा-एकादशी (गृहस्थ)\eventsep मकर-श्रवण-कार्त्तिकोत्सवः}
{Wed} 
\cfoot{\rygdata{12:18--13:31}{08:40--09:53}{11:06--12:18}}
\caldata{JANUARY}{17}{\sunmonth{मकरः}{4}{}{पौषः}{हेमन्तऋतुः}{गुरुः}{विलम्बः}{उत्तरायणम्}{हेमन्तऋतुः}}
{\sunmoonrsdata{07:27}{17:10}{13:48}{04:56(+1)}{12:19}
{\kalas{05:32 06:30 10:02 09:23 10:41 15:53 11:20 13:17 15:14 16:32 18:07 20:44 22:31 02:05(+1)}}}
{\tnykdata{\anga{\tithi{11}{शुक्ल-एकादशी}}{\time{4-3}{09:04}}\hspace{1ex}\anga{\tithi{12}{शुक्ल-द्वादशी}}{\time{58-33}{06:52(+1)}}\hspace{1ex}\avamA{}}%
{\anga{रोहिणी}{\time{38-35}{22:53}}\hspace{1ex}}{चन्द्रराशिः—\mbox{वृषभः}}%
{\anga{शुक्लः}{\time{10-32}{11:40}}\hspace{1ex}\uanga{ब्रह्म}}%
{\anga{विष्टिः}{\time{4-3}{09:04}}\hspace{1ex}\anga{बवः}{\time{31-30}{20:03}}\hspace{1ex}\anga{बालवः}{\time{58-33}{06:52(+1)}}\hspace{1ex}\uanga{कौलवः}}{}
}
{हरिवासरः\RIGHTarrow{}14:35\eventsep स्मार्त-पुत्रदा-एकादशी (सन्न्यस्थ)\eventsep तैत्तिरीय-उत्सर्गो रोहिण्याम्\eventsep त्रैलङ्ग-स्वामि-जयन्ती\eventsep त्रिस्पर्शा-महाद्वादशी\eventsep वैष्णव-पुत्रदा-एकादशी}
{Thu} 
\cfoot{\rygdata{13:32--14:45}{07:27--08:40}{09:53--11:06}}
\caldata{JANUARY}{18}{\sunmonth{मकरः}{5}{}{पौषः}{हेमन्तऋतुः}{शुक्रः}{विलम्बः}{उत्तरायणम्}{हेमन्तऋतुः}}
{\sunmoonrsdata{07:26}{17:11}{14:44}{05:59(+1)}{12:19}
{\kalas{05:32 06:29 10:02 09:23 10:41 15:54 11:20 13:17 15:14 16:33 18:08 20:45 22:32 02:06(+1)}}}
{\tnykdata{\anga{\tithi{13}{शुक्ल-त्रयोदशी}}{\time{51-34}{04:04(+1)}}\hspace{1ex}}%
{\anga{मृगशीर्षम्}{\time{33-51}{20:59}}\hspace{1ex}}{चन्द्रराशिः—\mbox{वृषभः\RIGHTarrow{10:00}}}%
{\anga{ब्रह्म}{\time{2-53}{08:36}}\hspace{1ex}\anga{इन्द्रः}{\time{54-2}{05:03(+1)}}\hspace{1ex}\uanga{वैधृतिः}}%
{\anga{कौलवः}{\time{25-14}{17:32}}\hspace{1ex}\anga{तैतिलः}{\time{51-34}{04:04(+1)}}\hspace{1ex}\uanga{गरः}}{}
}
{\tamil{கண்ணப்ப நாயனார் (9) குருபூஜை}\eventsep प्रदोष-व्रतम्~17:11\RIGHTarrow{}18:08\eventsep \tamil{தை~வெள்ளிக்கிழமை}}
{Fri} 
\cfoot{\rygdata{11:06--12:19}{14:45--15:58}{08:39--09:53}}
\caldata{JANUARY}{19}{\sunmonth{मकरः}{6}{}{पौषः}{हेमन्तऋतुः}{शनिः}{विलम्बः}{उत्तरायणम्}{हेमन्तऋतुः}}
{\sunmoonrsdata{07:26}{17:13}{15:49}{06:56(+1)}{12:19}
{\kalas{05:32 06:29 10:02 09:23 10:41 15:54 11:20 13:18 15:15 16:33 18:09 20:46 22:32 02:06(+1)}}}
{\tnykdata{\anga{\tithi{14}{शुक्ल-चतुर्दशी}}{\time{43-27}{00:49(+1)}}\hspace{1ex}}%
{\anga{आर्द्रा}{\time{27-52}{18:35}}\hspace{1ex}}{चन्द्रराशिः—\mbox{मिथुनम्}}%
{\anga{वैधृतिः}{\time{44-17}{01:09(+1)}}\hspace{1ex}\uanga{विष्कम्भः}}%
{\anga{गरः}{\time{17-38}{14:29}}\hspace{1ex}\anga{वणिजः}{\time{43-27}{00:49(+1)}}\hspace{1ex}\uanga{विष्टिः}}{}
}
{\tamil{அரிவாட்டாய நாயனார் (12) குருபூஜை}\eventsep काञ्ची ८ जगद्गुरु श्री-कैवल्यानन्दयोगेन्द्र सरस्वती आराधना~\#{१९९०}\eventsep \tamil{கபாலீ தெப்போத்ஸவம்}\eventsep सहस्य-मासः/हेमन्तऋतुः\RIGHTarrow{}00:59(+1)\eventsep वैधृति-श्राद्धम्\eventsep विष्णुपदी-पुण्यकालः~18:35\RIGHTarrow{}07:23(+1)}
{Sat} 
\cfoot{\rygdata{09:52--11:06}{13:33--14:46}{07:26--08:39}}
\caldata{JANUARY}{20}{\sunmonth{मकरः}{7}{}{पौषः}{हेमन्तऋतुः}{भानुः}{विलम्बः}{उत्तरायणम्}{हेमन्तऋतुः}}
{\sunmoonrsdata{07:25}{17:14}{17:01}{---}{12:20}
{\kalas{05:32 06:28 10:02 09:23 10:41 15:55 11:21 13:18 15:16 16:34 18:10 20:46 22:33 02:06(+1)}}}
{\tnykdata{\anga{\tithi{15}{पौर्णमासी}}{\time{34-35}{21:16}}\hspace{1ex}}%
{\anga{पुनर्वसुः}{\time{21-2}{15:50}}\hspace{1ex}}{चन्द्रराशिः—\mbox{मिथुनम्\RIGHTarrow{10:33}}}%
{\anga{विष्कम्भः}{\time{33-56}{21:00}}\hspace{1ex}\uanga{प्रीतिः}}%
{\anga{विष्टिः}{\time{9-6}{11:04}}\hspace{1ex}\anga{बवः}{\time{34-35}{21:16}}\hspace{1ex}\uanga{बालवः}}{}
}
{बदरी ज्योतिर्मठ-प्रतिष्ठापन-जयन्ती~\#{२५०४}\eventsep चन्द्र-ग्रहणम्-(राहुग्रस्त)~19:33\RIGHTarrow{}22:50\eventsep देवी-पर्व-१०\eventsep \tamil{கபாலீ தெப்போத்ஸவம்}\eventsep पार्वणव्रतम् पूर्णिमायाम्\eventsep पूर्णिमा-व्रतम्\eventsep पञ्च-पर्व-पूजा (पूर्णिमा)\eventsep रविपुष्ययोग-पुण्यकालः~15:50\RIGHTarrow{}\eventsep मकर-पुष्योत्सवः\eventsep तैत्तिरीय-उत्सर्गः पौर्णमास्याम्\eventsep वेङ्कटाचले पूर्णिमा-गरुड-सेवा\eventsep शाकम्भरी-जयन्ती\eventsep शृङ्गेरी शारदामठ-प्रतिष्ठापन-जयन्ती~\#{२५०२}}
{Sun} 
\cfoot{\rygdata{16:00--17:14}{12:20--13:33}{14:47--16:00}}
\caldata{JANUARY}{21}{\sunmonth{मकरः}{8}{}{पौषः}{हेमन्तऋतुः}{सोमः}{विलम्बः}{उत्तरायणम्}{हेमन्तऋतुः}}
{\sunmoonsrdata{07:25}{17:15}{18:17}{07:45}{12:20}
{\kalas{05:31 06:28 10:02 09:23 10:41 15:56 11:21 13:19 15:17 16:35 18:11 20:47 22:33 02:06(+1)}}}
{\tnykdata{\anga{\tithi{16}{कृष्ण-प्रथमा}}{\time{25-24}{17:35}}\hspace{1ex}}%
{\anga{पुष्यः}{\time{13-46}{12:55}}\hspace{1ex}}{चन्द्रराशिः—\mbox{कर्कटः}}%
{\anga{प्रीतिः}{\time{23-18}{16:44}}\hspace{1ex}\uanga{आयुष्मान्}}%
{\anga{बालवः}{\time{0-1}{07:25}}\hspace{1ex}\anga{कौलवः}{\time{25-24}{17:35}}\hspace{1ex}\anga{तैतिलः}{\time{50-48}{03:44(+1)}}\hspace{1ex}\uanga{गरः}}{}
}
{\tamil{கபாலீ தெப்போத்ஸவம்}\eventsep पार्वण-प्रायश्चित्तावकाशः दर्शे\eventsep पूर्णमासेष्टिः\eventsep स्थालीपाकः}
{Mon} 
\cfoot{\rygdata{08:39--09:52}{11:06--12:20}{13:34--14:47}}
\caldata{JANUARY}{22}{\sunmonth{मकरः}{9}{}{पौषः}{हेमन्तऋतुः}{मङ्गलः}{विलम्बः}{उत्तरायणम्}{हेमन्तऋतुः}}
{\sunmoonsrdata{07:24}{17:16}{19:32}{08:27}{12:20}
{\kalas{05:31 06:28 10:02 09:23 10:41 15:57 11:21 13:19 15:18 16:36 18:12 20:48 22:34 02:06(+1)}}}
{\tnykdata{\anga{\tithi{17}{कृष्ण-द्वितीया}}{\time{16-18}{13:56}}\hspace{1ex}}%
{\anga{आश्रेषा}{\time{6-28}{10:00}}\hspace{1ex}\anga{मघा}{\time{59-35}{07:15(+1)}}\hspace{1ex}}{चन्द्रराशिः—\mbox{कर्कटः\RIGHTarrow{10:00}}}%
{\anga{आयुष्मान्}{\time{12-45}{12:30}}\hspace{1ex}\uanga{सौभाग्यः}}%
{\anga{गरः}{\time{16-18}{13:56}}\hspace{1ex}\anga{वणिजः}{\time{41-54}{00:10(+1)}}\hspace{1ex}\uanga{विष्टिः}}{}
}
{काञ्ची ३७ जगद्गुरु श्री-विद्याघनेन्द्र सरस्वती ३ आराधना~\#{१२३१}\eventsep काञ्ची ६२ जगद्गुरु श्री-चन्द्रशेखरेन्द्र सरस्वती ४ आराधना~\#{२३६}\eventsep \tamil{திருமழிசையாழ்வார் திருநக்ஷத்திரம்}}
{Tue} 
\cfoot{\rygdata{14:48--16:02}{09:52--11:06}{12:20--13:34}}
\caldata{JANUARY}{23}{\sunmonth{मकरः}{10}{}{पौषः}{हेमन्तऋतुः}{बुधः}{विलम्बः}{उत्तरायणम्}{हेमन्तऋतुः}}
{\sunmoonsrdata{07:24}{17:17}{20:46}{09:04}{12:20}
{\kalas{05:31 06:27 10:02 09:22 10:41 15:58 11:21 13:20 15:18 16:37 18:13 20:48 22:34 02:06(+1)}}}
{\tnykdata{\anga{\tithi{18}{कृष्ण-तृतीया}}{\time{7-42}{10:29}}\hspace{1ex}}%
{\anga{पूर्वफल्गुनी}{\time{53-34}{04:49(+1)}}\hspace{1ex}}{चन्द्रराशिः—\mbox{सिंहः}}%
{\anga{सौभाग्यः}{\time{2-34}{08:26}}\hspace{1ex}\anga{शोभनः}{\time{53-5}{04:38(+1)}}\hspace{1ex}\uanga{अतिगण्डः}}%
{\anga{विष्टिः}{\time{7-42}{10:29}}\hspace{1ex}\anga{बवः}{\time{33-43}{20:53}}\hspace{1ex}\uanga{बालवः}}{}
}
{लम्बोदर-महागणपति सङ्कटहर-चतुर्थी-व्रतम्}
{Wed} 
\cfoot{\rygdata{12:20--13:35}{08:38--09:52}{11:06--12:20}}
\caldata{JANUARY}{24}{\sunmonth{मकरः}{11}{}{पौषः}{हेमन्तऋतुः}{गुरुः}{विलम्बः}{उत्तरायणम्}{हेमन्तऋतुः}}
{\sunmoonsrdata{07:23}{17:18}{21:57}{09:38}{12:21}
{\kalas{05:30 06:27 10:02 09:22 10:41 15:59 11:21 13:20 15:19 16:38 18:14 20:49 22:35 02:06(+1)}}}
{\tnykdata{\anga{\tithi{19}{कृष्ण-चतुर्थी}}{\time{0-1}{07:23}}\hspace{1ex}\anga{\tithi{20}{कृष्ण-पञ्चमी}}{\time{53-31}{04:48(+1)}}\hspace{1ex}\avamA{}}%
{\anga{उत्तरफल्गुनी}{\time{48-44}{02:53(+1)}}\hspace{1ex}}{चन्द्रराशिः—\mbox{सिंहः\RIGHTarrow{10:17}}}%
{\anga{अतिगण्डः}{\time{44-35}{01:13(+1)}}\hspace{1ex}\uanga{सुकर्म}}%
{\anga{बालवः}{\time{0-1}{07:23}}\hspace{1ex}\anga{कौलवः}{\time{26-36}{18:02}}\hspace{1ex}\anga{तैतिलः}{\time{53-31}{04:48(+1)}}\hspace{1ex}\uanga{गरः}}{}
}
{बहुल-पञ्चमी\eventsep \tamil{சண்டேஶ்வர நாயனார் (19) குருபூஜை}\eventsep त्यागराज-आराधना~\#{१७२}}
{Thu} 
\cfoot{\rygdata{13:35--14:49}{07:23--08:37}{09:52--11:06}}
\caldata{JANUARY}{25}{\sunmonth{मकरः}{12}{}{पौषः}{हेमन्तऋतुः}{शुक्रः}{विलम्बः}{उत्तरायणम्}{हेमन्तऋतुः}}
{\sunmoonsrdata{07:22}{17:19}{23:06}{10:10}{12:21}
{\kalas{05:30 06:26 10:02 09:22 10:41 16:00 11:21 13:20 15:20 16:39 18:15 20:50 22:35 02:06(+1)}}}
{\tnykdata{\anga{\tithi{21}{कृष्ण-षष्ठी}}{\time{48-36}{02:49(+1)}}\hspace{1ex}}%
{\anga{हस्तः}{\time{45-24}{01:32(+1)}}\hspace{1ex}}{चन्द्रराशिः—\mbox{कन्या}}%
{\anga{सुकर्म}{\time{37-18}{22:18}}\hspace{1ex}\uanga{धृतिः}}%
{\anga{गरः}{\time{20-52}{15:43}}\hspace{1ex}\anga{वणिजः}{\time{48-36}{02:49(+1)}}\hspace{1ex}\uanga{विष्टिः}}{}
}
{\tamil{தை~வெள்ளிக்கிழமை}\eventsep श्री-शेषाद्रि-स्वामि-जयन्ती~\#{१५०}}
{Fri} 
\cfoot{\rygdata{11:06--12:21}{14:50--16:05}{08:37--09:52}}
\caldata{JANUARY}{26}{\sunmonth{मकरः}{13}{}{पौषः}{हेमन्तऋतुः}{शनिः}{विलम्बः}{उत्तरायणम्}{हेमन्तऋतुः}}
{\sunmoonsrdata{07:22}{17:20}{00:13(+1)}{10:41}{12:21}
{\kalas{05:29 06:26 10:01 09:21 10:41 16:01 11:21 13:21 15:21 16:40 18:16 20:50 22:36 02:06(+1)}}}
{\tnykdata{\anga{\tithi{22}{कृष्ण-सप्तमी}}{\time{45-25}{01:32(+1)}}\hspace{1ex}}%
{\anga{चित्रा}{\time{43-46}{00:52(+1)}}\hspace{1ex}}{चन्द्रराशिः—\mbox{कन्या\RIGHTarrow{13:07}}}%
{\anga{धृतिः}{\time{31-22}{19:55}}\hspace{1ex}\uanga{शूलः}}%
{\anga{विष्टिः}{\time{16-48}{14:05}}\hspace{1ex}\anga{बवः}{\time{45-25}{01:32(+1)}}\hspace{1ex}\uanga{बालवः}}{}
}
{पौष-अष्टका-पूर्वेद्युः\eventsep विवेकानन्द-जयन्ती~\#{१५७}}
{Sat} 
\cfoot{\rygdata{09:51--11:06}{13:36--14:51}{07:22--08:37}}
\caldata{JANUARY}{27}{\sunmonth{मकरः}{14}{}{पौषः}{हेमन्तऋतुः}{भानुः}{विलम्बः}{उत्तरायणम्}{हेमन्तऋतुः}}
{\sunmoonsrdata{07:21}{17:21}{01:18(+1)}{11:14}{12:21}
{\kalas{05:29 06:25 10:01 09:21 10:41 16:01 11:21 13:21 15:21 16:41 18:17 20:51 22:36 02:06(+1)}}}
{\tnykdata{\anga{\tithi{23}{कृष्ण-अष्टमी}}{\time{44-5}{00:59(+1)}}\hspace{1ex}}%
{\anga{स्वाती}{\time{43-57}{00:56(+1)}}\hspace{1ex}}{चन्द्रराशिः—\mbox{तुला}}%
{\anga{शूलः}{\time{26-54}{18:07}}\hspace{1ex}\uanga{गण्डः}}%
{\anga{बालवः}{\time{14-32}{13:10}}\hspace{1ex}\anga{कौलवः}{\time{44-5}{00:59(+1)}}\hspace{1ex}\uanga{तैतिलः}}{}
}
{पञ्च-पर्व-पूजा (अष्टमी)\eventsep पौष-अष्टका-श्राद्धम्}
{Sun} 
\cfoot{\rygdata{16:06--17:21}{12:21--13:36}{14:51--16:06}}
\caldata{JANUARY}{28}{\sunmonth{मकरः}{15}{}{पौषः}{हेमन्तऋतुः}{सोमः}{विलम्बः}{उत्तरायणम्}{हेमन्तऋतुः}}
{\sunmoonsrdata{07:20}{17:23}{02:21(+1)}{11:50}{12:21}
{\kalas{05:28 06:24 10:01 09:21 10:41 16:02 11:21 13:22 15:22 16:42 18:18 20:52 22:36 02:06(+1)}}}
{\tnykdata{\anga{\tithi{24}{कृष्ण-नवमी}}{\time{44-35}{01:10(+1)}}\hspace{1ex}}%
{\anga{विशाखा}{\time{45-54}{01:42(+1)}}\hspace{1ex}}{चन्द्रराशिः—\mbox{तुला\RIGHTarrow{19:27}}}%
{\anga{गण्डः}{\time{23-52}{16:53}}\hspace{1ex}\uanga{वृद्धिः}}%
{\anga{तैतिलः}{\time{14-7}{12:59}}\hspace{1ex}\anga{गरः}{\time{44-35}{01:10(+1)}}\hspace{1ex}\uanga{वणिजः}}{}
}
{भीष्म-जयन्ती\eventsep पौष-अन्वष्टका-श्राद्धम्\eventsep \tamil{திருநீலகண்ட நாயனார் (1) குருபூஜை}}
{Mon} 
\cfoot{\rygdata{08:36--09:51}{11:06--12:21}{13:37--14:52}}
\caldata{JANUARY}{29}{\sunmonth{मकरः}{16}{}{पौषः}{हेमन्तऋतुः}{मङ्गलः}{विलम्बः}{उत्तरायणम्}{हेमन्तऋतुः}}
{\sunmoonsrdata{07:20}{17:24}{03:21(+1)}{12:29}{12:22}
{\kalas{05:28 06:24 10:01 09:20 10:41 16:03 11:21 13:22 15:23 16:43 18:19 20:52 22:37 02:06(+1)}}}
{\tnykdata{\anga{\tithi{25}{कृष्ण-दशमी}}{\time{46-48}{02:03(+1)}}\hspace{1ex}}%
{\anga{अनूराधा}{\time{49-30}{03:08(+1)}}\hspace{1ex}}{चन्द्रराशिः—\mbox{वृश्चिकः}}%
{\anga{वृद्धिः}{\time{22-11}{16:12}}\hspace{1ex}\uanga{ध्रुवः}}%
{\anga{वणिजः}{\time{15-30}{13:32}}\hspace{1ex}\anga{विष्टिः}{\time{46-48}{02:03(+1)}}\hspace{1ex}\uanga{बवः}}{}
}
{त्रैलोक्य-गौरी-व्रतम्}
{Tue} 
\cfoot{\rygdata{14:53--16:08}{09:51--11:06}{12:22--13:37}}
\caldata{JANUARY}{30}{\sunmonth{मकरः}{17}{}{पौषः}{हेमन्तऋतुः}{बुधः}{विलम्बः}{उत्तरायणम्}{हेमन्तऋतुः}}
{\sunmoonsrdata{07:19}{17:25}{04:17(+1)}{13:12}{12:22}
{\kalas{05:27 06:23 10:00 09:20 10:41 16:04 11:21 13:22 15:24 16:44 18:20 20:53 22:37 02:06(+1)}}}
{\tnykdata{\anga{\tithi{26}{कृष्ण-एकादशी}}{\time{50-32}{03:32(+1)}}\hspace{1ex}}%
{\anga{ज्येष्ठा}{\time{54-32}{05:08(+1)}}\hspace{1ex}}{चन्द्रराशिः—\mbox{वृश्चिकः\RIGHTarrow{05:08(+1)}}}%
{\anga{ध्रुवः}{\time{21-44}{16:00}}\hspace{1ex}\uanga{व्याघातः}}%
{\anga{बवः}{\time{18-31}{14:43}}\hspace{1ex}\anga{बालवः}{\time{50-32}{03:32(+1)}}\hspace{1ex}\uanga{कौलवः}}{}
}
{सर्व-षट्तिला-एकादशी}
{Wed} 
\cfoot{\rygdata{12:22--13:38}{08:34--09:50}{11:06--12:22}}
\caldata{JANUARY}{31}{\sunmonth{मकरः}{18}{}{पौषः}{हेमन्तऋतुः}{गुरुः}{विलम्बः}{उत्तरायणम्}{हेमन्तऋतुः}}
{\sunmoonsrdata{07:18}{17:26}{05:08(+1)}{14:00}{12:22}
{\kalas{05:27 06:22 10:00 09:20 10:41 16:05 11:21 13:23 15:24 16:45 18:21 20:54 22:38 02:05(+1)}}}
{\tnykdata{\anga{\tithi{27}{कृष्ण-द्वादशी}}{\time{55-28}{05:29(+1)}}\hspace{1ex}}%
{\fullanga{मूला}}{चन्द्रराशिः—\mbox{धनुः}}%
{\anga{व्याघातः}{\time{22-17}{16:13}}\hspace{1ex}\uanga{हर्षणः}}%
{\anga{कौलवः}{\time{22-53}{16:27}}\hspace{1ex}\anga{तैतिलः}{\time{55-28}{05:29(+1)}}\hspace{1ex}\uanga{गरः}}{}
}
{हरिवासरः\RIGHTarrow{}09:59\eventsep सेङ्गालिपुरम्-मुत्तण्णावाळ्-आराधना~\#{१२६}}
{Thu} 
\cfoot{\rygdata{13:38--14:54}{07:18--08:34}{09:50--11:06}}
\caldata{FEBRUARY}{1}{\sunmonth{मकरः}{19}{}{पौषः}{हेमन्तऋतुः}{शुक्रः}{विलम्बः}{उत्तरायणम्}{हेमन्तऋतुः}}
{\sunmoonsrdata{07:17}{17:27}{05:53(+1)}{14:52}{12:22}
{\kalas{05:26 06:22 10:00 09:19 10:40 16:06 11:21 13:23 15:25 16:46 18:22 20:54 22:38 02:05(+1)}}}
{\tnykdata{\fulltithi{\tithi{28}{कृष्ण-त्रयोदशी}}}%
{\anga{मूला}{\time{0-45}{07:35}}\hspace{1ex}}{चन्द्रराशिः—\mbox{धनुः}}%
{\anga{हर्षणः}{\time{23-39}{16:45}}\hspace{1ex}\uanga{वज्रम्}}%
{\anga{गरः}{\time{28-19}{18:37}}\hspace{1ex}\uanga{वणिजः}}{}
}
{प्रदोष-व्रतम्~17:27\RIGHTarrow{}18:22\eventsep \tamil{தை~வெள்ளிக்கிழமை}}
{Fri} 
\cfoot{\rygdata{11:06--12:22}{14:55--16:11}{08:33--09:50}}
\caldata{FEBRUARY}{2}{\sunmonth{मकरः}{20}{}{पौषः}{हेमन्तऋतुः}{शनिः}{विलम्बः}{उत्तरायणम्}{हेमन्तऋतुः}}
{\sunmoonsrdata{07:16}{17:28}{06:32(+1)}{15:47}{12:22}
{\kalas{05:26 06:21 09:59 09:19 10:40 16:07 11:21 13:23 15:26 16:48 18:23 20:55 22:38 02:05(+1)}}}
{\tnykdata{\anga{\tithi{28}{कृष्ण-त्रयोदशी}}{\time{1-21}{07:49}}\hspace{1ex}}%
{\anga{पूर्वाषाढा}{\time{7-45}{10:22}}\hspace{1ex}}{चन्द्रराशिः—\mbox{धनुः\RIGHTarrow{17:07}}}%
{\anga{वज्रम्}{\time{25-34}{17:30}}\hspace{1ex}\uanga{सिद्धिः}}%
{\anga{वणिजः}{\time{1-21}{07:49}}\hspace{1ex}\anga{विष्टिः}{\time{34-30}{21:04}}\hspace{1ex}\uanga{शकुनिः}}{}
}
{मासशिवरात्रिः\eventsep पञ्च-पर्व-पूजा (चतुर्दशी)}
{Sat} 
\cfoot{\rygdata{09:49--11:06}{13:39--14:55}{07:16--08:33}}
\caldata{FEBRUARY}{3}{\sunmonth{मकरः}{21}{}{पौषः}{हेमन्तऋतुः}{भानुः}{विलम्बः}{उत्तरायणम्}{हेमन्तऋतुः}}
{\sunmoonsrdata{07:15}{17:29}{07:06(+1)}{16:44}{12:22}
{\kalas{05:25 06:20 09:59 09:18 10:40 16:07 11:21 13:24 15:27 16:48 18:24 20:56 22:39 02:05(+1)}}}
{\tnykdata{\anga{\tithi{29}{कृष्ण-चतुर्दशी}}{\time{7-47}{10:22}}\hspace{1ex}}%
{\anga{उत्तराषाढा}{\time{15-18}{13:23}}\hspace{1ex}}{चन्द्रराशिः—\mbox{मकरः}}%
{\anga{सिद्धिः}{\time{27-52}{18:24}}\hspace{1ex}\uanga{व्यतीपातः}}%
{\anga{शकुनिः}{\time{7-47}{10:22}}\hspace{1ex}\anga{चतुष्पात्}{\time{41-7}{23:42}}\hspace{1ex}\uanga{नाग}}{}
}
{पञ्च-पर्व-पूजा (अमावास्या)\eventsep सर्व-मौनि (पौष/मकर) अमावास्या\eventsep व्यतीपात-योगः (अलभ्यम्)}
{Sun} 
\cfoot{\rygdata{16:13--17:29}{12:22--13:39}{14:56--16:13}}
\caldata{FEBRUARY}{4}{\sunmonth{मकरः}{22}{}{पौषः}{हेमन्तऋतुः}{सोमः}{विलम्बः}{उत्तरायणम्}{हेमन्तऋतुः}}
{\sunmoonsrdata{07:14}{17:31}{---}{17:41}{12:22}
{\kalas{05:24 06:19 09:59 09:18 10:40 16:08 11:21 13:24 15:27 16:49 18:25 20:56 22:39 02:05(+1)}}}
{\tnykdata{\anga{\tithi{30}{अमावास्या}}{\time{14-31}{13:03}}\hspace{1ex}}%
{\anga{श्रवणः}{\time{23-6}{16:29}}\hspace{1ex}}{चन्द्रराशिः—\mbox{मकरः\RIGHTarrow{06:03(+1)}}}%
{\anga{व्यतीपातः}{\time{30-19}{19:22}}\hspace{1ex}\uanga{वरीयान्}}%
{\anga{नाग}{\time{14-31}{13:03}}\hspace{1ex}\anga{किंस्तुघ्नः}{\time{47-54}{02:24(+1)}}\hspace{1ex}\uanga{बवः}}{}
}
{बोधायन-इष्टिः\eventsep महोदय-पुण्यकालः\eventsep पार्वणव्रतम् अमावास्यायाम्\eventsep सोमवती अमावास्या\eventsep सोमश्रावणी-पुण्यकालः\RIGHTarrow{}16:29\eventsep \tamil{திருநெல்வேலி நெல்லையப்பர் பத்ர தீப திருவிழா}\eventsep व्यतीपात-श्राद्धम्\eventsep श्रवण-व्रतम्}
{Mon} 
\cfoot{\rygdata{08:31--09:48}{11:05--12:22}{13:39--14:56}}
\caldata{FEBRUARY}{5}{\sunmonth{मकरः}{23}{}{माघः}{शिशिरऋतुः}{मङ्गलः}{विलम्बः}{उत्तरायणम्}{हेमन्तऋतुः}}
{\sunmoonrsdata{07:13}{17:32}{07:37}{18:38}{12:22}
{\kalas{05:24 06:19 09:58 09:17 10:39 16:09 11:21 13:24 15:28 16:50 18:26 20:57 22:39 02:05(+1)}}}
{\tnykdata{\anga{\tithi{1}{शुक्ल-प्रथमा}}{\time{21-19}{15:45}}\hspace{1ex}}%
{\anga{श्रविष्ठा}{\time{30-55}{19:36}}\hspace{1ex}}{चन्द्रराशिः—\mbox{कुम्भः}}%
{\anga{वरीयान्}{\time{32-46}{20:20}}\hspace{1ex}\uanga{परिघः}}%
{\anga{बवः}{\time{21-19}{15:45}}\hspace{1ex}\anga{बालवः}{\time{54-38}{05:05(+1)}}\hspace{1ex}\uanga{कौलवः}}{}
}
{चन्द्र-दर्शनम्~17:32\RIGHTarrow{}18:26\eventsep दर्शेष्टिः\eventsep पार्वण-प्रायश्चित्तावकाशः पौर्णमास्याम्\eventsep स्थालीपाकः\eventsep श्यामळानवरात्र-आरम्भः}
{Tue} 
\cfoot{\rygdata{14:57--16:14}{09:48--11:05}{12:23--13:40}}
\caldata{FEBRUARY}{6}{\sunmonth{मकरः}{24}{}{माघः}{शिशिरऋतुः}{बुधः}{विलम्बः}{उत्तरायणम्}{हेमन्तऋतुः}}
{\sunmoonrsdata{07:12}{17:33}{08:05}{19:35}{12:23}
{\kalas{05:23 06:18 09:58 09:16 10:39 16:10 11:20 13:25 15:29 16:51 18:27 20:57 22:40 02:04(+1)}}}
{\tnykdata{\anga{\tithi{2}{शुक्ल-द्वितीया}}{\time{27-54}{18:22}}\hspace{1ex}}%
{\anga{शतभिषक्}{\time{38-31}{22:37}}\hspace{1ex}}{चन्द्रराशिः—\mbox{कुम्भः}}%
{\anga{परिघः}{\time{35-2}{21:13}}\hspace{1ex}\uanga{शिवः}}%
{\anga{कौलवः}{\time{27-54}{18:22}}\hspace{1ex}\uanga{तैतिलः}}{}
}
{\tamil{அப்பூதியடிகள் நாயனார் (24) குருபூஜை}}
{Wed} 
\cfoot{\rygdata{12:23--13:40}{08:30--09:47}{11:05--12:23}}
\caldata{FEBRUARY}{7}{\sunmonth{मकरः}{25}{}{माघः}{शिशिरऋतुः}{गुरुः}{विलम्बः}{उत्तरायणम्}{हेमन्तऋतुः}}
{\sunmoonrsdata{07:11}{17:34}{08:31}{20:32}{12:23}
{\kalas{05:22 06:17 09:57 09:16 10:39 16:11 11:20 13:25 15:29 16:52 18:28 20:58 22:40 02:04(+1)}}}
{\tnykdata{\anga{\tithi{3}{शुक्ल-तृतीया}}{\time{34-0}{20:48}}\hspace{1ex}}%
{\anga{पूर्वप्रोष्ठपदा}{\time{45-37}{01:26(+1)}}\hspace{1ex}}{चन्द्रराशिः—\mbox{कुम्भः\RIGHTarrow{18:45}}}%
{\anga{शिवः}{\time{36-57}{21:58}}\hspace{1ex}\uanga{सिद्धः}}%
{\anga{तैतिलः}{\time{1-3}{07:37}}\hspace{1ex}\anga{गरः}{\time{34-0}{20:48}}\hspace{1ex}\uanga{वणिजः}}{}
}
{}
{Thu} 
\cfoot{\rygdata{13:40--14:58}{07:11--08:29}{09:47--11:05}}
\caldata{FEBRUARY}{8}{\sunmonth{मकरः}{26}{}{माघः}{शिशिरऋतुः}{शुक्रः}{विलम्बः}{उत्तरायणम्}{हेमन्तऋतुः}}
{\sunmoonrsdata{07:10}{17:35}{08:57}{21:29}{12:23}
{\kalas{05:21 06:16 09:57 09:15 10:39 16:12 11:20 13:25 15:30 16:53 18:29 20:59 22:40 02:04(+1)}}}
{\tnykdata{\anga{\tithi{4}{शुक्ल-चतुर्थी}}{\time{39-23}{22:56}}\hspace{1ex}}%
{\anga{उत्तरप्रोष्ठपदा}{\time{51-59}{03:58(+1)}}\hspace{1ex}}{चन्द्रराशिः—\mbox{मीनः}}%
{\anga{सिद्धः}{\time{38-18}{22:30}}\hspace{1ex}\uanga{साध्यः}}%
{\anga{वणिजः}{\time{6-50}{09:54}}\hspace{1ex}\anga{विष्टिः}{\time{39-23}{22:56}}\hspace{1ex}\uanga{बवः}}{}
}
{मार्कण्डेय-जयन्ती\eventsep \tamil{தை~வெள்ளிக்கிழமை}\eventsep \tamil{திருச்செந்தூர் முருகன் மாசித் திருவிழா தொடக்கம்}\eventsep शान्ता~वरकुन्द-चतुर्थी}
{Fri} 
\cfoot{\rygdata{11:05--12:23}{14:59--16:17}{08:28--09:47}}
\caldata{FEBRUARY}{9}{\sunmonth{मकरः}{27}{}{माघः}{शिशिरऋतुः}{शनिः}{विलम्बः}{उत्तरायणम्}{हेमन्तऋतुः}}
{\sunmoonrsdata{07:09}{17:36}{09:23}{22:27}{12:23}
{\kalas{05:21 06:15 09:57 09:15 10:38 16:13 11:20 13:25 15:31 16:54 18:30 20:59 22:41 02:04(+1)}}}
{\tnykdata{\anga{\tithi{5}{शुक्ल-पञ्चमी}}{\time{43-43}{00:39(+1)}}\hspace{1ex}}%
{\anga{रेवती}{\time{57-19}{06:05(+1)}}\hspace{1ex}}{चन्द्रराशिः—\mbox{मीनः\RIGHTarrow{06:05(+1)}}}%
{\anga{साध्यः}{\time{38-53}{22:43}}\hspace{1ex}\uanga{शुभः}}%
{\anga{बवः}{\time{11-43}{11:51}}\hspace{1ex}\anga{बालवः}{\time{43-43}{00:39(+1)}}\hspace{1ex}\uanga{कौलवः}}{}
}
{\tamil{கலிக்கம்ப நாயனார் (42) குருபூஜை}\eventsep माघी-सरस्वती-पूजा\eventsep प्रोक्लस्-जन्म~\#{१६०७}\eventsep सर्प-पूजा\eventsep \tamil{திருச்செந்தூர் முருகன் மாசித் திருவிழா 2ம் நாள்}\eventsep वसन्त-श्री-पञ्चमी\eventsep श्रीराम-वनवास-गमनम्}
{Sat} 
\cfoot{\rygdata{09:46--11:04}{13:41--14:59}{07:09--08:28}}
\caldata{FEBRUARY}{10}{\sunmonth{मकरः}{28}{}{माघः}{शिशिरऋतुः}{भानुः}{विलम्बः}{उत्तरायणम्}{हेमन्तऋतुः}}
{\sunmoonrsdata{07:08}{17:37}{09:51}{23:28}{12:23}
{\kalas{05:20 06:14 09:56 09:14 10:38 16:13 11:20 13:26 15:31 16:55 18:31 21:00 22:41 02:03(+1)}}}
{\tnykdata{\anga{\tithi{6}{शुक्ल-षष्ठी}}{\time{46-45}{01:50(+1)}}\hspace{1ex}}%
{\fullanga{अश्विनी}}{चन्द्रराशिः—\mbox{मेषः}}%
{\anga{शुभः}{\time{38-30}{22:33}}\hspace{1ex}\uanga{शुक्लः}}%
{\anga{कौलवः}{\time{15-26}{13:19}}\hspace{1ex}\anga{तैतिलः}{\time{46-45}{01:50(+1)}}\hspace{1ex}\uanga{गरः}}{}
}
{षष्ठी-व्रतम्\eventsep \tamil{திருச்செந்தூர் முருகன் மாசித் திருவிழா 3ம் நாள்—முருகன் பவனி}\eventsep \tamil{திருநெல்வேலி நெல்லையப்பர் நெல்லுக்கு வேலி கட்டிய லீலை}}
{Sun} 
\cfoot{\rygdata{16:19--17:37}{12:23--13:41}{15:00--16:19}}
\caldata{FEBRUARY}{11}{\sunmonth{मकरः}{29}{}{माघः}{शिशिरऋतुः}{सोमः}{विलम्बः}{उत्तरायणम्}{हेमन्तऋतुः}}
{\sunmoonrsdata{07:07}{17:38}{10:22}{00:30(+1)}{12:23}
{\kalas{05:19 06:13 09:56 09:13 10:38 16:14 11:20 13:26 15:32 16:56 18:32 21:00 22:41 02:03(+1)}}}
{\tnykdata{\anga{\tithi{7}{शुक्ल-सप्तमी}}{\time{48-12}{02:24(+1)}}\hspace{1ex}}%
{\anga{अश्विनी}{\time{1-23}{07:40}}\hspace{1ex}}{चन्द्रराशिः—\mbox{मेषः}}%
{\anga{शुक्लः}{\time{36-57}{21:54}}\hspace{1ex}\uanga{ब्रह्म}}%
{\anga{गरः}{\time{17-42}{14:12}}\hspace{1ex}\anga{वणिजः}{\time{48-12}{02:24(+1)}}\hspace{1ex}\uanga{विष्टिः}}{}
}
{अचला-सप्तमी-व्रतम्\eventsep द्वारका-मठ-प्रतिष्ठापन-जयन्ती~\#{२५०९}\eventsep मन्वादिः-(सावर्णिः-[८])\eventsep नर्मदा-जयन्ती\eventsep रथ-सप्तमी\eventsep \tamil{திருச்செந்தூர் முருகன் மாசித் திருவிழா 4ம் நாள்}}
{Mon} 
\cfoot{\rygdata{08:26--09:45}{11:04--12:23}{13:42--15:01}}
\caldata{FEBRUARY}{12}{\sunmonth{मकरः}{30}{\mbox{मकरः{\tiny\RIGHTarrow}{19:00}}}{माघः}{शिशिरऋतुः}{मङ्गलः}{विलम्बः}{उत्तरायणम्}{हेमन्तऋतुः}}
{\sunmoonrsdata{07:06}{17:39}{10:57}{01:33(+1)}{12:23}
{\kalas{05:18 06:12 09:55 09:13 10:37 16:15 11:19 13:26 15:33 16:57 18:33 21:01 22:41 02:03(+1)}}}
{\tnykdata{\anga{\tithi{8}{शुक्ल-अष्टमी}}{\time{47-55}{02:16(+1)}}\hspace{1ex}}%
{\anga{अपभरणी}{\time{3-51}{08:39}}\hspace{1ex}}{चन्द्रराशिः—\mbox{मेषः\RIGHTarrow{14:47}}}%
{\anga{ब्रह्म}{\time{34-6}{20:45}}\hspace{1ex}\uanga{इन्द्रः}}%
{\anga{विष्टिः}{\time{18-19}{14:26}}\hspace{1ex}\anga{बवः}{\time{47-55}{02:16(+1)}}\hspace{1ex}\uanga{बालवः}}{}
}
{भीष्माष्टमी\eventsep कृत्तिका-व्रतम्\eventsep खोडियार-माता-जयन्ती\eventsep कुम्भ-रवि-सङ्क्रमण-विष्णुपदी-पुण्यकालः~12:36\RIGHTarrow{}01:24(+1)\eventsep \tamil{திருச்செந்தூர் முருகன் மாசித் திருவிழா 5ம் நாள்}}
{Tue} 
\cfoot{\rygdata{15:01--16:20}{09:44--11:04}{12:23--13:42}}
\caldata{FEBRUARY}{13}{\sunmonth{कुम्भः}{1}{}{माघः}{शिशिरऋतुः}{बुधः}{विलम्बः}{उत्तरायणम्}{शिशिरऋतुः}}
{\sunmoonrsdata{07:05}{17:40}{11:39}{02:38(+1)}{12:23}
{\kalas{05:17 06:11 09:54 09:12 10:37 16:16 11:19 13:26 15:33 16:58 18:34 21:01 22:42 02:02(+1)}}}
{\tnykdata{\anga{\tithi{9}{शुक्ल-नवमी}}{\time{45-48}{01:24(+1)}}\hspace{1ex}}%
{\anga{कृत्तिका}{\time{4-36}{08:56}}\hspace{1ex}}{चन्द्रराशिः—\mbox{वृषभः}}%
{\anga{इन्द्रः}{\time{29-49}{19:01}}\hspace{1ex}\uanga{वैधृतिः}}%
{\anga{बालवः}{\time{17-7}{13:56}}\hspace{1ex}\anga{कौलवः}{\time{45-48}{01:24(+1)}}\hspace{1ex}\uanga{तैतिलः}}{}
}
{मध्व-नवमी\eventsep \tamil{திருச்செந்தூர் முருகன் மாசித் திருவிழா 6ம் நாள்—வெள்ளித் தேர் பவனி}\eventsep श्यामळानवरात्र-समापनम्}
{Wed} 
\cfoot{\rygdata{12:23--13:42}{08:24--09:44}{11:03--12:23}}
\caldata{FEBRUARY}{14}{\sunmonth{कुम्भः}{2}{}{माघः}{शिशिरऋतुः}{गुरुः}{विलम्बः}{उत्तरायणम्}{शिशिरऋतुः}}
{\sunmoonrsdata{07:04}{17:42}{12:28}{03:40(+1)}{12:23}
{\kalas{05:17 06:10 09:54 09:11 10:36 16:17 11:19 13:26 15:34 16:59 18:35 21:02 22:42 02:02(+1)}}}
{\tnykdata{\anga{\tithi{10}{शुक्ल-दशमी}}{\time{41-51}{23:48}}\hspace{1ex}}%
{\anga{रोहिणी}{\time{3-33}{08:29}}\hspace{1ex}}{चन्द्रराशिः—\mbox{वृषभः\RIGHTarrow{20:00}}}%
{\anga{वैधृतिः}{\time{24-5}{16:42}}\hspace{1ex}\uanga{विष्कम्भः}}%
{\anga{तैतिलः}{\time{14-4}{12:42}}\hspace{1ex}\anga{गरः}{\time{41-51}{23:48}}\hspace{1ex}\uanga{वणिजः}}{}
}
{काञ्ची ११ जगद्गुरु श्री-शिवानन्द चिद्घनेन्द्र सरस्वती आराधना~\#{१८४७}\eventsep साम्ब-दशमी (सूर्यपूजा)\eventsep \tamil{திருச்செந்தூர் முருகன் மாசித் திருவிழா 7ம் நாள்—உருகு சத்தச் சேவை/சிகப்பு சாத்தி அலங்காரம்}\eventsep वैधृति-श्राद्धम्}
{Thu} 
\cfoot{\rygdata{13:42--15:02}{07:04--08:24}{09:43--11:03}}
\caldata{FEBRUARY}{15}{\sunmonth{कुम्भः}{3}{}{माघः}{शिशिरऋतुः}{शुक्रः}{विलम्बः}{उत्तरायणम्}{शिशिरऋतुः}}
{\sunmoonrsdata{07:03}{17:43}{13:27}{04:38(+1)}{12:23}
{\kalas{05:16 06:09 09:53 09:11 10:36 16:17 11:19 13:27 15:35 17:00 18:36 21:02 22:42 02:02(+1)}}}
{\tnykdata{\anga{\tithi{11}{शुक्ल-एकादशी}}{\time{36-12}{21:32}}\hspace{1ex}}%
{\anga{मृगशीर्षम्}{\time{0-45}{07:21}}\hspace{1ex}\anga{आर्द्रा}{\time{56-17}{05:34(+1)}}\hspace{1ex}}{चन्द्रराशिः—\mbox{मिथुनम्}}%
{\anga{विष्कम्भः}{\time{16-57}{13:50}}\hspace{1ex}\uanga{प्रीतिः}}%
{\anga{वणिजः}{\time{9-15}{10:45}}\hspace{1ex}\anga{विष्टिः}{\time{36-12}{21:32}}\hspace{1ex}\uanga{बवः}}{}
}
{सर्व-जया/भैमी-एकादशी\eventsep \tamil{திருச்செந்தூர் முருகன் மாசித் திருவிழா 8ம் நாள்—பச்சை சாத்தி அலங்காரம்}}
{Fri} 
\cfoot{\rygdata{11:03--12:23}{15:03--16:23}{08:23--09:43}}
\caldata{FEBRUARY}{16}{\sunmonth{कुम्भः}{4}{}{माघः}{शिशिरऋतुः}{शनिः}{विलम्बः}{उत्तरायणम्}{शिशिरऋतुः}}
{\sunmoonrsdata{07:01}{17:44}{14:33}{05:30(+1)}{12:23}
{\kalas{05:15 06:08 09:53 09:10 10:36 16:18 11:18 13:27 15:35 17:01 18:37 21:03 22:42 02:01(+1)}}}
{\tnykdata{\anga{\tithi{12}{शुक्ल-द्वादशी}}{\time{29-6}{18:40}}\hspace{1ex}}%
{\anga{पुनर्वसुः}{\time{50-31}{03:14(+1)}}\hspace{1ex}}{चन्द्रराशिः—\mbox{मिथुनम्\RIGHTarrow{21:51}}}%
{\anga{प्रीतिः}{\time{8-34}{10:27}}\hspace{1ex}\anga{आयुष्मान्}{\time{59-6}{06:40(+1)}}\hspace{1ex}\uanga{सौभाग्यः}}%
{\anga{बवः}{\time{2-51}{08:10}}\hspace{1ex}\anga{बालवः}{\time{29-6}{18:40}}\hspace{1ex}\anga{कौलवः}{\time{55-3}{05:03(+1)}}\hspace{1ex}\uanga{तैतिलः}}{}
}
{भीष्म-द्वादशी\eventsep हरिवासरः\RIGHTarrow{}02:52\eventsep जयन्ती-महाद्वादशी\eventsep \tamil{குலஶேகர ஆழ்வார் திருநக்ஷத்திரம்}\eventsep तिलपद्म-द्वादशी/तिलोत्पत्ति\eventsep \tamil{திருச்செந்தூர் முருகன் மாசித் திருவிழா 9ம் நாள்—தங்க கைலாச வாஹனம்}\eventsep वराह-द्वादशी\eventsep शनि-प्रदोष-व्रतम्~17:44\RIGHTarrow{}18:37}
{Sat} 
\cfoot{\rygdata{09:42--11:02}{13:43--15:03}{07:01--08:22}}
\caldata{FEBRUARY}{17}{\sunmonth{कुम्भः}{5}{}{माघः}{शिशिरऋतुः}{भानुः}{विलम्बः}{उत्तरायणम्}{शिशिरऋतुः}}
{\sunmoonrsdata{07:00}{17:45}{15:46}{06:16(+1)}{12:23}
{\kalas{05:14 06:07 09:52 09:09 10:35 16:19 11:18 13:27 15:36 17:02 18:38 21:03 22:43 02:01(+1)}}}
{\tnykdata{\anga{\tithi{13}{शुक्ल-त्रयोदशी}}{\time{20-49}{15:20}}\hspace{1ex}}%
{\anga{पुष्यः}{\time{43-44}{00:30(+1)}}\hspace{1ex}}{चन्द्रराशिः—\mbox{कर्कटः}}%
{\anga{सौभाग्यः}{\time{48-53}{02:33(+1)}}\hspace{1ex}\uanga{शोभनः}}%
{\anga{तैतिलः}{\time{20-49}{15:20}}\hspace{1ex}\anga{गरः}{\time{46-20}{01:32(+1)}}\hspace{1ex}\uanga{वणिजः}}{}
}
{देवी-पर्व-११\eventsep \tamil{நடராஜர் மஹாபிஷேகம்}\eventsep रविपुष्ययोग-पुण्यकालः\eventsep \tamil{திருச்செந்தூர் முருகன் மாசித் திருவிழா 10ம் நாள்—தேர்}\eventsep वराह-कल्पादिः}
{Sun} 
\cfoot{\rygdata{16:24--17:45}{12:23--13:43}{15:04--16:24}}
\caldata{FEBRUARY}{18}{\sunmonth{कुम्भः}{6}{}{माघः}{शिशिरऋतुः}{सोमः}{विलम्बः}{उत्तरायणम्}{शिशिरऋतुः}}
{\sunmoonrsdata{06:59}{17:46}{17:02}{06:56(+1)}{12:22}
{\kalas{05:13 06:06 09:51 09:08 10:35 16:20 11:18 13:27 15:37 17:03 18:39 21:04 22:43 02:01(+1)}}}
{\tnykdata{\anga{\tithi{14}{शुक्ल-चतुर्दशी}}{\time{11-45}{11:41}}\hspace{1ex}}%
{\anga{आश्रेषा}{\time{36-20}{21:31}}\hspace{1ex}}{चन्द्रराशिः—\mbox{कर्कटः\RIGHTarrow{21:31}}}%
{\anga{शोभनः}{\time{38-12}{22:16}}\hspace{1ex}\uanga{अतिगण्डः}}%
{\anga{वणिजः}{\time{11-45}{11:41}}\hspace{1ex}\anga{विष्टिः}{\time{37-1}{21:48}}\hspace{1ex}\uanga{बवः}}{}
}
{षडशीति-पुण्यकालः~15:03\RIGHTarrow{}15:03(+1)\eventsep पार्वणव्रतम् पूर्णिमायाम्\eventsep पञ्च-पर्व-पूजा (पूर्णिमा)\eventsep तपो-मासः\RIGHTarrow{}15:03\eventsep \tamil{திருச்செந்தூர் முருகன் தெப்பம்}\eventsep वेङ्कटाचले पूर्णिमा-गरुड-सेवा}
{Mon} 
\cfoot{\rygdata{08:20--09:41}{11:02--12:22}{13:43--15:04}}
\caldata{FEBRUARY}{19}{\sunmonth{कुम्भः}{7}{}{माघः}{शिशिरऋतुः}{मङ्गलः}{विलम्बः}{उत्तरायणम्}{शिशिरऋतुः}}
{\sunmoonrsdata{06:58}{17:47}{18:18}{---}{12:22}
{\kalas{05:12 06:05 09:51 09:08 10:34 16:20 11:17 13:27 15:37 17:04 18:40 21:04 22:43 02:00(+1)}}}
{\tnykdata{\anga{\tithi{15}{पौर्णमासी}}{\time{2-18}{07:53}}\hspace{1ex}\anga{\tithi{16}{कृष्ण-प्रथमा}}{\time{52-51}{04:06(+1)}}\hspace{1ex}\avamA{}}%
{\anga{मघा}{\time{28-46}{18:28}}\hspace{1ex}}{चन्द्रराशिः—\mbox{सिंहः}}%
{\anga{अतिगण्डः}{\time{27-22}{17:55}}\hspace{1ex}\uanga{सुकर्म}}%
{\anga{बवः}{\time{2-18}{07:53}}\hspace{1ex}\anga{बालवः}{\time{27-32}{17:59}}\hspace{1ex}\anga{कौलवः}{\time{52-51}{04:06(+1)}}\hspace{1ex}\uanga{तैतिलः}}{}
}
{काञ्ची ५१ जगद्गुरु श्री-विद्यातीर्थेन्द्र सरस्वती आराधना~\#{६३४}\eventsep ललिता-जयन्ती\eventsep \tamil{மாசி~செவ்வாய்}\eventsep कुम्भमाघोत्सवः\eventsep माघ-पूर्णिमा\eventsep माघ-पूर्णिमा-स्नानम्\eventsep पार्वण-प्रायश्चित्तावकाशः दर्शे\eventsep पूर्णमासेष्टिः\eventsep पूर्णिमा-व्रतम्\eventsep स्थालीपाकः\eventsep \tamil{திருச்செந்தூர் மாசித் திருவிழா நிறைவு}}
{Tue} 
\cfoot{\rygdata{15:05--16:26}{09:40--11:01}{12:22--13:43}}
\caldata{FEBRUARY}{20}{\sunmonth{कुम्भः}{8}{}{माघः}{शिशिरऋतुः}{बुधः}{विलम्बः}{उत्तरायणम्}{शिशिरऋतुः}}
{\sunmoonsrdata{06:56}{17:48}{19:32}{07:32}{12:22}
{\kalas{05:11 06:04 09:50 09:07 10:34 16:21 11:17 13:27 15:38 17:05 18:41 21:05 22:43 02:00(+1)}}}
{\tnykdata{\anga{\tithi{17}{कृष्ण-द्वितीया}}{\time{43-58}{00:32(+1)}}\hspace{1ex}}%
{\anga{पूर्वफल्गुनी}{\time{21-29}{15:32}}\hspace{1ex}}{चन्द्रराशिः—\mbox{सिंहः\RIGHTarrow{20:51}}}%
{\anga{सुकर्म}{\time{16-47}{13:39}}\hspace{1ex}\uanga{धृतिः}}%
{\anga{तैतिलः}{\time{18-20}{14:17}}\hspace{1ex}\anga{गरः}{\time{43-58}{00:32(+1)}}\hspace{1ex}\uanga{वणिजः}}{}
}
{}
{Wed} 
\cfoot{\rygdata{12:22--13:44}{08:18--09:39}{11:01--12:22}}
\caldata{FEBRUARY}{21}{\sunmonth{कुम्भः}{9}{}{माघः}{शिशिरऋतुः}{गुरुः}{विलम्बः}{उत्तरायणम्}{शिशिरऋतुः}}
{\sunmoonsrdata{06:55}{17:49}{20:45}{08:06}{12:22}
{\kalas{05:10 06:03 09:49 09:06 10:33 16:22 11:17 13:27 15:38 17:06 18:42 21:05 22:43 02:00(+1)}}}
{\tnykdata{\anga{\tithi{18}{कृष्ण-तृतीया}}{\time{36-1}{21:20}}\hspace{1ex}}%
{\anga{उत्तरफल्गुनी}{\time{14-58}{12:54}}\hspace{1ex}}{चन्द्रराशिः—\mbox{कन्या}}%
{\anga{धृतिः}{\time{6-48}{09:38}}\hspace{1ex}\anga{शूलः}{\time{57-42}{06:00(+1)}}\hspace{1ex}\uanga{गण्डः}}%
{\anga{वणिजः}{\time{9-52}{10:52}}\hspace{1ex}\anga{विष्टिः}{\time{36-1}{21:20}}\hspace{1ex}\uanga{बवः}}{}
}
{}
{Thu} 
\cfoot{\rygdata{13:44--15:06}{06:55--08:17}{09:39--11:00}}
\caldata{FEBRUARY}{22}{\sunmonth{कुम्भः}{10}{}{माघः}{शिशिरऋतुः}{शुक्रः}{विलम्बः}{उत्तरायणम्}{शिशिरऋतुः}}
{\sunmoonsrdata{06:54}{17:50}{21:56}{08:39}{12:22}
{\kalas{05:09 06:02 09:49 09:05 10:33 16:23 11:16 13:28 15:39 17:07 18:42 21:06 22:44 01:59(+1)}}}
{\tnykdata{\anga{\tithi{19}{कृष्ण-चतुर्थी}}{\time{29-27}{18:41}}\hspace{1ex}}%
{\anga{हस्तः}{\time{9-39}{10:46}}\hspace{1ex}}{चन्द्रराशिः—\mbox{कन्या\RIGHTarrow{21:55}}}%
{\anga{गण्डः}{\time{49-53}{02:51(+1)}}\hspace{1ex}\uanga{वृद्धिः}}%
{\anga{बवः}{\time{2-34}{07:55}}\hspace{1ex}\anga{बालवः}{\time{29-27}{18:41}}\hspace{1ex}\anga{कौलवः}{\time{56-46}{05:36(+1)}}\hspace{1ex}\uanga{तैतिलः}}{}
}
{द्विजप्रिय-महागणपति सङ्कटहर-चतुर्थी-व्रतम्\eventsep \tamil{எறிபத்த நாயனார் (7) குருபூஜை}}
{Fri} 
\cfoot{\rygdata{11:00--12:22}{15:06--16:28}{08:16--09:38}}
\caldata{FEBRUARY}{23}{\sunmonth{कुम्भः}{11}{}{माघः}{शिशिरऋतुः}{शनिः}{विलम्बः}{उत्तरायणम्}{शिशिरऋतुः}}
{\sunmoonsrdata{06:52}{17:51}{23:04}{09:12}{12:22}
{\kalas{05:08 06:00 09:48 09:04 10:32 16:23 11:16 13:28 15:40 17:07 18:43 21:06 22:44 01:59(+1)}}}
{\tnykdata{\anga{\tithi{20}{कृष्ण-पञ्चमी}}{\time{24-36}{16:43}}\hspace{1ex}}%
{\anga{चित्रा}{\time{5-56}{09:15}}\hspace{1ex}}{चन्द्रराशिः—\mbox{तुला}}%
{\anga{वृद्धिः}{\time{43-35}{00:19(+1)}}\hspace{1ex}\uanga{ध्रुवः}}%
{\anga{तैतिलः}{\time{24-36}{16:43}}\hspace{1ex}\anga{गरः}{\time{52-54}{04:02(+1)}}\hspace{1ex}\uanga{वणिजः}}{}
}
{}
{Sat} 
\cfoot{\rygdata{09:37--11:00}{13:44--15:07}{06:52--08:15}}
\caldata{FEBRUARY}{24}{\sunmonth{कुम्भः}{12}{}{माघः}{शिशिरऋतुः}{भानुः}{विलम्बः}{उत्तरायणम्}{शिशिरऋतुः}}
{\sunmoonsrdata{06:51}{17:52}{00:11(+1)}{09:48}{12:22}
{\kalas{05:07 05:59 09:47 09:03 10:32 16:24 11:16 13:28 15:40 17:08 18:44 21:07 22:44 01:58(+1)}}}
{\tnykdata{\anga{\tithi{21}{कृष्ण-षष्ठी}}{\time{21-47}{15:34}}\hspace{1ex}}%
{\anga{स्वाती}{\time{4-8}{08:31}}\hspace{1ex}}{चन्द्रराशिः—\mbox{तुला\RIGHTarrow{02:30(+1)}}}%
{\anga{ध्रुवः}{\time{38-55}{22:25}}\hspace{1ex}\uanga{व्याघातः}}%
{\anga{वणिजः}{\time{21-47}{15:34}}\hspace{1ex}\anga{विष्टिः}{\time{51-9}{03:19(+1)}}\hspace{1ex}\uanga{बवः}}{}
}
{यशोदा-जयन्ती}
{Sun} 
\cfoot{\rygdata{16:30--17:52}{12:22--13:44}{15:07--16:30}}
\caldata{FEBRUARY}{25}{\sunmonth{कुम्भः}{13}{}{माघः}{शिशिरऋतुः}{सोमः}{विलम्बः}{उत्तरायणम्}{शिशिरऋतुः}}
{\sunmoonsrdata{06:50}{17:53}{01:14(+1)}{10:27}{12:22}
{\kalas{05:06 05:58 09:47 09:03 10:31 16:25 11:15 13:28 15:41 17:09 18:45 21:07 22:44 01:58(+1)}}}
{\tnykdata{\anga{\tithi{22}{कृष्ण-सप्तमी}}{\time{21-7}{15:17}}\hspace{1ex}}%
{\anga{विशाखा}{\time{4-25}{08:36}}\hspace{1ex}}{चन्द्रराशिः—\mbox{वृश्चिकः}}%
{\anga{व्याघातः}{\time{35-57}{21:13}}\hspace{1ex}\uanga{हर्षणः}}%
{\anga{बवः}{\time{21-7}{15:17}}\hspace{1ex}\anga{बालवः}{\time{51-33}{03:27(+1)}}\hspace{1ex}\uanga{कौलवः}}{}
}
{माघ-अष्टका-पूर्वेद्युः\eventsep निक्षुभार्क-सप्तमी\eventsep पञ्च-पर्व-पूजा (अष्टमी)\eventsep शबरी-जयन्ती}
{Mon} 
\cfoot{\rygdata{08:13--09:36}{10:59--12:22}{13:44--15:07}}
\caldata{FEBRUARY}{26}{\sunmonth{कुम्भः}{14}{}{माघः}{शिशिरऋतुः}{मङ्गलः}{विलम्बः}{उत्तरायणम्}{शिशिरऋतुः}}
{\sunmoonsrdata{06:48}{17:54}{02:12(+1)}{11:09}{12:21}
{\kalas{05:05 05:57 09:46 09:02 10:30 16:26 11:15 13:28 15:41 17:10 18:46 21:08 22:44 01:57(+1)}}}
{\tnykdata{\anga{\tithi{23}{कृष्ण-अष्टमी}}{\time{22-34}{15:50}}\hspace{1ex}}%
{\anga{अनूराधा}{\time{6-47}{09:31}}\hspace{1ex}}{चन्द्रराशिः—\mbox{वृश्चिकः}}%
{\anga{हर्षणः}{\time{34-37}{20:39}}\hspace{1ex}\uanga{वज्रम्}}%
{\anga{कौलवः}{\time{22-34}{15:50}}\hspace{1ex}\anga{तैतिलः}{\time{54-1}{04:25(+1)}}\hspace{1ex}\uanga{गरः}}{}
}
{काञ्ची ६६ जगद्गुरु श्री-चन्द्रशेखरेन्द्र सरस्वती ६ आराधना~\#{११२}\eventsep \tamil{மாசி~செவ்வாய்}\eventsep माघ-अष्टका-श्राद्धम्}
{Tue} 
\cfoot{\rygdata{15:08--16:31}{09:35--10:58}{12:21--13:45}}
\caldata{FEBRUARY}{27}{\sunmonth{कुम्भः}{15}{}{माघः}{शिशिरऋतुः}{बुधः}{विलम्बः}{उत्तरायणम्}{शिशिरऋतुः}}
{\sunmoonsrdata{06:47}{17:55}{03:05(+1)}{11:56}{12:21}
{\kalas{05:04 05:56 09:45 09:01 10:30 16:26 11:14 13:28 15:42 17:11 18:47 21:08 22:44 01:57(+1)}}}
{\tnykdata{\anga{\tithi{24}{कृष्ण-नवमी}}{\time{25-59}{17:11}}\hspace{1ex}}%
{\anga{ज्येष्ठा}{\time{11-6}{11:13}}\hspace{1ex}}{चन्द्रराशिः—\mbox{वृश्चिकः\RIGHTarrow{11:13}}}%
{\anga{वज्रम्}{\time{34-43}{20:40}}\hspace{1ex}\uanga{सिद्धिः}}%
{\anga{गरः}{\time{25-59}{17:11}}\hspace{1ex}\anga{वणिजः}{\time{58-16}{06:06(+1)}}\hspace{1ex}\uanga{विष्टिः}}{}
}
{माघ-अन्वष्टका-श्राद्धम्}
{Wed} 
\cfoot{\rygdata{12:21--13:45}{08:11--09:34}{10:58--12:21}}
\caldata{FEBRUARY}{28}{\sunmonth{कुम्भः}{16}{}{माघः}{शिशिरऋतुः}{गुरुः}{विलम्बः}{उत्तरायणम्}{शिशिरऋतुः}}
{\sunmoonsrdata{06:46}{17:56}{03:51(+1)}{12:47}{12:21}
{\kalas{05:03 05:54 09:45 09:00 10:29 16:27 11:14 13:28 15:42 17:12 18:48 21:08 22:44 01:56(+1)}}}
{\tnykdata{\anga{\tithi{25}{कृष्ण-दशमी}}{\time{30-57}{19:09}}\hspace{1ex}}%
{\anga{मूला}{\time{17-0}{13:34}}\hspace{1ex}}{चन्द्रराशिः—\mbox{धनुः}}%
{\anga{सिद्धिः}{\time{35-58}{21:09}}\hspace{1ex}\uanga{व्यतीपातः}}%
{\anga{विष्टिः}{\time{30-57}{19:09}}\hspace{1ex}\uanga{बवः}}{}
}
{}
{Thu} 
\cfoot{\rygdata{13:45--15:09}{06:46--08:10}{09:33--10:57}}
\caldata{MARCH}{1}{\sunmonth{कुम्भः}{17}{}{माघः}{शिशिरऋतुः}{शुक्रः}{विलम्बः}{उत्तरायणम्}{शिशिरऋतुः}}
{\sunmoonsrdata{06:44}{17:57}{04:32(+1)}{13:42}{12:21}
{\kalas{05:02 05:53 09:44 08:59 10:29 16:28 11:14 13:28 15:43 17:13 18:48 21:09 22:44 01:56(+1)}}}
{\tnykdata{\anga{\tithi{26}{कृष्ण-एकादशी}}{\time{37-4}{21:34}}\hspace{1ex}}%
{\anga{पूर्वाषाढा}{\time{24-5}{16:23}}\hspace{1ex}}{चन्द्रराशिः—\mbox{धनुः\RIGHTarrow{23:08}}}%
{\anga{व्यतीपातः}{\time{38-2}{21:57}}\hspace{1ex}\uanga{वरीयान्}}%
{\anga{बवः}{\time{3-56}{08:19}}\hspace{1ex}\anga{बालवः}{\time{37-4}{21:34}}\hspace{1ex}\uanga{कौलवः}}{}
}
{\tamil{காரி நாயனார் (47) குருபூஜை}\eventsep सर्व-विजया-एकादशी\eventsep व्यतीपात-श्राद्धम्}
{Fri} 
\cfoot{\rygdata{10:57--12:21}{15:09--16:33}{08:08--09:33}}
\caldata{MARCH}{2}{\sunmonth{कुम्भः}{18}{}{माघः}{शिशिरऋतुः}{शनिः}{विलम्बः}{उत्तरायणम्}{शिशिरऋतुः}}
{\sunmoonsrdata{06:43}{17:58}{05:08(+1)}{14:38}{12:21}
{\kalas{05:01 05:52 09:43 08:58 10:28 16:28 11:13 13:28 15:43 17:13 18:49 21:09 22:45 01:55(+1)}}}
{\tnykdata{\anga{\tithi{27}{कृष्ण-द्वादशी}}{\time{43-48}{00:14(+1)}}\hspace{1ex}}%
{\anga{उत्तराषाढा}{\time{31-51}{19:28}}\hspace{1ex}}{चन्द्रराशिः—\mbox{मकरः}}%
{\anga{वरीयान्}{\time{40-33}{22:56}}\hspace{1ex}\uanga{परिघः}}%
{\anga{कौलवः}{\time{10-25}{10:53}}\hspace{1ex}\anga{तैतिलः}{\time{43-48}{00:14(+1)}}\hspace{1ex}\uanga{गरः}}{}
}
{हरिवासरः\RIGHTarrow{}04:13\eventsep काञ्ची जगद्गुरु श्री-शङ्कर विजयेन्द्र सरस्वती जयन्ती~\#{५१}\eventsep पक्षवर्धिनी-महाद्वादशी\eventsep विजया/श्रवण-महाद्वादशी}
{Sat} 
\cfoot{\rygdata{09:32--10:56}{13:45--15:10}{06:43--08:07}}
\caldata{MARCH}{3}{\sunmonth{कुम्भः}{19}{}{माघः}{शिशिरऋतुः}{भानुः}{विलम्बः}{उत्तरायणम्}{शिशिरऋतुः}}
{\sunmoonsrdata{06:41}{18:00}{05:40(+1)}{15:34}{12:20}
{\kalas{05:00 05:51 09:42 08:57 10:27 16:29 11:13 13:28 15:44 17:14 18:50 21:10 22:45 01:55(+1)}}}
{\tnykdata{\anga{\tithi{28}{कृष्ण-त्रयोदशी}}{\time{50-42}{02:58(+1)}}\hspace{1ex}}%
{\anga{श्रवणः}{\time{39-51}{22:38}}\hspace{1ex}}{चन्द्रराशिः—\mbox{मकरः}}%
{\anga{परिघः}{\time{43-13}{23:59}}\hspace{1ex}\uanga{शिवः}}%
{\anga{गरः}{\time{17-17}{13:36}}\hspace{1ex}\anga{वणिजः}{\time{50-42}{02:58(+1)}}\hspace{1ex}\uanga{विष्टिः}}{}
}
{प्रदोष-व्रतम्~17:59\RIGHTarrow{}18:50\eventsep श्रवण-व्रतम्}
{Sun} 
\cfoot{\rygdata{16:35--18:00}{12:21--13:45}{15:10--16:35}}
\caldata{MARCH}{4}{\sunmonth{कुम्भः}{20}{}{माघः}{शिशिरऋतुः}{सोमः}{विलम्बः}{उत्तरायणम्}{शिशिरऋतुः}}
{\sunmoonsrdata{06:40}{18:01}{06:09(+1)}{16:32}{12:20}
{\kalas{04:59 05:49 09:42 08:56 10:27 16:30 11:12 13:28 15:44 17:15 18:51 21:10 22:45 01:54(+1)}}}
{\tnykdata{\anga{\tithi{29}{कृष्ण-चतुर्दशी}}{\time{57-22}{05:37(+1)}}\hspace{1ex}}%
{\anga{श्रविष्ठा}{\time{47-42}{01:45(+1)}}\hspace{1ex}}{चन्द्रराशिः—\mbox{मकरः\RIGHTarrow{12:12}}}%
{\anga{शिवः}{\time{45-45}{00:58(+1)}}\hspace{1ex}\uanga{सिद्धः}}%
{\anga{विष्टिः}{\time{24-6}{16:19}}\hspace{1ex}\anga{शकुनिः}{\time{57-22}{05:37(+1)}}\hspace{1ex}\uanga{चतुष्पात्}}{}
}
{मासशिवरात्रिः\eventsep महाशिवरात्रिः\eventsep पञ्च-पर्व-पूजा (चतुर्दशी)\eventsep माघ-यम-तर्पणम्}
{Mon} 
\cfoot{\rygdata{08:05--09:30}{10:55--12:20}{13:45--15:10}}
\caldata{MARCH}{5}{\sunmonth{कुम्भः}{21}{}{माघः}{शिशिरऋतुः}{मङ्गलः}{विलम्बः}{उत्तरायणम्}{शिशिरऋतुः}}
{\sunmoonsrdata{06:39}{18:02}{06:36(+1)}{17:29}{12:20}
{\kalas{04:58 05:48 09:41 08:55 10:26 16:31 11:12 13:28 15:45 17:16 18:52 21:10 22:45 01:54(+1)}}}
{\tnykdata{\fulltithi{\tithi{30}{अमावास्या}}}%
{\anga{शतभिषक्}{\time{55-5}{04:41(+1)}}\hspace{1ex}}{चन्द्रराशिः—\mbox{कुम्भः}}%
{\anga{सिद्धः}{\time{47-57}{01:50(+1)}}\hspace{1ex}\uanga{साध्यः}}%
{\anga{चतुष्पात्}{\time{30-33}{18:52}}\hspace{1ex}\uanga{नाग}}{}
}
{कलियुगादिः\eventsep \tamil{கொச்செங்கட் சோழ நாயனார் (59) குருபூஜை}\eventsep \tamil{மாசி~செவ்வாய்}\eventsep माघ-स्नानपूर्तिः\eventsep पार्वणव्रतम् अमावास्यायाम्\eventsep पञ्च-पर्व-पूजा (अमावास्या)\eventsep पुरन्दरदास-आराधना~\#{४५५}\eventsep सर्व-माघ-अमावास्या (अलभ्यम्–शतभिषक्, पुष्कला)}
{Tue} 
\cfoot{\rygdata{15:11--16:36}{09:29--10:55}{12:20--13:45}}
\caldata{MARCH}{6}{\sunmonth{कुम्भः}{22}{}{माघः}{शिशिरऋतुः}{बुधः}{विलम्बः}{उत्तरायणम्}{शिशिरऋतुः}}
{\sunmoonsrdata{06:37}{18:03}{---}{18:26}{12:20}
{\kalas{04:56 05:47 09:40 08:54 10:26 16:31 11:11 13:28 15:45 17:17 18:53 21:11 22:45 01:53(+1)}}}
{\tnykdata{\anga{\tithi{30}{अमावास्या}}{\time{3-36}{08:03}}\hspace{1ex}}%
{\fullanga{पूर्वप्रोष्ठपदा}}{चन्द्रराशिः—\mbox{कुम्भः\RIGHTarrow{00:43(+1)}}}%
{\anga{साध्यः}{\time{49-40}{02:29(+1)}}\hspace{1ex}\uanga{शुभः}}%
{\anga{नाग}{\time{3-36}{08:03}}\hspace{1ex}\anga{किंस्तुघ्नः}{\time{36-24}{21:11}}\hspace{1ex}\uanga{बवः}}{}
}
{दर्शेष्टिः\eventsep काञ्ची ६७ जगद्गुरु श्री-महादेवेन्द्र सरस्वती ५ आराधना~\#{११२}\eventsep पार्वण-प्रायश्चित्तावकाशः पौर्णमास्याम्\eventsep स्थालीपाकः}
{Wed} 
\cfoot{\rygdata{12:20--13:45}{08:03--09:28}{10:54--12:20}}
\caldata{MARCH}{7}{\sunmonth{कुम्भः}{23}{}{फाल्गुनः}{शिशिरऋतुः}{गुरुः}{विलम्बः}{उत्तरायणम्}{शिशिरऋतुः}}
{\sunmoonrsdata{06:36}{18:04}{07:01}{19:23}{12:20}
{\kalas{04:55 05:46 09:39 08:53 10:25 16:32 11:11 13:28 15:46 17:18 18:54 21:11 22:45 01:53(+1)}}}
{\tnykdata{\anga{\tithi{1}{शुक्ल-प्रथमा}}{\time{9-5}{10:14}}\hspace{1ex}}%
{\anga{पूर्वप्रोष्ठपदा}{\time{1-55}{07:22}}\hspace{1ex}}{चन्द्रराशिः—\mbox{मीनः}}%
{\anga{शुभः}{\time{50-48}{02:55(+1)}}\hspace{1ex}\uanga{शुक्लः}}%
{\anga{बवः}{\time{9-5}{10:14}}\hspace{1ex}\anga{बालवः}{\time{41-29}{23:11}}\hspace{1ex}\uanga{कौलवः}}{}
}
{पयोव्रत-आरम्भः\eventsep चन्द्र-दर्शनम्~18:03\RIGHTarrow{}18:53}
{Thu} 
\cfoot{\rygdata{13:45--15:12}{06:36--08:02}{09:28--10:54}}
\caldata{MARCH}{8}{\sunmonth{कुम्भः}{24}{}{फाल्गुनः}{शिशिरऋतुः}{शुक्रः}{विलम्बः}{उत्तरायणम्}{शिशिरऋतुः}}
{\sunmoonrsdata{06:34}{18:04}{07:28}{20:22}{12:19}
{\kalas{04:54 05:44 09:38 08:52 10:24 16:32 11:10 13:28 15:46 17:18 18:54 21:12 22:45 01:52(+1)}}}
{\tnykdata{\anga{\tithi{2}{शुक्ल-द्वितीया}}{\time{13-44}{12:04}}\hspace{1ex}}%
{\anga{उत्तरप्रोष्ठपदा}{\time{7-56}{09:45}}\hspace{1ex}}{चन्द्रराशिः—\mbox{मीनः}}%
{\anga{शुक्लः}{\time{51-16}{03:05(+1)}}\hspace{1ex}\uanga{ब्रह्म}}%
{\anga{कौलवः}{\time{13-44}{12:04}}\hspace{1ex}\anga{तैतिलः}{\time{45-42}{00:51(+1)}}\hspace{1ex}\uanga{गरः}}{}
}
{भृगुरेवती-पुण्यकालः~09:45\RIGHTarrow{}\eventsep फूलेरा-दूज्\eventsep रामकृष्ण-परमहंस-जयन्ती~\#{१८४}}
{Fri} 
\cfoot{\rygdata{10:53--12:19}{15:12--16:38}{08:00--09:27}}
\caldata{MARCH}{9}{\sunmonth{कुम्भः}{25}{}{फाल्गुनः}{शिशिरऋतुः}{शनिः}{विलम्बः}{उत्तरायणम्}{शिशिरऋतुः}}
{\sunmoonrsdata{06:33}{18:05}{07:55}{21:21}{12:19}
{\kalas{04:53 05:43 09:38 08:51 10:24 16:33 11:10 13:28 15:47 17:19 18:55 21:12 22:45 01:52(+1)}}}
{\tnykdata{\anga{\tithi{3}{शुक्ल-तृतीया}}{\time{17-29}{13:32}}\hspace{1ex}}%
{\anga{रेवती}{\time{13-4}{11:47}}\hspace{1ex}}{चन्द्रराशिः—\mbox{मीनः\RIGHTarrow{11:47}}}%
{\anga{ब्रह्म}{\time{51-0}{02:57(+1)}}\hspace{1ex}\uanga{इन्द्रः}}%
{\anga{गरः}{\time{17-29}{13:32}}\hspace{1ex}\anga{वणिजः}{\time{48-56}{02:08(+1)}}\hspace{1ex}\uanga{विष्टिः}}{}
}
{}
{Sat} 
\cfoot{\rygdata{09:26--10:53}{13:46--15:12}{06:33--07:59}}
\caldata{MARCH}{10}{\sunmonth{कुम्भः}{26}{}{फाल्गुनः}{शिशिरऋतुः}{भानुः}{विलम्बः}{उत्तरायणम्}{शिशिरऋतुः}}
{\sunmoonrsdata{06:31}{18:06}{08:25}{22:22}{12:19}
{\kalas{04:52 05:42 09:37 08:50 10:23 16:34 11:09 13:28 15:47 17:20 18:56 21:12 22:45 01:51(+1)}}}
{\tnykdata{\anga{\tithi{4}{शुक्ल-चतुर्थी}}{\time{20-12}{14:36}}\hspace{1ex}}%
{\anga{अश्विनी}{\time{17-14}{13:25}}\hspace{1ex}}{चन्द्रराशिः—\mbox{मेषः}}%
{\anga{इन्द्रः}{\time{49-55}{02:30(+1)}}\hspace{1ex}\uanga{वैधृतिः}}%
{\anga{विष्टिः}{\time{20-12}{14:36}}\hspace{1ex}\anga{बवः}{\time{51-7}{02:58(+1)}}\hspace{1ex}\uanga{बालवः}}{}
}
{कपालीश्वर-ध्वजारोहणम्\eventsep पून्तानं-जयन्ती~\#{४७२}}
{Sun} 
\cfoot{\rygdata{16:40--18:06}{12:19--13:46}{15:13--16:40}}
\caldata{MARCH}{11}{\sunmonth{कुम्भः}{27}{}{फाल्गुनः}{शिशिरऋतुः}{सोमः}{विलम्बः}{उत्तरायणम्}{शिशिरऋतुः}}
{\sunmoonrsdata{07:30}{19:07}{09:58}{00:25(+1)}{13:18}
{\kalas{05:51 06:40 10:36 09:49 11:22 17:34 12:09 14:28 16:48 18:21 19:57 22:13 23:45 02:50(+1)}}}
{\tnykdata{\anga{\tithi{5}{शुक्ल-पञ्चमी}}{\time{21-48}{16:13}}\hspace{1ex}}%
{\anga{अपभरणी}{\time{20-20}{15:38}}\hspace{1ex}}{चन्द्रराशिः—\mbox{मेषः\RIGHTarrow{21:52}}}%
{\anga{वैधृतिः}{\time{47-56}{02:40(+1)}}\hspace{1ex}\uanga{विष्कम्भः}}%
{\anga{बालवः}{\time{21-48}{16:13}}\hspace{1ex}\anga{कौलवः}{\time{52-6}{04:20(+1)}}\hspace{1ex}\uanga{तैतिलः}}{}
}
{कृत्तिका-व्रतम्\eventsep \tamil{கபாலீ ஸூர்ய~சந்த்ர~வட்டம்}\eventsep वैधृति-श्राद्धम्}
{Mon} 
\cfoot{\rygdata{08:57--10:24}{11:51--13:19}{14:46--16:13}}
\caldata{MARCH}{12}{\sunmonth{कुम्भः}{28}{}{फाल्गुनः}{शिशिरऋतुः}{मङ्गलः}{विलम्बः}{उत्तरायणम्}{शिशिरऋतुः}}
{\sunmoonrsdata{07:28}{19:08}{10:36}{01:27(+1)}{13:18}
{\kalas{05:49 06:39 10:35 09:48 11:21 17:35 12:08 14:28 16:48 18:22 19:57 22:13 23:45 02:50(+1)}}}
{\tnykdata{\anga{\tithi{6}{शुक्ल-षष्ठी}}{\time{22-8}{16:19}}\hspace{1ex}}%
{\anga{कृत्तिका}{\time{22-13}{16:21}}\hspace{1ex}}{चन्द्रराशिः—\mbox{वृषभः}}%
{\anga{विष्कम्भः}{\time{44-56}{01:27(+1)}}\hspace{1ex}\uanga{प्रीतिः}}%
{\anga{तैतिलः}{\time{22-8}{16:19}}\hspace{1ex}\anga{गरः}{\time{51-45}{04:10(+1)}}\hspace{1ex}\uanga{वणिजः}}{}
}
{षष्ठी-व्रतम्\eventsep कपाल्यधिकार-नन्दी\eventsep \tamil{கபாலீ பூதண் பூதகீ}\eventsep \tamil{மாசி~செவ்வாய்}}
{Tue} 
\cfoot{\rygdata{16:13--17:41}{10:23--11:51}{13:18--14:46}}
\caldata{MARCH}{13}{\sunmonth{कुम्भः}{29}{}{फाल्गुनः}{शिशिरऋतुः}{बुधः}{विलम्बः}{उत्तरायणम्}{शिशिरऋतुः}}
{\sunmoonrsdata{07:27}{19:09}{11:21}{02:29(+1)}{13:18}
{\kalas{05:48 06:38 10:34 09:47 11:21 17:36 12:08 14:28 16:49 18:22 19:58 22:13 23:45 02:49(+1)}}}
{\tnykdata{\anga{\tithi{7}{शुक्ल-सप्तमी}}{\time{21-5}{15:53}}\hspace{1ex}}%
{\anga{रोहिणी}{\time{22-45}{16:33}}\hspace{1ex}}{चन्द्रराशिः—\mbox{वृषभः\RIGHTarrow{04:26(+1)}}}%
{\anga{प्रीतिः}{\time{40-49}{23:47}}\hspace{1ex}\uanga{आयुष्मान्}}%
{\anga{वणिजः}{\time{21-5}{15:53}}\hspace{1ex}\anga{विष्टिः}{\time{49-59}{03:27(+1)}}\hspace{1ex}\uanga{बवः}}{}
}
{नन्दा-सप्तमी\eventsep श्री-राघवेन्द्र-स्वामि-जयन्ती~\#{४२५}}
{Wed} 
\cfoot{\rygdata{13:18--14:46}{08:54--10:22}{11:50--13:18}}
\caldata{MARCH}{14}{\sunmonth{मीनः}{1}{\mbox{कुम्भः{\tiny\RIGHTarrow}{16:53}}}{फाल्गुनः}{शिशिरऋतुः}{गुरुः}{विलम्बः}{उत्तरायणम्}{शिशिरऋतुः}}
{\sunmoonrsdata{07:25}{19:10}{12:15}{03:27(+1)}{13:18}
{\kalas{05:47 06:36 10:33 09:46 11:20 17:36 12:07 14:28 16:49 18:23 19:59 22:14 23:45 02:49(+1)}}}
{\tnykdata{\anga{\tithi{8}{शुक्ल-अष्टमी}}{\time{18-35}{14:51}}\hspace{1ex}}%
{\anga{मृगशीर्षम्}{\time{21-52}{16:10}}\hspace{1ex}}{चन्द्रराशिः—\mbox{मिथुनम्}}%
{\anga{आयुष्मान्}{\time{35-33}{21:38}}\hspace{1ex}\uanga{सौभाग्यः}}%
{\anga{बवः}{\time{18-35}{14:51}}\hspace{1ex}\anga{बालवः}{\time{46-44}{02:07(+1)}}\hspace{1ex}\uanga{कौलवः}}{}
}
{सावित्री-व्रतम्\eventsep \tamil{கபாலீ சவுடல் விமானம்}\eventsep कपालि-वृषभ-वाहनम्\eventsep मीन-रवि-सङ्क्रमण-षडशीति-पुण्यकालः~16:53\RIGHTarrow{}16:53(+1)}
{Thu} 
\cfoot{\rygdata{14:46--16:14}{07:25--08:53}{10:21--11:50}}
\caldata{MARCH}{15}{\sunmonth{मीनः}{2}{}{फाल्गुनः}{शिशिरऋतुः}{शुक्रः}{विलम्बः}{उत्तरायणम्}{शिशिरऋतुः}}
{\sunmoonrsdata{07:24}{19:11}{13:16}{04:20(+1)}{13:17}
{\kalas{05:46 06:35 10:32 09:45 11:19 17:37 12:07 14:28 16:50 18:24 20:00 22:14 23:45 02:48(+1)}}}
{\tnykdata{\anga{\tithi{9}{शुक्ल-नवमी}}{\time{14-35}{13:14}}\hspace{1ex}}%
{\anga{आर्द्रा}{\time{19-31}{15:12}}\hspace{1ex}}{चन्द्रराशिः—\mbox{मिथुनम्}}%
{\anga{सौभाग्यः}{\time{29-6}{19:02}}\hspace{1ex}\uanga{शोभनः}}%
{\anga{कौलवः}{\time{14-35}{13:14}}\hspace{1ex}\anga{तैतिलः}{\time{42-1}{00:12(+1)}}\hspace{1ex}\uanga{गरः}}{}
}
{\tamil{கபாலீ பல்லக்கு விழா}\eventsep वेङ्कटाचले प्लवोत्सव-प्रारम्भः}
{Fri} 
\cfoot{\rygdata{11:49--13:17}{16:14--17:43}{08:52--10:21}}
\caldata{MARCH}{16}{\sunmonth{मीनः}{3}{}{फाल्गुनः}{शिशिरऋतुः}{शनिः}{विलम्बः}{उत्तरायणम्}{शिशिरऋतुः}}
{\sunmoonrsdata{07:22}{19:12}{14:23}{05:06(+1)}{13:17}
{\kalas{05:45 06:33 10:32 09:44 11:19 17:37 12:06 14:28 16:50 18:25 20:01 22:14 23:45 02:47(+1)}}}
{\tnykdata{\anga{\tithi{10}{शुक्ल-दशमी}}{\time{9-11}{11:03}}\hspace{1ex}}%
{\anga{पुनर्वसुः}{\time{15-47}{13:41}}\hspace{1ex}}{चन्द्रराशिः—\mbox{मिथुनम्\RIGHTarrow{08:07}}}%
{\anga{शोभनः}{\time{21-31}{15:59}}\hspace{1ex}\uanga{अतिगण्डः}}%
{\anga{गरः}{\time{9-11}{11:03}}\hspace{1ex}\anga{वणिजः}{\time{35-57}{21:45}}\hspace{1ex}\uanga{विष्टिः}}{}
}
{कपालीश्वरयात्रा\eventsep स्मार्त-आमलकी-एकादशी (गृहस्थ)\eventsep वेङ्कटाचले प्लवोत्सवः}
{Sat} 
\cfoot{\rygdata{10:20--11:48}{14:46--16:15}{07:22--08:51}}
\caldata{MARCH}{17}{\sunmonth{मीनः}{4}{}{फाल्गुनः}{शिशिरऋतुः}{भानुः}{विलम्बः}{उत्तरायणम्}{शिशिरऋतुः}}
{\sunmoonrsdata{07:21}{19:13}{15:36}{05:48(+1)}{13:17}
{\kalas{05:44 06:32 10:31 09:43 11:18 17:38 12:06 14:28 16:51 18:26 20:01 22:15 23:45 02:47(+1)}}}
{\tnykdata{\anga{\tithi{11}{शुक्ल-एकादशी}}{\time{2-29}{08:20}}\hspace{1ex}\anga{\tithi{12}{शुक्ल-द्वादशी}}{\time{54-41}{05:13(+1)}}\hspace{1ex}\avamA{}}%
{\anga{पुष्यः}{\time{10-47}{11:40}}\hspace{1ex}}{चन्द्रराशिः—\mbox{कर्कटः}}%
{\anga{अतिगण्डः}{\time{12-56}{12:31}}\hspace{1ex}\uanga{सुकर्म}}%
{\anga{विष्टिः}{\time{2-29}{08:20}}\hspace{1ex}\anga{बवः}{\time{28-41}{18:49}}\hspace{1ex}\anga{बालवः}{\time{54-41}{05:13(+1)}}\hspace{1ex}\uanga{कौलवः}}{}
}
{पयोव्रत-समापनम्\eventsep हरिवासरः\RIGHTarrow{}13:36\eventsep \tamil{கபாலீ அறுபத்து மூவர்}\eventsep नरसिंह-द्वादशी\eventsep रंगभरी एकादशी\eventsep रविपुष्ययोग-पुण्यकालः\RIGHTarrow{}11:40\eventsep स्मार्त-आमलकी-एकादशी (सन्न्यस्थ)\eventsep त्रिस्पर्शा-महाद्वादशी\eventsep वेङ्कटाचले प्लवोत्सवः\eventsep वैष्णव-आमलकी-एकादशी}
{Sun} 
\cfoot{\rygdata{17:44--19:13}{13:17--14:46}{16:15--17:44}}
\caldata{MARCH}{18}{\sunmonth{मीनः}{5}{}{फाल्गुनः}{शिशिरऋतुः}{सोमः}{विलम्बः}{उत्तरायणम्}{शिशिरऋतुः}}
{\sunmoonrsdata{07:19}{19:14}{16:50}{06:25(+1)}{13:17}
{\kalas{05:42 06:31 10:30 09:42 11:17 17:39 12:05 14:28 16:51 18:26 20:02 22:15 23:45 02:46(+1)}}}
{\tnykdata{\anga{\tithi{13}{शुक्ल-त्रयोदशी}}{\time{46-12}{01:48(+1)}}\hspace{1ex}}%
{\anga{आश्रेषा}{\time{4-48}{09:14}}\hspace{1ex}\anga{मघा}{\time{58-5}{06:33(+1)}}\hspace{1ex}}{चन्द्रराशिः—\mbox{कर्कटः\RIGHTarrow{09:14}}}%
{\anga{सुकर्म}{\time{3-33}{08:45}}\hspace{1ex}\anga{धृतिः}{\time{53-33}{04:44(+1)}}\hspace{1ex}\uanga{शूलः}}%
{\anga{कौलवः}{\time{20-32}{15:32}}\hspace{1ex}\anga{तैतिलः}{\time{46-12}{01:48(+1)}}\hspace{1ex}\uanga{गरः}}{}
}
{काञ्ची ६९ जगद्गुरु श्री-जयेन्द्र सरस्वती आराधना~\#{१}\eventsep सोम-प्रदोष-व्रतम्~19:14\RIGHTarrow{}20:02\eventsep वेङ्कटाचले प्लवोत्सवः}
{Mon} 
\cfoot{\rygdata{08:49--10:18}{11:47--13:17}{14:46--16:15}}
\caldata{MARCH}{19}{\sunmonth{मीनः}{6}{}{फाल्गुनः}{शिशिरऋतुः}{मङ्गलः}{विलम्बः}{उत्तरायणम्}{शिशिरऋतुः}}
{\sunmoonrsdata{07:18}{19:15}{18:04}{06:59(+1)}{13:16}
{\kalas{05:41 06:29 10:29 09:41 11:17 17:39 12:05 14:28 16:51 18:27 20:03 22:15 23:45 02:46(+1)}}}
{\tnykdata{\anga{\tithi{14}{शुक्ल-चतुर्दशी}}{\time{37-22}{22:15}}\hspace{1ex}}%
{\anga{पूर्वफल्गुनी}{\time{51-9}{03:45(+1)}}\hspace{1ex}}{चन्द्रराशिः—\mbox{सिंहः}}%
{\anga{शूलः}{\time{43-21}{00:38(+1)}}\hspace{1ex}\uanga{गण्डः}}%
{\anga{गरः}{\time{11-50}{12:02}}\hspace{1ex}\anga{वणिजः}{\time{37-22}{22:15}}\hspace{1ex}\uanga{विष्टिः}}{}
}
{होलिका-पूर्णिमा\eventsep \tamil{கற்பகாம்பாள்–கபாலீஶ்வரர் திருக்கல்யாணம்}\eventsep पञ्च-पर्व-पूजा (पूर्णिमा)\eventsep वेङ्कटाचले पूर्णिमा-गरुड-सेवा\eventsep वेङ्कटाचले प्लवोत्सव-समापनम्}
{Tue} 
\cfoot{\rygdata{16:16--17:45}{10:17--11:47}{13:16--14:46}}
\caldata{MARCH}{20}{\sunmonth{मीनः}{7}{}{फाल्गुनः}{शिशिरऋतुः}{बुधः}{विलम्बः}{उत्तरायणम्}{शिशिरऋतुः}}
{\sunmoonrsdata{07:16}{19:16}{19:18}{---}{13:16}
{\kalas{05:40 06:28 10:28 09:40 11:16 17:40 12:04 14:28 16:52 18:28 20:04 22:16 23:45 02:45(+1)}}}
{\tnykdata{\anga{\tithi{15}{पौर्णमासी}}{\time{28-35}{18:42}}\hspace{1ex}}%
{\anga{उत्तरफल्गुनी}{\time{44-25}{01:02(+1)}}\hspace{1ex}}{चन्द्रराशिः—\mbox{सिंहः\RIGHTarrow{09:04}}}%
{\anga{गण्डः}{\time{33-15}{20:34}}\hspace{1ex}\uanga{वृद्धिः}}%
{\anga{विष्टिः}{\time{2-58}{08:28}}\hspace{1ex}\anga{बवः}{\time{28-35}{18:42}}\hspace{1ex}\anga{बालवः}{\time{54-19}{05:00(+1)}}\hspace{1ex}\uanga{कौलवः}}{}
}
{चैतन्य-महाप्रभु-जयन्ती~\#{५३४}\eventsep होलि\eventsep काम-दहनम्\eventsep उमा-कपालीश्वर-दर्शनम्\eventsep \tamil{கபாலீ விடையாற்றி தொடக்கம்}\eventsep मेष-विषु-पुण्यकालः~10:58\RIGHTarrow{}18:58\eventsep मन्वादिः-(रुद्रः-[१२])\eventsep पार्वणव्रतम् पूर्णिमायाम्\eventsep पूर्णिमा-व्रतम्\eventsep मीनोत्तरफाल्गुनोत्सवः\eventsep तपस्य-मासः/शिशिरऋतुः\RIGHTarrow{}14:58\eventsep विषुवदिनम्}
{Wed} 
\cfoot{\rygdata{13:16--14:46}{08:46--10:16}{11:46--13:16}}
\caldata{MARCH}{21}{\sunmonth{मीनः}{8}{}{फाल्गुनः}{शिशिरऋतुः}{गुरुः}{विलम्बः}{उत्तरायणम्}{शिशिरऋतुः}}
{\sunmoonsrdata{07:15}{19:17}{20:30}{07:33}{13:16}
{\kalas{05:39 06:27 10:27 09:39 11:15 17:40 12:03 14:28 16:52 18:29 20:04 22:16 23:45 02:44(+1)}}}
{\tnykdata{\anga{\tithi{16}{कृष्ण-प्रथमा}}{\time{20-19}{15:22}}\hspace{1ex}}%
{\anga{हस्तः}{\time{38-19}{22:34}}\hspace{1ex}}{चन्द्रराशिः—\mbox{कन्या}}%
{\anga{वृद्धिः}{\time{23-36}{16:41}}\hspace{1ex}\uanga{ध्रुवः}}%
{\anga{कौलवः}{\time{20-19}{15:22}}\hspace{1ex}\anga{तैतिलः}{\time{46-29}{01:50(+1)}}\hspace{1ex}\uanga{गरः}}{}
}
{आम्र-कुसुम-प्राशनम्\eventsep पार्वण-प्रायश्चित्तावकाशः दर्शे\eventsep पूर्णमासेष्टिः\eventsep स्थालीपाकः}
{Thu} 
\cfoot{\rygdata{14:46--16:16}{07:15--08:45}{10:15--11:45}}
\caldata{MARCH}{22}{\sunmonth{मीनः}{9}{}{फाल्गुनः}{शिशिरऋतुः}{शुक्रः}{विलम्बः}{उत्तरायणम्}{शिशिरऋतुः}}
{\sunmoonsrdata{07:13}{19:18}{21:42}{08:07}{13:15}
{\kalas{05:38 06:25 10:26 09:38 11:15 17:41 12:03 14:28 16:53 18:29 20:05 22:16 23:45 02:44(+1)}}}
{\tnykdata{\anga{\tithi{17}{कृष्ण-द्वितीया}}{\time{13-0}{12:25}}\hspace{1ex}}%
{\anga{चित्रा}{\time{33-20}{20:33}}\hspace{1ex}}{चन्द्रराशिः—\mbox{कन्या\RIGHTarrow{09:30}}}%
{\anga{ध्रुवः}{\time{14-47}{13:08}}\hspace{1ex}\uanga{व्याघातः}}%
{\anga{गरः}{\time{13-0}{12:25}}\hspace{1ex}\anga{वणिजः}{\time{39-48}{23:09}}\hspace{1ex}\uanga{विष्टिः}}{}
}
{ब्रह्म-कल्पादिः}
{Fri} 
\cfoot{\rygdata{11:45--13:15}{16:17--17:47}{08:44--10:14}}
\caldata{MARCH}{23}{\sunmonth{मीनः}{10}{}{फाल्गुनः}{शिशिरऋतुः}{शनिः}{विलम्बः}{उत्तरायणम्}{शिशिरऋतुः}}
{\sunmoonsrdata{07:12}{19:19}{22:52}{08:42}{13:15}
{\kalas{05:36 06:24 10:25 09:37 11:14 17:42 12:02 14:28 16:53 18:30 20:06 22:16 23:45 02:43(+1)}}}
{\tnykdata{\anga{\tithi{18}{कृष्ण-तृतीया}}{\time{7-5}{10:02}}\hspace{1ex}}%
{\anga{स्वाती}{\time{29-55}{19:10}}\hspace{1ex}}{चन्द्रराशिः—\mbox{तुला}}%
{\anga{व्याघातः}{\time{7-7}{10:03}}\hspace{1ex}\uanga{हर्षणः}}%
{\anga{विष्टिः}{\time{7-5}{10:02}}\hspace{1ex}\anga{बवः}{\time{34-45}{21:06}}\hspace{1ex}\uanga{बालवः}}{}
}
{भालचन्द्र-महागणपति सङ्कटहर-चतुर्थी-व्रतम्\eventsep छत्रपति-शिवाजी-जयन्ती~\#{३९०}\eventsep काञ्ची जगद्गुरु श्री-जयेन्द्र सरस्वती आश्रम-स्वीकार-जयन्ती~\#{६६}\eventsep \tamil{காரைக்கால் அம்மையார் (23) குருபூஜை}}
{Sat} 
\cfoot{\rygdata{10:13--11:44}{14:46--16:17}{07:12--08:42}}
\caldata{MARCH}{24}{\sunmonth{मीनः}{11}{}{फाल्गुनः}{शिशिरऋतुः}{भानुः}{विलम्बः}{उत्तरायणम्}{शिशिरऋतुः}}
{\sunmoonsrdata{07:10}{19:20}{23:58}{09:21}{13:15}
{\kalas{05:35 06:23 10:25 09:36 11:13 17:42 12:02 14:28 16:54 18:31 20:07 22:17 23:45 02:43(+1)}}}
{\tnykdata{\anga{\tithi{19}{कृष्ण-चतुर्थी}}{\time{2-57}{08:21}}\hspace{1ex}}%
{\anga{विशाखा}{\time{28-22}{18:31}}\hspace{1ex}}{चन्द्रराशिः—\mbox{तुला\RIGHTarrow{12:36}}}%
{\anga{हर्षणः}{\time{0-53}{07:32}}\hspace{1ex}\anga{वज्रम्}{\time{56-15}{05:40(+1)}}\hspace{1ex}\uanga{सिद्धिः}}%
{\anga{बालवः}{\time{2-57}{08:21}}\hspace{1ex}\anga{कौलवः}{\time{31-36}{19:49}}\hspace{1ex}\uanga{तैतिलः}}{}
}
{रङ्ग-पञ्चमी}
{Sun} 
\cfoot{\rygdata{17:48--19:20}{13:15--14:46}{16:17--17:48}}
\caldata{MARCH}{25}{\sunmonth{मीनः}{12}{}{फाल्गुनः}{शिशिरऋतुः}{सोमः}{विलम्बः}{उत्तरायणम्}{शिशिरऋतुः}}
{\sunmoonsrdata{07:08}{19:20}{01:01(+1)}{10:03}{13:14}
{\kalas{05:34 06:21 10:24 09:35 11:12 17:43 12:01 14:28 16:54 18:32 20:08 22:17 23:45 02:42(+1)}}}
{\tnykdata{\anga{\tithi{20}{कृष्ण-पञ्चमी}}{\time{0-52}{07:30}}\hspace{1ex}}%
{\anga{अनूराधा}{\time{28-57}{18:43}}\hspace{1ex}}{चन्द्रराशिः—\mbox{वृश्चिकः}}%
{\anga{सिद्धिः}{\time{53-24}{04:30(+1)}}\hspace{1ex}\uanga{व्यतीपातः}}%
{\anga{तैतिलः}{\time{0-52}{07:30}}\hspace{1ex}\anga{गरः}{\time{30-38}{19:24}}\hspace{1ex}\uanga{वणिजः}}{}
}
{}
{Mon} 
\cfoot{\rygdata{08:40--10:12}{11:43--13:15}{14:46--16:17}}
\caldata{MARCH}{26}{\sunmonth{मीनः}{13}{}{फाल्गुनः}{शिशिरऋतुः}{मङ्गलः}{विलम्बः}{उत्तरायणम्}{शिशिरऋतुः}}
{\sunmoonsrdata{07:07}{19:21}{01:57(+1)}{10:49}{13:14}
{\kalas{05:33 06:20 10:23 09:34 11:12 17:43 12:01 14:28 16:54 18:32 20:08 22:17 23:45 02:41(+1)}}}
{\tnykdata{\anga{\tithi{21}{कृष्ण-षष्ठी}}{\time{1-0}{07:31}}\hspace{1ex}}%
{\anga{ज्येष्ठा}{\time{31-40}{19:47}}\hspace{1ex}}{चन्द्रराशिः—\mbox{वृश्चिकः\RIGHTarrow{19:47}}}%
{\anga{व्यतीपातः}{\time{52-15}{04:01(+1)}}\hspace{1ex}\uanga{वरीयान्}}%
{\anga{वणिजः}{\time{1-0}{07:31}}\hspace{1ex}\anga{विष्टिः}{\time{31-51}{19:52}}\hspace{1ex}\uanga{बवः}}{}
}
{फाल्गुन-अष्टका-पूर्वेद्युः\eventsep व्यतीपात-श्राद्धम्}
{Tue} 
\cfoot{\rygdata{16:18--17:50}{10:10--11:42}{13:14--14:46}}
\caldata{MARCH}{27}{\sunmonth{मीनः}{14}{}{फाल्गुनः}{शिशिरऋतुः}{बुधः}{विलम्बः}{उत्तरायणम्}{शिशिरऋतुः}}
{\sunmoonsrdata{07:05}{19:22}{02:47(+1)}{11:40}{13:14}
{\kalas{05:32 06:19 10:22 09:33 11:11 17:44 12:00 14:28 16:55 18:33 20:09 22:18 23:45 02:41(+1)}}}
{\tnykdata{\anga{\tithi{22}{कृष्ण-सप्तमी}}{\time{3-18}{08:25}}\hspace{1ex}}%
{\anga{मूला}{\time{36-22}{21:39}}\hspace{1ex}}{चन्द्रराशिः—\mbox{धनुः}}%
{\anga{वरीयान्}{\time{52-37}{04:08(+1)}}\hspace{1ex}\uanga{परिघः}}%
{\anga{बवः}{\time{3-18}{08:25}}\hspace{1ex}\anga{बालवः}{\time{35-9}{21:09}}\hspace{1ex}\uanga{कौलवः}}{}
}
{पञ्च-पर्व-पूजा (अष्टमी)\eventsep फाल्गुन-अष्टका-श्राद्धम्}
{Wed} 
\cfoot{\rygdata{13:14--14:46}{08:38--10:10}{11:42--13:14}}
\caldata{MARCH}{28}{\sunmonth{मीनः}{15}{}{फाल्गुनः}{शिशिरऋतुः}{गुरुः}{विलम्बः}{उत्तरायणम्}{शिशिरऋतुः}}
{\sunmoonsrdata{07:04}{19:23}{03:31(+1)}{12:34}{13:14}
{\kalas{05:30 06:17 10:21 09:32 11:10 17:45 12:00 14:28 16:55 18:34 20:10 22:18 23:45 02:40(+1)}}}
{\tnykdata{\anga{\tithi{23}{कृष्ण-अष्टमी}}{\time{7-30}{10:04}}\hspace{1ex}}%
{\anga{पूर्वाषाढा}{\time{42-42}{00:09(+1)}}\hspace{1ex}}{चन्द्रराशिः—\mbox{धनुः\RIGHTarrow{06:51(+1)}}}%
{\anga{परिघः}{\time{54-10}{04:44(+1)}}\hspace{1ex}\uanga{शिवः}}%
{\anga{कौलवः}{\time{7-30}{10:04}}\hspace{1ex}\anga{तैतिलः}{\time{40-8}{23:07}}\hspace{1ex}\uanga{गरः}}{}
}
{फाल्गुन-अन्वष्टका-श्राद्धम्}
{Thu} 
\cfoot{\rygdata{14:46--16:18}{07:04--08:36}{10:09--11:41}}
\caldata{MARCH}{29}{\sunmonth{मीनः}{16}{}{फाल्गुनः}{शिशिरऋतुः}{शुक्रः}{विलम्बः}{उत्तरायणम्}{शिशिरऋतुः}}
{\sunmoonsrdata{07:02}{19:24}{04:08(+1)}{13:30}{13:13}
{\kalas{05:29 06:16 10:20 09:31 11:10 17:45 11:59 14:27 16:56 18:35 20:11 22:18 23:45 02:40(+1)}}}
{\tnykdata{\anga{\tithi{24}{कृष्ण-नवमी}}{\time{13-8}{12:18}}\hspace{1ex}}%
{\anga{उत्तराषाढा}{\time{50-8}{03:06(+1)}}\hspace{1ex}}{चन्द्रराशिः—\mbox{मकरः}}%
{\anga{शिवः}{\time{56-30}{05:39(+1)}}\hspace{1ex}\uanga{सिद्धः}}%
{\anga{गरः}{\time{13-8}{12:18}}\hspace{1ex}\anga{वणिजः}{\time{46-17}{01:34(+1)}}\hspace{1ex}\uanga{विष्टिः}}{}
}
{गुरु-सङ्क्रान्तिः~(वृश्चिकः\To{}धनुः)}
{Fri} 
\cfoot{\rygdata{11:40--13:13}{16:19--17:51}{08:35--10:08}}
\caldata{MARCH}{30}{\sunmonth{मीनः}{17}{}{फाल्गुनः}{शिशिरऋतुः}{शनिः}{विलम्बः}{उत्तरायणम्}{शिशिरऋतुः}}
{\sunmoonsrdata{07:01}{19:25}{04:41(+1)}{14:27}{13:13}
{\kalas{05:28 06:14 10:19 09:30 11:09 17:46 11:58 14:27 16:56 18:35 20:11 22:19 23:45 02:39(+1)}}}
{\tnykdata{\anga{\tithi{25}{कृष्ण-दशमी}}{\time{19-39}{14:53}}\hspace{1ex}}%
{\anga{श्रवणः}{\time{58-4}{06:15(+1)}}\hspace{1ex}}{चन्द्रराशिः—\mbox{मकरः}}%
{\anga{सिद्धः}{\time{59-12}{06:42(+1)}}\hspace{1ex}\uanga{साध्यः}}%
{\anga{विष्टिः}{\time{19-39}{14:53}}\hspace{1ex}\anga{बवः}{\time{53-1}{04:13(+1)}}\hspace{1ex}\uanga{बालवः}}{}
}
{देवी-पर्व-१२\eventsep \tamil{கபாலீ விடையாற்றி நிறைவு}\eventsep श्रवण-व्रतम्}
{Sat} 
\cfoot{\rygdata{10:07--11:40}{14:46--16:19}{07:01--08:34}}
\caldata{MARCH}{31}{\sunmonth{मीनः}{18}{}{फाल्गुनः}{शिशिरऋतुः}{भानुः}{विलम्बः}{उत्तरायणम्}{शिशिरऋतुः}}
{\sunmoonsrdata{06:59}{19:26}{05:11(+1)}{15:24}{13:13}
{\kalas{05:27 06:13 10:18 09:29 11:08 17:46 11:58 14:27 16:57 18:36 20:12 22:19 23:45 02:38(+1)}}}
{\tnykdata{\anga{\tithi{26}{कृष्ण-एकादशी}}{\time{26-26}{17:34}}\hspace{1ex}}%
{\fullanga{श्रविष्ठा}}{चन्द्रराशिः—\mbox{मकरः\RIGHTarrow{19:49}}}%
{\fullanga{साध्यः}}%
{\anga{बालवः}{\time{26-26}{17:34}}\hspace{1ex}\anga{कौलवः}{\time{59-43}{06:53(+1)}}\hspace{1ex}\uanga{तैतिलः}}{}
}
{सर्व-पापमोचनी-एकादशी}
{Sun} 
\cfoot{\rygdata{17:53--19:26}{13:13--14:46}{16:19--17:53}}
\caldata{APRIL}{1}{\sunmonth{मीनः}{19}{}{फाल्गुनः}{शिशिरऋतुः}{सोमः}{विलम्बः}{उत्तरायणम्}{शिशिरऋतुः}}
{\sunmoonsrdata{06:58}{19:27}{05:39(+1)}{16:21}{13:12}
{\kalas{05:26 06:12 10:18 09:28 11:08 17:47 11:57 14:27 16:57 18:37 20:13 22:19 23:45 02:38(+1)}}}
{\tnykdata{\anga{\tithi{27}{कृष्ण-द्वादशी}}{\time{32-56}{20:08}}\hspace{1ex}}%
{\anga{श्रविष्ठा}{\time{6-1}{09:22}}\hspace{1ex}}{चन्द्रराशिः—\mbox{कुम्भः}}%
{\anga{साध्यः}{\time{1-55}{07:44}}\hspace{1ex}\uanga{शुभः}}%
{\anga{तैतिलः}{\time{32-56}{20:08}}\hspace{1ex}\uanga{गरः}}{}
}
{हरिवासरः\RIGHTarrow{}00:14}
{Mon} 
\cfoot{\rygdata{08:32--10:05}{11:39--13:12}{14:46--16:20}}
\caldata{APRIL}{2}{\sunmonth{मीनः}{20}{}{फाल्गुनः}{शिशिरऋतुः}{मङ्गलः}{विलम्बः}{उत्तरायणम्}{शिशिरऋतुः}}
{\sunmoonsrdata{06:56}{19:28}{06:05(+1)}{17:18}{13:12}
{\kalas{05:24 06:10 10:17 09:27 11:07 17:48 11:57 14:27 16:57 18:38 20:14 22:20 23:45 02:37(+1)}}}
{\tnykdata{\anga{\tithi{28}{कृष्ण-त्रयोदशी}}{\time{38-43}{22:26}}\hspace{1ex}}%
{\anga{शतभिषक्}{\time{13-22}{12:17}}\hspace{1ex}}{चन्द्रराशिः—\mbox{कुम्भः}}%
{\anga{शुभः}{\time{4-10}{08:37}}\hspace{1ex}\uanga{शुक्लः}}%
{\anga{गरः}{\time{5-58}{09:20}}\hspace{1ex}\anga{वणिजः}{\time{38-43}{22:26}}\hspace{1ex}\uanga{विष्टिः}}{}
}
{\tamil{தண்டியடிகள் நாயனார் (30) குருபூஜை}\eventsep मासशिवरात्रिः\eventsep प्रदोष-व्रतम्~19:28\RIGHTarrow{}20:14\eventsep वारुणी-त्रयोदशी}
{Tue} 
\cfoot{\rygdata{16:20--17:54}{10:04--11:38}{13:12--14:46}}
\caldata{APRIL}{3}{\sunmonth{मीनः}{21}{}{फाल्गुनः}{शिशिरऋतुः}{बुधः}{विलम्बः}{उत्तरायणम्}{शिशिरऋतुः}}
{\sunmoonsrdata{06:55}{19:29}{06:31(+1)}{18:16}{13:12}
{\kalas{05:23 06:09 10:16 09:26 11:06 17:48 11:56 14:27 16:58 18:38 20:14 22:20 23:45 02:37(+1)}}}
{\tnykdata{\anga{\tithi{29}{कृष्ण-चतुर्दशी}}{\time{43-34}{00:21(+1)}}\hspace{1ex}}%
{\anga{पूर्वप्रोष्ठपदा}{\time{19-54}{14:53}}\hspace{1ex}}{चन्द्रराशिः—\mbox{कुम्भः\RIGHTarrow{08:16}}}%
{\anga{शुक्लः}{\time{5-49}{09:15}}\hspace{1ex}\uanga{ब्रह्म}}%
{\anga{विष्टिः}{\time{11-18}{11:27}}\hspace{1ex}\anga{शकुनिः}{\time{43-34}{00:21(+1)}}\hspace{1ex}\uanga{चतुष्पात्}}{}
}
{पञ्च-पर्व-पूजा (चतुर्दशी)}
{Wed} 
\cfoot{\rygdata{13:12--14:46}{08:29--10:03}{11:38--13:12}}
\caldata{APRIL}{4}{\sunmonth{मीनः}{22}{}{फाल्गुनः}{शिशिरऋतुः}{गुरुः}{विलम्बः}{उत्तरायणम्}{शिशिरऋतुः}}
{\sunmoonsrdata{06:53}{19:30}{---}{19:14}{13:12}
{\kalas{05:22 06:08 10:15 09:25 11:06 17:49 11:56 14:27 16:58 18:39 20:15 22:20 23:45 02:36(+1)}}}
{\tnykdata{\anga{\tithi{30}{अमावास्या}}{\time{47-21}{01:50(+1)}}\hspace{1ex}}%
{\anga{उत्तरप्रोष्ठपदा}{\time{25-27}{17:04}}\hspace{1ex}}{चन्द्रराशिः—\mbox{मीनः}}%
{\anga{ब्रह्म}{\time{6-41}{09:34}}\hspace{1ex}\uanga{इन्द्रः}}%
{\anga{चतुष्पात्}{\time{15-38}{13:09}}\hspace{1ex}\anga{नाग}{\time{47-21}{01:50(+1)}}\hspace{1ex}\uanga{किंस्तुघ्नः}}{}
}
{काञ्ची ६५ जगद्गुरु श्री-सुदर्शन महादेवेन्द्र सरस्वती आराधना~\#{१२९}\eventsep मन्वादिः-(रैवतः-[५])\eventsep पार्वणव्रतम् अमावास्यायाम्\eventsep पञ्च-पर्व-पूजा (अमावास्या)\eventsep सर्व-फाल्गुन-अमावास्या (अलभ्यम्–पुष्कला)}
{Thu} 
\cfoot{\rygdata{14:46--16:21}{06:53--08:28}{10:02--11:37}}
\caldata{APRIL}{5}{\sunmonth{मीनः}{23}{}{चैत्रः}{वसन्तऋतुः}{शुक्रः}{विलम्बः}{उत्तरायणम्}{शिशिरऋतुः}}
{\sunmoonrsdata{06:52}{19:31}{06:58}{20:14}{13:11}
{\kalas{05:21 06:06 10:14 09:24 11:05 17:49 11:55 14:27 16:59 18:40 20:16 22:20 23:46 02:35(+1)}}}
{\tnykdata{\anga{\tithi{1}{शुक्ल-प्रथमा}}{\time{50-2}{02:53(+1)}}\hspace{1ex}}%
{\anga{रेवती}{\time{29-56}{18:51}}\hspace{1ex}}{चन्द्रराशिः—\mbox{मीनः\RIGHTarrow{18:51}}}%
{\anga{इन्द्रः}{\time{6-44}{09:34}}\hspace{1ex}\uanga{वैधृतिः}}%
{\anga{किंस्तुघ्नः}{\time{18-51}{14:25}}\hspace{1ex}\anga{बवः}{\time{50-2}{02:53(+1)}}\hspace{1ex}\uanga{बालवः}}{}
}
{भृगुरेवती-पुण्यकालः\RIGHTarrow{}18:51\eventsep दर्शेष्टिः\eventsep काञ्ची १५ जगद्गुरु श्री-गङ्गाधरेन्द्र सरस्वती आराधना~\#{१६९१}\eventsep काञ्ची २७ जगद्गुरु श्री-चिद्विलासेन्द्र सरस्वती आराधना~\#{१४४३}\eventsep काञ्ची ५२ जगद्गुरु श्री-शङ्करानन्देन्द्र सरस्वती आराधना~\#{६०३}\eventsep पार्वण-प्रायश्चित्तावकाशः पौर्णमास्याम्\eventsep स्थालीपाकः\eventsep वैधृति-श्राद्धम्\eventsep वसन्तनवरात्र-आरम्भः\eventsep युगादिः\eventsep श्वेत-कल्पादिः}
{Fri} 
\cfoot{\rygdata{11:36--13:11}{16:21--17:56}{08:27--10:02}}
\caldata{APRIL}{6}{\sunmonth{मीनः}{24}{}{चैत्रः}{वसन्तऋतुः}{शनिः}{विलम्बः}{उत्तरायणम्}{शिशिरऋतुः}}
{\sunmoonrsdata{06:50}{19:31}{07:27}{21:15}{13:11}
{\kalas{05:20 06:05 10:13 09:23 11:04 17:50 11:55 14:27 16:59 18:41 20:17 22:21 23:46 02:35(+1)}}}
{\tnykdata{\anga{\tithi{2}{शुक्ल-द्वितीया}}{\time{51-41}{03:31(+1)}}\hspace{1ex}}%
{\anga{अश्विनी}{\time{33-25}{20:13}}\hspace{1ex}}{चन्द्रराशिः—\mbox{मेषः}}%
{\anga{वैधृतिः}{\time{5-58}{09:14}}\hspace{1ex}\uanga{विष्कम्भः}}%
{\anga{बालवः}{\time{21-1}{15:15}}\hspace{1ex}\anga{कौलवः}{\time{51-41}{03:31(+1)}}\hspace{1ex}\uanga{तैतिलः}}{}
}
{आन्दोलन-तृतीया\eventsep अरुन्धती-व्रत-आरम्भः\eventsep बालेन्दुव्रतम्\eventsep चन्द्र-दर्शनम्~19:31\RIGHTarrow{}20:17\eventsep झूलेलाल-जयन्ती}
{Sat} 
\cfoot{\rygdata{10:01--11:36}{14:46--16:21}{06:50--08:26}}
\caldata{APRIL}{7}{\sunmonth{मीनः}{25}{}{चैत्रः}{वसन्तऋतुः}{भानुः}{विलम्बः}{उत्तरायणम्}{शिशिरऋतुः}}
{\sunmoonrsdata{06:49}{19:32}{07:59}{22:18}{13:11}
{\kalas{05:19 06:04 10:13 09:22 11:03 17:51 11:54 14:27 17:00 18:41 20:17 22:21 23:46 02:34(+1)}}}
{\tnykdata{\anga{\tithi{3}{शुक्ल-तृतीया}}{\time{52-21}{03:45(+1)}}\hspace{1ex}}%
{\anga{अपभरणी}{\time{35-55}{21:11}}\hspace{1ex}}{चन्द्रराशिः—\mbox{मेषः\RIGHTarrow{03:22(+1)}}}%
{\anga{विष्कम्भः}{\time{4-24}{08:35}}\hspace{1ex}\uanga{प्रीतिः}}%
{\anga{तैतिलः}{\time{22-10}{15:41}}\hspace{1ex}\anga{गरः}{\time{52-21}{03:45(+1)}}\hspace{1ex}\uanga{वणिजः}}{}
}
{गौरी-तृतीया/सौभाग्य-गौरी-व्रतम्\eventsep मन्वादिः-(उत्तमः-[३])\eventsep पार्वतीश्वरयोरान्दोलनव्रतम्}
{Sun} 
\cfoot{\rygdata{17:57--19:32}{13:11--14:46}{16:21--17:57}}
\caldata{APRIL}{8}{\sunmonth{मीनः}{26}{}{चैत्रः}{वसन्तऋतुः}{सोमः}{विलम्बः}{उत्तरायणम्}{शिशिरऋतुः}}
{\sunmoonrsdata{06:47}{19:33}{08:37}{23:21}{13:10}
{\kalas{05:17 06:03 10:12 09:21 11:03 17:51 11:54 14:27 17:00 18:42 20:18 22:21 23:46 02:34(+1)}}}
{\tnykdata{\anga{\tithi{4}{शुक्ल-चतुर्थी}}{\time{52-3}{03:37(+1)}}\hspace{1ex}}%
{\anga{कृत्तिका}{\time{37-30}{21:47}}\hspace{1ex}}{चन्द्रराशिः—\mbox{वृषभः}}%
{\anga{प्रीतिः}{\time{2-5}{07:38}}\hspace{1ex}\anga{आयुष्मान्}{\time{58-58}{06:23(+1)}}\hspace{1ex}\uanga{सौभाग्यः}}%
{\anga{वणिजः}{\time{22-21}{15:44}}\hspace{1ex}\anga{विष्टिः}{\time{52-3}{03:37(+1)}}\hspace{1ex}\uanga{बवः}}{}
}
{कृत्तिका-व्रतम्\eventsep मुत्तुस्वामि-दीक्षित-जयन्ती~\#{२४५}}
{Mon} 
\cfoot{\rygdata{08:23--09:59}{11:35--13:10}{14:46--16:22}}
\caldata{APRIL}{9}{\sunmonth{मीनः}{27}{}{चैत्रः}{वसन्तऋतुः}{मङ्गलः}{विलम्बः}{उत्तरायणम्}{शिशिरऋतुः}}
{\sunmoonrsdata{06:46}{19:34}{09:20}{00:23(+1)}{13:10}
{\kalas{05:16 06:01 10:11 09:20 11:02 17:52 11:53 14:27 17:00 18:43 20:19 22:22 23:46 02:33(+1)}}}
{\tnykdata{\anga{\tithi{5}{शुक्ल-पञ्चमी}}{\time{50-49}{03:06(+1)}}\hspace{1ex}}%
{\anga{रोहिणी}{\time{38-9}{22:02}}\hspace{1ex}}{चन्द्रराशिः—\mbox{वृषभः}}%
{\anga{सौभाग्यः}{\time{55-10}{04:50(+1)}}\hspace{1ex}\uanga{शोभनः}}%
{\anga{बवः}{\time{21-35}{15:24}}\hspace{1ex}\anga{बालवः}{\time{50-49}{03:06(+1)}}\hspace{1ex}\uanga{कौलवः}}{}
}
{हय-पूजा\eventsep कूर्म-कल्पादिः\eventsep लक्ष्मी-पञ्चमी\eventsep \tamil{நேச நாயனார் (58) குருபூஜை}\eventsep शालिहोत्र-व्रत-आरम्भः}
{Tue} 
\cfoot{\rygdata{16:22--17:58}{09:58--11:34}{13:10--14:46}}
\caldata{APRIL}{10}{\sunmonth{मीनः}{28}{}{चैत्रः}{वसन्तऋतुः}{बुधः}{विलम्बः}{उत्तरायणम्}{शिशिरऋतुः}}
{\sunmoonrsdata{06:45}{19:35}{10:10}{01:22(+1)}{13:10}
{\kalas{05:15 06:00 10:10 09:19 11:01 17:52 11:53 14:27 17:01 18:44 20:20 22:22 23:46 02:33(+1)}}}
{\tnykdata{\anga{\tithi{6}{शुक्ल-षष्ठी}}{\time{48-37}{02:11(+1)}}\hspace{1ex}}%
{\anga{मृगशीर्षम्}{\time{37-52}{21:54}}\hspace{1ex}}{चन्द्रराशिः—\mbox{वृषभः\RIGHTarrow{10:01}}}%
{\anga{शोभनः}{\time{50-37}{03:00(+1)}}\hspace{1ex}\uanga{अतिगण्डः}}%
{\anga{कौलवः}{\time{19-52}{14:42}}\hspace{1ex}\anga{तैतिलः}{\time{48-37}{02:11(+1)}}\hspace{1ex}\uanga{गरः}}{}
}
{षष्ठी-व्रतम्\eventsep यमुना-जयन्ती}
{Wed} 
\cfoot{\rygdata{13:10--14:46}{08:21--09:57}{11:33--13:10}}
\caldata{APRIL}{11}{\sunmonth{मीनः}{29}{}{चैत्रः}{वसन्तऋतुः}{गुरुः}{विलम्बः}{उत्तरायणम्}{शिशिरऋतुः}}
{\sunmoonrsdata{06:43}{19:36}{11:08}{02:15(+1)}{13:10}
{\kalas{05:14 05:59 10:09 09:18 11:01 17:53 11:52 14:27 17:01 18:44 20:20 22:22 23:46 02:32(+1)}}}
{\tnykdata{\anga{\tithi{7}{शुक्ल-सप्तमी}}{\time{45-25}{00:53(+1)}}\hspace{1ex}}%
{\anga{आर्द्रा}{\time{36-37}{21:22}}\hspace{1ex}}{चन्द्रराशिः—\mbox{मिथुनम्}}%
{\anga{अतिगण्डः}{\time{45-16}{00:50(+1)}}\hspace{1ex}\uanga{सुकर्म}}%
{\anga{गरः}{\time{17-10}{13:35}}\hspace{1ex}\anga{वणिजः}{\time{45-25}{00:53(+1)}}\hspace{1ex}\uanga{विष्टिः}}{}
}
{ऋषीणां दमनकपूजा\eventsep \tamil{கணநாத நாயனார் (37) குருபூஜை}\eventsep सूर्यस्य दमनकपूजा}
{Thu} 
\cfoot{\rygdata{14:46--16:23}{06:43--08:20}{09:56--11:33}}
\caldata{APRIL}{12}{\sunmonth{मीनः}{30}{}{चैत्रः}{वसन्तऋतुः}{शुक्रः}{विलम्बः}{उत्तरायणम्}{शिशिरऋतुः}}
{\sunmoonrsdata{06:42}{19:37}{12:12}{03:02(+1)}{13:09}
{\kalas{05:13 05:57 10:08 09:17 11:00 17:53 11:52 14:27 17:02 18:45 20:21 22:23 23:46 02:32(+1)}}}
{\tnykdata{\anga{\tithi{8}{शुक्ल-अष्टमी}}{\time{41-13}{23:11}}\hspace{1ex}}%
{\anga{पुनर्वसुः}{\time{34-23}{20:27}}\hspace{1ex}}{चन्द्रराशिः—\mbox{मिथुनम्\RIGHTarrow{14:43}}}%
{\anga{सुकर्म}{\time{39-7}{22:21}}\hspace{1ex}\uanga{धृतिः}}%
{\anga{विष्टिः}{\time{13-28}{12:05}}\hspace{1ex}\anga{बवः}{\time{41-13}{23:11}}\hspace{1ex}\uanga{बालवः}}{}
}
{अशोकाष्टमी\eventsep भवान्युत्पत्तिः}
{Fri} 
\cfoot{\rygdata{11:32--13:09}{16:23--18:00}{08:19--09:56}}
\caldata{APRIL}{13}{\sunmonth{मीनः}{31}{\mbox{मीनः{\tiny\RIGHTarrow}{01:23(+1)}}}{चैत्रः}{वसन्तऋतुः}{शनिः}{विलम्बः}{उत्तरायणम्}{शिशिरऋतुः}}
{\sunmoonrsdata{06:40}{19:38}{13:20}{03:44(+1)}{13:09}
{\kalas{05:12 05:56 10:08 09:16 10:59 17:54 11:51 14:27 17:02 18:46 20:22 22:23 23:46 02:31(+1)}}}
{\tnykdata{\anga{\tithi{9}{शुक्ल-नवमी}}{\time{36-2}{21:05}}\hspace{1ex}}%
{\anga{पुष्यः}{\time{31-9}{19:08}}\hspace{1ex}}{चन्द्रराशिः—\mbox{कर्कटः}}%
{\anga{धृतिः}{\time{32-10}{19:32}}\hspace{1ex}\uanga{शूलः}}%
{\anga{बालवः}{\time{8-47}{10:11}}\hspace{1ex}\anga{कौलवः}{\time{36-2}{21:05}}\hspace{1ex}\uanga{तैतिलः}}{}
}
{काञ्ची ४३ जगद्गुरु श्री-आनन्दघनेन्द्र सरस्वती आराधना~\#{१००६}\eventsep मेष-सङ्क्रमण-पुण्यकालः~21:24\RIGHTarrow{}05:24(+1)\eventsep महातारा-जयन्ती\eventsep \tamil{முனையடுவார் நாயனார் (50) குருபூஜை}\eventsep वसन्तनवरात्र-समापनम्\eventsep श्रीरामनवमी}
{Sat} 
\cfoot{\rygdata{09:55--11:32}{14:46--16:23}{06:40--08:17}}
\caldata{APRIL}{14}{\sunmonth{मेषः}{1}{}{चैत्रः}{वसन्तऋतुः}{भानुः}{विकारी}{उत्तरायणम्}{वसन्तऋतुः}}
{\sunmoonrsdata{06:39}{19:39}{14:31}{04:21(+1)}{13:09}
{\kalas{05:11 05:55 10:07 09:15 10:59 17:55 11:51 14:27 17:03 18:47 20:23 22:23 23:46 02:30(+1)}}}
{\tnykdata{\anga{\tithi{10}{शुक्ल-दशमी}}{\time{29-57}{18:38}}\hspace{1ex}}%
{\anga{आश्रेषा}{\time{27-2}{17:28}}\hspace{1ex}}{चन्द्रराशिः—\mbox{कर्कटः\RIGHTarrow{17:28}}}%
{\anga{शूलः}{\time{24-28}{16:26}}\hspace{1ex}\uanga{गण्डः}}%
{\anga{तैतिलः}{\time{3-7}{07:54}}\hspace{1ex}\anga{गरः}{\time{29-57}{18:38}}\hspace{1ex}\anga{वणिजः}{\time{56-35}{05:17(+1)}}\hspace{1ex}\uanga{विष्टिः}}{}
}
{धर्मराज-दशमी\eventsep मेष-सङ्क्रान्तिः (विकारी-संवत्सरः)\eventsep पञ्चाङ्ग-पठनम्\eventsep \tamil{விஷுக்கனி}}
{Sun} 
\cfoot{\rygdata{18:01--19:39}{13:09--14:46}{16:24--18:01}}
\caldata{APRIL}{15}{\sunmonth{मेषः}{2}{}{चैत्रः}{वसन्तऋतुः}{सोमः}{विकारी}{उत्तरायणम्}{वसन्तऋतुः}}
{\sunmoonrsdata{06:38}{19:40}{15:43}{04:56(+1)}{13:09}
{\kalas{05:10 05:54 10:06 09:14 10:58 17:55 11:50 14:27 17:03 18:47 20:23 22:24 23:46 02:30(+1)}}}
{\tnykdata{\anga{\tithi{11}{शुक्ल-एकादशी}}{\time{23-7}{15:53}}\hspace{1ex}}%
{\anga{मघा}{\time{22-9}{15:29}}\hspace{1ex}}{चन्द्रराशिः—\mbox{सिंहः}}%
{\anga{गण्डः}{\time{16-10}{13:06}}\hspace{1ex}\uanga{वृद्धिः}}%
{\anga{विष्टिः}{\time{23-7}{15:53}}\hspace{1ex}\anga{बवः}{\time{49-29}{02:25(+1)}}\hspace{1ex}\uanga{बालवः}}{}
}
{हरिवासरः\RIGHTarrow{}21:09\eventsep समुद्र-मन्थनम्\eventsep सर्व-कामदा-एकादशी\eventsep श्रीकृष्णदोलोत्सवः}
{Mon} 
\cfoot{\rygdata{08:15--09:53}{11:31--13:08}{14:46--16:24}}
\caldata{APRIL}{16}{\sunmonth{मेषः}{3}{}{चैत्रः}{वसन्तऋतुः}{मङ्गलः}{विकारी}{उत्तरायणम्}{वसन्तऋतुः}}
{\sunmoonrsdata{06:36}{19:41}{16:54}{05:28(+1)}{13:08}
{\kalas{05:09 05:52 10:05 09:13 10:58 17:56 11:50 14:27 17:04 18:48 20:24 22:24 23:46 02:29(+1)}}}
{\tnykdata{\anga{\tithi{12}{शुक्ल-द्वादशी}}{\time{15-48}{12:56}}\hspace{1ex}}%
{\anga{पूर्वफल्गुनी}{\time{16-46}{13:19}}\hspace{1ex}}{चन्द्रराशिः—\mbox{सिंहः\RIGHTarrow{18:45}}}%
{\anga{वृद्धिः}{\time{7-25}{09:34}}\hspace{1ex}\anga{ध्रुवः}{\time{58-25}{05:58(+1)}}\hspace{1ex}\uanga{व्याघातः}}%
{\anga{बालवः}{\time{15-48}{12:56}}\hspace{1ex}\anga{कौलवः}{\time{42-1}{23:25}}\hspace{1ex}\uanga{तैतिलः}}{}
}
{भ्रातृप्राप्ति-व्रत-आरम्भः\eventsep दमनकारोपण-द्वादशी\eventsep प्रदोष-व्रतम्~19:41\RIGHTarrow{}20:24\eventsep तुलसी-जननं-क्षीरसागरतः\eventsep वेङ्कटाचले वसन्तोत्सव-प्रारम्भः\eventsep विष्णु-दमनकोत्सवः}
{Tue} 
\cfoot{\rygdata{16:24--18:03}{09:52--11:30}{13:08--14:46}}
\caldata{APRIL}{17}{\sunmonth{मेषः}{4}{}{चैत्रः}{वसन्तऋतुः}{बुधः}{विकारी}{उत्तरायणम्}{वसन्तऋतुः}}
{\sunmoonrsdata{06:35}{19:41}{18:06}{06:01(+1)}{13:08}
{\kalas{05:08 05:51 10:04 09:12 10:57 17:56 11:49 14:27 17:04 18:49 20:25 22:24 23:46 02:29(+1)}}}
{\tnykdata{\anga{\tithi{13}{शुक्ल-त्रयोदशी}}{\time{8-17}{09:54}}\hspace{1ex}}%
{\anga{उत्तरफल्गुनी}{\time{11-13}{11:04}}\hspace{1ex}}{चन्द्रराशिः—\mbox{कन्या}}%
{\anga{व्याघातः}{\time{49-33}{02:24(+1)}}\hspace{1ex}\uanga{हर्षणः}}%
{\anga{तैतिलः}{\time{8-17}{09:54}}\hspace{1ex}\anga{गरः}{\time{34-32}{20:24}}\hspace{1ex}\uanga{वणिजः}}{}
}
{दमनक-चोरी-उत्सवः\eventsep मदन-त्रयोदशी\eventsep वेङ्कटाचले वसन्तोत्सवः}
{Wed} 
\cfoot{\rygdata{13:08--14:46}{08:13--09:51}{11:30--13:08}}
\caldata{APRIL}{18}{\sunmonth{मेषः}{5}{}{चैत्रः}{वसन्तऋतुः}{गुरुः}{विकारी}{उत्तरायणम्}{वसन्तऋतुः}}
{\sunmoonrsdata{06:33}{19:42}{19:18}{---}{13:08}
{\kalas{05:06 05:50 10:04 09:11 10:56 17:57 11:49 14:27 17:04 18:50 20:26 22:25 23:46 02:28(+1)}}}
{\tnykdata{\anga{\tithi{14}{शुक्ल-चतुर्दशी}}{\time{0-56}{06:56}}\hspace{1ex}\anga{\tithi{15}{पौर्णमासी}}{\time{54-5}{04:12(+1)}}\hspace{1ex}\avamA{}}%
{\anga{हस्तः}{\time{5-51}{08:54}}\hspace{1ex}}{चन्द्रराशिः—\mbox{कन्या\RIGHTarrow{19:53}}}%
{\anga{हर्षणः}{\time{41-5}{22:59}}\hspace{1ex}\uanga{वज्रम्}}%
{\anga{वणिजः}{\time{0-56}{06:56}}\hspace{1ex}\anga{विष्टिः}{\time{27-25}{17:32}}\hspace{1ex}\anga{बवः}{\time{54-5}{04:12(+1)}}\hspace{1ex}\uanga{बालवः}}{}
}
{चैत्र-पूर्णिमा\eventsep चित्रा-पूर्णिमा\eventsep चित्रगुप्त-व्रतम्\eventsep दमनक-चतुर्दशी\eventsep गजेन्द्र-मोक्षः\eventsep \tamil{இசைஞானியார் நாயனார் (62) குருபூஜை}\eventsep मदन-चतुर्दशी\eventsep \tamil{மதுரகவி ஆழ்வார் திருநக்ஷத்திரம்}\eventsep मन्वादिः-(रौच्यः-[१३])\eventsep नृसिंह-दोलोत्सवः\eventsep पार्वणव्रतम् पूर्णिमायाम्\eventsep पूर्णिमा-व्रतम्\eventsep पञ्च-पर्व-पूजा (पूर्णिमा)\eventsep प्रोक्लस्-मृत्युः~\#{१५३४}\eventsep वेङ्कटाचले पूर्णिमा-गरुड-सेवा\eventsep वेङ्कटाचले वसन्तोत्सव-समापनम्\eventsep श्री-हनूमत्-जयन्ती}
{Thu} 
\cfoot{\rygdata{14:47--16:25}{06:33--08:12}{09:51--11:29}}
\caldata{APRIL}{19}{\sunmonth{मेषः}{6}{}{चैत्रः}{वसन्तऋतुः}{शुक्रः}{विकारी}{उत्तरायणम्}{वसन्तऋतुः}}
{\sunmoonsrdata{06:32}{19:43}{20:28}{06:36}{13:08}
{\kalas{05:05 05:49 10:03 09:10 10:56 17:58 11:49 14:27 17:05 18:50 20:26 22:25 23:46 02:28(+1)}}}
{\tnykdata{\anga{\tithi{16}{कृष्ण-प्रथमा}}{\time{48-16}{01:51(+1)}}\hspace{1ex}}%
{\anga{चित्रा}{\time{1-4}{06:58}}\hspace{1ex}\anga{स्वाती}{\time{57-16}{05:27(+1)}}\hspace{1ex}}{चन्द्रराशिः—\mbox{तुला}}%
{\anga{वज्रम्}{\time{33-20}{19:52}}\hspace{1ex}\uanga{सिद्धिः}}%
{\anga{बालवः}{\time{21-3}{14:58}}\hspace{1ex}\anga{कौलवः}{\time{48-16}{01:51(+1)}}\hspace{1ex}\uanga{तैतिलः}}{}
}
{मधु-मासः\RIGHTarrow{}01:55(+1)\eventsep पार्वण-प्रायश्चित्तावकाशः दर्शे\eventsep पूर्णमासेष्टिः\eventsep स्थालीपाकः\eventsep \tamil{திருக்குறிப்புத் தொண்ட நாயனார் (18) குருபூஜை}\eventsep विष्णुपदी-पुण्यकालः~19:31\RIGHTarrow{}08:19(+1)}
{Fri} 
\cfoot{\rygdata{11:29--13:08}{16:25--18:04}{08:11--09:50}}
\caldata{APRIL}{20}{\sunmonth{मेषः}{7}{}{चैत्रः}{वसन्तऋतुः}{शनिः}{विकारी}{उत्तरायणम्}{वसन्तऋतुः}}
{\sunmoonsrdata{06:31}{19:44}{21:38}{07:13}{13:07}
{\kalas{05:04 05:48 10:02 09:09 10:55 17:58 11:48 14:27 17:06 18:51 20:27 22:25 23:46 02:27(+1)}}}
{\tnykdata{\anga{\tithi{17}{कृष्ण-द्वितीया}}{\time{43-49}{00:02(+1)}}\hspace{1ex}}%
{\anga{विशाखा}{\time{54-56}{04:29(+1)}}\hspace{1ex}}{चन्द्रराशिः—\mbox{तुला\RIGHTarrow{22:40}}}%
{\anga{सिद्धिः}{\time{26-36}{17:09}}\hspace{1ex}\uanga{व्यतीपातः}}%
{\anga{तैतिलः}{\time{15-52}{12:52}}\hspace{1ex}\anga{गरः}{\time{43-49}{00:02(+1)}}\hspace{1ex}\uanga{वणिजः}}{}
}
{काञ्ची ६० जगद्गुरु श्री-अद्वैतात्मप्रकाशेन्द्र सरस्वती आराधना~\#{३१६}}
{Sat} 
\cfoot{\rygdata{09:49--11:28}{14:47--16:26}{06:31--08:10}}
\caldata{APRIL}{21}{\sunmonth{मेषः}{8}{}{चैत्रः}{वसन्तऋतुः}{भानुः}{विकारी}{उत्तरायणम्}{वसन्तऋतुः}}
{\sunmoonsrdata{06:29}{19:45}{22:44}{07:54}{13:07}
{\kalas{05:03 05:46 10:02 09:08 10:55 17:59 11:48 14:27 17:06 18:52 20:28 22:26 23:46 02:27(+1)}}}
{\tnykdata{\anga{\tithi{18}{कृष्ण-तृतीया}}{\time{41-3}{22:55}}\hspace{1ex}}%
{\anga{अनूराधा}{\time{54-20}{04:14(+1)}}\hspace{1ex}}{चन्द्रराशिः—\mbox{वृश्चिकः}}%
{\anga{व्यतीपातः}{\time{21-11}{14:58}}\hspace{1ex}\uanga{वरीयान्}}%
{\anga{वणिजः}{\time{12-13}{11:23}}\hspace{1ex}\anga{विष्टिः}{\time{41-3}{22:55}}\hspace{1ex}\uanga{बवः}}{}
}
{पद्मक-योगः~11:23\RIGHTarrow{}14:58\eventsep व्यतीपात-श्राद्धम्}
{Sun} 
\cfoot{\rygdata{18:06--19:45}{13:07--14:47}{16:26--18:06}}
\caldata{APRIL}{22}{\sunmonth{मेषः}{9}{}{चैत्रः}{वसन्तऋतुः}{सोमः}{विकारी}{उत्तरायणम्}{वसन्तऋतुः}}
{\sunmoonsrdata{06:28}{19:46}{23:44}{08:39}{13:07}
{\kalas{05:02 05:45 10:01 09:08 10:54 18:00 11:47 14:27 17:06 18:53 20:29 22:26 23:46 02:26(+1)}}}
{\tnykdata{\anga{\tithi{19}{कृष्ण-चतुर्थी}}{\time{40-14}{22:34}}\hspace{1ex}}%
{\anga{ज्येष्ठा}{\time{55-41}{04:45(+1)}}\hspace{1ex}}{चन्द्रराशिः—\mbox{वृश्चिकः\RIGHTarrow{04:45(+1)}}}%
{\anga{वरीयान्}{\time{17-16}{13:23}}\hspace{1ex}\uanga{परिघः}}%
{\anga{बवः}{\time{10-24}{10:38}}\hspace{1ex}\anga{बालवः}{\time{40-14}{22:34}}\hspace{1ex}\uanga{कौलवः}}{}
}
{विकट-महागणपति सङ्कटहर-चतुर्थी-व्रतम्}
{Mon} 
\cfoot{\rygdata{08:08--09:47}{11:27--13:07}{14:47--16:26}}
\caldata{APRIL}{23}{\sunmonth{मेषः}{10}{}{चैत्रः}{वसन्तऋतुः}{मङ्गलः}{विकारी}{उत्तरायणम्}{वसन्तऋतुः}}
{\sunmoonsrdata{06:27}{19:47}{00:39(+1)}{09:29}{13:07}
{\kalas{05:01 05:44 10:00 09:07 10:53 18:00 11:47 14:27 17:07 18:54 20:30 22:26 23:46 02:26(+1)}}}
{\tnykdata{\anga{\tithi{20}{कृष्ण-पञ्चमी}}{\time{41-27}{23:02}}\hspace{1ex}}%
{\anga{मूला}{\time{59-1}{06:03(+1)}}\hspace{1ex}}{चन्द्रराशिः—\mbox{धनुः}}%
{\anga{परिघः}{\time{14-57}{12:26}}\hspace{1ex}\uanga{शिवः}}%
{\anga{कौलवः}{\time{10-37}{10:42}}\hspace{1ex}\anga{तैतिलः}{\time{41-27}{23:02}}\hspace{1ex}\uanga{गरः}}{}
}
{गुरु-सङ्क्रान्तिः~(धनुः\To{}वृश्चिकः)\eventsep वराह-जयन्ती}
{Tue} 
\cfoot{\rygdata{16:27--18:07}{09:47--11:27}{13:07--14:47}}
\caldata{APRIL}{24}{\sunmonth{मेषः}{11}{}{चैत्रः}{वसन्तऋतुः}{बुधः}{विकारी}{उत्तरायणम्}{वसन्तऋतुः}}
{\sunmoonsrdata{06:25}{19:48}{01:26(+1)}{10:23}{13:07}
{\kalas{05:00 05:43 09:59 09:06 10:53 18:01 11:46 14:27 17:07 18:54 20:30 22:27 23:46 02:26(+1)}}}
{\tnykdata{\anga{\tithi{21}{कृष्ण-षष्ठी}}{\time{44-36}{00:16(+1)}}\hspace{1ex}}%
{\fullanga{पूर्वाषाढा}}{चन्द्रराशिः—\mbox{धनुः}}%
{\anga{शिवः}{\time{14-12}{12:06}}\hspace{1ex}\uanga{सिद्धः}}%
{\anga{गरः}{\time{12-50}{11:33}}\hspace{1ex}\anga{वणिजः}{\time{44-36}{00:16(+1)}}\hspace{1ex}\uanga{विष्टिः}}{}
}
{}
{Wed} 
\cfoot{\rygdata{13:07--14:47}{08:06--09:46}{11:26--13:07}}
\caldata{APRIL}{25}{\sunmonth{मेषः}{12}{}{चैत्रः}{वसन्तऋतुः}{गुरुः}{विकारी}{उत्तरायणम्}{वसन्तऋतुः}}
{\sunmoonsrdata{06:24}{19:49}{02:06(+1)}{11:19}{13:06}
{\kalas{04:59 05:42 09:59 09:05 10:52 18:02 11:46 14:27 17:08 18:55 20:31 22:27 23:47 02:25(+1)}}}
{\tnykdata{\anga{\tithi{22}{कृष्ण-सप्तमी}}{\time{49-24}{02:10(+1)}}\hspace{1ex}}%
{\anga{पूर्वाषाढा}{\time{4-13}{08:05}}\hspace{1ex}}{चन्द्रराशिः—\mbox{धनुः\RIGHTarrow{14:42}}}%
{\anga{सिद्धः}{\time{14-51}{12:21}}\hspace{1ex}\uanga{साध्यः}}%
{\anga{विष्टिः}{\time{16-51}{13:09}}\hspace{1ex}\anga{बवः}{\time{49-24}{02:10(+1)}}\hspace{1ex}\uanga{बालवः}}{}
}
{}
{Thu} 
\cfoot{\rygdata{14:47--16:28}{06:24--08:05}{09:45--11:26}}
\caldata{APRIL}{26}{\sunmonth{मेषः}{13}{}{चैत्रः}{वसन्तऋतुः}{शुक्रः}{विकारी}{उत्तरायणम्}{वसन्तऋतुः}}
{\sunmoonsrdata{06:23}{19:50}{02:41(+1)}{12:17}{13:06}
{\kalas{04:58 05:41 09:58 09:04 10:52 18:02 11:46 14:27 17:08 18:56 20:32 22:28 23:47 02:25(+1)}}}
{\tnykdata{\anga{\tithi{23}{कृष्ण-अष्टमी}}{\time{55-19}{04:31(+1)}}\hspace{1ex}}%
{\anga{उत्तराषाढा}{\time{10-48}{10:42}}\hspace{1ex}}{चन्द्रराशिः—\mbox{मकरः}}%
{\anga{साध्यः}{\time{16-34}{13:01}}\hspace{1ex}\uanga{शुभः}}%
{\anga{बालवः}{\time{22-17}{15:18}}\hspace{1ex}\anga{कौलवः}{\time{55-19}{04:31(+1)}}\hspace{1ex}\uanga{तैतिलः}}{}
}
{काञ्ची ५६ जगद्गुरु श्री-सर्वज्ञ सदाशिव बोधेन्द्र सरस्वती आराधना~\#{४८१}\eventsep \tamil{நடராஜர் சித்திரை ஓணம் மஹாபிஷேகம்}\eventsep पञ्च-पर्व-पूजा (अष्टमी)\eventsep श्रवण-व्रतम्}
{Fri} 
\cfoot{\rygdata{11:25--13:06}{16:28--18:09}{08:04--09:45}}
\caldata{APRIL}{27}{\sunmonth{मेषः}{14}{}{चैत्रः}{वसन्तऋतुः}{शनिः}{विकारी}{उत्तरायणम्}{वसन्तऋतुः}}
{\sunmoonsrdata{06:22}{19:51}{03:12(+1)}{13:14}{13:06}
{\kalas{04:57 05:39 09:57 09:03 10:51 18:03 11:45 14:27 17:09 18:57 20:33 22:28 23:47 02:24(+1)}}}
{\tnykdata{\fulltithi{\tithi{24}{कृष्ण-नवमी}}}%
{\anga{श्रवणः}{\time{18-17}{13:41}}\hspace{1ex}}{चन्द्रराशिः—\mbox{मकरः\RIGHTarrow{03:13(+1)}}}%
{\anga{शुभः}{\time{18-58}{13:57}}\hspace{1ex}\uanga{शुक्लः}}%
{\anga{तैतिलः}{\time{28-32}{17:47}}\hspace{1ex}\uanga{गरः}}{}
}
{}
{Sat} 
\cfoot{\rygdata{09:44--11:25}{14:47--16:28}{06:22--08:03}}
\caldata{APRIL}{28}{\sunmonth{मेषः}{15}{}{चैत्रः}{वसन्तऋतुः}{भानुः}{विकारी}{उत्तरायणम्}{वसन्तऋतुः}}
{\sunmoonsrdata{06:21}{19:52}{03:41(+1)}{14:12}{13:06}
{\kalas{04:56 05:38 09:57 09:03 10:51 18:03 11:45 14:27 17:09 18:57 20:33 22:28 23:47 02:24(+1)}}}
{\tnykdata{\anga{\tithi{24}{कृष्ण-नवमी}}{\time{1-48}{07:04}}\hspace{1ex}}%
{\anga{श्रविष्ठा}{\time{26-2}{16:46}}\hspace{1ex}}{चन्द्रराशिः—\mbox{कुम्भः}}%
{\anga{शुक्लः}{\time{21-35}{14:59}}\hspace{1ex}\uanga{ब्रह्म}}%
{\anga{गरः}{\time{1-48}{07:04}}\hspace{1ex}\anga{वणिजः}{\time{34-59}{20:20}}\hspace{1ex}\uanga{विष्टिः}}{}
}
{}
{Sun} 
\cfoot{\rygdata{18:10--19:52}{13:06--14:47}{16:29--18:10}}
\caldata{APRIL}{29}{\sunmonth{मेषः}{16}{}{चैत्रः}{वसन्तऋतुः}{सोमः}{विकारी}{उत्तरायणम्}{वसन्तऋतुः}}
{\sunmoonsrdata{06:19}{19:52}{04:07(+1)}{15:09}{13:06}
{\kalas{04:56 05:37 09:56 09:02 10:50 18:04 11:45 14:27 17:10 18:58 20:34 22:29 23:47 02:24(+1)}}}
{\tnykdata{\anga{\tithi{25}{कृष्ण-दशमी}}{\time{8-6}{09:34}}\hspace{1ex}}%
{\anga{शतभिषक्}{\time{33-28}{19:43}}\hspace{1ex}}{चन्द्रराशिः—\mbox{कुम्भः}}%
{\anga{ब्रह्म}{\time{23-58}{15:55}}\hspace{1ex}\uanga{इन्द्रः}}%
{\anga{विष्टिः}{\time{8-6}{09:34}}\hspace{1ex}\anga{बवः}{\time{41-0}{22:43}}\hspace{1ex}\uanga{बालवः}}{}
}
{\tamil{திருநாவுக்கரச நாயனார் (20) குருபூஜை}}
{Mon} 
\cfoot{\rygdata{08:01--09:43}{11:24--13:06}{14:48--16:29}}
\caldata{APRIL}{30}{\sunmonth{मेषः}{17}{}{चैत्रः}{वसन्तऋतुः}{मङ्गलः}{विकारी}{उत्तरायणम्}{वसन्तऋतुः}}
{\sunmoonsrdata{06:18}{19:53}{04:33(+1)}{16:06}{13:06}
{\kalas{04:55 05:36 09:56 09:01 10:50 18:05 11:44 14:27 17:10 18:59 20:35 22:29 23:47 02:23(+1)}}}
{\tnykdata{\anga{\tithi{26}{कृष्ण-एकादशी}}{\time{13-43}{11:47}}\hspace{1ex}}%
{\anga{पूर्वप्रोष्ठपदा}{\time{40-5}{22:20}}\hspace{1ex}}{चन्द्रराशिः—\mbox{कुम्भः\RIGHTarrow{15:43}}}%
{\anga{इन्द्रः}{\time{25-45}{16:36}}\hspace{1ex}\uanga{वैधृतिः}}%
{\anga{बालवः}{\time{13-43}{11:47}}\hspace{1ex}\anga{कौलवः}{\time{46-6}{00:45(+1)}}\hspace{1ex}\uanga{तैतिलः}}{}
}
{हरिवासरः\RIGHTarrow{}18:17\eventsep सर्व-वरूथिनी-एकादशी\eventsep वल्लभाचार्य-जयन्ती~\#{५४१}}
{Tue} 
\cfoot{\rygdata{16:30--18:11}{09:42--11:24}{13:06--14:48}}
\caldata{MAY}{1}{\sunmonth{मेषः}{18}{}{चैत्रः}{वसन्तऋतुः}{बुधः}{विकारी}{उत्तरायणम्}{वसन्तऋतुः}}
{\sunmoonsrdata{06:17}{19:54}{05:00(+1)}{17:04}{13:06}
{\kalas{04:54 05:35 09:55 09:00 10:49 18:05 11:44 14:27 17:11 19:00 20:36 22:30 23:47 02:23(+1)}}}
{\tnykdata{\anga{\tithi{27}{कृष्ण-द्वादशी}}{\time{18-14}{13:35}}\hspace{1ex}}%
{\anga{उत्तरप्रोष्ठपदा}{\time{45-32}{00:30(+1)}}\hspace{1ex}}{चन्द्रराशिः—\mbox{मीनः}}%
{\anga{वैधृतिः}{\time{26-40}{16:57}}\hspace{1ex}\uanga{विष्कम्भः}}%
{\anga{तैतिलः}{\time{18-14}{13:35}}\hspace{1ex}\anga{गरः}{\time{49-59}{02:17(+1)}}\hspace{1ex}\uanga{वणिजः}}{}
}
{देवी-पर्व-१\eventsep मत्स्य-जयन्ती\eventsep प्रदोष-व्रतम्~19:54\RIGHTarrow{}20:36\eventsep रमण-महर्षि-आराधना~\#{६९}\eventsep वैधृति-श्राद्धम्}
{Wed} 
\cfoot{\rygdata{13:06--14:48}{07:59--09:41}{11:23--13:06}}
\caldata{MAY}{2}{\sunmonth{मेषः}{19}{}{चैत्रः}{वसन्तऋतुः}{गुरुः}{विकारी}{उत्तरायणम्}{वसन्तऋतुः}}
{\sunmoonsrdata{06:16}{19:55}{05:28(+1)}{18:03}{13:05}
{\kalas{04:53 05:34 09:54 09:00 10:49 18:06 11:44 14:27 17:11 19:01 20:36 22:30 23:47 02:22(+1)}}}
{\tnykdata{\anga{\tithi{28}{कृष्ण-त्रयोदशी}}{\time{21-27}{14:51}}\hspace{1ex}}%
{\anga{रेवती}{\time{49-41}{02:08(+1)}}\hspace{1ex}}{चन्द्रराशिः—\mbox{मीनः\RIGHTarrow{02:08(+1)}}}%
{\anga{विष्कम्भः}{\time{26-34}{16:53}}\hspace{1ex}\uanga{प्रीतिः}}%
{\anga{वणिजः}{\time{21-27}{14:51}}\hspace{1ex}\anga{विष्टिः}{\time{52-30}{03:16(+1)}}\hspace{1ex}\uanga{शकुनिः}}{}
}
{मासशिवरात्रिः\eventsep पञ्च-पर्व-पूजा (चतुर्दशी)}
{Thu} 
\cfoot{\rygdata{14:48--16:30}{06:16--07:58}{09:41--11:23}}
\caldata{MAY}{3}{\sunmonth{मेषः}{20}{}{चैत्रः}{वसन्तऋतुः}{शुक्रः}{विकारी}{उत्तरायणम्}{वसन्तऋतुः}}
{\sunmoonsrdata{06:15}{19:56}{06:00(+1)}{19:05}{13:05}
{\kalas{04:52 05:33 09:54 08:59 10:49 18:07 11:43 14:28 17:12 19:01 20:37 22:30 23:48 02:22(+1)}}}
{\tnykdata{\anga{\tithi{29}{कृष्ण-चतुर्दशी}}{\time{23-17}{15:33}}\hspace{1ex}}%
{\anga{अश्विनी}{\time{52-31}{03:15(+1)}}\hspace{1ex}}{चन्द्रराशिः—\mbox{मेषः}}%
{\anga{प्रीतिः}{\time{25-23}{16:24}}\hspace{1ex}\uanga{आयुष्मान्}}%
{\anga{शकुनिः}{\time{23-17}{15:33}}\hspace{1ex}\anga{चतुष्पात्}{\time{53-40}{03:43(+1)}}\hspace{1ex}\uanga{नाग}}{}
}
{गङ्गा-स्नानम्\eventsep काञ्ची ४७ जगद्गुरु श्री-चन्द्रशेखरेन्द्र सरस्वती ३ आराधना~\#{८५४}\eventsep पञ्च-पर्व-पूजा (अमावास्या)\eventsep सर्व-चैत्र-अमावास्या}
{Fri} 
\cfoot{\rygdata{11:23--13:05}{16:31--18:13}{07:57--09:40}}
\caldata{MAY}{4}{\sunmonth{मेषः}{21}{}{चैत्रः}{वसन्तऋतुः}{शनिः}{विकारी}{उत्तरायणम्}{वसन्तऋतुः}}
{\sunmoonsrdata{06:14}{19:57}{---}{20:08}{13:05}
{\kalas{04:51 05:32 09:53 08:58 10:48 18:07 11:43 14:28 17:12 19:02 20:38 22:31 23:48 02:22(+1)}}}
{\tnykdata{\anga{\tithi{30}{अमावास्या}}{\time{23-48}{15:45}}\hspace{1ex}}%
{\anga{अपभरणी}{\time{54-7}{03:53(+1)}}\hspace{1ex}}{चन्द्रराशिः—\mbox{मेषः}}%
{\anga{आयुष्मान्}{\time{23-12}{15:31}}\hspace{1ex}\uanga{सौभाग्यः}}%
{\anga{नाग}{\time{23-48}{15:45}}\hspace{1ex}\anga{किंस्तुघ्नः}{\time{53-35}{03:40(+1)}}\hspace{1ex}\uanga{बवः}}{}
}
{अग्निनक्षत्र-आरम्भः\eventsep बोधायन-इष्टिः\eventsep भार्गव-राम-पूजा\eventsep \tamil{சிறுத்தொண்ட நாயனார் (35) குருபூஜை}\eventsep पार्वणव्रतम् अमावास्यायाम्\eventsep वह्नि-व्रतम्}
{Sat} 
\cfoot{\rygdata{09:39--11:22}{14:48--16:31}{06:14--07:56}}
\caldata{MAY}{5}{\sunmonth{मेषः}{22}{}{वैशाखः}{वसन्तऋतुः}{भानुः}{विकारी}{उत्तरायणम्}{वसन्तऋतुः}}
{\sunmoonrsdata{06:12}{19:58}{06:35}{21:12}{13:05}
{\kalas{04:50 05:31 09:53 08:58 10:48 18:08 11:43 14:28 17:13 19:03 20:39 22:31 23:48 02:21(+1)}}}
{\tnykdata{\anga{\tithi{1}{शुक्ल-प्रथमा}}{\time{23-9}{15:28}}\hspace{1ex}}%
{\anga{कृत्तिका}{\time{54-40}{04:05(+1)}}\hspace{1ex}}{चन्द्रराशिः—\mbox{मेषः\RIGHTarrow{09:58}}}%
{\anga{सौभाग्यः}{\time{20-6}{14:15}}\hspace{1ex}\uanga{शोभनः}}%
{\anga{बवः}{\time{23-9}{15:28}}\hspace{1ex}\anga{बालवः}{\time{52-25}{03:11(+1)}}\hspace{1ex}\uanga{कौलवः}}{}
}
{चन्द्र-दर्शनम्~19:58\RIGHTarrow{}20:39\eventsep दर्शेष्टिः\eventsep कृत्तिका-व्रतम्\eventsep पार्वण-प्रायश्चित्तावकाशः पौर्णमास्याम्\eventsep पराशर-महर्षि-जयन्ती\eventsep स्थालीपाकः\eventsep वैशाख-मास-आरम्भः\eventsep श्यामा-शास्त्री-जयन्ती~\#{२५८}}
{Sun} 
\cfoot{\rygdata{18:15--19:58}{13:05--14:48}{16:32--18:15}}
\caldata{MAY}{6}{\sunmonth{मेषः}{23}{}{वैशाखः}{वसन्तऋतुः}{सोमः}{विकारी}{उत्तरायणम्}{वसन्तऋतुः}}
{\sunmoonrsdata{06:11}{19:59}{07:17}{22:16}{13:05}
{\kalas{04:50 05:31 09:52 08:57 10:47 18:08 11:42 14:28 17:13 19:04 20:39 22:32 23:48 02:21(+1)}}}
{\tnykdata{\anga{\tithi{2}{शुक्ल-द्वितीया}}{\time{21-30}{14:48}}\hspace{1ex}}%
{\anga{रोहिणी}{\time{54-19}{03:55(+1)}}\hspace{1ex}}{चन्द्रराशिः—\mbox{वृषभः}}%
{\anga{शोभनः}{\time{16-12}{12:40}}\hspace{1ex}\uanga{अतिगण्डः}}%
{\anga{कौलवः}{\time{21-30}{14:48}}\hspace{1ex}\anga{तैतिलः}{\time{50-19}{02:19(+1)}}\hspace{1ex}\uanga{गरः}}{}
}
{देवी-पर्व-२\eventsep कृतयुगादिः\eventsep \tamil{மங்கையர்க்கரசியார் நாயனார் (57) குருபூஜை}}
{Mon} 
\cfoot{\rygdata{07:55--09:38}{11:22--13:05}{14:49--16:32}}
\caldata{MAY}{7}{\sunmonth{मेषः}{24}{}{वैशाखः}{वसन्तऋतुः}{मङ्गलः}{विकारी}{उत्तरायणम्}{वसन्तऋतुः}}
{\sunmoonrsdata{06:10}{20:00}{08:06}{23:17}{13:05}
{\kalas{04:49 05:30 09:51 08:56 10:47 18:09 11:42 14:28 17:14 19:04 20:40 22:32 23:48 02:21(+1)}}}
{\tnykdata{\anga{\tithi{3}{शुक्ल-तृतीया}}{\time{19-0}{13:47}}\hspace{1ex}}%
{\anga{मृगशीर्षम्}{\time{53-13}{03:28(+1)}}\hspace{1ex}}{चन्द्रराशिः—\mbox{वृषभः\RIGHTarrow{15:44}}}%
{\anga{अतिगण्डः}{\time{11-36}{10:49}}\hspace{1ex}\uanga{सुकर्म}}%
{\anga{गरः}{\time{19-0}{13:47}}\hspace{1ex}\anga{वणिजः}{\time{47-27}{01:10(+1)}}\hspace{1ex}\uanga{विष्टिः}}{}
}
{अक्षय्य-तृतीया\eventsep बगलामुखी-जयन्ती\eventsep बलराम-जयन्ती\eventsep चन्दन-पूजा\eventsep राज-मातङ्गी-जयन्ती}
{Tue} 
\cfoot{\rygdata{16:32--18:16}{09:38--11:21}{13:05--14:49}}
\caldata{MAY}{8}{\sunmonth{मेषः}{25}{}{वैशाखः}{वसन्तऋतुः}{बुधः}{विकारी}{उत्तरायणम्}{वसन्तऋतुः}}
{\sunmoonrsdata{06:09}{20:01}{09:02}{00:12(+1)}{13:05}
{\kalas{04:48 05:29 09:51 08:56 10:46 18:10 11:42 14:28 17:14 19:05 20:41 22:33 23:49 02:20(+1)}}}
{\tnykdata{\anga{\tithi{4}{शुक्ल-चतुर्थी}}{\time{15-48}{12:29}}\hspace{1ex}}%
{\anga{आर्द्रा}{\time{51-29}{02:45(+1)}}\hspace{1ex}}{चन्द्रराशिः—\mbox{मिथुनम्}}%
{\anga{सुकर्म}{\time{6-27}{08:44}}\hspace{1ex}\uanga{धृतिः}}%
{\anga{विष्टिः}{\time{15-48}{12:29}}\hspace{1ex}\anga{बवः}{\time{43-57}{23:44}}\hspace{1ex}\uanga{बालवः}}{}
}
{रामानुज-जन्म-नक्षत्रम्~\#{१००३}\eventsep वार्ता-गौरी-व्रतम्\eventsep \tamil{விறன்மிண்ட நாயனார் (5) குருபூஜை}}
{Wed} 
\cfoot{\rygdata{13:05--14:49}{07:53--09:37}{11:21--13:05}}
\caldata{MAY}{9}{\sunmonth{मेषः}{26}{}{वैशाखः}{वसन्तऋतुः}{गुरुः}{विकारी}{उत्तरायणम्}{वसन्तऋतुः}}
{\sunmoonrsdata{06:08}{20:01}{10:05}{01:02(+1)}{13:05}
{\kalas{04:47 05:28 09:50 08:55 10:46 18:10 11:42 14:28 17:15 19:06 20:42 22:33 23:49 02:20(+1)}}}
{\tnykdata{\anga{\tithi{5}{शुक्ल-पञ्चमी}}{\time{12-0}{10:56}}\hspace{1ex}}%
{\anga{पुनर्वसुः}{\time{49-12}{01:49(+1)}}\hspace{1ex}}{चन्द्रराशिः—\mbox{मिथुनम्\RIGHTarrow{20:05}}}%
{\anga{धृतिः}{\time{0-47}{06:27}}\hspace{1ex}\anga{शूलः}{\time{54-38}{04:00(+1)}}\hspace{1ex}\uanga{गण्डः}}%
{\anga{बालवः}{\time{12-0}{10:56}}\hspace{1ex}\anga{कौलवः}{\time{39-52}{22:05}}\hspace{1ex}\uanga{तैतिलः}}{}
}
{षष्ठी-व्रतम्\eventsep काञ्ची ४० जगद्गुरु श्री-महादेवेन्द्र सरस्वती २ आराधना~\#{११०५}\eventsep लावण्य-गौरी-व्रतम्\eventsep सूरदास-जयन्ती~\#{५४२}\eventsep सर्प-पूजा\eventsep शङ्कर-जयन्ती~\#{२५२८}}
{Thu} 
\cfoot{\rygdata{14:49--16:33}{06:08--07:53}{09:37--11:21}}
\caldata{MAY}{10}{\sunmonth{मेषः}{27}{}{वैशाखः}{वसन्तऋतुः}{शुक्रः}{विकारी}{उत्तरायणम्}{वसन्तऋतुः}}
{\sunmoonrsdata{06:07}{20:02}{11:12}{01:44(+1)}{13:05}
{\kalas{04:47 05:27 09:50 08:54 10:46 18:11 11:41 14:28 17:15 19:07 20:43 22:33 23:49 02:20(+1)}}}
{\tnykdata{\anga{\tithi{6}{शुक्ल-षष्ठी}}{\time{7-39}{09:11}}\hspace{1ex}}%
{\anga{पुष्यः}{\time{46-25}{00:42(+1)}}\hspace{1ex}}{चन्द्रराशिः—\mbox{कर्कटः}}%
{\anga{गण्डः}{\time{48-8}{01:23(+1)}}\hspace{1ex}\uanga{वृद्धिः}}%
{\anga{तैतिलः}{\time{7-39}{09:11}}\hspace{1ex}\anga{गरः}{\time{35-16}{20:14}}\hspace{1ex}\uanga{वणिजः}}{}
}
{गङ्गा-सप्तमी\eventsep रामानुज-जयन्ती~\#{१००३}}
{Fri} 
\cfoot{\rygdata{11:20--13:05}{16:34--18:18}{07:52--09:36}}
\caldata{MAY}{11}{\sunmonth{मेषः}{28}{}{वैशाखः}{वसन्तऋतुः}{शनिः}{विकारी}{उत्तरायणम्}{वसन्तऋतुः}}
{\sunmoonrsdata{06:06}{20:03}{12:21}{02:22(+1)}{13:05}
{\kalas{04:46 05:26 09:50 08:54 10:45 18:12 11:41 14:29 17:16 19:07 20:43 22:34 23:49 02:20(+1)}}}
{\tnykdata{\anga{\tithi{7}{शुक्ल-सप्तमी}}{\time{2-49}{07:14}}\hspace{1ex}\anga{\tithi{8}{शुक्ल-अष्टमी}}{\time{57-30}{05:07(+1)}}\hspace{1ex}\avamA{}}%
{\anga{आश्रेषा}{\time{43-11}{23:23}}\hspace{1ex}}{चन्द्रराशिः—\mbox{कर्कटः\RIGHTarrow{23:23}}}%
{\anga{वृद्धिः}{\time{41-16}{22:37}}\hspace{1ex}\uanga{ध्रुवः}}%
{\anga{वणिजः}{\time{2-49}{07:14}}\hspace{1ex}\anga{विष्टिः}{\time{30-13}{18:12}}\hspace{1ex}\anga{बवः}{\time{57-30}{05:07(+1)}}\hspace{1ex}\uanga{बालवः}}{}
}
{काञ्ची २६ जगद्गुरु श्री-प्रज्ञाघनेन्द्र सरस्वती आराधना~\#{१४५६}\eventsep त्यागराज-जयन्ती~\#{२५३}\eventsep विद्यारण्य-स्वामि-जयन्ती\eventsep शर्करा-सप्तमी}
{Sat} 
\cfoot{\rygdata{09:36--11:20}{14:49--16:34}{06:06--07:51}}
\caldata{MAY}{12}{\sunmonth{मेषः}{29}{}{वैशाखः}{वसन्तऋतुः}{भानुः}{विकारी}{उत्तरायणम्}{वसन्तऋतुः}}
{\sunmoonrsdata{06:06}{20:04}{13:31}{02:56(+1)}{13:05}
{\kalas{04:45 05:25 09:49 08:53 10:45 18:12 11:41 14:29 17:16 19:08 20:44 22:34 23:49 02:19(+1)}}}
{\tnykdata{\anga{\tithi{9}{शुक्ल-नवमी}}{\time{51-53}{02:51(+1)}}\hspace{1ex}}%
{\anga{मघा}{\time{39-34}{21:55}}\hspace{1ex}}{चन्द्रराशिः—\mbox{सिंहः}}%
{\anga{ध्रुवः}{\time{34-5}{19:44}}\hspace{1ex}\uanga{व्याघातः}}%
{\anga{बालवः}{\time{24-45}{16:00}}\hspace{1ex}\anga{कौलवः}{\time{51-53}{02:51(+1)}}\hspace{1ex}\uanga{तैतिलः}}{}
}
{पुरी गोवर्धन-मठ-प्रतिष्ठापन-जयन्ती~\#{२५०४}\eventsep सीतानवमी\eventsep सिंहाचलं-चन्दन-महोत्सवः\eventsep वेङ्कटाचले पद्मावती-परिणयोत्सव-प्रारम्भः (गज-वाहनम्)\eventsep वसिष्ठ-महर्षि-जयन्ती}
{Sun} 
\cfoot{\rygdata{18:19--20:04}{13:05--14:50}{16:34--18:19}}
\caldata{MAY}{13}{\sunmonth{मेषः}{30}{}{वैशाखः}{वसन्तऋतुः}{सोमः}{विकारी}{उत्तरायणम्}{वसन्तऋतुः}}
{\sunmoonrsdata{06:05}{20:05}{14:41}{03:29(+1)}{13:05}
{\kalas{04:45 05:25 09:49 08:53 10:45 18:13 11:41 14:29 17:17 19:09 20:45 22:35 23:50 02:19(+1)}}}
{\tnykdata{\anga{\tithi{10}{शुक्ल-दशमी}}{\time{46-0}{00:29(+1)}}\hspace{1ex}}%
{\anga{पूर्वफल्गुनी}{\time{35-41}{20:21}}\hspace{1ex}}{चन्द्रराशिः—\mbox{सिंहः\RIGHTarrow{01:57(+1)}}}%
{\anga{व्याघातः}{\time{26-39}{16:45}}\hspace{1ex}\uanga{हर्षणः}}%
{\anga{तैतिलः}{\time{18-59}{13:40}}\hspace{1ex}\anga{गरः}{\time{46-0}{00:29(+1)}}\hspace{1ex}\uanga{वणिजः}}{}
}
{नॆरूर्-श्री-सदाशिव-ब्रह्मेन्द्र-आराधना~\#{१०५}\eventsep निमिषाम्बा-जयन्ती\eventsep वेङ्कटाचले पद्मावती-परिणयम् (अश्व-वाहनम्)\eventsep श्री-वासवी-जयन्ती}
{Mon} 
\cfoot{\rygdata{07:50--09:35}{11:20--13:05}{14:50--16:35}}
\caldata{MAY}{14}{\sunmonth{मेषः}{31}{\mbox{मेषः{\tiny\RIGHTarrow}{22:14}}}{वैशाखः}{वसन्तऋतुः}{मङ्गलः}{विकारी}{उत्तरायणम्}{वसन्तऋतुः}}
{\sunmoonrsdata{06:04}{20:06}{15:50}{04:00(+1)}{13:05}
{\kalas{04:44 05:24 09:48 08:52 10:44 18:14 11:41 14:29 17:17 19:10 20:46 22:35 23:50 02:19(+1)}}}
{\tnykdata{\anga{\tithi{11}{शुक्ल-एकादशी}}{\time{40-3}{22:05}}\hspace{1ex}}%
{\anga{उत्तरफल्गुनी}{\time{31-41}{18:44}}\hspace{1ex}}{चन्द्रराशिः—\mbox{कन्या}}%
{\anga{हर्षणः}{\time{19-8}{13:43}}\hspace{1ex}\uanga{वज्रम्}}%
{\anga{वणिजः}{\time{13-3}{11:17}}\hspace{1ex}\anga{विष्टिः}{\time{40-3}{22:05}}\hspace{1ex}\uanga{बवः}}{}
}
{बुध-जयन्ती\eventsep \tamil{மீனாக்ஷீ திருக்கல்யாணம்}\eventsep सर्व-मोहिनी-एकादशी\eventsep वेङ्कटाचले पद्मावती-परिणयोत्सव-समापनम् (गरुड-वाहनम्)\eventsep वृषभ-रवि-सङ्क्रमण-विष्णुपदी-पुण्यकालः~15:50\RIGHTarrow{}04:38(+1)}
{Tue} 
\cfoot{\rygdata{16:35--18:21}{09:34--11:19}{13:05--14:50}}
\caldata{MAY}{15}{\sunmonth{वृषभः}{1}{}{वैशाखः}{वसन्तऋतुः}{बुधः}{विकारी}{उत्तरायणम्}{वसन्तऋतुः}}
{\sunmoonrsdata{06:03}{20:07}{17:00}{04:33(+1)}{13:05}
{\kalas{04:43 05:23 09:48 08:52 10:44 18:14 11:40 14:29 17:18 19:10 20:46 22:36 23:50 02:19(+1)}}}
{\tnykdata{\anga{\tithi{12}{शुक्ल-द्वादशी}}{\time{34-14}{19:45}}\hspace{1ex}}%
{\anga{हस्तः}{\time{27-47}{17:10}}\hspace{1ex}}{चन्द्रराशिः—\mbox{कन्या\RIGHTarrow{04:26(+1)}}}%
{\anga{वज्रम्}{\time{11-38}{10:42}}\hspace{1ex}\uanga{सिद्धिः}}%
{\anga{बवः}{\time{7-8}{08:54}}\hspace{1ex}\anga{बालवः}{\time{34-14}{19:45}}\hspace{1ex}\uanga{कौलवः}}{}
}
{गिरिजा-कल्याणम्\eventsep हरिवासरः\RIGHTarrow{}03:30\eventsep मधुसूदन-पूजा\eventsep परशुराम-द्वादशी\eventsep प्रदोष-व्रतम्~20:07\RIGHTarrow{}20:46\eventsep रुक्मिणी-द्वादशी}
{Wed} 
\cfoot{\rygdata{13:05--14:50}{07:48--09:34}{11:19--13:05}}
\caldata{MAY}{16}{\sunmonth{वृषभः}{2}{}{वैशाखः}{वसन्तऋतुः}{गुरुः}{विकारी}{उत्तरायणम्}{वसन्तऋतुः}}
{\sunmoonrsdata{06:02}{20:08}{18:09}{05:08(+1)}{13:05}
{\kalas{04:43 05:22 09:48 08:51 10:44 18:15 11:40 14:29 17:18 19:11 20:47 22:36 23:50 02:19(+1)}}}
{\tnykdata{\anga{\tithi{13}{शुक्ल-त्रयोदशी}}{\time{28-50}{17:34}}\hspace{1ex}}%
{\anga{चित्रा}{\time{24-16}{15:45}}\hspace{1ex}}{चन्द्रराशिः—\mbox{तुला}}%
{\anga{सिद्धिः}{\time{4-23}{07:48}}\hspace{1ex}\anga{व्यतीपातः}{\time{57-34}{05:04(+1)}}\hspace{1ex}\uanga{वरीयान्}}%
{\anga{कौलवः}{\time{1-29}{06:38}}\hspace{1ex}\anga{तैतिलः}{\time{28-50}{17:34}}\hspace{1ex}\anga{गरः}{\time{56-21}{04:35(+1)}}\hspace{1ex}\uanga{वणिजः}}{}
}
{नृसिंह-जयन्ती\eventsep व्यतीपात-श्राद्धम्}
{Thu} 
\cfoot{\rygdata{14:51--16:36}{06:02--07:48}{09:33--11:19}}
\caldata{MAY}{17}{\sunmonth{वृषभः}{3}{}{वैशाखः}{वसन्तऋतुः}{शुक्रः}{विकारी}{उत्तरायणम्}{वसन्तऋतुः}}
{\sunmoonrsdata{06:01}{20:08}{19:18}{05:46(+1)}{13:05}
{\kalas{04:42 05:22 09:47 08:51 10:44 18:15 11:40 14:30 17:19 19:12 20:48 22:36 23:50 02:18(+1)}}}
{\tnykdata{\anga{\tithi{14}{शुक्ल-चतुर्दशी}}{\time{24-7}{15:40}}\hspace{1ex}}%
{\anga{स्वाती}{\time{21-25}{14:35}}\hspace{1ex}}{चन्द्रराशिः—\mbox{तुला}}%
{\anga{वरीयान्}{\time{51-29}{02:37(+1)}}\hspace{1ex}\uanga{परिघः}}%
{\anga{वणिजः}{\time{24-7}{15:40}}\hspace{1ex}\anga{विष्टिः}{\time{52-7}{02:52(+1)}}\hspace{1ex}\uanga{बवः}}{}
}
{काञ्ची ३९ जगद्गुरु श्री-सच्चिद्विलासेन्द्र सरस्वती आराधना~\#{११४७}\eventsep पञ्च-पर्व-पूजा (पूर्णिमा)\eventsep वेङ्कटाचले पूर्णिमा-गरुड-सेवा}
{Fri} 
\cfoot{\rygdata{11:19--13:05}{16:37--18:23}{07:47--09:33}}
\caldata{MAY}{18}{\sunmonth{वृषभः}{4}{}{वैशाखः}{वसन्तऋतुः}{शनिः}{विकारी}{उत्तरायणम्}{वसन्तऋतुः}}
{\sunmoonrsdata{06:01}{20:09}{20:26}{---}{13:05}
{\kalas{04:42 05:21 09:47 08:50 10:43 18:16 11:40 14:30 17:19 19:13 20:49 22:37 23:51 02:18(+1)}}}
{\tnykdata{\anga{\tithi{15}{पौर्णमासी}}{\time{20-25}{14:11}}\hspace{1ex}}%
{\anga{विशाखा}{\time{19-33}{13:50}}\hspace{1ex}}{चन्द्रराशिः—\mbox{तुला\RIGHTarrow{07:59}}}%
{\anga{परिघः}{\time{46-21}{00:33(+1)}}\hspace{1ex}\uanga{शिवः}}%
{\anga{बवः}{\time{20-25}{14:11}}\hspace{1ex}\anga{बालवः}{\time{49-1}{01:37(+1)}}\hspace{1ex}\uanga{कौलवः}}{}
}
{अन्नमाचार्य-जयन्ती\eventsep अर्धनारीश्वर-व्रतम्\eventsep छिन्नमस्ता-जयन्ती\eventsep काञ्ची कामकोटि-मठ-प्रतिष्ठापन-जयन्ती~\#{२५०१}\eventsep पार्वणव्रतम् पूर्णिमायाम्\eventsep पूर्णिमा-व्रतम्\eventsep सम्पत्-गौरी-व्रतम्\eventsep वैशाख-पूर्णिमा-स्नानम्\eventsep शरभ-जयन्ती}
{Sat} 
\cfoot{\rygdata{09:33--11:19}{14:51--16:37}{06:01--07:47}}
\caldata{MAY}{19}{\sunmonth{वृषभः}{5}{}{वैशाखः}{वसन्तऋतुः}{भानुः}{विकारी}{उत्तरायणम्}{वसन्तऋतुः}}
{\sunmoonsrdata{06:00}{20:10}{21:29}{06:29}{13:05}
{\kalas{04:41 05:20 09:46 08:50 10:43 18:17 11:40 14:30 17:20 19:13 20:49 22:37 23:51 02:18(+1)}}}
{\tnykdata{\anga{\tithi{16}{कृष्ण-प्रथमा}}{\time{18-1}{13:12}}\hspace{1ex}}%
{\anga{अनूराधा}{\time{18-59}{13:35}}\hspace{1ex}}{चन्द्रराशिः—\mbox{वृश्चिकः}}%
{\anga{शिवः}{\time{42-20}{22:56}}\hspace{1ex}\uanga{सिद्धः}}%
{\anga{कौलवः}{\time{18-1}{13:12}}\hspace{1ex}\anga{तैतिलः}{\time{47-22}{00:57(+1)}}\hspace{1ex}\uanga{गरः}}{}
}
{काञ्ची ६८ जगद्गुरु श्री-चन्द्रशेखरेन्द्र सरस्वती ७ जयन्ती~\#{१२६}\eventsep पार्वण-प्रायश्चित्तावकाशः दर्शे\eventsep पूर्णमासेष्टिः\eventsep स्थालीपाकः}
{Sun} 
\cfoot{\rygdata{18:24--20:10}{13:05--14:51}{16:37--18:24}}
\caldata{MAY}{20}{\sunmonth{वृषभः}{6}{}{वैशाखः}{वसन्तऋतुः}{सोमः}{विकारी}{उत्तरायणम्}{वसन्तऋतुः}}
{\sunmoonsrdata{05:59}{20:11}{22:27}{07:18}{13:05}
{\kalas{04:40 05:20 09:46 08:49 10:43 18:17 11:40 14:30 17:20 19:14 20:50 22:38 23:51 02:18(+1)}}}
{\tnykdata{\anga{\tithi{17}{कृष्ण-द्वितीया}}{\time{17-9}{12:51}}\hspace{1ex}}%
{\anga{ज्येष्ठा}{\time{19-55}{13:57}}\hspace{1ex}}{चन्द्रराशिः—\mbox{वृश्चिकः\RIGHTarrow{13:57}}}%
{\anga{सिद्धः}{\time{39-38}{21:50}}\hspace{1ex}\uanga{साध्यः}}%
{\anga{गरः}{\time{17-9}{12:51}}\hspace{1ex}\anga{वणिजः}{\time{47-20}{00:55(+1)}}\hspace{1ex}\uanga{विष्टिः}}{}
}
{माधव-मासः/वसन्तऋतुः\RIGHTarrow{}00:59(+1)\eventsep नारद-जयन्ती\eventsep पार्थिव-कल्पादिः}
{Mon} 
\cfoot{\rygdata{07:45--09:32}{11:18--13:05}{14:51--16:38}}
\caldata{MAY}{21}{\sunmonth{वृषभः}{7}{}{वैशाखः}{वसन्तऋतुः}{मङ्गलः}{विकारी}{उत्तरायणम्}{वसन्तऋतुः}}
{\sunmoonsrdata{05:58}{20:12}{23:18}{08:11}{13:05}
{\kalas{04:40 05:19 09:46 08:49 10:43 18:18 11:40 14:30 17:21 19:15 20:51 22:38 23:51 02:18(+1)}}}
{\tnykdata{\anga{\tithi{18}{कृष्ण-तृतीया}}{\time{17-59}{13:10}}\hspace{1ex}}%
{\anga{मूला}{\time{22-32}{14:59}}\hspace{1ex}}{चन्द्रराशिः—\mbox{धनुः}}%
{\anga{साध्यः}{\time{38-16}{21:17}}\hspace{1ex}\uanga{शुभः}}%
{\anga{विष्टिः}{\time{17-59}{13:10}}\hspace{1ex}\anga{बवः}{\time{49-2}{01:35(+1)}}\hspace{1ex}\uanga{बालवः}}{}
}
{षडशीति-पुण्यकालः~00:59\RIGHTarrow{}00:59(+1)\eventsep अङ्गारकी एकदन्त-महागणपति सङ्कटहर-चतुर्थी-व्रतम्\eventsep काञ्ची ३० जगद्गुरु श्री-बोधेन्द्र सरस्वती २ आराधना~\#{१३६५}\eventsep काञ्ची जगद्गुरु श्री-शङ्कर विजयेन्द्र सरस्वती आश्रम-स्वीकार-जयन्ती~\#{३७}\eventsep \tamil{முருக நாயனார் (15) குருபூஜை}\eventsep \tamil{திருஞானஸம்பந்தமூர்த்தி நாயனார் (27) குருபூஜை}\eventsep \tamil{திருநீலகண்ட யாழ்ப்பாண நாயனார் (60) குருபூஜை}\eventsep \tamil{திருநீலநக்க நாயனார் (25) குருபூஜை}}
{Tue} 
\cfoot{\rygdata{16:38--18:25}{09:32--11:18}{13:05--14:52}}
\caldata{MAY}{22}{\sunmonth{वृषभः}{8}{}{वैशाखः}{वसन्तऋतुः}{बुधः}{विकारी}{उत्तरायणम्}{वसन्तऋतुः}}
{\sunmoonsrdata{05:58}{20:13}{00:02(+1)}{09:07}{13:05}
{\kalas{04:40 05:19 09:46 08:49 10:43 18:19 11:40 14:31 17:21 19:16 20:51 22:39 23:52 02:18(+1)}}}
{\tnykdata{\anga{\tithi{19}{कृष्ण-चतुर्थी}}{\time{20-32}{14:11}}\hspace{1ex}}%
{\anga{पूर्वाषाढा}{\time{26-48}{16:41}}\hspace{1ex}}{चन्द्रराशिः—\mbox{धनुः\RIGHTarrow{23:12}}}%
{\anga{शुभः}{\time{38-13}{21:15}}\hspace{1ex}\uanga{शुक्लः}}%
{\anga{बालवः}{\time{20-32}{14:11}}\hspace{1ex}\anga{कौलवः}{\time{52-23}{02:55(+1)}}\hspace{1ex}\uanga{तैतिलः}}{}
}
{\tamil{மயிலை~வெள்ளீஶ்வரர்~ப்ரஹ்மோத்ஸவம்}\eventsep सावित्री-व्रतम्\eventsep सती-अनसूया-जयन्ती}
{Wed} 
\cfoot{\rygdata{13:05--14:52}{07:44--09:31}{11:18--13:05}}
\caldata{MAY}{23}{\sunmonth{वृषभः}{9}{}{वैशाखः}{वसन्तऋतुः}{गुरुः}{विकारी}{उत्तरायणम्}{वसन्तऋतुः}}
{\sunmoonsrdata{05:57}{20:13}{00:40(+1)}{10:05}{13:05}
{\kalas{04:39 05:18 09:45 08:48 10:42 18:19 11:39 14:31 17:22 19:16 20:52 22:39 23:52 02:18(+1)}}}
{\tnykdata{\anga{\tithi{20}{कृष्ण-पञ्चमी}}{\time{24-37}{15:48}}\hspace{1ex}}%
{\anga{उत्तराषाढा}{\time{32-33}{18:58}}\hspace{1ex}}{चन्द्रराशिः—\mbox{मकरः}}%
{\anga{शुक्लः}{\time{39-19}{21:41}}\hspace{1ex}\uanga{ब्रह्म}}%
{\anga{तैतिलः}{\time{24-37}{15:48}}\hspace{1ex}\anga{गरः}{\time{57-9}{04:49(+1)}}\hspace{1ex}\uanga{वणिजः}}{}
}
{कश्यप-महर्षि-जयन्ती}
{Thu} 
\cfoot{\rygdata{14:52--16:39}{05:57--07:44}{09:31--11:18}}
\caldata{MAY}{24}{\sunmonth{वृषभः}{10}{}{वैशाखः}{वसन्तऋतुः}{शुक्रः}{विकारी}{उत्तरायणम्}{वसन्तऋतुः}}
{\sunmoonsrdata{05:56}{20:14}{01:12(+1)}{11:03}{13:05}
{\kalas{04:39 05:18 09:45 08:48 10:42 18:20 11:39 14:31 17:22 19:17 20:53 22:39 23:52 02:18(+1)}}}
{\tnykdata{\anga{\tithi{21}{कृष्ण-षष्ठी}}{\time{29-56}{17:55}}\hspace{1ex}}%
{\anga{श्रवणः}{\time{39-25}{21:43}}\hspace{1ex}}{चन्द्रराशिः—\mbox{मकरः}}%
{\anga{ब्रह्म}{\time{41-16}{22:27}}\hspace{1ex}\uanga{इन्द्रः}}%
{\anga{वणिजः}{\time{29-56}{17:55}}\hspace{1ex}\uanga{विष्टिः}}{}
}
{श्रवण-व्रतम्}
{Fri} 
\cfoot{\rygdata{11:18--13:05}{16:40--18:27}{07:44--09:31}}
\caldata{MAY}{25}{\sunmonth{वृषभः}{11}{}{वैशाखः}{वसन्तऋतुः}{शनिः}{विकारी}{उत्तरायणम्}{वसन्तऋतुः}}
{\sunmoonsrdata{05:56}{20:15}{01:42(+1)}{12:01}{13:05}
{\kalas{04:38 05:17 09:45 08:48 10:42 18:20 11:39 14:31 17:23 19:18 20:54 22:40 23:52 02:18(+1)}}}
{\tnykdata{\anga{\tithi{22}{कृष्ण-सप्तमी}}{\time{35-57}{20:19}}\hspace{1ex}}%
{\anga{श्रविष्ठा}{\time{46-54}{00:42(+1)}}\hspace{1ex}}{चन्द्रराशिः—\mbox{मकरः\RIGHTarrow{11:11}}}%
{\anga{इन्द्रः}{\time{43-42}{23:25}}\hspace{1ex}\uanga{वैधृतिः}}%
{\anga{विष्टिः}{\time{2-54}{07:06}}\hspace{1ex}\anga{बवः}{\time{35-57}{20:19}}\hspace{1ex}\uanga{बालवः}}{}
}
{पञ्च-पर्व-पूजा (अष्टमी)}
{Sat} 
\cfoot{\rygdata{09:31--11:18}{14:53--16:40}{05:56--07:43}}
\caldata{MAY}{26}{\sunmonth{वृषभः}{12}{}{वैशाखः}{वसन्तऋतुः}{भानुः}{विकारी}{उत्तरायणम्}{वसन्तऋतुः}}
{\sunmoonsrdata{05:55}{20:16}{02:09(+1)}{12:58}{13:05}
{\kalas{04:38 05:17 09:45 08:47 10:42 18:21 11:39 14:31 17:24 19:18 20:54 22:40 23:53 02:18(+1)}}}
{\tnykdata{\anga{\tithi{23}{कृष्ण-अष्टमी}}{\time{42-6}{22:46}}\hspace{1ex}}%
{\anga{शतभिषक्}{\time{54-23}{03:41(+1)}}\hspace{1ex}}{चन्द्रराशिः—\mbox{कुम्भः}}%
{\anga{वैधृतिः}{\time{46-9}{00:23(+1)}}\hspace{1ex}\uanga{विष्कम्भः}}%
{\anga{बालवः}{\time{9-4}{09:33}}\hspace{1ex}\anga{कौलवः}{\time{42-6}{22:46}}\hspace{1ex}\uanga{तैतिलः}}{}
}
{वैधृति-श्राद्धम्}
{Sun} 
\cfoot{\rygdata{18:28--20:16}{13:05--14:53}{16:40--18:28}}
\caldata{MAY}{27}{\sunmonth{वृषभः}{13}{}{वैशाखः}{वसन्तऋतुः}{सोमः}{विकारी}{उत्तरायणम्}{वसन्तऋतुः}}
{\sunmoonsrdata{05:55}{20:16}{02:35(+1)}{13:55}{13:06}
{\kalas{04:38 05:16 09:45 08:47 10:42 18:21 11:39 14:32 17:24 19:19 20:55 22:41 23:53 02:18(+1)}}}
{\tnykdata{\anga{\tithi{24}{कृष्ण-नवमी}}{\time{47-44}{01:01(+1)}}\hspace{1ex}}%
{\fullanga{पूर्वप्रोष्ठपदा}}{चन्द्रराशिः—\mbox{कुम्भः\RIGHTarrow{23:47}}}%
{\anga{विष्कम्भः}{\time{48-12}{01:12(+1)}}\hspace{1ex}\uanga{प्रीतिः}}%
{\anga{तैतिलः}{\time{15-2}{11:56}}\hspace{1ex}\anga{गरः}{\time{47-44}{01:01(+1)}}\hspace{1ex}\uanga{वणिजः}}{}
}
{काञ्ची ७ जगद्गुरु श्री-आनन्दज्ञानेन्द्र सरस्वती आराधना~\#{२०७४}}
{Mon} 
\cfoot{\rygdata{07:42--09:30}{11:18--13:06}{14:53--16:41}}
\caldata{MAY}{28}{\sunmonth{वृषभः}{14}{}{वैशाखः}{वसन्तऋतुः}{मङ्गलः}{विकारी}{उत्तरायणम्}{वसन्तऋतुः}}
{\sunmoonsrdata{05:54}{20:17}{03:01(+1)}{14:52}{13:06}
{\kalas{04:37 05:16 09:44 08:47 10:42 18:22 11:39 14:32 17:25 19:20 20:55 22:41 23:53 02:17(+1)}}}
{\tnykdata{\anga{\tithi{25}{कृष्ण-दशमी}}{\time{52-21}{02:51(+1)}}\hspace{1ex}}%
{\anga{पूर्वप्रोष्ठपदा}{\time{1-19}{06:26}}\hspace{1ex}}{चन्द्रराशिः—\mbox{मीनः}}%
{\anga{प्रीतिः}{\time{49-29}{01:42(+1)}}\hspace{1ex}\uanga{आयुष्मान्}}%
{\anga{वणिजः}{\time{20-13}{14:00}}\hspace{1ex}\anga{विष्टिः}{\time{52-21}{02:51(+1)}}\hspace{1ex}\uanga{बवः}}{}
}
{अग्निनक्षत्र-समापनम्}
{Tue} 
\cfoot{\rygdata{16:41--18:29}{09:30--11:18}{13:06--14:53}}
\caldata{MAY}{29}{\sunmonth{वृषभः}{15}{}{वैशाखः}{वसन्तऋतुः}{बुधः}{विकारी}{उत्तरायणम्}{वसन्तऋतुः}}
{\sunmoonsrdata{05:54}{20:18}{03:28(+1)}{15:50}{13:06}
{\kalas{04:37 05:15 09:44 08:47 10:42 18:23 11:39 14:32 17:25 19:20 20:56 22:42 23:54 02:17(+1)}}}
{\tnykdata{\anga{\tithi{26}{कृष्ण-एकादशी}}{\time{55-35}{04:08(+1)}}\hspace{1ex}}%
{\anga{उत्तरप्रोष्ठपदा}{\time{7-9}{08:46}}\hspace{1ex}}{चन्द्रराशिः—\mbox{मीनः}}%
{\anga{आयुष्मान्}{\time{49-41}{01:46(+1)}}\hspace{1ex}\uanga{सौभाग्यः}}%
{\anga{बवः}{\time{24-10}{15:34}}\hspace{1ex}\anga{बालवः}{\time{55-35}{04:08(+1)}}\hspace{1ex}\uanga{कौलवः}}{}
}
{आयुष्मद्-बव-सौम्य-संयॊगः\eventsep भद्रकाळी-जयन्ती\eventsep सर्व-अपरा-एकादशी}
{Wed} 
\cfoot{\rygdata{13:06--14:54}{07:42--09:30}{11:18--13:06}}
\caldata{MAY}{30}{\sunmonth{वृषभः}{16}{}{वैशाखः}{वसन्तऋतुः}{गुरुः}{विकारी}{उत्तरायणम्}{वसन्तऋतुः}}
{\sunmoonsrdata{05:53}{20:18}{03:58(+1)}{16:51}{13:06}
{\kalas{04:37 05:15 09:44 08:46 10:42 18:23 11:39 14:32 17:26 19:21 20:57 22:42 23:54 02:17(+1)}}}
{\tnykdata{\anga{\tithi{27}{कृष्ण-द्वादशी}}{\time{57-12}{04:46(+1)}}\hspace{1ex}}%
{\anga{रेवती}{\time{11-35}{10:32}}\hspace{1ex}}{चन्द्रराशिः—\mbox{मीनः\RIGHTarrow{10:32}}}%
{\anga{सौभाग्यः}{\time{48-40}{01:21(+1)}}\hspace{1ex}\uanga{शोभनः}}%
{\anga{कौलवः}{\time{26-36}{16:32}}\hspace{1ex}\anga{तैतिलः}{\time{57-12}{04:46(+1)}}\hspace{1ex}\uanga{गरः}}{}
}
{हरिवासरः\RIGHTarrow{}10:21}
{Thu} 
\cfoot{\rygdata{14:54--16:42}{05:53--07:41}{09:30--11:18}}
\caldata{MAY}{31}{\sunmonth{वृषभः}{17}{}{वैशाखः}{वसन्तऋतुः}{शुक्रः}{विकारी}{उत्तरायणम्}{वसन्तऋतुः}}
{\sunmoonsrdata{05:53}{20:19}{04:32(+1)}{17:53}{13:06}
{\kalas{04:36 05:15 09:44 08:46 10:42 18:24 11:39 14:33 17:26 19:21 20:57 22:42 23:54 02:18(+1)}}}
{\tnykdata{\anga{\tithi{28}{कृष्ण-त्रयोदशी}}{\time{57-13}{04:46(+1)}}\hspace{1ex}}%
{\anga{अश्विनी}{\time{14-27}{11:40}}\hspace{1ex}}{चन्द्रराशिः—\mbox{मेषः}}%
{\anga{शोभनः}{\time{46-22}{00:26(+1)}}\hspace{1ex}\uanga{अतिगण्डः}}%
{\anga{गरः}{\time{27-25}{16:51}}\hspace{1ex}\anga{वणिजः}{\time{57-13}{04:46(+1)}}\hspace{1ex}\uanga{विष्टिः}}{}
}
{प्रदोष-व्रतम्~20:19\RIGHTarrow{}20:57}
{Fri} 
\cfoot{\rygdata{11:18--13:06}{16:43--18:31}{07:41--09:29}}
\caldata{JUNE}{1}{\sunmonth{वृषभः}{18}{}{वैशाखः}{वसन्तऋतुः}{शनिः}{विकारी}{उत्तरायणम्}{वसन्तऋतुः}}
{\sunmoonsrdata{05:53}{20:20}{05:11(+1)}{18:58}{13:06}
{\kalas{04:36 05:14 09:44 08:46 10:42 18:24 11:39 14:33 17:26 19:22 20:58 22:43 23:54 02:18(+1)}}}
{\tnykdata{\anga{\tithi{29}{कृष्ण-चतुर्दशी}}{\time{55-42}{04:10(+1)}}\hspace{1ex}}%
{\anga{अपभरणी}{\time{15-45}{12:11}}\hspace{1ex}}{चन्द्रराशिः—\mbox{मेषः\RIGHTarrow{18:13}}}%
{\anga{अतिगण्डः}{\time{42-50}{23:01}}\hspace{1ex}\uanga{सुकर्म}}%
{\anga{विष्टिः}{\time{26-39}{16:32}}\hspace{1ex}\anga{शकुनिः}{\time{55-42}{04:10(+1)}}\hspace{1ex}\uanga{चतुष्पात्}}{}
}
{काञ्ची ३ जगद्गुरु श्री-सर्वज्ञात्मेन्द्र सरस्वती आराधना~\#{२३८३}\eventsep कृत्तिका-व्रतम्\eventsep \tamil{கழற்சிங்க நாயனார் (51) குருபூஜை}\eventsep मासशिवरात्रिः\eventsep पञ्च-पर्व-पूजा (चतुर्दशी)}
{Sat} 
\cfoot{\rygdata{09:29--11:18}{14:55--16:43}{05:53--07:41}}
\caldata{JUNE}{2}{\sunmonth{वृषभः}{19}{}{वैशाखः}{वसन्तऋतुः}{भानुः}{विकारी}{उत्तरायणम्}{वसन्तऋतुः}}
{\sunmoonsrdata{05:52}{20:20}{---}{20:03}{13:06}
{\kalas{04:36 05:14 09:44 08:46 10:42 18:25 11:39 14:33 17:27 19:23 20:59 22:43 23:55 02:18(+1)}}}
{\tnykdata{\anga{\tithi{30}{अमावास्या}}{\time{52-53}{03:01(+1)}}\hspace{1ex}}%
{\anga{कृत्तिका}{\time{15-36}{12:07}}\hspace{1ex}}{चन्द्रराशिः—\mbox{वृषभः}}%
{\anga{सुकर्म}{\time{38-13}{21:10}}\hspace{1ex}\uanga{धृतिः}}%
{\anga{चतुष्पात्}{\time{24-27}{15:39}}\hspace{1ex}\anga{नाग}{\time{52-53}{03:01(+1)}}\hspace{1ex}\uanga{किंस्तुघ्नः}}{}
}
{पार्वणव्रतम् अमावास्यायाम्\eventsep पञ्च-पर्व-पूजा (अमावास्या)\eventsep सर्व-वैशाख-अमावास्या\eventsep वैशाख-मास-समापनम्\eventsep वैशाख-स्नानपूर्तिः\eventsep शनि-जयन्ती\eventsep शुक-महर्षि-जयन्ती}
{Sun} 
\cfoot{\rygdata{18:32--20:20}{13:06--14:55}{16:43--18:32}}
\caldata{JUNE}{3}{\sunmonth{वृषभः}{20}{}{ज्यैष्ठः}{ग्रीष्मऋतुः}{सोमः}{विकारी}{उत्तरायणम्}{वसन्तऋतुः}}
{\sunmoonrsdata{05:52}{20:21}{05:58}{21:06}{13:06}
{\kalas{04:36 05:14 09:44 08:46 10:42 18:25 11:40 14:33 17:27 19:23 20:59 22:44 23:55 02:18(+1)}}}
{\tnykdata{\anga{\tithi{1}{शुक्ल-प्रथमा}}{\time{48-57}{01:27(+1)}}\hspace{1ex}}%
{\anga{रोहिणी}{\time{14-13}{11:33}}\hspace{1ex}}{चन्द्रराशिः—\mbox{वृषभः\RIGHTarrow{23:08}}}%
{\anga{धृतिः}{\time{32-40}{18:56}}\hspace{1ex}\uanga{शूलः}}%
{\anga{किंस्तुघ्नः}{\time{21-3}{14:17}}\hspace{1ex}\anga{बवः}{\time{48-57}{01:27(+1)}}\hspace{1ex}\uanga{बालवः}}{}
}
{भद्र-चतुष्टय-व्रतम्\eventsep दर्शेष्टिः\eventsep करवीर-व्रतम्\eventsep पार्वण-प्रायश्चित्तावकाशः पौर्णमास्याम्\eventsep पुन्नाग-गौरी-व्रतम्\eventsep सोममृगशीर्ष-पुण्यकालः~11:33\RIGHTarrow{}\eventsep स्थालीपाकः}
{Mon} 
\cfoot{\rygdata{07:41--09:29}{11:18--13:06}{14:55--16:44}}
\caldata{JUNE}{4}{\sunmonth{वृषभः}{21}{}{ज्यैष्ठः}{ग्रीष्मऋतुः}{मङ्गलः}{विकारी}{उत्तरायणम्}{वसन्तऋतुः}}
{\sunmoonrsdata{05:52}{20:22}{06:52}{22:05}{13:07}
{\kalas{04:36 05:14 09:44 08:46 10:42 18:26 11:40 14:34 17:28 19:24 21:00 22:44 23:55 02:18(+1)}}}
{\tnykdata{\anga{\tithi{2}{शुक्ल-द्वितीया}}{\time{44-13}{23:33}}\hspace{1ex}}%
{\anga{मृगशीर्षम्}{\time{11-52}{10:37}}\hspace{1ex}}{चन्द्रराशिः—\mbox{मिथुनम्}}%
{\anga{शूलः}{\time{26-24}{16:25}}\hspace{1ex}\uanga{गण्डः}}%
{\anga{बालवः}{\time{16-41}{12:32}}\hspace{1ex}\anga{कौलवः}{\time{44-13}{23:33}}\hspace{1ex}\uanga{तैतिलः}}{}
}
{चन्द्र-दर्शनम्~20:22\RIGHTarrow{}21:00\eventsep शृङ्गेरी ३२ जगद्गुरु श्री-नृसिंह भारती आराधना}
{Tue} 
\cfoot{\rygdata{16:44--18:33}{09:29--11:18}{13:07--14:55}}
\caldata{JUNE}{5}{\sunmonth{वृषभः}{22}{}{ज्यैष्ठः}{ग्रीष्मऋतुः}{बुधः}{विकारी}{उत्तरायणम्}{वसन्तऋतुः}}
{\sunmoonrsdata{05:51}{20:22}{07:55}{22:58}{13:07}
{\kalas{04:35 05:13 09:44 08:46 10:42 18:26 11:40 14:34 17:28 19:24 21:00 22:44 23:56 02:18(+1)}}}
{\tnykdata{\anga{\tithi{3}{शुक्ल-तृतीया}}{\time{38-53}{21:25}}\hspace{1ex}}%
{\anga{आर्द्रा}{\time{8-47}{09:22}}\hspace{1ex}}{चन्द्रराशिः—\mbox{मिथुनम्\RIGHTarrow{02:19(+1)}}}%
{\anga{गण्डः}{\time{19-34}{13:41}}\hspace{1ex}\uanga{वृद्धिः}}%
{\anga{तैतिलः}{\time{11-37}{10:30}}\hspace{1ex}\anga{गरः}{\time{38-53}{21:25}}\hspace{1ex}\uanga{वणिजः}}{}
}
{रम्भा-तृतीया}
{Wed} 
\cfoot{\rygdata{13:07--14:56}{07:40--09:29}{11:18--13:07}}
\caldata{JUNE}{6}{\sunmonth{वृषभः}{23}{}{ज्यैष्ठः}{ग्रीष्मऋतुः}{गुरुः}{विकारी}{उत्तरायणम्}{वसन्तऋतुः}}
{\sunmoonrsdata{05:51}{20:23}{09:02}{23:44}{13:07}
{\kalas{04:35 05:13 09:44 08:46 10:42 18:27 11:40 14:34 17:29 19:25 21:01 22:45 23:56 02:18(+1)}}}
{\tnykdata{\anga{\tithi{4}{शुक्ल-चतुर्थी}}{\time{33-10}{19:07}}\hspace{1ex}}%
{\anga{पुनर्वसुः}{\time{5-13}{07:57}}\hspace{1ex}}{चन्द्रराशिः—\mbox{कर्कटः}}%
{\anga{वृद्धिः}{\time{12-23}{10:48}}\hspace{1ex}\uanga{ध्रुवः}}%
{\anga{वणिजः}{\time{6-4}{08:17}}\hspace{1ex}\anga{विष्टिः}{\time{33-10}{19:07}}\hspace{1ex}\uanga{बवः}}{}
}
{गुरुपुष्य-पुण्यकालः~07:57\RIGHTarrow{}\eventsep कदली-गौरी-व्रतम्/पूजा\eventsep \tamil{நமிநந்தியடிகள் நாயனார் (26) குருபூஜை}\eventsep उमा-अवतारः}
{Thu} 
\cfoot{\rygdata{14:56--16:45}{05:51--07:40}{09:29--11:18}}
\caldata{JUNE}{7}{\sunmonth{वृषभः}{24}{}{ज्यैष्ठः}{ग्रीष्मऋतुः}{शुक्रः}{विकारी}{उत्तरायणम्}{वसन्तऋतुः}}
{\sunmoonrsdata{05:51}{20:24}{10:12}{00:24(+1)}{13:07}
{\kalas{04:35 05:13 09:44 08:45 10:42 18:27 11:40 14:34 17:29 19:25 21:01 22:45 23:56 02:18(+1)}}}
{\tnykdata{\anga{\tithi{5}{शुक्ल-पञ्चमी}}{\time{27-18}{16:46}}\hspace{1ex}}%
{\anga{पुष्यः}{\time{1-23}{06:24}}\hspace{1ex}\anga{आश्रेषा}{\time{57-27}{04:50(+1)}}\hspace{1ex}}{चन्द्रराशिः—\mbox{कर्कटः\RIGHTarrow{04:50(+1)}}}%
{\anga{ध्रुवः}{\time{5-0}{07:51}}\hspace{1ex}\anga{व्याघातः}{\time{57-32}{04:52(+1)}}\hspace{1ex}\uanga{हर्षणः}}%
{\anga{बवः}{\time{0-15}{05:57}}\hspace{1ex}\anga{बालवः}{\time{27-18}{16:46}}\hspace{1ex}\anga{कौलवः}{\time{54-20}{03:35(+1)}}\hspace{1ex}\uanga{तैतिलः}}{}
}
{काञ्ची ५० जगद्गुरु श्री-चन्द्रचूडेन्द्र सरस्वती २ आराधना~\#{७२३}\eventsep काञ्ची ६ जगद्गुरु श्री-शुद्धानन्देन्द्र सरस्वती आराधना~\#{२१४३}\eventsep \tamil{ஸோமாஸிமார நாயனார் (32) குருபூஜை}\eventsep श्रीनिवासमङ्गापुरे साक्षात्कार-वैभवोत्सव-आरम्भः}
{Fri} 
\cfoot{\rygdata{11:18--13:07}{16:45--18:34}{07:40--09:29}}
\caldata{JUNE}{8}{\sunmonth{वृषभः}{25}{}{ज्यैष्ठः}{ग्रीष्मऋतुः}{शनिः}{विकारी}{उत्तरायणम्}{वसन्तऋतुः}}
{\sunmoonrsdata{05:51}{20:24}{11:23}{00:59(+1)}{13:07}
{\kalas{04:35 05:13 09:44 08:45 10:42 18:28 11:40 14:35 17:29 19:26 21:02 22:46 23:56 02:18(+1)}}}
{\tnykdata{\anga{\tithi{6}{शुक्ल-षष्ठी}}{\time{21-24}{14:25}}\hspace{1ex}}%
{\anga{मघा}{\time{53-35}{03:17(+1)}}\hspace{1ex}}{चन्द्रराशिः—\mbox{सिंहः}}%
{\anga{हर्षणः}{\time{50-8}{01:54(+1)}}\hspace{1ex}\uanga{वज्रम्}}%
{\anga{तैतिलः}{\time{21-24}{14:25}}\hspace{1ex}\anga{गरः}{\time{48-30}{01:15(+1)}}\hspace{1ex}\uanga{वणिजः}}{}
}
{आरण्य-गौरी-व्रतम्\eventsep षष्ठी-व्रतम्\eventsep काञ्ची २३ जगद्गुरु श्री-सच्चित्सुखेन्द्र सरस्वती आराधना~\#{१५०८}\eventsep विन्ध्यावासिनी-देवी-पूजा\eventsep श्रीनिवासमङ्गापुरे साक्षात्कार-वैभवोत्सवः}
{Sat} 
\cfoot{\rygdata{09:29--11:18}{14:56--16:46}{05:51--07:40}}
\caldata{JUNE}{9}{\sunmonth{वृषभः}{26}{}{ज्यैष्ठः}{ग्रीष्मऋतुः}{भानुः}{विकारी}{उत्तरायणम्}{वसन्तऋतुः}}
{\sunmoonrsdata{05:51}{20:25}{12:33}{01:32(+1)}{13:08}
{\kalas{04:35 05:13 09:44 08:45 10:42 18:28 11:40 14:35 17:30 19:26 21:02 22:46 23:57 02:18(+1)}}}
{\tnykdata{\anga{\tithi{7}{शुक्ल-सप्तमी}}{\time{15-38}{12:06}}\hspace{1ex}}%
{\anga{पूर्वफल्गुनी}{\time{49-56}{01:49(+1)}}\hspace{1ex}}{चन्द्रराशिः—\mbox{सिंहः}}%
{\anga{वज्रम्}{\time{42-54}{23:00}}\hspace{1ex}\uanga{सिद्धिः}}%
{\anga{वणिजः}{\time{15-38}{12:06}}\hspace{1ex}\anga{विष्टिः}{\time{42-50}{22:59}}\hspace{1ex}\uanga{बवः}}{}
}
{काञ्ची १६ जगद्गुरु श्री-उज्ज्वल शङ्करेन्द्र सरस्वती आराधना~\#{१६५३}\eventsep वरुण-पूजा\eventsep विजया-भानुसप्तमी\eventsep श्रीनिवासमङ्गापुरे साक्षात्कार-वैभवोत्सव-समापनम्}
{Sun} 
\cfoot{\rygdata{18:35--20:25}{13:08--14:57}{16:46--18:35}}
\caldata{JUNE}{10}{\sunmonth{वृषभः}{27}{}{ज्यैष्ठः}{ग्रीष्मऋतुः}{सोमः}{विकारी}{उत्तरायणम्}{वसन्तऋतुः}}
{\sunmoonrsdata{05:51}{20:25}{13:42}{02:03(+1)}{13:08}
{\kalas{04:35 05:13 09:44 08:45 10:42 18:28 11:40 14:35 17:30 19:27 21:03 22:46 23:57 02:18(+1)}}}
{\tnykdata{\anga{\tithi{8}{शुक्ल-अष्टमी}}{\time{10-6}{09:53}}\hspace{1ex}}%
{\anga{उत्तरफल्गुनी}{\time{46-35}{00:29(+1)}}\hspace{1ex}}{चन्द्रराशिः—\mbox{सिंहः\RIGHTarrow{07:28}}}%
{\anga{सिद्धिः}{\time{35-55}{20:13}}\hspace{1ex}\uanga{व्यतीपातः}}%
{\anga{बवः}{\time{10-6}{09:53}}\hspace{1ex}\anga{बालवः}{\time{37-28}{20:50}}\hspace{1ex}\uanga{कौलवः}}{}
}
{धूमावती-जयन्ती\eventsep ज्येष्ठाष्टमी\eventsep काञ्ची ६१ जगद्गुरु श्री-महादेवेन्द्र सरस्वती ४ आराधना~\#{२७४}}
{Mon} 
\cfoot{\rygdata{07:40--09:29}{11:18--13:08}{14:57--16:46}}
\caldata{JUNE}{11}{\sunmonth{वृषभः}{28}{}{ज्यैष्ठः}{ग्रीष्मऋतुः}{मङ्गलः}{विकारी}{उत्तरायणम्}{वसन्तऋतुः}}
{\sunmoonrsdata{05:50}{20:26}{14:50}{02:34(+1)}{13:08}
{\kalas{04:35 05:13 09:44 08:45 10:42 18:29 11:40 14:35 17:30 19:27 21:03 22:47 23:57 02:19(+1)}}}
{\tnykdata{\anga{\tithi{9}{शुक्ल-नवमी}}{\time{4-57}{07:49}}\hspace{1ex}}%
{\anga{हस्तः}{\time{43-41}{23:19}}\hspace{1ex}}{चन्द्रराशिः—\mbox{कन्या}}%
{\anga{व्यतीपातः}{\time{29-16}{17:33}}\hspace{1ex}\uanga{वरीयान्}}%
{\anga{कौलवः}{\time{4-57}{07:49}}\hspace{1ex}\anga{तैतिलः}{\time{32-32}{18:51}}\hspace{1ex}\uanga{गरः}}{}
}
{ब्रह्माणी-पूजा\eventsep देवी-पर्व-३\eventsep काञ्ची १७ जगद्गुरु श्री-सदाशिवेन्द्र सरस्वती आराधना~\#{१६४५}\eventsep काञ्ची ५३ जगद्गुरु श्री-पूर्णानन्द सदाशिवेन्द्र सरस्वती आराधना~\#{५२२}\eventsep महेश-नवमी\eventsep व्यतीपात-श्राद्धम्\eventsep शुक्ल-देवी-पूजा}
{Tue} 
\cfoot{\rygdata{16:47--18:36}{09:29--11:19}{13:08--14:57}}
\caldata{JUNE}{12}{\sunmonth{वृषभः}{29}{}{ज्यैष्ठः}{ग्रीष्मऋतुः}{बुधः}{विकारी}{उत्तरायणम्}{वसन्तऋतुः}}
{\sunmoonrsdata{05:50}{20:26}{15:58}{03:07(+1)}{13:08}
{\kalas{04:35 05:13 09:44 08:46 10:42 18:29 11:41 14:36 17:31 19:28 21:04 22:47 23:58 02:19(+1)}}}
{\tnykdata{\anga{\tithi{10}{शुक्ल-दशमी}}{\time{0-15}{05:57}}\hspace{1ex}\anga{\tithi{11}{शुक्ल-एकादशी}}{\time{56-11}{04:19(+1)}}\hspace{1ex}\avamA{}}%
{\anga{चित्रा}{\time{41-21}{22:23}}\hspace{1ex}}{चन्द्रराशिः—\mbox{कन्या\RIGHTarrow{10:49}}}%
{\anga{वरीयान्}{\time{23-4}{15:04}}\hspace{1ex}\uanga{परिघः}}%
{\anga{गरः}{\time{0-15}{05:57}}\hspace{1ex}\anga{वणिजः}{\time{28-8}{17:06}}\hspace{1ex}\anga{विष्टिः}{\time{56-11}{04:19(+1)}}\hspace{1ex}\uanga{बवः}}{}
}
{अलर्मेल्मङ्गापुरे प्लवोत्सव-प्रारम्भः\eventsep दशहरा/गङ्गावतरणम्/दशपापहरा-दशमी\eventsep काञ्ची १ जगद्गुरु श्री-आदि-शङ्कर भगवत्पाद आराधना~\#{२४९५}\eventsep स्मार्त-पाण्डव-निर्जला-एकादशी}
{Wed} 
\cfoot{\rygdata{13:08--14:58}{07:40--09:29}{11:19--13:08}}
\caldata{JUNE}{13}{\sunmonth{वृषभः}{30}{}{ज्यैष्ठः}{ग्रीष्मऋतुः}{गुरुः}{विकारी}{उत्तरायणम्}{वसन्तऋतुः}}
{\sunmoonrsdata{05:50}{20:26}{17:05}{03:43(+1)}{13:08}
{\kalas{04:35 05:13 09:44 08:46 10:42 18:30 11:41 14:36 17:31 19:28 21:04 22:47 23:58 02:19(+1)}}}
{\tnykdata{\anga{\tithi{12}{शुक्ल-द्वादशी}}{\time{52-53}{03:00(+1)}}\hspace{1ex}}%
{\anga{स्वाती}{\time{39-45}{21:44}}\hspace{1ex}}{चन्द्रराशिः—\mbox{तुला}}%
{\anga{परिघः}{\time{17-26}{12:49}}\hspace{1ex}\uanga{शिवः}}%
{\anga{बवः}{\time{24-26}{15:37}}\hspace{1ex}\anga{बालवः}{\time{52-53}{03:00(+1)}}\hspace{1ex}\uanga{कौलवः}}{}
}
{अलर्मेल्मङ्गापुरे प्लवोत्सवः\eventsep चम्पक-द्वादशी\eventsep गवामयन-द्वादशी\eventsep हरिवासरः\RIGHTarrow{}09:57\eventsep काञ्ची २ जगद्गुरु श्री-सुरेश्वराचार्य आराधना~\#{२४२५}\eventsep रामलक्ष्मण-द्वादशी\eventsep वैष्णव-पाण्डव-निर्जला-एकादशी}
{Thu} 
\cfoot{\rygdata{14:58--16:47}{05:50--07:40}{09:29--11:19}}
\caldata{JUNE}{14}{\sunmonth{वृषभः}{31}{\mbox{वृषभः{\tiny\RIGHTarrow}{04:48(+1)}}}{ज्यैष्ठः}{ग्रीष्मऋतुः}{शुक्रः}{विकारी}{उत्तरायणम्}{वसन्तऋतुः}}
{\sunmoonrsdata{05:50}{20:27}{18:12}{04:24(+1)}{13:09}
{\kalas{04:35 05:13 09:44 08:46 10:42 18:30 11:41 14:36 17:31 19:28 21:04 22:48 23:58 02:19(+1)}}}
{\tnykdata{\anga{\tithi{13}{शुक्ल-त्रयोदशी}}{\time{50-30}{02:03(+1)}}\hspace{1ex}}%
{\anga{विशाखा}{\time{39-1}{21:27}}\hspace{1ex}}{चन्द्रराशिः—\mbox{तुला\RIGHTarrow{15:29}}}%
{\anga{शिवः}{\time{12-27}{10:49}}\hspace{1ex}\uanga{सिद्धः}}%
{\anga{कौलवः}{\time{21-34}{14:28}}\hspace{1ex}\anga{तैतिलः}{\time{50-30}{02:03(+1)}}\hspace{1ex}\uanga{गरः}}{}
}
{अलर्मेल्मङ्गापुरे प्लवोत्सवः\eventsep छत्रपति-शिवाजी-राज्याभिषेकः~\#{३४६}\eventsep दुर्गन्ध-दौर्भाग्य-नाशक-त्रयोदशी\eventsep कृत्तिकावैषाखोत्सवः\eventsep मिथुन-रवि-सङ्क्रमण-षडशीति-पुण्यकालः~04:48(+1)\RIGHTarrow{}28:48(+1)\eventsep \tamil{நம்மாழ்வார் திருநக்ஷத்திரம்}\eventsep प्रदोष-व्रतम्~20:27\RIGHTarrow{}21:04\eventsep वेङ्कटाचले ज्येष्ठ-अभिद्येयकाभिषेकः (वज्र-कवचम्)\eventsep \tamil{வைகாசி~விஶாகம்}\eventsep विद्यारण्य-स्वामि-आराधना~\#{६२८}}
{Fri} 
\cfoot{\rygdata{11:19--13:09}{16:48--18:37}{07:40--09:29}}
\caldata{JUNE}{15}{\sunmonth{मिथुनम्}{1}{}{ज्यैष्ठः}{ग्रीष्मऋतुः}{शनिः}{विकारी}{उत्तरायणम्}{ग्रीष्मऋतुः}}
{\sunmoonrsdata{05:50}{20:27}{19:16}{05:09(+1)}{13:09}
{\kalas{04:35 05:13 09:44 08:46 10:43 18:30 11:41 14:36 17:32 19:29 21:05 22:48 23:58 02:19(+1)}}}
{\tnykdata{\anga{\tithi{14}{शुक्ल-चतुर्दशी}}{\time{49-13}{01:32(+1)}}\hspace{1ex}}%
{\anga{अनूराधा}{\time{39-20}{21:35}}\hspace{1ex}}{चन्द्रराशिः—\mbox{वृश्चिकः}}%
{\anga{सिद्धः}{\time{8-15}{09:09}}\hspace{1ex}\uanga{साध्यः}}%
{\anga{गरः}{\time{19-42}{13:44}}\hspace{1ex}\anga{वणिजः}{\time{49-13}{01:32(+1)}}\hspace{1ex}\uanga{विष्टिः}}{}
}
{अलर्मेल्मङ्गापुरे प्लवोत्सवः\eventsep वेङ्कटाचले ज्येष्ठ-अभिद्येयकाभिषेकः (मुत्यल-कवचम्)}
{Sat} 
\cfoot{\rygdata{09:30--11:19}{14:58--16:48}{05:50--07:40}}
\caldata{JUNE}{16}{\sunmonth{मिथुनम्}{2}{}{ज्यैष्ठः}{ग्रीष्मऋतुः}{भानुः}{विकारी}{उत्तरायणम्}{ग्रीष्मऋतुः}}
{\sunmoonrsdata{05:50}{20:28}{20:16}{---}{13:09}
{\kalas{04:35 05:13 09:44 08:46 10:43 18:31 11:41 14:37 17:32 19:29 21:05 22:48 23:59 02:19(+1)}}}
{\tnykdata{\anga{\tithi{15}{पौर्णमासी}}{\time{49-9}{01:30(+1)}}\hspace{1ex}}%
{\anga{ज्येष्ठा}{\time{40-51}{22:11}}\hspace{1ex}}{चन्द्रराशिः—\mbox{वृश्चिकः\RIGHTarrow{22:11}}}%
{\anga{साध्यः}{\time{4-59}{07:50}}\hspace{1ex}\uanga{शुभः}}%
{\anga{विष्टिः}{\time{19-1}{13:27}}\hspace{1ex}\anga{बवः}{\time{49-9}{01:30(+1)}}\hspace{1ex}\uanga{बालवः}}{}
}
{अलर्मेल्मङ्गापुरे प्लवोत्सव-समापनम्\eventsep ऎरुवक-पूर्णिमा\eventsep कबीरदास-जयन्ती\eventsep मन्वादिः-(भौत्यः-[१४])\eventsep पार्वणव्रतम् पूर्णिमायाम्\eventsep पूर्णिमा-व्रतम्\eventsep पञ्च-पर्व-पूजा (पूर्णिमा)\eventsep वेङ्कटाचले ज्येष्ठ-अभिद्येयकाभिषेकः (स्वर्ण-कवचम्)\eventsep वेङ्कटाचले पूर्णिमा-गरुड-सेवा\eventsep वट-पूर्णिमा/वट-सावित्री-व्रतम्}
{Sun} 
\cfoot{\rygdata{18:38--20:28}{13:09--14:59}{16:48--18:38}}
\caldata{JUNE}{17}{\sunmonth{मिथुनम्}{3}{}{ज्यैष्ठः}{ग्रीष्मऋतुः}{सोमः}{विकारी}{उत्तरायणम्}{ग्रीष्मऋतुः}}
{\sunmoonsrdata{05:51}{20:28}{21:09}{06:00}{13:09}
{\kalas{04:35 05:13 09:44 08:46 10:43 18:31 11:42 14:37 17:32 19:29 21:05 22:49 23:59 02:20(+1)}}}
{\tnykdata{\anga{\tithi{16}{कृष्ण-प्रथमा}}{\time{50-24}{02:00(+1)}}\hspace{1ex}}%
{\anga{मूला}{\time{43-39}{23:18}}\hspace{1ex}}{चन्द्रराशिः—\mbox{धनुः}}%
{\anga{शुभः}{\time{2-42}{06:56}}\hspace{1ex}\uanga{शुक्लः}}%
{\anga{बालवः}{\time{19-36}{13:41}}\hspace{1ex}\anga{कौलवः}{\time{50-24}{02:00(+1)}}\hspace{1ex}\uanga{तैतिलः}}{}
}
{पार्वण-प्रायश्चित्तावकाशः दर्शे\eventsep पूर्णमासेष्टिः\eventsep स्थालीपाकः}
{Mon} 
\cfoot{\rygdata{07:40--09:30}{11:19--13:09}{14:59--16:49}}
\caldata{JUNE}{18}{\sunmonth{मिथुनम्}{4}{}{ज्यैष्ठः}{ग्रीष्मऋतुः}{मङ्गलः}{विकारी}{उत्तरायणम्}{ग्रीष्मऋतुः}}
{\sunmoonsrdata{05:51}{20:28}{21:56}{06:55}{13:09}
{\kalas{04:36 05:13 09:45 08:46 10:43 18:31 11:42 14:37 17:33 19:30 21:06 22:49 23:59 02:20(+1)}}}
{\tnykdata{\anga{\tithi{17}{कृष्ण-द्वितीया}}{\time{53-2}{03:04(+1)}}\hspace{1ex}}%
{\anga{पूर्वाषाढा}{\time{47-46}{00:57(+1)}}\hspace{1ex}}{चन्द्रराशिः—\mbox{धनुः}}%
{\anga{शुक्लः}{\time{1-29}{06:26}}\hspace{1ex}\uanga{ब्रह्म}}%
{\anga{तैतिलः}{\time{21-33}{14:28}}\hspace{1ex}\anga{गरः}{\time{53-2}{03:04(+1)}}\hspace{1ex}\uanga{वणिजः}}{}
}
{}
{Tue} 
\cfoot{\rygdata{16:49--18:38}{09:30--11:20}{13:09--14:59}}
\caldata{JUNE}{19}{\sunmonth{मिथुनम्}{5}{}{ज्यैष्ठः}{ग्रीष्मऋतुः}{बुधः}{विकारी}{उत्तरायणम्}{ग्रीष्मऋतुः}}
{\sunmoonsrdata{05:51}{20:28}{22:37}{07:53}{13:10}
{\kalas{04:36 05:13 09:45 08:46 10:43 18:31 11:42 14:37 17:33 19:30 21:06 22:49 23:59 02:20(+1)}}}
{\tnykdata{\anga{\tithi{18}{कृष्ण-तृतीया}}{\time{56-57}{04:38(+1)}}\hspace{1ex}}%
{\anga{उत्तराषाढा}{\time{53-10}{03:07(+1)}}\hspace{1ex}}{चन्द्रराशिः—\mbox{धनुः\RIGHTarrow{07:27}}}%
{\anga{ब्रह्म}{\time{1-19}{06:23}}\hspace{1ex}\uanga{इन्द्रः}}%
{\anga{वणिजः}{\time{24-50}{15:47}}\hspace{1ex}\anga{विष्टिः}{\time{56-57}{04:38(+1)}}\hspace{1ex}\uanga{बवः}}{}
}
{}
{Wed} 
\cfoot{\rygdata{13:10--14:59}{07:40--09:30}{11:20--13:10}}
\caldata{JUNE}{20}{\sunmonth{मिथुनम्}{6}{}{ज्यैष्ठः}{ग्रीष्मऋतुः}{गुरुः}{विकारी}{उत्तरायणम्}{ग्रीष्मऋतुः}}
{\sunmoonsrdata{05:51}{20:29}{23:12}{08:51}{13:10}
{\kalas{04:36 05:13 09:45 08:47 10:43 18:32 11:42 14:38 17:33 19:30 21:06 22:49 00:00(+1) 02:20(+1)}}}
{\tnykdata{\fulltithi{\tithi{19}{कृष्ण-चतुर्थी}}}%
{\anga{श्रवणः}{\time{59-38}{05:42(+1)}}\hspace{1ex}}{चन्द्रराशिः—\mbox{मकरः}}%
{\anga{इन्द्रः}{\time{2-10}{06:43}}\hspace{1ex}\uanga{वैधृतिः}}%
{\anga{बवः}{\time{29-20}{17:35}}\hspace{1ex}\uanga{बालवः}}{}
}
{दक्षिणायन-पुण्यकालः~20:54\RIGHTarrow{}08:54(+1)\eventsep कृष्णपिङ्गल-महागणपति सङ्कटहर-चतुर्थी-व्रतम्\eventsep वैधृति-श्राद्धम्\eventsep श्रवण-व्रतम्}
{Thu} 
\cfoot{\rygdata{15:00--16:49}{05:51--07:41}{09:30--11:20}}
\caldata{JUNE}{21}{\sunmonth{मिथुनम्}{7}{}{ज्यैष्ठः}{ग्रीष्मऋतुः}{शुक्रः}{विकारी}{उत्तरायणम्}{ग्रीष्मऋतुः}}
{\sunmoonsrdata{05:51}{20:29}{23:42}{09:49}{13:10}
{\kalas{04:36 05:14 09:45 08:47 10:44 18:32 11:42 14:38 17:33 19:30 21:06 22:50 00:00(+1) 02:20(+1)}}}
{\tnykdata{\anga{\tithi{19}{कृष्ण-चतुर्थी}}{\time{1-58}{06:38}}\hspace{1ex}}%
{\fullanga{श्रविष्ठा}}{चन्द्रराशिः—\mbox{मकरः\RIGHTarrow{19:07}}}%
{\anga{वैधृतिः}{\time{3-50}{07:23}}\hspace{1ex}\uanga{विष्कम्भः}}%
{\anga{बालवः}{\time{1-58}{06:38}}\hspace{1ex}\anga{कौलवः}{\time{34-46}{19:46}}\hspace{1ex}\uanga{तैतिलः}}{}
}
{दक्षिणायनारम्भः\eventsep शुक्र-मासः/उत्तरायणम्\RIGHTarrow{}08:54}
{Fri} 
\cfoot{\rygdata{11:20--13:10}{16:49--18:39}{07:41--09:31}}
\caldata{JUNE}{22}{\sunmonth{मिथुनम्}{8}{}{ज्यैष्ठः}{ग्रीष्मऋतुः}{शनिः}{विकारी}{उत्तरायणम्}{ग्रीष्मऋतुः}}
{\sunmoonsrdata{05:51}{20:29}{00:10(+1)}{10:47}{13:10}
{\kalas{04:36 05:14 09:45 08:47 10:44 18:32 11:43 14:38 17:33 19:31 21:07 22:50 00:00(+1) 02:21(+1)}}}
{\tnykdata{\anga{\tithi{20}{कृष्ण-पञ्चमी}}{\time{7-43}{08:57}}\hspace{1ex}}%
{\anga{श्रविष्ठा}{\time{6-49}{08:35}}\hspace{1ex}}{चन्द्रराशिः—\mbox{कुम्भः}}%
{\anga{विष्कम्भः}{\time{6-3}{08:17}}\hspace{1ex}\uanga{प्रीतिः}}%
{\anga{तैतिलः}{\time{7-43}{08:57}}\hspace{1ex}\anga{गरः}{\time{40-45}{22:10}}\hspace{1ex}\uanga{वणिजः}}{}
}
{}
{Sat} 
\cfoot{\rygdata{09:31--11:20}{15:00--16:50}{05:51--07:41}}
\caldata{JUNE}{23}{\sunmonth{मिथुनम्}{9}{}{ज्यैष्ठः}{ग्रीष्मऋतुः}{भानुः}{विकारी}{उत्तरायणम्}{ग्रीष्मऋतुः}}
{\sunmoonsrdata{05:52}{20:29}{00:36(+1)}{11:44}{13:10}
{\kalas{04:37 05:14 09:46 08:47 10:44 18:32 11:43 14:38 17:34 19:31 21:07 22:50 00:00(+1) 02:21(+1)}}}
{\tnykdata{\anga{\tithi{21}{कृष्ण-षष्ठी}}{\time{13-47}{11:22}}\hspace{1ex}}%
{\anga{शतभिषक्}{\time{14-18}{11:35}}\hspace{1ex}}{चन्द्रराशिः—\mbox{कुम्भः}}%
{\anga{प्रीतिः}{\time{8-31}{09:16}}\hspace{1ex}\uanga{आयुष्मान्}}%
{\anga{वणिजः}{\time{13-47}{11:22}}\hspace{1ex}\anga{विष्टिः}{\time{46-45}{00:34(+1)}}\hspace{1ex}\uanga{बवः}}{}
}
{}
{Sun} 
\cfoot{\rygdata{18:40--20:29}{13:10--15:00}{16:50--18:40}}
\caldata{JUNE}{24}{\sunmonth{मिथुनम्}{10}{}{ज्यैष्ठः}{ग्रीष्मऋतुः}{सोमः}{विकारी}{उत्तरायणम्}{ग्रीष्मऋतुः}}
{\sunmoonsrdata{05:52}{20:29}{01:02(+1)}{12:41}{13:11}
{\kalas{04:37 05:14 09:46 08:47 10:44 18:32 11:43 14:38 17:34 19:31 21:07 22:50 00:00(+1) 02:21(+1)}}}
{\tnykdata{\anga{\tithi{22}{कृष्ण-सप्तमी}}{\time{19-36}{13:42}}\hspace{1ex}}%
{\anga{पूर्वप्रोष्ठपदा}{\time{21-33}{14:29}}\hspace{1ex}}{चन्द्रराशिः—\mbox{कुम्भः\RIGHTarrow{07:47}}}%
{\anga{आयुष्मान्}{\time{10-49}{10:12}}\hspace{1ex}\uanga{सौभाग्यः}}%
{\anga{बवः}{\time{19-36}{13:42}}\hspace{1ex}\anga{बालवः}{\time{52-14}{02:46(+1)}}\hspace{1ex}\uanga{कौलवः}}{}
}
{पञ्च-पर्व-पूजा (अष्टमी)}
{Mon} 
\cfoot{\rygdata{07:42--09:31}{11:21--13:11}{15:00--16:50}}
\caldata{JUNE}{25}{\sunmonth{मिथुनम्}{11}{}{ज्यैष्ठः}{ग्रीष्मऋतुः}{मङ्गलः}{विकारी}{उत्तरायणम्}{ग्रीष्मऋतुः}}
{\sunmoonsrdata{05:52}{20:29}{01:28(+1)}{13:38}{13:11}
{\kalas{04:37 05:15 09:46 08:48 10:45 18:32 11:43 14:38 17:34 19:31 21:07 22:50 00:01(+1) 02:21(+1)}}}
{\tnykdata{\anga{\tithi{23}{कृष्ण-अष्टमी}}{\time{24-37}{15:43}}\hspace{1ex}}%
{\anga{उत्तरप्रोष्ठपदा}{\time{28-2}{17:05}}\hspace{1ex}}{चन्द्रराशिः—\mbox{मीनः}}%
{\anga{सौभाग्यः}{\time{12-35}{10:54}}\hspace{1ex}\uanga{शोभनः}}%
{\anga{कौलवः}{\time{24-37}{15:43}}\hspace{1ex}\anga{तैतिलः}{\time{56-41}{04:33(+1)}}\hspace{1ex}\uanga{गरः}}{}
}
{तिन्दुकाष्टमी\eventsep त्रिलोचनाष्टमी\eventsep विनायकाष्टमी\eventsep शीतलाष्टमी}
{Tue} 
\cfoot{\rygdata{16:50--18:40}{09:31--11:21}{13:11--15:01}}
\caldata{JUNE}{26}{\sunmonth{मिथुनम्}{12}{}{ज्यैष्ठः}{ग्रीष्मऋतुः}{बुधः}{विकारी}{उत्तरायणम्}{ग्रीष्मऋतुः}}
{\sunmoonsrdata{05:53}{20:30}{01:56(+1)}{14:36}{13:11}
{\kalas{04:37 05:15 09:46 08:48 10:45 18:33 11:43 14:39 17:34 19:31 21:07 22:50 00:01(+1) 02:22(+1)}}}
{\tnykdata{\anga{\tithi{24}{कृष्ण-नवमी}}{\time{28-23}{17:14}}\hspace{1ex}}%
{\anga{रेवती}{\time{33-16}{19:11}}\hspace{1ex}}{चन्द्रराशिः—\mbox{मीनः\RIGHTarrow{19:11}}}%
{\anga{शोभनः}{\time{13-26}{11:15}}\hspace{1ex}\uanga{अतिगण्डः}}%
{\anga{गरः}{\time{28-23}{17:14}}\hspace{1ex}\anga{वणिजः}{\time{59-41}{05:45(+1)}}\hspace{1ex}\uanga{विष्टिः}}{}
}
{\tamil{ஏயர்கோன் கலிக்காம நாயனார் (28) குருபூஜை}\eventsep दुर्गा-स्वापनम्}
{Wed} 
\cfoot{\rygdata{13:11--15:01}{07:42--09:32}{11:21--13:11}}
\caldata{JUNE}{27}{\sunmonth{मिथुनम्}{13}{}{ज्यैष्ठः}{ग्रीष्मऋतुः}{गुरुः}{विकारी}{उत्तरायणम्}{ग्रीष्मऋतुः}}
{\sunmoonsrdata{05:53}{20:30}{02:28(+1)}{15:37}{13:11}
{\kalas{04:38 05:15 09:47 08:48 10:45 18:33 11:44 14:39 17:34 19:31 21:07 22:51 00:01(+1) 02:22(+1)}}}
{\tnykdata{\anga{\tithi{25}{कृष्ण-दशमी}}{\time{30-32}{18:06}}\hspace{1ex}}%
{\anga{अश्विनी}{\time{36-56}{20:39}}\hspace{1ex}}{चन्द्रराशिः—\mbox{मेषः}}%
{\anga{अतिगण्डः}{\time{13-5}{11:07}}\hspace{1ex}\uanga{सुकर्म}}%
{\anga{विष्टिः}{\time{30-32}{18:06}}\hspace{1ex}\uanga{बवः}}{}
}
{}
{Thu} 
\cfoot{\rygdata{15:01--16:50}{05:53--07:42}{09:32--11:22}}
\caldata{JUNE}{28}{\sunmonth{मिथुनम्}{14}{}{ज्यैष्ठः}{ग्रीष्मऋतुः}{शुक्रः}{विकारी}{उत्तरायणम्}{ग्रीष्मऋतुः}}
{\sunmoonsrdata{05:53}{20:30}{03:04(+1)}{16:40}{13:11}
{\kalas{04:38 05:16 09:47 08:49 10:45 18:33 11:44 14:39 17:34 19:31 21:07 22:51 00:01(+1) 02:22(+1)}}}
{\tnykdata{\anga{\tithi{26}{कृष्ण-एकादशी}}{\time{30-54}{18:15}}\hspace{1ex}}%
{\anga{अपभरणी}{\time{38-50}{21:25}}\hspace{1ex}}{चन्द्रराशिः—\mbox{मेषः\RIGHTarrow{03:30(+1)}}}%
{\anga{सुकर्म}{\time{11-22}{10:26}}\hspace{1ex}\uanga{धृतिः}}%
{\anga{बवः}{\time{0-56}{06:16}}\hspace{1ex}\anga{बालवः}{\time{30-54}{18:15}}\hspace{1ex}\uanga{कौलवः}}{}
}
{चिदम्बरे ध्वजारोहणम्/पञ्चमूर्ति रथोत्सवः\eventsep कूर्म-जयन्ती\eventsep सर्व-योगिनी-एकादशी}
{Fri} 
\cfoot{\rygdata{11:22--13:11}{16:51--18:40}{07:43--09:32}}
\caldata{JUNE}{29}{\sunmonth{मिथुनम्}{15}{}{ज्यैष्ठः}{ग्रीष्मऋतुः}{शनिः}{विकारी}{उत्तरायणम्}{ग्रीष्मऋतुः}}
{\sunmoonsrdata{05:54}{20:30}{03:47(+1)}{17:45}{13:12}
{\kalas{04:38 05:16 09:47 08:49 10:46 18:33 11:44 14:39 17:34 19:31 21:07 22:51 00:01(+1) 02:22(+1)}}}
{\tnykdata{\anga{\tithi{27}{कृष्ण-द्वादशी}}{\time{29-28}{17:41}}\hspace{1ex}}%
{\anga{कृत्तिका}{\time{38-57}{21:29}}\hspace{1ex}}{चन्द्रराशिः—\mbox{वृषभः}}%
{\anga{धृतिः}{\time{8-12}{09:11}}\hspace{1ex}\uanga{शूलः}}%
{\anga{कौलवः}{\time{0-24}{06:03}}\hspace{1ex}\anga{तैतिलः}{\time{29-28}{17:41}}\hspace{1ex}\anga{गरः}{\time{58-6}{05:08(+1)}}\hspace{1ex}\uanga{वणिजः}}{}
}
{चिदम्बरे रजत चन्द्रप्रभ वाहनम्\eventsep हरिवासरः\RIGHTarrow{}00:11\eventsep कृत्तिका-व्रतम्\eventsep शनि-प्रदोष-व्रतम्~20:30\RIGHTarrow{}21:07}
{Sat} 
\cfoot{\rygdata{09:33--11:22}{15:01--16:51}{05:54--07:43}}
\caldata{JUNE}{30}{\sunmonth{मिथुनम्}{16}{}{ज्यैष्ठः}{ग्रीष्मऋतुः}{भानुः}{विकारी}{उत्तरायणम्}{ग्रीष्मऋतुः}}
{\sunmoonsrdata{05:54}{20:30}{04:38(+1)}{18:49}{13:12}
{\kalas{04:39 05:16 09:48 08:49 10:46 18:33 11:44 14:39 17:34 19:31 21:07 22:51 00:01(+1) 02:23(+1)}}}
{\tnykdata{\anga{\tithi{28}{कृष्ण-त्रयोदशी}}{\time{26-19}{16:26}}\hspace{1ex}}%
{\anga{रोहिणी}{\time{37-26}{20:53}}\hspace{1ex}}{चन्द्रराशिः—\mbox{वृषभः}}%
{\anga{शूलः}{\time{3-38}{07:22}}\hspace{1ex}\anga{गण्डः}{\time{57-48}{05:02(+1)}}\hspace{1ex}\uanga{वृद्धिः}}%
{\anga{वणिजः}{\time{26-19}{16:26}}\hspace{1ex}\anga{विष्टिः}{\time{54-11}{03:35(+1)}}\hspace{1ex}\uanga{शकुनिः}}{}
}
{चिदम्बरे स्वर्ण-सूर्यप्रभ वाहनम्\eventsep मासशिवरात्रिः\eventsep पञ्च-पर्व-पूजा (चतुर्दशी)}
{Sun} 
\cfoot{\rygdata{18:40--20:30}{13:12--15:01}{16:51--18:40}}
\caldata{JULY}{1}{\sunmonth{मिथुनम्}{17}{}{ज्यैष्ठः}{ग्रीष्मऋतुः}{सोमः}{विकारी}{उत्तरायणम्}{ग्रीष्मऋतुः}}
{\sunmoonsrdata{05:55}{20:29}{05:38(+1)}{19:51}{13:12}
{\kalas{04:39 05:17 09:48 08:50 10:46 18:33 11:45 14:39 17:34 19:31 21:07 22:51 00:02(+1) 02:23(+1)}}}
{\tnykdata{\anga{\tithi{29}{कृष्ण-चतुर्दशी}}{\time{21-42}{14:35}}\hspace{1ex}}%
{\anga{मृगशीर्षम्}{\time{34-28}{19:42}}\hspace{1ex}}{चन्द्रराशिः—\mbox{वृषभः\RIGHTarrow{08:21}}}%
{\anga{वृद्धिः}{\time{50-51}{02:15(+1)}}\hspace{1ex}\uanga{ध्रुवः}}%
{\anga{शकुनिः}{\time{21-42}{14:35}}\hspace{1ex}\anga{चतुष्पात्}{\time{48-55}{01:29(+1)}}\hspace{1ex}\uanga{नाग}}{}
}
{चिदम्बरे रजत भूत वाहनम्\eventsep पार्वणव्रतम् अमावास्यायाम्\eventsep पञ्च-पर्व-पूजा (अमावास्या)\eventsep सोममृगशीर्ष-पुण्यकालः\RIGHTarrow{}19:42\eventsep सर्व-ज्यैष्ठ-अमावास्या (अलभ्यम्–पुष्कला)}
{Mon} 
\cfoot{\rygdata{07:44--09:33}{11:23--13:12}{15:01--16:51}}
\caldata{JULY}{2}{\sunmonth{मिथुनम्}{18}{}{ज्यैष्ठः}{ग्रीष्मऋतुः}{मङ्गलः}{विकारी}{उत्तरायणम्}{ग्रीष्मऋतुः}}
{\sunmoonsrdata{05:55}{20:29}{---}{20:48}{13:12}
{\kalas{04:40 05:17 09:48 08:50 10:46 18:33 11:45 14:40 17:34 19:31 21:07 22:51 00:02(+1) 02:23(+1)}}}
{\tnykdata{\anga{\tithi{30}{अमावास्या}}{\time{15-51}{12:16}}\hspace{1ex}}%
{\anga{आर्द्रा}{\time{30-22}{18:04}}\hspace{1ex}}{चन्द्रराशिः—\mbox{मिथुनम्}}%
{\anga{ध्रुवः}{\time{43-3}{23:08}}\hspace{1ex}\uanga{व्याघातः}}%
{\anga{नाग}{\time{15-51}{12:16}}\hspace{1ex}\anga{किंस्तुघ्नः}{\time{42-35}{22:57}}\hspace{1ex}\uanga{बवः}}{}
}
{भोगशायि-पूजा\eventsep चिदम्बरे रजत ऋषभ वाहनम्\eventsep दर्शेष्टिः\eventsep काञ्ची २५ जगद्गुरु श्री-सच्चिदानन्दघनेन्द्र सरस्वती आराधना~\#{१४७२}\eventsep पार्वण-प्रायश्चित्तावकाशः पौर्णमास्याम्\eventsep स्थालीपाकः}
{Tue} 
\cfoot{\rygdata{16:51--18:40}{09:34--11:23}{13:12--15:02}}
\caldata{JULY}{3}{\sunmonth{मिथुनम्}{19}{}{आषाढः}{ग्रीष्मऋतुः}{बुधः}{विकारी}{उत्तरायणम्}{ग्रीष्मऋतुः}}
{\sunmoonrsdata{05:55}{20:29}{06:45}{21:38}{13:12}
{\kalas{04:40 05:18 09:48 08:50 10:47 18:33 11:45 14:40 17:34 19:31 21:07 22:51 00:02(+1) 02:24(+1)}}}
{\tnykdata{\anga{\tithi{1}{शुक्ल-प्रथमा}}{\time{9-7}{09:34}}\hspace{1ex}}%
{\anga{पुनर्वसुः}{\time{25-28}{16:07}}\hspace{1ex}}{चन्द्रराशिः—\mbox{मिथुनम्\RIGHTarrow{10:37}}}%
{\anga{व्याघातः}{\time{34-38}{19:47}}\hspace{1ex}\uanga{हर्षणः}}%
{\anga{बवः}{\time{9-7}{09:34}}\hspace{1ex}\anga{बालवः}{\time{35-31}{20:08}}\hspace{1ex}\uanga{कौलवः}}{}
}
{चन्द्र-दर्शनम्~20:29\RIGHTarrow{}21:07\eventsep चिदम्बरे रजत-गजवाहनम्\eventsep वाराही-नवरात्र-आरम्भः}
{Wed} 
\cfoot{\rygdata{13:12--15:02}{07:45--09:34}{11:23--13:12}}
\caldata{JULY}{4}{\sunmonth{मिथुनम्}{20}{}{आषाढः}{ग्रीष्मऋतुः}{गुरुः}{विकारी}{उत्तरायणम्}{ग्रीष्मऋतुः}}
{\sunmoonrsdata{05:56}{20:29}{07:57}{22:22}{13:13}
{\kalas{04:40 05:18 09:49 08:51 10:47 18:33 11:45 14:40 17:34 19:31 21:07 22:51 00:02(+1) 02:24(+1)}}}
{\tnykdata{\anga{\tithi{2}{शुक्ल-द्वितीया}}{\time{1-48}{06:40}}\hspace{1ex}\anga{\tithi{3}{शुक्ल-तृतीया}}{\time{54-16}{03:39(+1)}}\hspace{1ex}\avamA{}}%
{\anga{पुष्यः}{\time{20-4}{13:58}}\hspace{1ex}}{चन्द्रराशिः—\mbox{कर्कटः}}%
{\anga{हर्षणः}{\time{25-52}{16:17}}\hspace{1ex}\uanga{वज्रम्}}%
{\anga{कौलवः}{\time{1-48}{06:40}}\hspace{1ex}\anga{तैतिलः}{\time{28-3}{17:09}}\hspace{1ex}\anga{गरः}{\time{54-16}{03:39(+1)}}\hspace{1ex}\uanga{वणिजः}}{}
}
{अमृतलक्ष्मी-व्रतम्\eventsep चिदम्बरे कैलास वाहनम्\eventsep गुरुपुष्य-पुण्यकालः\RIGHTarrow{}13:58\eventsep जगन्नाथ-रथ-यात्रा}
{Thu} 
\cfoot{\rygdata{15:02--16:51}{05:56--07:45}{09:34--11:23}}
\caldata{JULY}{5}{\sunmonth{मिथुनम्}{21}{}{आषाढः}{ग्रीष्मऋतुः}{शुक्रः}{विकारी}{उत्तरायणम्}{ग्रीष्मऋतुः}}
{\sunmoonrsdata{05:57}{20:29}{09:09}{23:00}{13:13}
{\kalas{04:41 05:19 09:49 08:51 10:47 18:32 11:46 14:40 17:34 19:31 21:07 22:51 00:02(+1) 02:24(+1)}}}
{\tnykdata{\anga{\tithi{4}{शुक्ल-चतुर्थी}}{\time{46-46}{00:39(+1)}}\hspace{1ex}}%
{\anga{आश्रेषा}{\time{14-33}{11:46}}\hspace{1ex}}{चन्द्रराशिः—\mbox{कर्कटः\RIGHTarrow{11:46}}}%
{\anga{वज्रम्}{\time{17-1}{12:45}}\hspace{1ex}\uanga{सिद्धिः}}%
{\anga{वणिजः}{\time{20-29}{14:09}}\hspace{1ex}\anga{विष्टिः}{\time{46-46}{00:39(+1)}}\hspace{1ex}\uanga{बवः}}{}
}
{चिदम्बरे भिक्षाटन स्वर्णरथः\eventsep \tamil{புகழ்த்துணை நாயனார் (54) குருபூஜை}}
{Fri} 
\cfoot{\rygdata{11:24--13:13}{16:51--18:40}{07:46--09:35}}
\caldata{JULY}{6}{\sunmonth{मिथुनम्}{22}{}{आषाढः}{ग्रीष्मऋतुः}{शनिः}{विकारी}{उत्तरायणम्}{ग्रीष्मऋतुः}}
{\sunmoonrsdata{05:57}{20:29}{10:22}{23:34}{13:13}
{\kalas{04:41 05:19 09:49 08:51 10:48 18:32 11:46 14:40 17:34 19:31 21:07 22:51 00:02(+1) 02:24(+1)}}}
{\tnykdata{\anga{\tithi{5}{शुक्ल-पञ्चमी}}{\time{39-38}{21:48}}\hspace{1ex}}%
{\anga{मघा}{\time{9-12}{09:38}}\hspace{1ex}}{चन्द्रराशिः—\mbox{सिंहः}}%
{\anga{सिद्धिः}{\time{8-20}{09:17}}\hspace{1ex}\uanga{व्यतीपातः}}%
{\anga{बवः}{\time{13-8}{11:13}}\hspace{1ex}\anga{बालवः}{\time{39-38}{21:48}}\hspace{1ex}\uanga{कौलवः}}{}
}
{\tamil{அமரநீதி நாயனார் (6) குருபூஜை}\eventsep चिदम्बरे रथोत्सवः\eventsep \tamil{மாணிக்கவாசகர் குருபூஜை}\eventsep स्कन्द-पञ्चमी\eventsep व्यतीपात-श्राद्धम्\eventsep शमी-गौरी-व्रतम्}
{Sat} 
\cfoot{\rygdata{09:35--11:24}{15:02--16:51}{05:57--07:46}}
\caldata{JULY}{7}{\sunmonth{मिथुनम्}{23}{}{आषाढः}{ग्रीष्मऋतुः}{भानुः}{विकारी}{उत्तरायणम्}{ग्रीष्मऋतुः}}
{\sunmoonrsdata{05:58}{20:28}{11:33}{00:06(+1)}{13:13}
{\kalas{04:42 05:20 09:50 08:52 10:48 18:32 11:46 14:40 17:34 19:30 21:06 22:51 00:02(+1) 02:25(+1)}}}
{\tnykdata{\anga{\tithi{6}{शुक्ल-षष्ठी}}{\time{33-5}{19:12}}\hspace{1ex}}%
{\anga{पूर्वफल्गुनी}{\time{4-19}{07:41}}\hspace{1ex}}{चन्द्रराशिः—\mbox{सिंहः\RIGHTarrow{13:15}}}%
{\anga{व्यतीपातः}{\time{0-1}{05:58}}\hspace{1ex}\anga{वरीयान्}{\time{52-18}{02:53(+1)}}\hspace{1ex}\uanga{परिघः}}%
{\anga{कौलवः}{\time{6-15}{08:28}}\hspace{1ex}\anga{तैतिलः}{\time{33-5}{19:12}}\hspace{1ex}\uanga{गरः}}{}
}
{चिदम्बरे नटराजस्य राजसभायां महाभिषेकः\eventsep काञ्ची ३५ जगद्गुरु श्री-चित्सुखेन्द्र सरस्वती आराधना~\#{१२८३}\eventsep कुमार-षष्ठी-व्रतम्\eventsep \tamil{நடராஜர் ஆனி திருமஞ்சனம்}\eventsep पद्मक-योगः}
{Sun} 
\cfoot{\rygdata{18:40--20:28}{13:13--15:02}{16:51--18:40}}
\caldata{JULY}{8}{\sunmonth{मिथुनम्}{24}{}{आषाढः}{ग्रीष्मऋतुः}{सोमः}{विकारी}{उत्तरायणम्}{ग्रीष्मऋतुः}}
{\sunmoonrsdata{05:58}{20:28}{12:42}{00:37(+1)}{13:13}
{\kalas{04:42 05:20 09:50 08:52 10:48 18:32 11:46 14:40 17:34 19:30 21:06 22:51 00:02(+1) 02:25(+1)}}}
{\tnykdata{\anga{\tithi{7}{शुक्ल-सप्तमी}}{\time{27-20}{16:54}}\hspace{1ex}}%
{\anga{उत्तरफल्गुनी}{\time{0-8}{06:02}}\hspace{1ex}\anga{हस्तः}{\time{56-52}{04:43(+1)}}\hspace{1ex}}{चन्द्रराशिः—\mbox{कन्या}}%
{\anga{परिघः}{\time{45-16}{00:05(+1)}}\hspace{1ex}\uanga{शिवः}}%
{\anga{गरः}{\time{0-5}{06:01}}\hspace{1ex}\anga{वणिजः}{\time{27-20}{16:54}}\hspace{1ex}\anga{विष्टिः}{\time{54-50}{03:54(+1)}}\hspace{1ex}\uanga{बवः}}{}
}
{चिदम्बरे मुत्तुप्पल्लक्कु\eventsep वैवस्वत-सप्तमी}
{Mon} 
\cfoot{\rygdata{07:47--09:36}{11:25--13:13}{15:02--16:51}}
\caldata{JULY}{9}{\sunmonth{मिथुनम्}{25}{}{आषाढः}{ग्रीष्मऋतुः}{मङ्गलः}{विकारी}{उत्तरायणम्}{ग्रीष्मऋतुः}}
{\sunmoonrsdata{05:59}{20:28}{13:50}{01:10(+1)}{13:13}
{\kalas{04:43 05:21 09:51 08:53 10:49 18:32 11:46 14:40 17:34 19:30 21:06 22:51 00:02(+1) 02:25(+1)}}}
{\tnykdata{\anga{\tithi{8}{शुक्ल-अष्टमी}}{\time{22-33}{15:00}}\hspace{1ex}}%
{\anga{चित्रा}{\time{54-36}{03:49(+1)}}\hspace{1ex}}{चन्द्रराशिः—\mbox{कन्या\RIGHTarrow{16:13}}}%
{\anga{शिवः}{\time{39-3}{21:37}}\hspace{1ex}\uanga{सिद्धः}}%
{\anga{बवः}{\time{22-33}{15:00}}\hspace{1ex}\anga{बालवः}{\time{50-34}{02:13(+1)}}\hspace{1ex}\uanga{कौलवः}}{}
}
{काञ्ची १२ जगद्गुरु श्री-चन्द्रशेखरेन्द्र सरस्वती आराधना~\#{१७८५}\eventsep महिषघ्नी-पूजा\eventsep सुदर्शन-जयन्ती}
{Tue} 
\cfoot{\rygdata{16:51--18:39}{09:36--11:25}{13:13--15:02}}
\caldata{JULY}{10}{\sunmonth{मिथुनम्}{26}{}{आषाढः}{ग्रीष्मऋतुः}{बुधः}{विकारी}{उत्तरायणम्}{ग्रीष्मऋतुः}}
{\sunmoonrsdata{06:00}{20:27}{14:57}{01:44(+1)}{13:13}
{\kalas{04:43 05:21 09:51 08:53 10:49 18:32 11:47 14:40 17:34 19:30 21:06 22:51 00:02(+1) 02:25(+1)}}}
{\tnykdata{\anga{\tithi{9}{शुक्ल-नवमी}}{\time{18-51}{13:32}}\hspace{1ex}}%
{\anga{स्वाती}{\time{53-28}{03:23(+1)}}\hspace{1ex}}{चन्द्रराशिः—\mbox{तुला}}%
{\anga{सिद्धः}{\time{33-46}{19:30}}\hspace{1ex}\uanga{साध्यः}}%
{\anga{कौलवः}{\time{18-51}{13:32}}\hspace{1ex}\anga{तैतिलः}{\time{47-27}{00:59(+1)}}\hspace{1ex}\uanga{गरः}}{}
}
{ऐन्द्री-दुर्गा-पूजा\eventsep काञ्ची ४८ जगद्गुरु श्री-अद्वैतानन्दबोधेन्द्र सरस्वती आराधना~\#{८२०}\eventsep मन्वादिः-(वैवस्वतः-[७])\eventsep \tamil{பெரியாழ்வார் திருநக்ஷத்திரம்}\eventsep उपेन्द्र-नवमी\eventsep वाराही-नवरात्र-समापनम्}
{Wed} 
\cfoot{\rygdata{13:14--15:02}{07:48--09:37}{11:25--13:14}}
\caldata{JULY}{11}{\sunmonth{मिथुनम्}{27}{}{आषाढः}{ग्रीष्मऋतुः}{गुरुः}{विकारी}{उत्तरायणम्}{ग्रीष्मऋतुः}}
{\sunmoonrsdata{06:00}{20:27}{16:04}{02:23(+1)}{13:14}
{\kalas{04:44 05:22 09:51 08:53 10:49 18:31 11:47 14:40 17:34 19:29 21:05 22:51 00:02(+1) 02:26(+1)}}}
{\tnykdata{\anga{\tithi{10}{शुक्ल-दशमी}}{\time{16-19}{12:32}}\hspace{1ex}}%
{\anga{विशाखा}{\time{53-31}{03:25(+1)}}\hspace{1ex}}{चन्द्रराशिः—\mbox{तुला\RIGHTarrow{21:22}}}%
{\anga{साध्यः}{\time{29-25}{17:46}}\hspace{1ex}\uanga{शुभः}}%
{\anga{गरः}{\time{16-19}{12:32}}\hspace{1ex}\anga{वणिजः}{\time{45-31}{00:13(+1)}}\hspace{1ex}\uanga{विष्टिः}}{}
}
{आशा-दशमी\eventsep चातुर्मास्यव्रत-आरम्भः}
{Thu} 
\cfoot{\rygdata{15:02--16:50}{06:00--07:49}{09:37--11:25}}
\caldata{JULY}{12}{\sunmonth{मिथुनम्}{28}{}{आषाढः}{ग्रीष्मऋतुः}{शुक्रः}{विकारी}{उत्तरायणम्}{ग्रीष्मऋतुः}}
{\sunmoonrsdata{06:01}{20:27}{17:08}{03:06(+1)}{13:14}
{\kalas{04:44 05:22 09:52 08:54 10:49 18:31 11:47 14:40 17:33 19:29 21:05 22:50 00:02(+1) 02:26(+1)}}}
{\tnykdata{\anga{\tithi{11}{शुक्ल-एकादशी}}{\time{14-59}{12:01}}\hspace{1ex}}%
{\anga{अनूराधा}{\time{54-45}{03:55(+1)}}\hspace{1ex}}{चन्द्रराशिः—\mbox{वृश्चिकः}}%
{\anga{शुभः}{\time{26-2}{16:26}}\hspace{1ex}\uanga{शुक्लः}}%
{\anga{विष्टिः}{\time{14-59}{12:01}}\hspace{1ex}\anga{बवः}{\time{44-47}{23:56}}\hspace{1ex}\uanga{बालवः}}{}
}
{गोपद्म-व्रत-आरम्भः\eventsep हरिवासरः\RIGHTarrow{}17:57\eventsep काञ्ची ३१ जगद्गुरु श्री-ब्रह्मानन्दघनेन्द्र सरस्वती आराधना~\#{१३५२}\eventsep काञ्ची ६३ जगद्गुरु श्री-महादेवेन्द्र सरस्वती ५ आराधना~\#{२०६}\eventsep सर्व-शयन-एकादशी\eventsep विष्णु-शयनोत्सवः}
{Fri} 
\cfoot{\rygdata{11:26--13:14}{16:50--18:38}{07:49--09:37}}
\caldata{JULY}{13}{\sunmonth{मिथुनम्}{29}{}{आषाढः}{ग्रीष्मऋतुः}{शनिः}{विकारी}{उत्तरायणम्}{ग्रीष्मऋतुः}}
{\sunmoonrsdata{06:01}{20:26}{18:08}{03:54(+1)}{13:14}
{\kalas{04:45 05:23 09:52 08:54 10:50 18:31 11:47 14:40 17:33 19:28 21:05 22:50 00:02(+1) 02:26(+1)}}}
{\tnykdata{\anga{\tithi{12}{शुक्ल-द्वादशी}}{\time{14-52}{11:58}}\hspace{1ex}}%
{\anga{ज्येष्ठा}{\time{57-10}{04:53(+1)}}\hspace{1ex}}{चन्द्रराशिः—\mbox{वृश्चिकः\RIGHTarrow{04:53(+1)}}}%
{\anga{शुक्लः}{\time{23-37}{15:28}}\hspace{1ex}\uanga{ब्रह्म}}%
{\anga{बालवः}{\time{14-52}{11:58}}\hspace{1ex}\anga{कौलवः}{\time{45-15}{00:08(+1)}}\hspace{1ex}\uanga{तैतिलः}}{}
}
{ज्येष्ठाभिषेकः\eventsep वासुदेव-द्वादशी\eventsep शाकव्रत-आरम्भः\eventsep शनि-प्रदोष-व्रतम्~20:26\RIGHTarrow{}21:05}
{Sat} 
\cfoot{\rygdata{09:38--11:26}{15:02--16:50}{06:01--07:50}}
\caldata{JULY}{14}{\sunmonth{मिथुनम्}{30}{}{आषाढः}{ग्रीष्मऋतुः}{भानुः}{विकारी}{उत्तरायणम्}{ग्रीष्मऋतुः}}
{\sunmoonrsdata{06:02}{20:26}{19:03}{04:47(+1)}{13:14}
{\kalas{04:45 05:24 09:52 08:55 10:50 18:31 11:48 14:40 17:33 19:28 21:04 22:50 00:02(+1) 02:26(+1)}}}
{\tnykdata{\anga{\tithi{13}{शुक्ल-त्रयोदशी}}{\time{15-55}{12:24}}\hspace{1ex}}%
{\fullanga{मूला}}{चन्द्रराशिः—\mbox{धनुः}}%
{\anga{ब्रह्म}{\time{22-8}{14:53}}\hspace{1ex}\uanga{इन्द्रः}}%
{\anga{तैतिलः}{\time{15-55}{12:24}}\hspace{1ex}\anga{गरः}{\time{46-54}{00:48(+1)}}\hspace{1ex}\uanga{वणिजः}}{}
}
{}
{Sun} 
\cfoot{\rygdata{18:38--20:26}{13:14--15:02}{16:50--18:38}}
\caldata{JULY}{15}{\sunmonth{मिथुनम्}{31}{}{आषाढः}{ग्रीष्मऋतुः}{सोमः}{विकारी}{उत्तरायणम्}{ग्रीष्मऋतुः}}
{\sunmoonrsdata{06:03}{20:25}{19:52}{05:43(+1)}{13:14}
{\kalas{04:46 05:24 09:53 08:55 10:50 18:30 11:48 14:40 17:33 19:28 21:04 22:50 00:02(+1) 02:27(+1)}}}
{\tnykdata{\anga{\tithi{14}{शुक्ल-चतुर्दशी}}{\time{18-7}{13:18}}\hspace{1ex}}%
{\anga{मूला}{\time{0-40}{06:19}}\hspace{1ex}}{चन्द्रराशिः—\mbox{धनुः}}%
{\anga{इन्द्रः}{\time{21-32}{14:40}}\hspace{1ex}\uanga{वैधृतिः}}%
{\anga{वणिजः}{\time{18-7}{13:18}}\hspace{1ex}\anga{विष्टिः}{\time{49-39}{01:55(+1)}}\hspace{1ex}\uanga{बवः}}{}
}
{काञ्ची १० जगद्गुरु श्री-सुरेश्वरेन्द्र सरस्वती आराधना~\#{१८९३}\eventsep कोकिल-व्रतम्\eventsep मन्वादिः-(ब्रह्मः-[१०])\eventsep पञ्च-पर्व-पूजा (पूर्णिमा)\eventsep पवित्र-चतुर्दशी\eventsep वेङ्कटाचले पूर्णिमा-गरुड-सेवा\eventsep वैधृति-श्राद्धम्}
{Mon} 
\cfoot{\rygdata{07:51--09:38}{11:26--13:14}{15:02--16:50}}
\caldata{JULY}{16}{\sunmonth{कर्कटः}{1}{\mbox{मिथुनम्{\tiny\RIGHTarrow}{15:39}}}{आषाढः}{ग्रीष्मऋतुः}{मङ्गलः}{विकारी}{दक्षिणायनम्}{ग्रीष्मऋतुः}}
{\sunmoonrsdata{06:04}{20:25}{20:35}{---}{13:14}
{\kalas{04:46 05:25 09:53 08:56 10:51 18:30 11:48 14:40 17:32 19:27 21:03 22:50 00:02(+1) 02:27(+1)}}}
{\tnykdata{\anga{\tithi{15}{पौर्णमासी}}{\time{21-25}{14:38}}\hspace{1ex}}%
{\anga{पूर्वाषाढा}{\time{5-17}{08:11}}\hspace{1ex}}{चन्द्रराशिः—\mbox{धनुः\RIGHTarrow{14:42}}}%
{\anga{वैधृतिः}{\time{21-47}{14:47}}\hspace{1ex}\uanga{विष्कम्भः}}%
{\anga{बवः}{\time{21-25}{14:38}}\hspace{1ex}\anga{बालवः}{\time{53-27}{03:27(+1)}}\hspace{1ex}\uanga{कौलवः}}{}
}
{आषाढ-पूर्णिमा-स्नानम्\eventsep गुरु-पूर्णिमा/व्यास-पूजा\eventsep काञ्ची ५४ जगद्गुरु श्री-व्यासाचल महादेवेन्द्र सरस्वती आराधना~\#{५१३}\eventsep कर्कट-सङ्क्रमण-पुण्यकालः~03:39\RIGHTarrow{}15:39\eventsep पार्वणव्रतम् पूर्णिमायाम्\eventsep पूर्णिमा-व्रतम्\eventsep सर्वनदी-रजस्वला\eventsep यतिचातुर्मास्यव्रत-आरम्भः\eventsep शिव-शयनोत्सवः}
{Tue} 
\cfoot{\rygdata{16:49--18:37}{09:39--11:26}{13:14--15:02}}
\caldata{JULY}{17}{\sunmonth{कर्कटः}{2}{}{आषाढः}{ग्रीष्मऋतुः}{बुधः}{विकारी}{दक्षिणायनम्}{ग्रीष्मऋतुः}}
{\sunmoonsrdata{06:04}{20:24}{21:11}{06:41}{13:14}
{\kalas{04:47 05:26 09:54 08:56 10:51 18:29 11:48 14:40 17:32 19:27 21:03 22:49 00:02(+1) 02:27(+1)}}}
{\tnykdata{\anga{\tithi{16}{कृष्ण-प्रथमा}}{\time{25-42}{16:21}}\hspace{1ex}}%
{\anga{उत्तराषाढा}{\time{10-54}{10:26}}\hspace{1ex}}{चन्द्रराशिः—\mbox{मकरः}}%
{\anga{विष्कम्भः}{\time{22-48}{15:11}}\hspace{1ex}\uanga{प्रीतिः}}%
{\anga{कौलवः}{\time{25-42}{16:21}}\hspace{1ex}\anga{तैतिलः}{\time{58-11}{05:21(+1)}}\hspace{1ex}\uanga{गरः}}{}
}
{अशून्यशयन-व्रतम्\eventsep पार्वण-प्रायश्चित्तावकाशः दर्शे\eventsep पूर्णमासेष्टिः\eventsep सर्वनदी-रजस्वला\eventsep स्थालीपाकः\eventsep श्रवण-व्रतम्}
{Wed} 
\cfoot{\rygdata{13:14--15:02}{07:52--09:39}{11:27--13:14}}
\caldata{JULY}{18}{\sunmonth{कर्कटः}{3}{}{आषाढः}{ग्रीष्मऋतुः}{गुरुः}{विकारी}{दक्षिणायनम्}{ग्रीष्मऋतुः}}
{\sunmoonsrdata{06:05}{20:24}{21:44}{07:40}{13:14}
{\kalas{04:48 05:26 09:54 08:57 10:51 18:29 11:48 14:40 17:32 19:26 21:02 22:49 00:02(+1) 02:27(+1)}}}
{\tnykdata{\anga{\tithi{17}{कृष्ण-द्वितीया}}{\time{30-49}{18:25}}\hspace{1ex}}%
{\anga{श्रवणः}{\time{17-21}{13:02}}\hspace{1ex}}{चन्द्रराशिः—\mbox{मकरः\RIGHTarrow{02:26(+1)}}}%
{\anga{प्रीतिः}{\time{24-27}{15:52}}\hspace{1ex}\uanga{आयुष्मान्}}%
{\anga{गरः}{\time{30-49}{18:25}}\hspace{1ex}\uanga{वणिजः}}{}
}
{अष्टनाग-पूजा\eventsep सर्वनदी-रजस्वला}
{Thu} 
\cfoot{\rygdata{15:02--16:49}{06:05--07:52}{09:40--11:27}}
\caldata{JULY}{19}{\sunmonth{कर्कटः}{4}{}{आषाढः}{ग्रीष्मऋतुः}{शुक्रः}{विकारी}{दक्षिणायनम्}{ग्रीष्मऋतुः}}
{\sunmoonsrdata{06:06}{20:23}{22:12}{08:38}{13:14}
{\kalas{04:48 05:27 09:54 08:57 10:51 18:29 11:49 14:40 17:32 19:26 21:02 22:49 00:02(+1) 02:28(+1)}}}
{\tnykdata{\anga{\tithi{18}{कृष्ण-तृतीया}}{\time{36-33}{20:43}}\hspace{1ex}}%
{\anga{श्रविष्ठा}{\time{24-27}{15:52}}\hspace{1ex}}{चन्द्रराशिः—\mbox{कुम्भः}}%
{\anga{आयुष्मान्}{\time{26-36}{16:44}}\hspace{1ex}\uanga{सौभाग्यः}}%
{\anga{वणिजः}{\time{3-37}{07:33}}\hspace{1ex}\anga{विष्टिः}{\time{36-33}{20:43}}\hspace{1ex}\uanga{बवः}}{}
}
{\tamil{ஆடி~வெள்ளிக்கிழமை}\eventsep गजानन-महागणपति सङ्कटहर-चतुर्थी-व्रतम्}
{Fri} 
\cfoot{\rygdata{11:27--13:14}{16:49--18:36}{07:53--09:40}}
\caldata{JULY}{20}{\sunmonth{कर्कटः}{5}{}{आषाढः}{ग्रीष्मऋतुः}{शनिः}{विकारी}{दक्षिणायनम्}{ग्रीष्मऋतुः}}
{\sunmoonsrdata{06:06}{20:22}{22:39}{09:35}{13:14}
{\kalas{04:49 05:28 09:55 08:58 10:52 18:28 11:49 14:40 17:31 19:25 21:01 22:49 00:02(+1) 02:28(+1)}}}
{\tnykdata{\anga{\tithi{19}{कृष्ण-चतुर्थी}}{\time{42-37}{23:09}}\hspace{1ex}}%
{\anga{शतभिषक्}{\time{31-54}{18:52}}\hspace{1ex}}{चन्द्रराशिः—\mbox{कुम्भः}}%
{\anga{सौभाग्यः}{\time{29-2}{17:44}}\hspace{1ex}\uanga{शोभनः}}%
{\anga{बवः}{\time{9-33}{09:56}}\hspace{1ex}\anga{बालवः}{\time{42-37}{23:09}}\hspace{1ex}\uanga{कौलवः}}{}
}
{}
{Sat} 
\cfoot{\rygdata{09:40--11:27}{15:01--16:48}{06:06--07:53}}
\caldata{JULY}{21}{\sunmonth{कर्कटः}{6}{}{आषाढः}{ग्रीष्मऋतुः}{भानुः}{विकारी}{दक्षिणायनम्}{ग्रीष्मऋतुः}}
{\sunmoonsrdata{06:07}{20:22}{23:05}{10:31}{13:14}
{\kalas{04:49 05:28 09:55 08:58 10:52 18:28 11:49 14:40 17:31 19:25 21:01 22:48 00:02(+1) 02:28(+1)}}}
{\tnykdata{\anga{\tithi{20}{कृष्ण-पञ्चमी}}{\time{48-36}{01:34(+1)}}\hspace{1ex}}%
{\anga{पूर्वप्रोष्ठपदा}{\time{39-21}{21:52}}\hspace{1ex}}{चन्द्रराशिः—\mbox{कुम्भः\RIGHTarrow{15:08}}}%
{\anga{शोभनः}{\time{31-29}{18:43}}\hspace{1ex}\uanga{अतिगण्डः}}%
{\anga{कौलवः}{\time{15-37}{12:22}}\hspace{1ex}\anga{तैतिलः}{\time{48-36}{01:34(+1)}}\hspace{1ex}\uanga{गरः}}{}
}
{}
{Sun} 
\cfoot{\rygdata{18:35--20:22}{13:14--15:01}{16:48--18:35}}
\caldata{JULY}{22}{\sunmonth{कर्कटः}{7}{}{आषाढः}{ग्रीष्मऋतुः}{सोमः}{विकारी}{दक्षिणायनम्}{ग्रीष्मऋतुः}}
{\sunmoonsrdata{06:08}{20:21}{23:30}{11:28}{13:14}
{\kalas{04:50 05:29 09:55 08:58 10:52 18:27 11:49 14:40 17:30 19:24 21:00 22:48 00:01(+1) 02:28(+1)}}}
{\tnykdata{\anga{\tithi{21}{कृष्ण-षष्ठी}}{\time{54-4}{03:46(+1)}}\hspace{1ex}}%
{\anga{उत्तरप्रोष्ठपदा}{\time{46-22}{00:41(+1)}}\hspace{1ex}}{चन्द्रराशिः—\mbox{मीनः}}%
{\anga{अतिगण्डः}{\time{33-37}{19:35}}\hspace{1ex}\uanga{सुकर्म}}%
{\anga{गरः}{\time{21-24}{14:42}}\hspace{1ex}\anga{वणिजः}{\time{54-4}{03:46(+1)}}\hspace{1ex}\uanga{विष्टिः}}{}
}
{विष्णुपदी-पुण्यकालः~13:26\RIGHTarrow{}02:14(+1)\eventsep शुचि-मासः/ग्रीष्मऋतुः\RIGHTarrow{}19:50}
{Mon} 
\cfoot{\rygdata{07:55--09:41}{11:28--13:14}{15:01--16:48}}
\caldata{JULY}{23}{\sunmonth{कर्कटः}{8}{}{आषाढः}{ग्रीष्मऋतुः}{मङ्गलः}{विकारी}{दक्षिणायनम्}{ग्रीष्मऋतुः}}
{\sunmoonsrdata{06:09}{20:20}{23:57}{12:25}{13:14}
{\kalas{04:50 05:30 09:56 08:59 10:53 18:27 11:49 14:40 17:30 19:23 21:00 22:48 00:01(+1) 02:29(+1)}}}
{\tnykdata{\anga{\tithi{22}{कृष्ण-सप्तमी}}{\time{58-35}{05:35(+1)}}\hspace{1ex}}%
{\anga{रेवती}{\time{52-31}{03:09(+1)}}\hspace{1ex}}{चन्द्रराशिः—\mbox{मीनः\RIGHTarrow{03:09(+1)}}}%
{\anga{सुकर्म}{\time{35-8}{20:12}}\hspace{1ex}\uanga{धृतिः}}%
{\anga{विष्टिः}{\time{26-27}{16:44}}\hspace{1ex}\anga{बवः}{\time{58-35}{05:35(+1)}}\hspace{1ex}\uanga{बालवः}}{}
}
{चामुण्डेश्वरी-जयन्ती}
{Tue} 
\cfoot{\rygdata{16:47--18:34}{09:42--11:28}{13:15--15:01}}
\caldata{JULY}{24}{\sunmonth{कर्कटः}{9}{}{आषाढः}{ग्रीष्मऋतुः}{बुधः}{विकारी}{दक्षिणायनम्}{ग्रीष्मऋतुः}}
{\sunmoonsrdata{06:09}{20:19}{00:26(+1)}{13:24}{13:14}
{\kalas{04:51 05:30 09:56 09:00 10:53 18:26 11:49 14:39 17:29 19:23 20:59 22:47 00:01(+1) 02:29(+1)}}}
{\tnykdata{\fulltithi{\tithi{23}{कृष्ण-अष्टमी}}}%
{\anga{अश्विनी}{\time{57-22}{05:06(+1)}}\hspace{1ex}}{चन्द्रराशिः—\mbox{मेषः}}%
{\anga{धृतिः}{\time{35-41}{20:26}}\hspace{1ex}\uanga{शूलः}}%
{\anga{बालवः}{\time{30-20}{18:18}}\hspace{1ex}\uanga{कौलवः}}{}
}
{पञ्च-पर्व-पूजा (अष्टमी)}
{Wed} 
\cfoot{\rygdata{13:15--15:01}{07:56--09:42}{11:28--13:15}}
\caldata{JULY}{25}{\sunmonth{कर्कटः}{10}{}{आषाढः}{ग्रीष्मऋतुः}{गुरुः}{विकारी}{दक्षिणायनम्}{ग्रीष्मऋतुः}}
{\sunmoonsrdata{06:10}{20:19}{00:59(+1)}{14:24}{13:15}
{\kalas{04:52 05:31 09:57 09:00 10:53 18:26 11:50 14:39 17:29 19:22 20:58 22:47 00:01(+1) 02:29(+1)}}}
{\tnykdata{\anga{\tithi{23}{कृष्ण-अष्टमी}}{\time{1-41}{06:51}}\hspace{1ex}}%
{\fullanga{अपभरणी}}{चन्द्रराशिः—\mbox{मेषः}}%
{\anga{शूलः}{\time{35-0}{20:11}}\hspace{1ex}\uanga{गण्डः}}%
{\anga{कौलवः}{\time{1-41}{06:51}}\hspace{1ex}\anga{तैतिलः}{\time{32-38}{19:14}}\hspace{1ex}\uanga{गरः}}{}
}
{}
{Thu} 
\cfoot{\rygdata{15:01--16:47}{06:10--07:56}{09:42--11:28}}
\caldata{JULY}{26}{\sunmonth{कर्कटः}{11}{}{आषाढः}{ग्रीष्मऋतुः}{शुक्रः}{विकारी}{दक्षिणायनम्}{ग्रीष्मऋतुः}}
{\sunmoonsrdata{06:11}{20:18}{01:38(+1)}{15:27}{13:15}
{\kalas{04:52 05:32 09:57 09:01 10:53 18:25 11:50 14:39 17:29 19:21 20:57 22:46 00:01(+1) 02:29(+1)}}}
{\tnykdata{\anga{\tithi{24}{कृष्ण-नवमी}}{\time{3-6}{07:26}}\hspace{1ex}}%
{\anga{अपभरणी}{\time{0-32}{06:24}}\hspace{1ex}}{चन्द्रराशिः—\mbox{मेषः\RIGHTarrow{12:37}}}%
{\anga{गण्डः}{\time{32-53}{19:21}}\hspace{1ex}\uanga{वृद्धिः}}%
{\anga{गरः}{\time{3-6}{07:26}}\hspace{1ex}\anga{वणिजः}{\time{33-8}{19:27}}\hspace{1ex}\uanga{विष्टिः}}{}
}
{\tamil{ஆடி~வெள்ளிக்கிழமை}\eventsep कृत्तिका-व्रतम्\eventsep \tamil{மூர்த்தி நாயனார் (15) குருபூஜை}\eventsep \tamil{புகழ்ச்சோழ நாயனார் (39) குருபூஜை}}
{Fri} 
\cfoot{\rygdata{11:29--13:15}{16:46--18:32}{07:57--09:43}}
\caldata{JULY}{27}{\sunmonth{कर्कटः}{12}{}{आषाढः}{ग्रीष्मऋतुः}{शनिः}{विकारी}{दक्षिणायनम्}{ग्रीष्मऋतुः}}
{\sunmoonsrdata{06:12}{20:17}{02:25(+1)}{16:30}{13:14}
{\kalas{04:53 05:32 09:57 09:01 10:54 18:24 11:50 14:39 17:28 19:21 20:57 22:46 00:00(+1) 02:29(+1)}}}
{\tnykdata{\anga{\tithi{25}{कृष्ण-दशमी}}{\time{2-39}{07:16}}\hspace{1ex}}%
{\anga{कृत्तिका}{\time{1-54}{06:58}}\hspace{1ex}}{चन्द्रराशिः—\mbox{वृषभः}}%
{\anga{वृद्धिः}{\time{29-15}{17:54}}\hspace{1ex}\uanga{ध्रुवः}}%
{\anga{विष्टिः}{\time{2-39}{07:16}}\hspace{1ex}\anga{बवः}{\time{31-43}{18:53}}\hspace{1ex}\uanga{बालवः}}{}
}
{कर्कट-कार्त्तिक-पूजा\eventsep स्मार्त-कामिका-एकादशी (गृहस्थ)\eventsep \tamil{திருப்பாணாழ்வார் திருநக்ஷத்திரம்}\eventsep शनिरोहिणी-पुण्यकालः~06:58\RIGHTarrow{}}
{Sat} 
\cfoot{\rygdata{09:43--11:29}{15:00--16:46}{06:12--07:58}}
\caldata{JULY}{28}{\sunmonth{कर्कटः}{13}{}{आषाढः}{ग्रीष्मऋतुः}{भानुः}{विकारी}{दक्षिणायनम्}{ग्रीष्मऋतुः}}
{\sunmoonsrdata{06:13}{20:16}{03:19(+1)}{17:33}{13:14}
{\kalas{04:53 05:33 09:58 09:01 10:54 18:24 11:50 14:39 17:28 19:20 20:56 22:45 00:00(+1) 02:30(+1)}}}
{\tnykdata{\anga{\tithi{26}{कृष्ण-एकादशी}}{\time{0-16}{06:19}}\hspace{1ex}\anga{\tithi{27}{कृष्ण-द्वादशी}}{\time{56-4}{04:39(+1)}}\hspace{1ex}\avamA{}}%
{\anga{रोहिणी}{\time{1-21}{06:45}}\hspace{1ex}\anga{मृगशीर्षम्}{\time{59-2}{05:50(+1)}}\hspace{1ex}}{चन्द्रराशिः—\mbox{वृषभः\RIGHTarrow{18:23}}}%
{\anga{ध्रुवः}{\time{24-4}{15:51}}\hspace{1ex}\uanga{व्याघातः}}%
{\anga{बालवः}{\time{0-16}{06:19}}\hspace{1ex}\anga{कौलवः}{\time{28-23}{17:34}}\hspace{1ex}\anga{तैतिलः}{\time{56-4}{04:39(+1)}}\hspace{1ex}\uanga{गरः}}{}
}
{देवी-पर्व-४\eventsep हरिवासरः\RIGHTarrow{}11:58\eventsep स्मार्त-कामिका-एकादशी (सन्न्यस्थ)\eventsep त्रिस्पर्शा-महाद्वादशी\eventsep वैष्णव-कामिका-एकादशी}
{Sun} 
\cfoot{\rygdata{18:31--20:16}{13:14--15:00}{16:45--18:31}}
\caldata{JULY}{29}{\sunmonth{कर्कटः}{14}{}{आषाढः}{ग्रीष्मऋतुः}{सोमः}{विकारी}{दक्षिणायनम्}{ग्रीष्मऋतुः}}
{\sunmoonsrdata{06:13}{20:15}{04:23(+1)}{18:32}{13:14}
{\kalas{04:54 05:34 09:58 09:02 10:54 18:23 11:50 14:39 17:27 19:19 20:55 22:45 00:00(+1) 02:30(+1)}}}
{\tnykdata{\anga{\tithi{28}{कृष्ण-त्रयोदशी}}{\time{50-13}{02:19(+1)}}\hspace{1ex}}%
{\anga{आर्द्रा}{\time{55-3}{04:15(+1)}}\hspace{1ex}}{चन्द्रराशिः—\mbox{मिथुनम्}}%
{\anga{व्याघातः}{\time{17-27}{13:13}}\hspace{1ex}\uanga{हर्षणः}}%
{\anga{गरः}{\time{23-19}{15:33}}\hspace{1ex}\anga{वणिजः}{\time{50-13}{02:19(+1)}}\hspace{1ex}\uanga{विष्टिः}}{}
}
{\tamil{கூற்றுவ நாயனார் (38) குருபூஜை}\eventsep मासशिवरात्रिः\eventsep सोम-प्रदोष-व्रतम्~20:15\RIGHTarrow{}20:55}
{Mon} 
\cfoot{\rygdata{07:59--09:44}{11:29--13:14}{15:00--16:45}}
\caldata{JULY}{30}{\sunmonth{कर्कटः}{15}{}{आषाढः}{ग्रीष्मऋतुः}{मङ्गलः}{विकारी}{दक्षिणायनम्}{ग्रीष्मऋतुः}}
{\sunmoonsrdata{06:14}{20:14}{05:33(+1)}{19:26}{13:14}
{\kalas{04:55 05:34 09:58 09:02 10:54 18:22 11:50 14:38 17:26 19:18 20:54 22:45 00:00(+1) 02:30(+1)}}}
{\tnykdata{\anga{\tithi{29}{कृष्ण-चतुर्दशी}}{\time{43-1}{23:27}}\hspace{1ex}}%
{\anga{पुनर्वसुः}{\time{49-45}{02:09(+1)}}\hspace{1ex}}{चन्द्रराशिः—\mbox{मिथुनम्\RIGHTarrow{20:43}}}%
{\anga{हर्षणः}{\time{9-35}{10:04}}\hspace{1ex}\uanga{वज्रम्}}%
{\anga{विष्टिः}{\time{16-45}{12:57}}\hspace{1ex}\anga{शकुनिः}{\time{43-1}{23:27}}\hspace{1ex}\uanga{चतुष्पात्}}{}
}
{बोधायन आषाढ (कर्कट) अमावास्या\eventsep कृष्णाङ्गारक-चतुर्दशी-पुण्यकालः/यमतर्पणम्\eventsep पञ्च-पर्व-पूजा (अमावास्या)\eventsep पञ्च-पर्व-पूजा (चतुर्दशी)}
{Tue} 
\cfoot{\rygdata{16:44--18:29}{09:44--11:29}{13:14--14:59}}
\caldata{JULY}{31}{\sunmonth{कर्कटः}{16}{}{आषाढः}{ग्रीष्मऋतुः}{बुधः}{विकारी}{दक्षिणायनम्}{ग्रीष्मऋतुः}}
{\sunmoonsrdata{06:15}{20:13}{---}{20:14}{13:14}
{\kalas{04:55 05:35 09:59 09:03 10:55 18:22 11:50 14:38 17:26 19:18 20:54 22:44 23:59 02:30(+1)}}}
{\tnykdata{\anga{\tithi{30}{अमावास्या}}{\time{34-50}{20:11}}\hspace{1ex}}%
{\anga{पुष्यः}{\time{43-30}{23:39}}\hspace{1ex}}{चन्द्रराशिः—\mbox{कर्कटः}}%
{\anga{वज्रम्}{\time{0-41}{06:32}}\hspace{1ex}\anga{सिद्धिः}{\time{51-6}{02:42(+1)}}\hspace{1ex}\uanga{व्यतीपातः}}%
{\anga{चतुष्पात्}{\time{9-1}{09:52}}\hspace{1ex}\anga{नाग}{\time{34-50}{20:11}}\hspace{1ex}\uanga{किंस्तुघ्नः}}{}
}
{आषाढ-स्नानपूर्तिः\eventsep आषाढ (कर्कट) अमावास्या (अलभ्यम्–पुष्यः)\eventsep बोधायन-इष्टिः\eventsep दीप-पूजा\eventsep काञ्ची ३८ जगद्गुरु श्री-अभिनवशङ्करेन्द्र सरस्वती आराधना~\#{११८०}\eventsep काञ्ची ४६ जगद्गुरु श्री-सान्द्रानन्दबोधेन्द्र सरस्वती आराधना~\#{९२२}\eventsep पार्वणव्रतम् अमावास्यायाम्\eventsep पति-सञ्जीवनी-व्रतम्}
{Wed} 
\cfoot{\rygdata{13:14--14:59}{08:00--09:45}{11:30--13:14}}
\caldata{AUGUST}{1}{\sunmonth{कर्कटः}{17}{}{श्रावणः}{वर्षऋतुः}{गुरुः}{विकारी}{दक्षिणायनम्}{ग्रीष्मऋतुः}}
{\sunmoonrsdata{06:16}{20:13}{06:47}{20:55}{13:14}
{\kalas{04:56 05:36 09:59 09:03 10:55 18:21 11:51 14:38 17:25 19:17 20:53 22:44 23:59 02:30(+1)}}}
{\tnykdata{\anga{\tithi{1}{शुक्ल-प्रथमा}}{\time{26-2}{16:41}}\hspace{1ex}}%
{\anga{आश्रेषा}{\time{36-42}{20:57}}\hspace{1ex}}{चन्द्रराशिः—\mbox{कर्कटः\RIGHTarrow{20:57}}}%
{\anga{व्यतीपातः}{\time{41-5}{22:42}}\hspace{1ex}\uanga{वरीयान्}}%
{\anga{किंस्तुघ्नः}{\time{0-28}{06:27}}\hspace{1ex}\anga{बवः}{\time{26-2}{16:41}}\hspace{1ex}\anga{बालवः}{\time{51-33}{02:53(+1)}}\hspace{1ex}\uanga{कौलवः}}{}
}
{चन्द्र-दर्शनम्~20:13\RIGHTarrow{}20:53\eventsep दर्शेष्टिः\eventsep काञ्ची ४१ जगद्गुरु श्री-गङ्गाधरेन्द्र सरस्वती २ आराधना~\#{१०७०}\eventsep पार्वण-प्रायश्चित्तावकाशः पौर्णमास्याम्\eventsep स्थालीपाकः\eventsep व्यतीपात-श्राद्धम्}
{Thu} 
\cfoot{\rygdata{14:59--16:43}{06:16--08:01}{09:45--11:30}}
\caldata{AUGUST}{2}{\sunmonth{कर्कटः}{18}{}{श्रावणः}{वर्षऋतुः}{शुक्रः}{विकारी}{दक्षिणायनम्}{ग्रीष्मऋतुः}}
{\sunmoonrsdata{06:17}{20:12}{08:02}{21:32}{13:14}
{\kalas{04:56 05:37 09:59 09:04 10:55 18:20 11:51 14:38 17:25 19:16 20:52 22:43 23:59 02:30(+1)}}}
{\tnykdata{\anga{\tithi{2}{शुक्ल-द्वितीया}}{\time{17-2}{13:06}}\hspace{1ex}}%
{\anga{मघा}{\time{29-46}{18:11}}\hspace{1ex}}{चन्द्रराशिः—\mbox{सिंहः}}%
{\anga{वरीयान्}{\time{30-59}{18:41}}\hspace{1ex}\uanga{परिघः}}%
{\anga{कौलवः}{\time{17-2}{13:06}}\hspace{1ex}\anga{तैतिलः}{\time{42-36}{23:19}}\hspace{1ex}\uanga{गरः}}{}
}
{\tamil{ஆடிப்~பெருக்கு}\eventsep \tamil{ஆடி~வெள்ளிக்கிழமை}\eventsep मनोरथ-द्वितीया\eventsep सत्यनारायण-जयन्ती}
{Fri} 
\cfoot{\rygdata{11:30--13:14}{16:43--18:27}{08:01--09:46}}
\caldata{AUGUST}{3}{\sunmonth{कर्कटः}{19}{}{श्रावणः}{वर्षऋतुः}{शनिः}{विकारी}{दक्षिणायनम्}{ग्रीष्मऋतुः}}
{\sunmoonrsdata{06:18}{20:11}{09:16}{22:06}{13:14}
{\kalas{04:57 05:37 10:00 09:04 10:55 18:20 11:51 14:37 17:24 19:15 20:51 22:42 23:58 02:31(+1)}}}
{\tnykdata{\anga{\tithi{3}{शुक्ल-तृतीया}}{\time{8-13}{09:35}}\hspace{1ex}\anga{\tithi{4}{शुक्ल-चतुर्थी}}{\time{60-2}{06:19(+1)}}\hspace{1ex}\avamA{}}%
{\anga{पूर्वफल्गुनी}{\time{23-8}{15:33}}\hspace{1ex}}{चन्द्रराशिः—\mbox{सिंहः\RIGHTarrow{20:56}}}%
{\anga{परिघः}{\time{21-9}{14:45}}\hspace{1ex}\uanga{शिवः}}%
{\anga{गरः}{\time{8-13}{09:35}}\hspace{1ex}\anga{वणिजः}{\time{34-2}{19:55}}\hspace{1ex}\anga{विष्टिः}{\time{60-2}{06:19(+1)}}\hspace{1ex}\uanga{बवः}}{}
}
{एकविंशति-दिवस-गणपति-व्रत-आरम्भः\eventsep दूर्वा-गणपति-व्रतम्\eventsep हरियाली-तृतीया\eventsep मधुश्रावणि-व्रतम्\eventsep पार्वती-पवित्रारोपणम्\eventsep स्वर्ण-गौरी-व्रतम्\eventsep \tamil{திருவாடிப்பூரம்}}
{Sat} 
\cfoot{\rygdata{09:46--11:30}{14:58--16:42}{06:18--08:02}}
\caldata{AUGUST}{4}{\sunmonth{कर्कटः}{20}{}{श्रावणः}{वर्षऋतुः}{भानुः}{विकारी}{दक्षिणायनम्}{ग्रीष्मऋतुः}}
{\sunmoonrsdata{06:19}{20:10}{10:28}{22:38}{13:14}
{\kalas{04:57 05:38 10:00 09:05 10:56 18:19 11:51 14:37 17:23 19:14 20:50 22:42 23:58 02:31(+1)}}}
{\tnykdata{\anga{\tithi{5}{शुक्ल-पञ्चमी}}{\time{52-44}{03:24(+1)}}\hspace{1ex}}%
{\anga{उत्तरफल्गुनी}{\time{17-12}{13:12}}\hspace{1ex}}{चन्द्रराशिः—\mbox{कन्या}}%
{\anga{शिवः}{\time{11-52}{11:03}}\hspace{1ex}\uanga{सिद्धः}}%
{\anga{बवः}{\time{26-13}{16:48}}\hspace{1ex}\anga{बालवः}{\time{52-44}{03:24(+1)}}\hspace{1ex}\uanga{कौलवः}}{}
}
{आदित्यहस्त-पुण्यकालः~13:12\RIGHTarrow{}\eventsep गरुड-पञ्चमी\eventsep नाग-पञ्चमी}
{Sun} 
\cfoot{\rygdata{18:26--20:10}{13:14--14:58}{16:42--18:26}}
\caldata{AUGUST}{5}{\sunmonth{कर्कटः}{21}{}{श्रावणः}{वर्षऋतुः}{सोमः}{विकारी}{दक्षिणायनम्}{ग्रीष्मऋतुः}}
{\sunmoonrsdata{06:19}{20:08}{11:39}{23:11}{13:14}
{\kalas{04:58 05:39 10:00 09:05 10:56 18:18 11:51 14:37 17:23 19:13 20:49 22:41 23:58 02:31(+1)}}}
{\tnykdata{\anga{\tithi{6}{शुक्ल-षष्ठी}}{\time{46-40}{01:00(+1)}}\hspace{1ex}}%
{\anga{हस्तः}{\time{12-19}{11:15}}\hspace{1ex}}{चन्द्रराशिः—\mbox{कन्या\RIGHTarrow{22:28}}}%
{\anga{सिद्धः}{\time{3-25}{07:42}}\hspace{1ex}\anga{साध्यः}{\time{56-5}{04:46(+1)}}\hspace{1ex}\uanga{शुभः}}%
{\anga{कौलवः}{\time{19-31}{14:08}}\hspace{1ex}\anga{तैतिलः}{\time{46-40}{01:00(+1)}}\hspace{1ex}\uanga{गरः}}{}
}
{षष्ठी-व्रतम्\eventsep सूपौदन-व्रतम्\eventsep तैत्तिरीय-उपाकर्म हस्ते}
{Mon} 
\cfoot{\rygdata{08:03--09:47}{11:30--13:14}{14:57--16:41}}
\caldata{AUGUST}{6}{\sunmonth{कर्कटः}{22}{}{श्रावणः}{वर्षऋतुः}{मङ्गलः}{विकारी}{दक्षिणायनम्}{ग्रीष्मऋतुः}}
{\sunmoonrsdata{06:20}{20:07}{12:48}{23:46}{13:14}
{\kalas{04:59 05:39 10:01 09:06 10:56 18:17 11:51 14:36 17:22 19:12 20:48 22:41 23:57 02:31(+1)}}}
{\tnykdata{\anga{\tithi{7}{शुक्ल-सप्तमी}}{\time{42-5}{23:11}}\hspace{1ex}}%
{\anga{चित्रा}{\time{8-45}{09:51}}\hspace{1ex}}{चन्द्रराशिः—\mbox{तुला}}%
{\anga{शुभः}{\time{49-57}{02:19(+1)}}\hspace{1ex}\uanga{शुक्लः}}%
{\anga{गरः}{\time{14-10}{12:01}}\hspace{1ex}\anga{वणिजः}{\time{42-5}{23:11}}\hspace{1ex}\uanga{विष्टिः}}{}
}
{अव्यङ्ग-सप्तमी\eventsep द्वादश-सप्तमी\eventsep \tamil{கழறிற்றறிவார்/சேரமான் பெருமாள் நாயனார் (36) குருபூஜை}\eventsep पापनाशनी-सप्तमी\eventsep \tamil{பெருமிழலைக் குறும்ப நாயனார் (22) குருபூஜை}\eventsep \tamil{ஸுந்தரமூர்த்தி நாயனார் (63) குருபூஜை/திருவாடி ஸ்வாதி}\eventsep तुलसीदास-जयन्ती\eventsep शीतला-सप्तमी}
{Tue} 
\cfoot{\rygdata{16:41--18:24}{09:47--11:30}{13:14--14:57}}
\caldata{AUGUST}{7}{\sunmonth{कर्कटः}{23}{}{श्रावणः}{वर्षऋतुः}{बुधः}{विकारी}{दक्षिणायनम्}{ग्रीष्मऋतुः}}
{\sunmoonrsdata{06:21}{20:06}{13:56}{00:23(+1)}{13:14}
{\kalas{04:59 05:40 10:01 09:06 10:56 18:16 11:51 14:36 17:21 19:11 20:47 22:40 23:57 02:31(+1)}}}
{\tnykdata{\anga{\tithi{8}{शुक्ल-अष्टमी}}{\time{39-7}{22:00}}\hspace{1ex}}%
{\anga{स्वाती}{\time{6-45}{09:03}}\hspace{1ex}}{चन्द्रराशिः—\mbox{तुला\RIGHTarrow{02:53(+1)}}}%
{\anga{शुक्लः}{\time{45-8}{00:24(+1)}}\hspace{1ex}\uanga{ब्रह्म}}%
{\anga{विष्टिः}{\time{10-23}{10:30}}\hspace{1ex}\anga{बवः}{\time{39-7}{22:00}}\hspace{1ex}\uanga{बालवः}}{}
}
{दुर्गा-व्रत-आरम्भः}
{Wed} 
\cfoot{\rygdata{13:14--14:57}{08:04--09:47}{11:31--13:14}}
\caldata{AUGUST}{8}{\sunmonth{कर्कटः}{24}{}{श्रावणः}{वर्षऋतुः}{गुरुः}{विकारी}{दक्षिणायनम्}{ग्रीष्मऋतुः}}
{\sunmoonrsdata{06:22}{20:05}{15:01}{01:05(+1)}{13:14}
{\kalas{05:00 05:41 10:01 09:07 10:56 18:15 11:51 14:36 17:20 19:10 20:46 22:40 23:57 02:31(+1)}}}
{\tnykdata{\anga{\tithi{9}{शुक्ल-नवमी}}{\time{37-49}{21:30}}\hspace{1ex}}%
{\anga{विशाखा}{\time{6-22}{08:55}}\hspace{1ex}}{चन्द्रराशिः—\mbox{वृश्चिकः}}%
{\anga{ब्रह्म}{\time{41-39}{23:02}}\hspace{1ex}\uanga{इन्द्रः}}%
{\anga{बालवः}{\time{8-15}{09:40}}\hspace{1ex}\anga{कौलवः}{\time{37-49}{21:30}}\hspace{1ex}\uanga{तैतिलः}}{}
}
{कौमारी-पूजा}
{Thu} 
\cfoot{\rygdata{14:56--16:39}{06:22--08:05}{09:48--11:31}}
\caldata{AUGUST}{9}{\sunmonth{कर्कटः}{25}{}{श्रावणः}{वर्षऋतुः}{शुक्रः}{विकारी}{दक्षिणायनम्}{ग्रीष्मऋतुः}}
{\sunmoonrsdata{06:23}{20:04}{16:03}{01:51(+1)}{13:13}
{\kalas{05:00 05:42 10:02 09:07 10:57 18:14 11:51 14:36 17:20 19:09 20:45 22:39 23:56 02:31(+1)}}}
{\tnykdata{\anga{\tithi{10}{शुक्ल-दशमी}}{\time{38-8}{21:38}}\hspace{1ex}}%
{\anga{अनूराधा}{\time{7-36}{09:26}}\hspace{1ex}}{चन्द्रराशिः—\mbox{वृश्चिकः}}%
{\anga{इन्द्रः}{\time{39-26}{22:09}}\hspace{1ex}\uanga{वैधृतिः}}%
{\anga{तैतिलः}{\time{7-46}{09:29}}\hspace{1ex}\anga{गरः}{\time{38-8}{21:38}}\hspace{1ex}\uanga{वणिजः}}{}
}
{\tamil{ஆடி~வெள்ளிக்கிழமை}\eventsep काञ्ची ५७ जगद्गुरु श्री-परमशिवेन्द्र सरस्वती २ आराधना~\#{४३४}\eventsep सेङ्गालिपुरम् अनन्तराम-दीक्षित-जयन्ती~\#{११७}\eventsep वेद/दधि-व्रत-आरम्भः\eventsep वरलक्ष्मी-व्रतम्}
{Fri} 
\cfoot{\rygdata{11:31--13:13}{16:39--18:21}{08:06--09:48}}
\caldata{AUGUST}{10}{\sunmonth{कर्कटः}{26}{}{श्रावणः}{वर्षऋतुः}{शनिः}{विकारी}{दक्षिणायनम्}{ग्रीष्मऋतुः}}
{\sunmoonrsdata{06:24}{20:03}{16:59}{02:42(+1)}{13:13}
{\kalas{05:01 05:42 10:02 09:07 10:57 18:14 11:51 14:35 17:19 19:08 20:44 22:38 23:56 02:31(+1)}}}
{\tnykdata{\anga{\tithi{11}{शुक्ल-एकादशी}}{\time{39-55}{22:22}}\hspace{1ex}}%
{\anga{ज्येष्ठा}{\time{10-23}{10:33}}\hspace{1ex}}{चन्द्रराशिः—\mbox{वृश्चिकः\RIGHTarrow{10:33}}}%
{\anga{वैधृतिः}{\time{38-22}{21:45}}\hspace{1ex}\uanga{विष्कम्भः}}%
{\anga{वणिजः}{\time{8-50}{09:56}}\hspace{1ex}\anga{विष्टिः}{\time{39-55}{22:22}}\hspace{1ex}\uanga{बवः}}{}
}
{काञ्ची २९ जगद्गुरु श्री-पूर्णबोधेन्द्र सरस्वती आराधना~\#{१४०२}\eventsep \tamil{கோட்புலி நாயனார் (55) குருபூஜை}\eventsep \tamil{கலிய நாயனார் (43) குருபூஜை}\eventsep सर्व-पवित्रोपान-एकादशी\eventsep वैधृति-श्राद्धम्}
{Sat} 
\cfoot{\rygdata{09:48--11:31}{14:56--16:38}{06:24--08:06}}
\caldata{AUGUST}{11}{\sunmonth{कर्कटः}{27}{}{श्रावणः}{वर्षऋतुः}{भानुः}{विकारी}{दक्षिणायनम्}{ग्रीष्मऋतुः}}
{\sunmoonrsdata{06:25}{20:02}{17:50}{03:37(+1)}{13:13}
{\kalas{05:02 05:43 10:02 09:08 10:57 18:13 11:51 14:35 17:18 19:07 20:43 22:38 23:56 02:31(+1)}}}
{\tnykdata{\anga{\tithi{12}{शुक्ल-द्वादशी}}{\time{42-59}{23:36}}\hspace{1ex}}%
{\anga{मूला}{\time{14-29}{12:12}}\hspace{1ex}}{चन्द्रराशिः—\mbox{धनुः}}%
{\anga{विष्कम्भः}{\time{38-19}{21:44}}\hspace{1ex}\uanga{प्रीतिः}}%
{\anga{बवः}{\time{11-17}{10:56}}\hspace{1ex}\anga{बालवः}{\time{42-59}{23:36}}\hspace{1ex}\uanga{कौलवः}}{}
}
{दामोदर-द्वादशी\eventsep दधि-व्रत-आरम्भः\eventsep हरिवासरः\RIGHTarrow{}04:38\eventsep शाकव्रत-समापनम्}
{Sun} 
\cfoot{\rygdata{18:20--20:02}{13:13--14:55}{16:37--18:20}}
\caldata{AUGUST}{12}{\sunmonth{कर्कटः}{28}{}{श्रावणः}{वर्षऋतुः}{सोमः}{विकारी}{दक्षिणायनम्}{ग्रीष्मऋतुः}}
{\sunmoonrsdata{06:25}{20:00}{18:34}{04:34(+1)}{13:13}
{\kalas{05:02 05:44 10:03 09:08 10:57 18:12 11:51 14:34 17:17 19:06 20:42 22:37 23:55 02:32(+1)}}}
{\tnykdata{\anga{\tithi{13}{शुक्ल-त्रयोदशी}}{\time{47-6}{01:16(+1)}}\hspace{1ex}}%
{\anga{पूर्वाषाढा}{\time{19-42}{14:19}}\hspace{1ex}}{चन्द्रराशिः—\mbox{धनुः\RIGHTarrow{20:54}}}%
{\anga{प्रीतिः}{\time{39-5}{22:03}}\hspace{1ex}\uanga{आयुष्मान्}}%
{\anga{कौलवः}{\time{14-54}{12:23}}\hspace{1ex}\anga{तैतिलः}{\time{47-6}{01:16(+1)}}\hspace{1ex}\uanga{गरः}}{}
}
{अनङ्ग-त्रयोदशी\eventsep सोम-प्रदोष-व्रतम्~20:01\RIGHTarrow{}20:42}
{Mon} 
\cfoot{\rygdata{08:07--09:49}{11:31--13:13}{14:55--16:37}}
\caldata{AUGUST}{13}{\sunmonth{कर्कटः}{29}{}{श्रावणः}{वर्षऋतुः}{मङ्गलः}{विकारी}{दक्षिणायनम्}{ग्रीष्मऋतुः}}
{\sunmoonrsdata{06:26}{19:59}{19:12}{05:32(+1)}{13:13}
{\kalas{05:03 05:45 10:03 09:09 10:57 18:11 11:51 14:34 17:17 19:05 20:41 22:36 23:55 02:32(+1)}}}
{\tnykdata{\anga{\tithi{14}{शुक्ल-चतुर्दशी}}{\time{52-2}{03:15(+1)}}\hspace{1ex}}%
{\anga{उत्तराषाढा}{\time{25-49}{16:46}}\hspace{1ex}}{चन्द्रराशिः—\mbox{मकरः}}%
{\anga{आयुष्मान्}{\time{40-29}{22:38}}\hspace{1ex}\uanga{सौभाग्यः}}%
{\anga{गरः}{\time{19-28}{14:13}}\hspace{1ex}\anga{वणिजः}{\time{52-2}{03:15(+1)}}\hspace{1ex}\uanga{विष्टिः}}{}
}
{}
{Tue} 
\cfoot{\rygdata{16:36--18:18}{09:49--11:31}{13:13--14:54}}
\caldata{AUGUST}{14}{\sunmonth{कर्कटः}{30}{}{श्रावणः}{वर्षऋतुः}{बुधः}{विकारी}{दक्षिणायनम्}{ग्रीष्मऋतुः}}
{\sunmoonrsdata{06:27}{19:58}{19:45}{---}{13:13}
{\kalas{05:03 05:45 10:03 09:09 10:57 18:10 11:51 14:34 17:16 19:04 20:40 22:36 23:54 02:32(+1)}}}
{\tnykdata{\anga{\tithi{15}{पौर्णमासी}}{\time{57-34}{05:29(+1)}}\hspace{1ex}}%
{\anga{श्रवणः}{\time{32-35}{19:29}}\hspace{1ex}}{चन्द्रराशिः—\mbox{मकरः}}%
{\anga{सौभाग्यः}{\time{42-23}{23:24}}\hspace{1ex}\uanga{शोभनः}}%
{\anga{विष्टिः}{\time{24-43}{16:20}}\hspace{1ex}\anga{बवः}{\time{57-34}{05:29(+1)}}\hspace{1ex}\uanga{बालवः}}{}
}
{ऋग्वेद-उपाकर्म\eventsep गायत्री-जयन्ती\eventsep हयग्रीव-जयन्ती\eventsep काञ्ची २० जगद्गुरु श्री-मूकशङ्करेन्द्र सरस्वती आराधना~\#{१५८३}\eventsep नारिकेल-पूर्णिमा\eventsep पार्वणव्रतम् पूर्णिमायाम्\eventsep पूर्णिमा-व्रतम्\eventsep पञ्च-पर्व-पूजा (पूर्णिमा)\eventsep रक्षाबन्धनम्\eventsep संस्कृत-दिवसः\eventsep सर्प-बलि-प्रारम्भः\eventsep श्रावण्युपवासः प्रायश्चित्तार्थः\eventsep वेङ्कटाचले पूर्णिमा-गरुड-सेवा\eventsep वैखानस-महर्षि-जयन्ती\eventsep यजुर्वेद-उपाकर्म\eventsep श्रवण-व्रतम्}
{Wed} 
\cfoot{\rygdata{13:13--14:54}{08:08--09:50}{11:31--13:13}}
\caldata{AUGUST}{15}{\sunmonth{कर्कटः}{31}{}{श्रावणः}{वर्षऋतुः}{गुरुः}{विकारी}{दक्षिणायनम्}{ग्रीष्मऋतुः}}
{\sunmoonsrdata{06:28}{19:57}{20:15}{06:30}{13:12}
{\kalas{05:04 05:46 10:04 09:10 10:58 18:09 11:51 14:33 17:15 19:03 20:39 22:35 23:54 02:32(+1)}}}
{\tnykdata{\fulltithi{\tithi{16}{कृष्ण-प्रथमा}}}%
{\anga{श्रविष्ठा}{\time{39-47}{22:23}}\hspace{1ex}}{चन्द्रराशिः—\mbox{मकरः\RIGHTarrow{08:55}}}%
{\anga{शोभनः}{\time{44-35}{00:18(+1)}}\hspace{1ex}\uanga{अतिगण्डः}}%
{\anga{बालवः}{\time{30-28}{18:39}}\hspace{1ex}\uanga{कौलवः}}{}
}
{काञ्ची जगद्गुरु श्री-जयेन्द्र सरस्वती जयन्ती~\#{८५}\eventsep पार्वण-प्रायश्चित्तावकाशः दर्शे\eventsep पूर्णमासेष्टिः\eventsep सहस्रगायत्रीजपः प्रायश्चित्तार्थः\eventsep स्थालीपाकः}
{Thu} 
\cfoot{\rygdata{14:53--16:35}{06:28--08:09}{09:50--11:31}}
\caldata{AUGUST}{16}{\sunmonth{कर्कटः}{32}{\mbox{कर्कटः{\tiny\RIGHTarrow}{00:03(+1)}}}{श्रावणः}{वर्षऋतुः}{शुक्रः}{विकारी}{दक्षिणायनम्}{ग्रीष्मऋतुः}}
{\sunmoonsrdata{06:29}{19:56}{20:43}{07:28}{13:12}
{\kalas{05:04 05:47 10:04 09:10 10:58 18:08 11:51 14:33 17:14 19:02 20:38 22:34 23:53 02:32(+1)}}}
{\tnykdata{\anga{\tithi{16}{कृष्ण-प्रथमा}}{\time{3-26}{07:51}}\hspace{1ex}}%
{\anga{शतभिषक्}{\time{47-13}{01:22(+1)}}\hspace{1ex}}{चन्द्रराशिः—\mbox{कुम्भः}}%
{\anga{अतिगण्डः}{\time{46-58}{01:16(+1)}}\hspace{1ex}\uanga{सुकर्म}}%
{\anga{कौलवः}{\time{3-26}{07:51}}\hspace{1ex}\anga{तैतिलः}{\time{36-29}{21:05}}\hspace{1ex}\uanga{गरः}}{}
}
{\tamil{ஆடி~வெள்ளிக்கிழமை}\eventsep अशून्यशयन-व्रतम्\eventsep सिंह-रवि-सङ्क्रमण-विष्णुपदी-पुण्यकालः~17:39\RIGHTarrow{}06:27(+1)}
{Fri} 
\cfoot{\rygdata{11:31--13:12}{16:34--18:15}{08:10--09:50}}
\caldata{AUGUST}{17}{\sunmonth{सिंहः}{1}{}{श्रावणः}{वर्षऋतुः}{शनिः}{विकारी}{दक्षिणायनम्}{वर्षऋतुः}}
{\sunmoonsrdata{06:30}{19:54}{21:08}{08:24}{13:12}
{\kalas{05:05 05:47 10:04 09:11 10:58 18:07 11:51 14:32 17:13 19:01 20:37 22:33 23:53 02:32(+1)}}}
{\tnykdata{\anga{\tithi{17}{कृष्ण-द्वितीया}}{\time{9-31}{10:18}}\hspace{1ex}}%
{\anga{पूर्वप्रोष्ठपदा}{\time{54-40}{04:22(+1)}}\hspace{1ex}}{चन्द्रराशिः—\mbox{कुम्भः\RIGHTarrow{21:37}}}%
{\anga{सुकर्म}{\time{49-22}{02:15(+1)}}\hspace{1ex}\uanga{धृतिः}}%
{\anga{गरः}{\time{9-31}{10:18}}\hspace{1ex}\anga{वणिजः}{\time{42-34}{23:31}}\hspace{1ex}\uanga{विष्टिः}}{}
}
{बृहती-वृक्षक-पूजा\eventsep भीम-चण्डी-जयन्ती\eventsep श्री-राघवेन्द्र-स्वामि-आराधना~\#{३४८}}
{Sat} 
\cfoot{\rygdata{09:51--11:31}{14:53--16:33}{06:30--08:10}}
\caldata{AUGUST}{18}{\sunmonth{सिंहः}{2}{}{श्रावणः}{वर्षऋतुः}{भानुः}{विकारी}{दक्षिणायनम्}{वर्षऋतुः}}
{\sunmoonsrdata{06:30}{19:53}{21:34}{09:21}{13:12}
{\kalas{05:06 05:48 10:04 09:11 10:58 18:06 11:51 14:32 17:12 18:59 20:35 22:33 23:52 02:32(+1)}}}
{\tnykdata{\anga{\tithi{18}{कृष्ण-तृतीया}}{\time{15-32}{12:43}}\hspace{1ex}}%
{\fullanga{उत्तरप्रोष्ठपदा}}{चन्द्रराशिः—\mbox{मीनः}}%
{\anga{धृतिः}{\time{51-34}{03:08(+1)}}\hspace{1ex}\uanga{शूलः}}%
{\anga{विष्टिः}{\time{15-32}{12:43}}\hspace{1ex}\anga{बवः}{\time{48-26}{01:53(+1)}}\hspace{1ex}\uanga{बालवः}}{}
}
{\tamil{ஆவணி~ஞாயிற்றுக்கிழமை}\eventsep बहुला-चतुर्थी\eventsep हेरम्ब-महागणपति महासङ्कटहर-चतुर्थी-व्रतम्\eventsep कज्जली-तृतीया\eventsep तुष्टि-प्राप्ति-तृतीया}
{Sun} 
\cfoot{\rygdata{18:13--19:53}{13:12--14:52}{16:32--18:13}}
\caldata{AUGUST}{19}{\sunmonth{सिंहः}{3}{}{श्रावणः}{वर्षऋतुः}{सोमः}{विकारी}{दक्षिणायनम्}{वर्षऋतुः}}
{\sunmoonsrdata{06:31}{19:52}{22:00}{10:17}{13:12}
{\kalas{05:06 05:49 10:05 09:11 10:58 18:05 11:51 14:32 17:12 18:58 20:34 22:32 23:52 02:32(+1)}}}
{\tnykdata{\anga{\tithi{19}{कृष्ण-चतुर्थी}}{\time{21-11}{15:00}}\hspace{1ex}}%
{\anga{उत्तरप्रोष्ठपदा}{\time{1-50}{07:15}}\hspace{1ex}}{चन्द्रराशिः—\mbox{मीनः}}%
{\anga{शूलः}{\time{53-23}{03:53(+1)}}\hspace{1ex}\uanga{गण्डः}}%
{\anga{बालवः}{\time{21-11}{15:00}}\hspace{1ex}\anga{कौलवः}{\time{53-47}{04:02(+1)}}\hspace{1ex}\uanga{तैतिलः}}{}
}
{}
{Mon} 
\cfoot{\rygdata{08:11--09:51}{11:31--13:12}{14:52--16:32}}
\caldata{AUGUST}{20}{\sunmonth{सिंहः}{4}{}{श्रावणः}{वर्षऋतुः}{मङ्गलः}{विकारी}{दक्षिणायनम्}{वर्षऋतुः}}
{\sunmoonsrdata{06:32}{19:50}{22:28}{11:15}{13:11}
{\kalas{05:07 05:49 10:05 09:12 10:58 18:04 11:51 14:31 17:11 18:57 20:33 22:31 23:51 02:32(+1)}}}
{\tnykdata{\anga{\tithi{20}{कृष्ण-पञ्चमी}}{\time{26-9}{17:00}}\hspace{1ex}}%
{\anga{रेवती}{\time{8-28}{09:56}}\hspace{1ex}}{चन्द्रराशिः—\mbox{मीनः\RIGHTarrow{09:56}}}%
{\anga{गण्डः}{\time{54-32}{04:21(+1)}}\hspace{1ex}\uanga{वृद्धिः}}%
{\anga{तैतिलः}{\time{26-9}{17:00}}\hspace{1ex}\anga{गरः}{\time{58-18}{05:52(+1)}}\hspace{1ex}\uanga{वणिजः}}{}
}
{भौमाश्विनी-पुण्यकालः~09:56\RIGHTarrow{}\eventsep रक्षा-पञ्चमी}
{Tue} 
\cfoot{\rygdata{16:31--18:11}{09:52--11:31}{13:11--14:51}}
\caldata{AUGUST}{21}{\sunmonth{सिंहः}{5}{}{श्रावणः}{वर्षऋतुः}{बुधः}{विकारी}{दक्षिणायनम्}{वर्षऋतुः}}
{\sunmoonsrdata{06:33}{19:49}{22:59}{12:14}{13:11}
{\kalas{05:07 05:50 10:05 09:12 10:58 18:03 11:51 14:31 17:10 18:56 20:32 22:30 23:51 02:32(+1)}}}
{\tnykdata{\anga{\tithi{21}{कृष्ण-षष्ठी}}{\time{30-6}{18:36}}\hspace{1ex}}%
{\anga{अश्विनी}{\time{14-13}{12:14}}\hspace{1ex}}{चन्द्रराशिः—\mbox{मेषः}}%
{\anga{वृद्धिः}{\time{54-47}{04:28(+1)}}\hspace{1ex}\uanga{ध्रुवः}}%
{\anga{वणिजः}{\time{30-6}{18:36}}\hspace{1ex}\uanga{विष्टिः}}{}
}
{हल-षष्ठी}
{Wed} 
\cfoot{\rygdata{13:11--14:50}{08:13--09:52}{11:32--13:11}}
\caldata{AUGUST}{22}{\sunmonth{सिंहः}{6}{}{श्रावणः}{वर्षऋतुः}{गुरुः}{विकारी}{दक्षिणायनम्}{वर्षऋतुः}}
{\sunmoonsrdata{06:34}{19:48}{23:34}{13:14}{13:11}
{\kalas{05:08 05:51 10:06 09:13 10:58 18:02 11:51 14:30 17:09 18:55 20:31 22:29 23:50 02:32(+1)}}}
{\tnykdata{\anga{\tithi{22}{कृष्ण-सप्तमी}}{\time{32-42}{19:39}}\hspace{1ex}}%
{\anga{अपभरणी}{\time{18-43}{14:03}}\hspace{1ex}}{चन्द्रराशिः—\mbox{मेषः\RIGHTarrow{20:25}}}%
{\anga{ध्रुवः}{\time{53-52}{04:07(+1)}}\hspace{1ex}\uanga{व्याघातः}}%
{\anga{विष्टिः}{\time{1-34}{07:12}}\hspace{1ex}\anga{बवः}{\time{32-42}{19:39}}\hspace{1ex}\uanga{बालवः}}{}
}
{कृत्तिका-व्रतम्\eventsep महाकाली-जयन्ती\eventsep नभो-मासः\RIGHTarrow{}03:02(+1)\eventsep पञ्च-पर्व-पूजा (अष्टमी)\eventsep श्रीकृष्णजन्माष्टमी}
{Thu} 
\cfoot{\rygdata{14:50--16:29}{06:34--08:13}{09:52--11:32}}
\caldata{AUGUST}{23}{\sunmonth{सिंहः}{7}{}{श्रावणः}{वर्षऋतुः}{शुक्रः}{विकारी}{दक्षिणायनम्}{वर्षऋतुः}}
{\sunmoonsrdata{06:35}{19:46}{00:16(+1)}{14:16}{13:10}
{\kalas{05:08 05:52 10:06 09:13 10:59 18:01 11:51 14:30 17:08 18:54 20:30 22:29 23:50 02:32(+1)}}}
{\tnykdata{\anga{\tithi{23}{कृष्ण-अष्टमी}}{\time{33-37}{20:02}}\hspace{1ex}}%
{\anga{कृत्तिका}{\time{21-39}{15:15}}\hspace{1ex}}{चन्द्रराशिः—\mbox{वृषभः}}%
{\anga{व्याघातः}{\time{51-34}{03:13(+1)}}\hspace{1ex}\uanga{हर्षणः}}%
{\anga{बालवः}{\time{3-21}{07:56}}\hspace{1ex}\anga{कौलवः}{\time{33-37}{20:02}}\hspace{1ex}\uanga{तैतिलः}}{}
}
{एकविंशति-दिवस-गणपति-व्रत-समापनम्\eventsep षडशीति-पुण्यकालः~03:02\RIGHTarrow{}03:02(+1)\eventsep काञ्ची २१ जगद्गुरु श्री-सार्वभौमगुरुः चन्द्रचूडेन्द्र सरस्वती आराधना~\#{१५७३}\eventsep मन्वादिः-(दक्षः-[९])\eventsep नन्दोत्सवः\eventsep \tamil{வரகூர் உறியடி உத்ஸவம்}\eventsep श्री-जयन्ती\eventsep श्रीकृष्णदेवराय-राज्याभिषेकः}
{Fri} 
\cfoot{\rygdata{11:32--13:10}{16:28--18:07}{08:14--09:53}}
\caldata{AUGUST}{24}{\sunmonth{सिंहः}{8}{}{श्रावणः}{वर्षऋतुः}{शनिः}{विकारी}{दक्षिणायनम्}{वर्षऋतुः}}
{\sunmoonsrdata{06:36}{19:45}{01:05(+1)}{15:17}{13:10}
{\kalas{05:09 05:52 10:06 09:13 10:59 18:00 11:51 14:29 17:07 18:52 20:28 22:28 23:49 02:32(+1)}}}
{\tnykdata{\anga{\tithi{24}{कृष्ण-नवमी}}{\time{32-41}{19:40}}\hspace{1ex}}%
{\anga{रोहिणी}{\time{22-49}{15:43}}\hspace{1ex}}{चन्द्रराशिः—\mbox{वृषभः\RIGHTarrow{03:40(+1)}}}%
{\anga{हर्षणः}{\time{47-48}{01:43(+1)}}\hspace{1ex}\uanga{वज्रम्}}%
{\anga{तैतिलः}{\time{3-23}{07:57}}\hspace{1ex}\anga{गरः}{\time{32-41}{19:40}}\hspace{1ex}\uanga{वणिजः}}{}
}
{अरविन्द-जयन्ती\eventsep चण्डिका-पूजा\eventsep काञ्ची २४ जगद्गुरु श्री-चित्सुखेन्द्र सरस्वती आराधना~\#{१४९३}\eventsep कौमार-पूजा\eventsep \tamil{திருச்செந்தூர் முருகன் ஆவணித் திருவிழா தொடக்கம்/கொடியேற்றம்}\eventsep शनिरोहिणी-पुण्यकालः\RIGHTarrow{}15:43}
{Sat} 
\cfoot{\rygdata{09:53--11:32}{14:49--16:27}{06:36--08:14}}
\caldata{AUGUST}{25}{\sunmonth{सिंहः}{9}{}{श्रावणः}{वर्षऋतुः}{भानुः}{विकारी}{दक्षिणायनम्}{वर्षऋतुः}}
{\sunmoonsrdata{06:36}{19:43}{02:03(+1)}{16:17}{13:10}
{\kalas{05:10 05:53 10:06 09:14 10:59 17:58 11:51 14:29 17:06 18:51 20:27 22:27 23:49 02:32(+1)}}}
{\tnykdata{\anga{\tithi{25}{कृष्ण-दशमी}}{\time{29-49}{18:32}}\hspace{1ex}}%
{\anga{मृगशीर्षम्}{\time{22-4}{15:26}}\hspace{1ex}}{चन्द्रराशिः—\mbox{मिथुनम्}}%
{\anga{वज्रम्}{\time{42-27}{23:35}}\hspace{1ex}\uanga{सिद्धिः}}%
{\anga{वणिजः}{\time{1-29}{07:12}}\hspace{1ex}\anga{विष्टिः}{\time{29-49}{18:32}}\hspace{1ex}\anga{बवः}{\time{57-42}{05:41(+1)}}\hspace{1ex}\uanga{बालवः}}{}
}
{\tamil{ஆவணி~ஞாயிற்றுக்கிழமை}\eventsep \tamil{திருச்செந்தூர் முருகன் ஆவணித் திருவிழா 2ம் நாள்}}
{Sun} 
\cfoot{\rygdata{18:05--19:43}{13:10--14:48}{16:27--18:05}}
\caldata{AUGUST}{26}{\sunmonth{सिंहः}{10}{}{श्रावणः}{वर्षऋतुः}{सोमः}{विकारी}{दक्षिणायनम्}{वर्षऋतुः}}
{\sunmoonsrdata{06:37}{19:42}{03:08(+1)}{17:12}{13:10}
{\kalas{05:10 05:54 10:07 09:14 10:59 17:57 11:51 14:28 17:05 18:50 20:26 22:26 23:48 02:32(+1)}}}
{\tnykdata{\anga{\tithi{26}{कृष्ण-एकादशी}}{\time{25-5}{16:39}}\hspace{1ex}}%
{\anga{आर्द्रा}{\time{19-26}{14:24}}\hspace{1ex}}{चन्द्रराशिः—\mbox{मिथुनम्}}%
{\anga{सिद्धिः}{\time{35-36}{20:52}}\hspace{1ex}\uanga{व्यतीपातः}}%
{\anga{बालवः}{\time{25-5}{16:39}}\hspace{1ex}\anga{कौलवः}{\time{52-5}{03:27(+1)}}\hspace{1ex}\uanga{तैतिलः}}{}
}
{हरिवासरः\RIGHTarrow{}22:05\eventsep सर्व-अजा-एकादशी\eventsep \tamil{திருச்செந்தூர் முருகன் ஆவணித் திருவிழா 3ம் நாள்—முருகன் பவனி}}
{Mon} 
\cfoot{\rygdata{08:15--09:54}{11:32--13:10}{14:48--16:26}}
\caldata{AUGUST}{27}{\sunmonth{सिंहः}{11}{}{श्रावणः}{वर्षऋतुः}{मङ्गलः}{विकारी}{दक्षिणायनम्}{वर्षऋतुः}}
{\sunmoonsrdata{06:38}{19:41}{04:20(+1)}{18:02}{13:09}
{\kalas{05:11 05:54 10:07 09:15 10:59 17:56 11:51 14:28 17:04 18:48 20:25 22:25 23:48 02:32(+1)}}}
{\tnykdata{\anga{\tithi{27}{कृष्ण-द्वादशी}}{\time{18-39}{14:06}}\hspace{1ex}}%
{\anga{पुनर्वसुः}{\time{15-6}{12:41}}\hspace{1ex}}{चन्द्रराशिः—\mbox{मिथुनम्\RIGHTarrow{07:10}}}%
{\anga{व्यतीपातः}{\time{27-24}{17:36}}\hspace{1ex}\uanga{वरीयान्}}%
{\anga{तैतिलः}{\time{18-39}{14:06}}\hspace{1ex}\anga{गरः}{\time{44-53}{00:36(+1)}}\hspace{1ex}\uanga{वणिजः}}{}
}
{जयन्ती-महाद्वादशी\eventsep प्रदोष-व्रतम्~19:41\RIGHTarrow{}20:25\eventsep रोहिणी-द्वादशी\eventsep \tamil{திருச்செந்தூர் முருகன் ஆவணித் திருவிழா 4ம் நாள்—யானை வாஹநத்தில் முருகன்-அம்பாள் பவனி}\eventsep व्यतीपात-श्राद्धम्}
{Tue} 
\cfoot{\rygdata{16:25--18:03}{09:54--11:32}{13:09--14:47}}
\caldata{AUGUST}{28}{\sunmonth{सिंहः}{12}{}{श्रावणः}{वर्षऋतुः}{बुधः}{विकारी}{दक्षिणायनम्}{वर्षऋतुः}}
{\sunmoonsrdata{06:39}{19:39}{05:35(+1)}{18:46}{13:09}
{\kalas{05:11 05:55 10:07 09:15 10:59 17:55 11:51 14:27 17:03 18:47 20:23 22:24 23:47 02:32(+1)}}}
{\tnykdata{\anga{\tithi{28}{कृष्ण-त्रयोदशी}}{\time{10-48}{10:58}}\hspace{1ex}}%
{\anga{पुष्यः}{\time{9-18}{10:22}}\hspace{1ex}}{चन्द्रराशिः—\mbox{कर्कटः}}%
{\anga{वरीयान्}{\time{18-4}{13:53}}\hspace{1ex}\uanga{परिघः}}%
{\anga{वणिजः}{\time{10-48}{10:58}}\hspace{1ex}\anga{विष्टिः}{\time{36-28}{21:14}}\hspace{1ex}\uanga{शकुनिः}}{}
}
{\tamil{அதிபத்த நாயனார் (41) குருபூஜை}\eventsep \tamil{செருத்துணை நாயனார் (53) குருபூஜை}\eventsep मासशिवरात्रिः\eventsep पञ्च-पर्व-पूजा (चतुर्दशी)\eventsep \tamil{திருச்செந்தூர் முருகன் ஆவணித் திருவிழா 5ம் நாள்}}
{Wed} 
\cfoot{\rygdata{13:09--14:47}{08:16--09:54}{11:32--13:09}}
\caldata{AUGUST}{29}{\sunmonth{सिंहः}{13}{}{श्रावणः}{वर्षऋतुः}{गुरुः}{विकारी}{दक्षिणायनम्}{वर्षऋतुः}}
{\sunmoonsrdata{06:40}{19:38}{---}{19:25}{13:09}
{\kalas{05:12 05:56 10:07 09:15 10:59 17:54 11:51 14:27 17:02 18:46 20:22 22:23 23:46 02:32(+1)}}}
{\tnykdata{\anga{\tithi{29}{कृष्ण-चतुर्दशी}}{\time{1-53}{07:25}}\hspace{1ex}\anga{\tithi{30}{अमावास्या}}{\time{52-22}{03:37(+1)}}\hspace{1ex}\avamA{}}%
{\anga{आश्रेषा}{\time{2-26}{07:38}}\hspace{1ex}\anga{मघा}{\time{54-57}{04:39(+1)}}\hspace{1ex}}{चन्द्रराशिः—\mbox{कर्कटः\RIGHTarrow{07:38}}}%
{\anga{परिघः}{\time{7-53}{09:49}}\hspace{1ex}\anga{शिवः}{\time{57-14}{05:34(+1)}}\hspace{1ex}\uanga{सिद्धः}}%
{\anga{शकुनिः}{\time{1-53}{07:25}}\hspace{1ex}\anga{चतुष्पात्}{\time{27-10}{17:32}}\hspace{1ex}\anga{नाग}{\time{52-22}{03:37(+1)}}\hspace{1ex}\uanga{किंस्तुघ्नः}}{}
}
{६४ योगिनी-पूजा\eventsep अघोर-चतुर्दशी\eventsep दर्भ-सङ्ग्रहः\eventsep देवी-पर्व-५\eventsep \tamil{இளையான்குடி மாற நாயனார் (3) குருபூஜை}\eventsep पार्वणव्रतम् अमावास्यायाम्\eventsep पञ्च-पर्व-पूजा (अमावास्या)\eventsep सर्व-श्रावण-अमावास्या (अलभ्यम्–पुष्कला)\eventsep \tamil{திருச்செந்தூர் முருகன் ஆவணித் திருவிழா 6ம் நாள்—வெள்ளித் தேர் பவனி}\eventsep वृषभ-पूजा}
{Thu} 
\cfoot{\rygdata{14:46--16:23}{06:40--08:17}{09:54--11:32}}
\caldata{AUGUST}{30}{\sunmonth{सिंहः}{14}{}{भाद्रपदः}{वर्षऋतुः}{शुक्रः}{विकारी}{दक्षिणायनम्}{वर्षऋतुः}}
{\sunmoonrsdata{06:41}{19:36}{06:50}{20:01}{13:08}
{\kalas{05:12 05:56 10:07 09:16 10:59 17:53 11:51 14:26 17:01 18:45 20:21 22:23 23:46 02:32(+1)}}}
{\tnykdata{\anga{\tithi{1}{शुक्ल-प्रथमा}}{\time{42-36}{23:43}}\hspace{1ex}}%
{\anga{पूर्वफल्गुनी}{\time{47-15}{01:35(+1)}}\hspace{1ex}}{चन्द्रराशिः—\mbox{सिंहः}}%
{\anga{सिद्धः}{\time{46-25}{01:15(+1)}}\hspace{1ex}\uanga{साध्यः}}%
{\anga{किंस्तुघ्नः}{\time{17-27}{13:40}}\hspace{1ex}\anga{बवः}{\time{42-36}{23:43}}\hspace{1ex}\uanga{बालवः}}{}
}
{दर्शेष्टिः\eventsep मृगशीर्ष-व्रतम्\eventsep पार्वण-प्रायश्चित्तावकाशः पौर्णमास्याम्\eventsep स्थालीपाकः\eventsep \tamil{திருச்செந்தூர் முருகன் ஆவணித் திருவிழா 7ம் நாள்—சிகப்பு சாத்தி அலங்காரம்}}
{Fri} 
\cfoot{\rygdata{11:32--13:08}{16:22--17:59}{08:18--09:55}}
\caldata{AUGUST}{31}{\sunmonth{सिंहः}{15}{}{भाद्रपदः}{वर्षऋतुः}{शनिः}{विकारी}{दक्षिणायनम्}{वर्षऋतुः}}
{\sunmoonrsdata{06:42}{19:35}{08:05}{20:35}{13:08}
{\kalas{05:13 05:57 10:08 09:16 10:59 17:52 11:51 14:26 17:00 18:43 20:19 22:22 23:45 02:32(+1)}}}
{\tnykdata{\anga{\tithi{2}{शुक्ल-द्वितीया}}{\time{33-6}{19:56}}\hspace{1ex}}%
{\anga{उत्तरफल्गुनी}{\time{39-51}{22:38}}\hspace{1ex}}{चन्द्रराशिः—\mbox{सिंहः\RIGHTarrow{06:50}}}%
{\anga{साध्यः}{\time{35-51}{21:02}}\hspace{1ex}\uanga{शुभः}}%
{\anga{बालवः}{\time{7-46}{09:48}}\hspace{1ex}\anga{कौलवः}{\time{33-6}{19:56}}\hspace{1ex}\anga{तैतिलः}{\time{58-36}{06:08(+1)}}\hspace{1ex}\uanga{गरः}}{}
}
{अङ्गारक-जयन्ती\eventsep चन्द्र-दर्शनम्~19:35\RIGHTarrow{}20:19\eventsep कल्कि-जयन्ती\eventsep \tamil{திருச்செந்தூர் முருகன் ஆவணித் திருவிழா 8ம் நாள்—பச்சை சாத்தி அலங்காரம்}}
{Sat} 
\cfoot{\rygdata{09:55--11:32}{14:45--16:21}{06:42--08:18}}
\caldata{SEPTEMBER}{1}{\sunmonth{सिंहः}{16}{}{भाद्रपदः}{वर्षऋतुः}{भानुः}{विकारी}{दक्षिणायनम्}{वर्षऋतुः}}
{\sunmoonrsdata{06:42}{19:33}{09:19}{21:09}{13:08}
{\kalas{05:13 05:58 10:08 09:16 10:59 17:51 11:51 14:25 16:59 18:42 20:18 22:21 23:45 02:32(+1)}}}
{\tnykdata{\anga{\tithi{3}{शुक्ल-तृतीया}}{\time{24-19}{16:26}}\hspace{1ex}}%
{\anga{हस्तः}{\time{33-14}{20:00}}\hspace{1ex}}{चन्द्रराशिः—\mbox{कन्या}}%
{\anga{शुभः}{\time{25-53}{17:04}}\hspace{1ex}\uanga{शुक्लः}}%
{\anga{गरः}{\time{24-19}{16:26}}\hspace{1ex}\anga{वणिजः}{\time{50-20}{02:51(+1)}}\hspace{1ex}\uanga{विष्टिः}}{}
}
{आदित्यहस्त-पुण्यकालः\eventsep \tamil{ஆவணி~ஞாயிற்றுக்கிழமை}\eventsep हरितालिका-व्रतम्\eventsep मन्वादिः-(तामसः-[४])\eventsep सामवेद-उपाकर्म\eventsep \tamil{திருச்செந்தூர் முருகன் ஆவணித் திருவிழா 9ம் நாள்}\eventsep विपत्तार-गौरी-व्रतम्}
{Sun} 
\cfoot{\rygdata{17:57--19:33}{13:08--14:44}{16:21--17:57}}
\caldata{SEPTEMBER}{2}{\sunmonth{सिंहः}{17}{}{भाद्रपदः}{वर्षऋतुः}{सोमः}{विकारी}{दक्षिणायनम्}{वर्षऋतुः}}
{\sunmoonrsdata{06:43}{19:32}{10:32}{21:44}{13:07}
{\kalas{05:14 05:58 10:08 09:17 10:59 17:49 11:51 14:24 16:58 18:41 20:17 22:20 23:44 02:32(+1)}}}
{\tnykdata{\anga{\tithi{4}{शुक्ल-चतुर्थी}}{\time{16-40}{13:23}}\hspace{1ex}}%
{\anga{चित्रा}{\time{27-50}{17:51}}\hspace{1ex}}{चन्द्रराशिः—\mbox{कन्या\RIGHTarrow{06:51}}}%
{\anga{शुक्लः}{\time{16-54}{13:29}}\hspace{1ex}\uanga{ब्रह्म}}%
{\anga{विष्टिः}{\time{16-40}{13:23}}\hspace{1ex}\anga{बवः}{\time{43-24}{00:05(+1)}}\hspace{1ex}\uanga{बालवः}}{}
}
{ऋषि-पञ्चमी-व्रतम्\eventsep \tamil{திருச்செந்தூர் முருகன் ஆவணித் திருவிழா 10ம் நாள்—தேர்}\eventsep श्रीविनायक-चतुर्थी}
{Mon} 
\cfoot{\rygdata{08:19--09:55}{11:31--13:07}{14:44--16:20}}
\caldata{SEPTEMBER}{3}{\sunmonth{सिंहः}{18}{}{भाद्रपदः}{वर्षऋतुः}{मङ्गलः}{विकारी}{दक्षिणायनम्}{वर्षऋतुः}}
{\sunmoonrsdata{06:44}{19:30}{11:43}{22:21}{13:07}
{\kalas{05:14 05:59 10:08 09:17 10:59 17:48 11:51 14:24 16:57 18:39 20:15 22:19 23:43 02:32(+1)}}}
{\tnykdata{\anga{\tithi{5}{शुक्ल-पञ्चमी}}{\time{10-32}{10:57}}\hspace{1ex}}%
{\anga{स्वाती}{\time{24-1}{16:21}}\hspace{1ex}}{चन्द्रराशिः—\mbox{तुला}}%
{\anga{ब्रह्म}{\time{9-10}{10:24}}\hspace{1ex}\uanga{इन्द्रः}}%
{\anga{बालवः}{\time{10-32}{10:57}}\hspace{1ex}\anga{कौलवः}{\time{38-9}{22:00}}\hspace{1ex}\uanga{तैतिलः}}{}
}
{षष्ठीदेवी-षष्ठी-व्रतम्\eventsep \tamil{திருச்செந்தூர் முருகன் ஆவணித் திருவிழா 11ம் நாள்}}
{Tue} 
\cfoot{\rygdata{16:19--17:55}{09:56--11:31}{13:07--14:43}}
\caldata{SEPTEMBER}{4}{\sunmonth{सिंहः}{19}{}{भाद्रपदः}{वर्षऋतुः}{बुधः}{विकारी}{दक्षिणायनम्}{वर्षऋतुः}}
{\sunmoonrsdata{06:45}{19:29}{12:51}{23:02}{13:07}
{\kalas{05:15 06:00 10:09 09:18 11:00 17:47 11:50 14:23 16:56 18:38 20:14 22:18 23:43 02:32(+1)}}}
{\tnykdata{\anga{\tithi{6}{शुक्ल-षष्ठी}}{\time{6-13}{09:14}}\hspace{1ex}}%
{\anga{विशाखा}{\time{22-4}{15:34}}\hspace{1ex}}{चन्द्रराशिः—\mbox{तुला\RIGHTarrow{09:42}}}%
{\anga{इन्द्रः}{\time{2-53}{07:54}}\hspace{1ex}\anga{वैधृतिः}{\time{58-15}{06:03(+1)}}\hspace{1ex}\uanga{विष्कम्भः}}%
{\anga{तैतिलः}{\time{6-13}{09:14}}\hspace{1ex}\anga{गरः}{\time{34-49}{20:41}}\hspace{1ex}\uanga{वणिजः}}{}
}
{अमुक्ताभरण-सप्तमी\eventsep बुधानुराधा-पुण्यकालः~15:35\RIGHTarrow{}\eventsep कुमारिका-स्वपनम्\eventsep ललिता-षष्ठी\eventsep मन्थन-षष्ठी\eventsep सूर्य-षष्ठी\eventsep \tamil{திருச்செந்தூர் ஆவணித் திருவிழா நிறைவு}\eventsep वैधृति-श्राद्धम्}
{Wed} 
\cfoot{\rygdata{13:07--14:42}{08:20--09:56}{11:31--13:07}}
\caldata{SEPTEMBER}{5}{\sunmonth{सिंहः}{20}{}{भाद्रपदः}{वर्षऋतुः}{गुरुः}{विकारी}{दक्षिणायनम्}{वर्षऋतुः}}
{\sunmoonrsdata{06:46}{19:27}{13:55}{23:48}{13:07}
{\kalas{05:15 06:01 10:09 09:18 11:00 17:46 11:50 14:23 16:55 18:37 20:13 22:17 23:42 02:32(+1)}}}
{\tnykdata{\anga{\tithi{7}{शुक्ल-सप्तमी}}{\time{3-53}{08:19}}\hspace{1ex}}%
{\anga{अनूराधा}{\time{22-5}{15:36}}\hspace{1ex}}{चन्द्रराशिः—\mbox{वृश्चिकः}}%
{\anga{विष्कम्भः}{\time{55-11}{04:50(+1)}}\hspace{1ex}\uanga{प्रीतिः}}%
{\anga{वणिजः}{\time{3-53}{08:19}}\hspace{1ex}\anga{विष्टिः}{\time{33-30}{20:10}}\hspace{1ex}\uanga{बवः}}{}
}
{अनन्तफल-सप्तमी\eventsep दूर्वाष्टमी\eventsep देवी-पर्व-६\eventsep कुक्कुटी-व्रतम्\eventsep \tamil{குலச்சிரை நாயனார் (21) குருபூஜை}}
{Thu} 
\cfoot{\rygdata{14:42--16:17}{06:46--08:21}{09:56--11:31}}
\caldata{SEPTEMBER}{6}{\sunmonth{सिंहः}{21}{}{भाद्रपदः}{वर्षऋतुः}{शुक्रः}{विकारी}{दक्षिणायनम्}{वर्षऋतुः}}
{\sunmoonrsdata{06:47}{19:26}{14:54}{00:38(+1)}{13:06}
{\kalas{05:16 06:01 10:09 09:18 11:00 17:45 11:50 14:22 16:54 18:35 20:11 22:16 23:42 02:32(+1)}}}
{\tnykdata{\anga{\tithi{8}{शुक्ल-अष्टमी}}{\time{3-34}{08:13}}\hspace{1ex}}%
{\anga{ज्येष्ठा}{\time{24-5}{16:25}}\hspace{1ex}}{चन्द्रराशिः—\mbox{वृश्चिकः\RIGHTarrow{16:25}}}%
{\anga{प्रीतिः}{\time{53-38}{04:14(+1)}}\hspace{1ex}\uanga{आयुष्मान्}}%
{\anga{बवः}{\time{3-34}{08:13}}\hspace{1ex}\anga{बालवः}{\time{34-10}{20:27}}\hspace{1ex}\uanga{कौलवः}}{}
}
{दधीचि-महर्षि-जयन्ति\eventsep राधाष्टमी}
{Fri} 
\cfoot{\rygdata{11:31--13:06}{16:16--17:51}{08:21--09:56}}
\caldata{SEPTEMBER}{7}{\sunmonth{सिंहः}{22}{}{भाद्रपदः}{वर्षऋतुः}{शनिः}{विकारी}{दक्षिणायनम्}{वर्षऋतुः}}
{\sunmoonrsdata{06:47}{19:24}{15:47}{01:32(+1)}{13:06}
{\kalas{05:17 06:02 10:09 09:19 11:00 17:43 11:50 14:22 16:53 18:34 20:10 22:15 23:41 02:32(+1)}}}
{\tnykdata{\anga{\tithi{9}{शुक्ल-नवमी}}{\time{5-11}{08:52}}\hspace{1ex}}%
{\anga{मूला}{\time{27-51}{17:56}}\hspace{1ex}}{चन्द्रराशिः—\mbox{धनुः}}%
{\anga{आयुष्मान्}{\time{53-25}{04:10(+1)}}\hspace{1ex}\uanga{सौभाग्यः}}%
{\anga{कौलवः}{\time{5-11}{08:52}}\hspace{1ex}\anga{तैतिलः}{\time{36-38}{21:27}}\hspace{1ex}\uanga{गरः}}{}
}
{अदुःखनवमी\eventsep गोधूमा-नवमी\eventsep गजेन्द्र-मोक्षः\eventsep \tamil{குங்கிலியக்கலய நாயனார் (10) குருபூஜை}\eventsep महालक्ष्मी-व्रत-आरम्भः\eventsep नन्दा-नवमी\eventsep \tamil{பிட்டுக்கு மண் சுமந்த லீலை}\eventsep तालनवमी}
{Sat} 
\cfoot{\rygdata{09:57--11:31}{14:40--16:15}{06:47--08:22}}
\caldata{SEPTEMBER}{8}{\sunmonth{सिंहः}{23}{}{भाद्रपदः}{वर्षऋतुः}{भानुः}{विकारी}{दक्षिणायनम्}{वर्षऋतुः}}
{\sunmoonrsdata{06:48}{19:23}{16:33}{02:29(+1)}{13:06}
{\kalas{05:17 06:03 10:09 09:19 11:00 17:42 11:50 14:21 16:52 18:32 20:09 22:14 23:40 02:32(+1)}}}
{\tnykdata{\anga{\tithi{10}{शुक्ल-दशमी}}{\time{8-26}{10:11}}\hspace{1ex}}%
{\anga{पूर्वाषाढा}{\time{33-6}{20:03}}\hspace{1ex}}{चन्द्रराशिः—\mbox{धनुः\RIGHTarrow{02:39(+1)}}}%
{\anga{सौभाग्यः}{\time{54-15}{04:31(+1)}}\hspace{1ex}\uanga{शोभनः}}%
{\anga{गरः}{\time{8-26}{10:11}}\hspace{1ex}\anga{वणिजः}{\time{40-35}{23:02}}\hspace{1ex}\uanga{विष्टिः}}{}
}
{\tamil{ஆவணி~ஞாயிற்றுக்கிழமை}\eventsep दशावतार-व्रतम्\eventsep वितस्तोत्सवः}
{Sun} 
\cfoot{\rygdata{17:49--19:23}{13:05--14:40}{16:14--17:49}}
\caldata{SEPTEMBER}{9}{\sunmonth{सिंहः}{24}{}{भाद्रपदः}{वर्षऋतुः}{सोमः}{विकारी}{दक्षिणायनम्}{वर्षऋतुः}}
{\sunmoonrsdata{06:49}{19:21}{17:13}{03:26(+1)}{13:05}
{\kalas{05:18 06:03 10:10 09:20 11:00 17:41 11:50 14:20 16:51 18:31 20:07 22:13 23:39 02:32(+1)}}}
{\tnykdata{\anga{\tithi{11}{शुक्ल-एकादशी}}{\time{12-58}{12:01}}\hspace{1ex}}%
{\anga{उत्तराषाढा}{\time{39-27}{22:36}}\hspace{1ex}}{चन्द्रराशिः—\mbox{मकरः}}%
{\anga{शोभनः}{\time{55-51}{05:10(+1)}}\hspace{1ex}\uanga{अतिगण्डः}}%
{\anga{विष्टिः}{\time{12-58}{12:01}}\hspace{1ex}\anga{बवः}{\time{45-38}{01:04(+1)}}\hspace{1ex}\uanga{बालवः}}{}
}
{हरिवासरः\RIGHTarrow{}18:32\eventsep कटदानोत्सवः\eventsep सर्व-परिवर्तिनी-एकादशी\eventsep वामन-जयन्ती}
{Mon} 
\cfoot{\rygdata{08:23--09:57}{11:31--13:05}{14:39--16:13}}
\caldata{SEPTEMBER}{10}{\sunmonth{सिंहः}{25}{}{भाद्रपदः}{वर्षऋतुः}{मङ्गलः}{विकारी}{दक्षिणायनम्}{वर्षऋतुः}}
{\sunmoonrsdata{06:50}{19:20}{17:48}{04:24(+1)}{13:05}
{\kalas{05:18 06:04 10:10 09:20 11:00 17:40 11:50 14:20 16:50 18:30 20:06 22:13 23:39 02:32(+1)}}}
{\tnykdata{\anga{\tithi{12}{शुक्ल-द्वादशी}}{\time{18-25}{14:12}}\hspace{1ex}}%
{\anga{श्रवणः}{\time{46-31}{01:26(+1)}}\hspace{1ex}}{चन्द्रराशिः—\mbox{मकरः}}%
{\anga{अतिगण्डः}{\time{57-57}{06:01(+1)}}\hspace{1ex}\uanga{सुकर्म}}%
{\anga{बालवः}{\time{18-25}{14:12}}\hspace{1ex}\anga{कौलवः}{\time{51-23}{03:23(+1)}}\hspace{1ex}\uanga{तैतिलः}}{}
}
{\tamil{ஓணம்}\eventsep अनन्त-द्वादशी\eventsep भुवनेश्वरी-जयन्ती\eventsep दधि-व्रत-समापनम्\eventsep पयोव्रत-आरम्भः\eventsep प्रदोष-व्रतम्~19:20\RIGHTarrow{}20:06\eventsep विजया/श्रवण-महाद्वादशी\eventsep श्रवण-व्रतम्}
{Tue} 
\cfoot{\rygdata{16:12--17:46}{09:57--11:31}{13:05--14:39}}
\caldata{SEPTEMBER}{11}{\sunmonth{सिंहः}{26}{}{भाद्रपदः}{वर्षऋतुः}{बुधः}{विकारी}{दक्षिणायनम्}{वर्षऋतुः}}
{\sunmoonrsdata{06:51}{19:18}{18:18}{05:21(+1)}{13:05}
{\kalas{05:19 06:05 10:10 09:20 11:00 17:38 11:50 14:19 16:49 18:28 20:04 22:12 23:38 02:32(+1)}}}
{\tnykdata{\anga{\tithi{13}{शुक्ल-त्रयोदशी}}{\time{24-23}{16:36}}\hspace{1ex}}%
{\anga{श्रविष्ठा}{\time{53-55}{04:25(+1)}}\hspace{1ex}}{चन्द्रराशिः—\mbox{मकरः\RIGHTarrow{14:55}}}%
{\fullanga{सुकर्म}}%
{\anga{तैतिलः}{\time{24-23}{16:36}}\hspace{1ex}\anga{गरः}{\time{57-28}{05:50(+1)}}\hspace{1ex}\uanga{वणिजः}}{}
}
{दूर्व-त्रि-व्रतम्\eventsep गो-त्रिरात्रि-व्रतम्\eventsep \tamil{நடராஜர் மஹாபிஷேகம்}\eventsep विवेकानन्द-भाषणं चिकागोनगरे~\#{१२६}}
{Wed} 
\cfoot{\rygdata{13:04--14:38}{08:24--09:58}{11:31--13:04}}
\caldata{SEPTEMBER}{12}{\sunmonth{सिंहः}{27}{}{भाद्रपदः}{वर्षऋतुः}{गुरुः}{विकारी}{दक्षिणायनम्}{वर्षऋतुः}}
{\sunmoonrsdata{06:52}{19:17}{18:46}{06:18(+1)}{13:04}
{\kalas{05:19 06:05 10:10 09:21 11:00 17:37 11:50 14:19 16:48 18:27 20:03 22:11 23:38 02:32(+1)}}}
{\tnykdata{\anga{\tithi{14}{शुक्ल-चतुर्दशी}}{\time{30-33}{19:05}}\hspace{1ex}}%
{\fullanga{शतभिषक्}}{चन्द्रराशिः—\mbox{कुम्भः}}%
{\anga{सुकर्म}{\time{0-14}{06:57}}\hspace{1ex}\uanga{धृतिः}}%
{\anga{वणिजः}{\time{30-33}{19:05}}\hspace{1ex}\uanga{विष्टिः}}{}
}
{अनन्त-चतुर्दशी\eventsep अनन्त-पद्मनाभ-व्रतम्\eventsep पञ्च-पर्व-पूजा (पूर्णिमा)}
{Thu} 
\cfoot{\rygdata{14:37--16:10}{06:52--08:25}{09:58--11:31}}
\caldata{SEPTEMBER}{13}{\sunmonth{सिंहः}{28}{}{भाद्रपदः}{वर्षऋतुः}{शुक्रः}{विकारी}{दक्षिणायनम्}{वर्षऋतुः}}
{\sunmoonrsdata{06:52}{19:15}{19:13}{---}{13:04}
{\kalas{05:20 06:06 10:10 09:21 11:00 17:36 11:50 14:18 16:47 18:26 20:02 22:10 23:37 02:31(+1)}}}
{\tnykdata{\anga{\tithi{15}{पौर्णमासी}}{\time{36-39}{21:32}}\hspace{1ex}}%
{\anga{शतभिषक्}{\time{1-22}{07:25}}\hspace{1ex}}{चन्द्रराशिः—\mbox{कुम्भः\RIGHTarrow{03:39(+1)}}}%
{\anga{धृतिः}{\time{2-35}{07:55}}\hspace{1ex}\uanga{शूलः}}%
{\anga{विष्टिः}{\time{3-36}{08:19}}\hspace{1ex}\anga{बवः}{\time{36-39}{21:32}}\hspace{1ex}\uanga{बालवः}}{}
}
{दिक्पाल-पूजा\eventsep काञ्ची ५९ जगद्गुरु श्री-भगवन्नाम बोधेन्द्र सरस्वती आराधना~\#{३२८}\eventsep पार्वणव्रतम् पूर्णिमायाम्\eventsep पूर्णिमा-व्रतम्\eventsep उमा-महेश्वर-व्रतम्\eventsep उपाङ्ग-ललिता-गौरी-व्रतम्\eventsep वेङ्कटाचले पूर्णिमा-गरुड-सेवा\eventsep विश्वरूप-यात्रा\eventsep यतिचातुर्मास्यव्रत-समापनम्}
{Fri} 
\cfoot{\rygdata{11:31--13:04}{16:09--17:42}{08:25--09:58}}
\caldata{SEPTEMBER}{14}{\sunmonth{सिंहः}{29}{}{भाद्रपदः}{वर्षऋतुः}{शनिः}{विकारी}{दक्षिणायनम्}{वर्षऋतुः}}
{\sunmoonsrdata{06:53}{19:14}{19:38}{07:15}{13:03}
{\kalas{05:20 06:07 10:11 09:21 11:00 17:35 11:49 14:17 16:45 18:24 20:00 22:09 23:36 02:31(+1)}}}
{\tnykdata{\anga{\tithi{16}{कृष्ण-प्रथमा}}{\time{42-30}{23:54}}\hspace{1ex}}%
{\anga{पूर्वप्रोष्ठपदा}{\time{8-42}{10:22}}\hspace{1ex}}{चन्द्रराशिः—\mbox{मीनः}}%
{\anga{शूलः}{\time{4-50}{08:49}}\hspace{1ex}\uanga{गण्डः}}%
{\anga{बालवः}{\time{9-36}{10:44}}\hspace{1ex}\anga{कौलवः}{\time{42-30}{23:54}}\hspace{1ex}\uanga{तैतिलः}}{}
}
{महालय-पक्ष-आरम्भः\eventsep पार्वण-प्रायश्चित्तावकाशः दर्शे\eventsep पूर्णमासेष्टिः\eventsep स्थालीपाकः}
{Sat} 
\cfoot{\rygdata{09:58--11:31}{14:36--16:08}{06:53--08:26}}
\caldata{SEPTEMBER}{15}{\sunmonth{सिंहः}{30}{}{भाद्रपदः}{वर्षऋतुः}{भानुः}{विकारी}{दक्षिणायनम्}{वर्षऋतुः}}
{\sunmoonsrdata{06:54}{19:12}{20:04}{08:12}{13:03}
{\kalas{05:21 06:07 10:11 09:22 11:00 17:34 11:49 14:17 16:44 18:23 19:59 22:08 23:36 02:31(+1)}}}
{\tnykdata{\anga{\tithi{17}{कृष्ण-द्वितीया}}{\time{47-57}{02:05(+1)}}\hspace{1ex}}%
{\anga{उत्तरप्रोष्ठपदा}{\time{15-43}{13:11}}\hspace{1ex}}{चन्द्रराशिः—\mbox{मीनः}}%
{\anga{गण्डः}{\time{6-49}{09:38}}\hspace{1ex}\uanga{वृद्धिः}}%
{\anga{तैतिलः}{\time{15-16}{13:01}}\hspace{1ex}\anga{गरः}{\time{47-57}{02:05(+1)}}\hspace{1ex}\uanga{वणिजः}}{}
}
{\tamil{ஆவணி~ஞாயிற்றுக்கிழமை}\eventsep अशून्यशयन-व्रतम्}
{Sun} 
\cfoot{\rygdata{17:40--19:12}{13:03--14:35}{16:07--17:40}}
\caldata{SEPTEMBER}{16}{\sunmonth{सिंहः}{31}{\mbox{सिंहः{\tiny\RIGHTarrow}{00:01(+1)}}}{भाद्रपदः}{वर्षऋतुः}{सोमः}{विकारी}{दक्षिणायनम्}{वर्षऋतुः}}
{\sunmoonsrdata{06:55}{19:10}{20:31}{09:09}{13:03}
{\kalas{05:21 06:08 10:11 09:22 11:00 17:32 11:49 14:16 16:43 18:21 19:57 22:07 23:35 02:31(+1)}}}
{\tnykdata{\anga{\tithi{18}{कृष्ण-तृतीया}}{\time{52-48}{04:02(+1)}}\hspace{1ex}}%
{\anga{रेवती}{\time{22-14}{15:49}}\hspace{1ex}}{चन्द्रराशिः—\mbox{मीनः\RIGHTarrow{15:49}}}%
{\anga{वृद्धिः}{\time{8-26}{10:18}}\hspace{1ex}\uanga{ध्रुवः}}%
{\anga{वणिजः}{\time{20-26}{15:06}}\hspace{1ex}\anga{विष्टिः}{\time{52-48}{04:02(+1)}}\hspace{1ex}\uanga{बवः}}{}
}
{गौरी-व्रतम्\eventsep कजरी-तृतीया\eventsep कन्या-रवि-सङ्क्रमण-षडशीति-पुण्यकालः~00:01(+1)\RIGHTarrow{}24:01(+1)}
{Mon} 
\cfoot{\rygdata{08:27--09:59}{11:31--13:03}{14:35--16:07}}
\caldata{SEPTEMBER}{17}{\sunmonth{कन्या}{1}{}{भाद्रपदः}{वर्षऋतुः}{मङ्गलः}{विकारी}{दक्षिणायनम्}{वर्षऋतुः}}
{\sunmoonsrdata{06:56}{19:09}{21:01}{10:07}{13:02}
{\kalas{05:22 06:09 10:11 09:22 11:00 17:31 11:49 14:16 16:42 18:20 19:56 22:06 23:34 02:31(+1)}}}
{\tnykdata{\anga{\tithi{19}{कृष्ण-चतुर्थी}}{\time{56-53}{05:41(+1)}}\hspace{1ex}}%
{\anga{अश्विनी}{\time{28-7}{18:11}}\hspace{1ex}}{चन्द्रराशिः—\mbox{मेषः}}%
{\anga{ध्रुवः}{\time{9-34}{10:45}}\hspace{1ex}\uanga{व्याघातः}}%
{\anga{बवः}{\time{24-56}{16:54}}\hspace{1ex}\anga{बालवः}{\time{56-53}{05:41(+1)}}\hspace{1ex}\uanga{कौलवः}}{}
}
{अङ्गारकी विघ्नराज-महागणपति सङ्कटहर-चतुर्थी-व्रतम्\eventsep अङ्गारक-चतुर्थी\eventsep भौमाश्विनी-पुण्यकालः\RIGHTarrow{}18:11\eventsep दिक्पाल-पूजा\eventsep विश्वकर्म-जयन्ती}
{Tue} 
\cfoot{\rygdata{16:06--17:37}{09:59--11:31}{13:02--14:34}}
\caldata{SEPTEMBER}{18}{\sunmonth{कन्या}{2}{}{भाद्रपदः}{वर्षऋतुः}{बुधः}{विकारी}{दक्षिणायनम्}{वर्षऋतुः}}
{\sunmoonsrdata{06:57}{19:07}{21:34}{11:07}{13:02}
{\kalas{05:22 06:09 10:12 09:23 11:00 17:30 11:49 14:15 16:41 18:19 19:55 22:05 23:34 02:31(+1)}}}
{\tnykdata{\anga{\tithi{20}{कृष्ण-पञ्चमी}}{\time{59-58}{06:56(+1)}}\hspace{1ex}}%
{\anga{अपभरणी}{\time{33-8}{20:12}}\hspace{1ex}}{चन्द्रराशिः—\mbox{मेषः\RIGHTarrow{02:38(+1)}}}%
{\anga{व्याघातः}{\time{10-3}{10:58}}\hspace{1ex}\uanga{हर्षणः}}%
{\anga{कौलवः}{\time{28-33}{18:22}}\hspace{1ex}\anga{तैतिलः}{\time{59-58}{06:56(+1)}}\hspace{1ex}\uanga{गरः}}{}
}
{महाभरणी\eventsep नाग-पूजा\eventsep सप्तर्षि-पूजा/अर्घ्यम्}
{Wed} 
\cfoot{\rygdata{13:02--14:33}{08:28--09:59}{11:31--13:02}}
\caldata{SEPTEMBER}{19}{\sunmonth{कन्या}{3}{}{भाद्रपदः}{वर्षऋतुः}{गुरुः}{विकारी}{दक्षिणायनम्}{वर्षऋतुः}}
{\sunmoonsrdata{06:58}{19:06}{22:13}{12:07}{13:02}
{\kalas{05:23 06:10 10:12 09:23 11:00 17:29 11:49 14:14 16:40 18:17 19:53 22:04 23:33 02:31(+1)}}}
{\tnykdata{\fulltithi{\tithi{21}{कृष्ण-षष्ठी}}}%
{\anga{कृत्तिका}{\time{37-2}{21:47}}\hspace{1ex}}{चन्द्रराशिः—\mbox{वृषभः}}%
{\anga{हर्षणः}{\time{9-44}{10:51}}\hspace{1ex}\uanga{वज्रम्}}%
{\anga{गरः}{\time{31-3}{19:23}}\hspace{1ex}\uanga{वणिजः}}{}
}
{चन्द्र-षष्ठी\eventsep काञ्ची ३३ जगद्गुरु श्री-सच्चिदानन्दघनेन्द्र सरस्वती २ आराधना~\#{१३२८}\eventsep कृत्तिका-व्रतम्\eventsep कपिल-षष्ठी}
{Thu} 
\cfoot{\rygdata{14:33--16:04}{06:58--08:29}{10:00--11:31}}
\caldata{SEPTEMBER}{20}{\sunmonth{कन्या}{4}{}{भाद्रपदः}{वर्षऋतुः}{शुक्रः}{विकारी}{दक्षिणायनम्}{वर्षऋतुः}}
{\sunmoonsrdata{06:58}{19:04}{22:58}{13:08}{13:01}
{\kalas{05:23 06:11 10:12 09:24 11:00 17:27 11:49 14:14 16:39 18:16 19:52 22:03 23:32 02:31(+1)}}}
{\tnykdata{\anga{\tithi{21}{कृष्ण-षष्ठी}}{\time{1-46}{07:41}}\hspace{1ex}}%
{\anga{रोहिणी}{\time{39-36}{22:49}}\hspace{1ex}}{चन्द्रराशिः—\mbox{वृषभः}}%
{\anga{वज्रम्}{\time{8-25}{10:20}}\hspace{1ex}\uanga{सिद्धिः}}%
{\anga{वणिजः}{\time{1-46}{07:41}}\hspace{1ex}\anga{विष्टिः}{\time{32-10}{19:51}}\hspace{1ex}\uanga{बवः}}{}
}
{\tamil{திருநாளைப்போவார் நாயனார் (17) குருபூஜை}\eventsep शृङ्गेरी ३५ जगद्गुरु श्री-अभिनव विद्यातीर्थ महास्वामी आराधना}
{Fri} 
\cfoot{\rygdata{11:31--13:01}{16:03--17:33}{08:29--10:00}}
\caldata{SEPTEMBER}{21}{\sunmonth{कन्या}{5}{}{भाद्रपदः}{वर्षऋतुः}{शनिः}{विकारी}{दक्षिणायनम्}{वर्षऋतुः}}
{\sunmoonsrdata{06:59}{19:03}{23:51}{14:06}{13:01}
{\kalas{05:24 06:12 10:12 09:24 11:00 17:26 11:49 14:13 16:38 18:14 19:51 22:02 23:32 02:31(+1)}}}
{\tnykdata{\anga{\tithi{22}{कृष्ण-सप्तमी}}{\time{2-7}{07:50}}\hspace{1ex}}%
{\anga{मृगशीर्षम्}{\time{40-35}{23:13}}\hspace{1ex}}{चन्द्रराशिः—\mbox{वृषभः\RIGHTarrow{11:06}}}%
{\anga{सिद्धिः}{\time{5-55}{09:21}}\hspace{1ex}\uanga{व्यतीपातः}}%
{\anga{बवः}{\time{2-7}{07:50}}\hspace{1ex}\anga{बालवः}{\time{31-42}{19:40}}\hspace{1ex}\uanga{कौलवः}}{}
}
{महालक्ष्मी-व्रत-समापनम्\eventsep महाव्यतीपात-श्राद्धम्\eventsep पञ्च-पर्व-पूजा (अष्टमी)\eventsep \tamil{புரட்டாசி~சனிக்கிழமை}}
{Sat} 
\cfoot{\rygdata{10:00--11:30}{14:31--16:02}{06:59--08:30}}
\caldata{SEPTEMBER}{22}{\sunmonth{कन्या}{6}{}{भाद्रपदः}{वर्षऋतुः}{भानुः}{विकारी}{दक्षिणायनम्}{वर्षऋतुः}}
{\sunmoonsrdata{07:00}{19:01}{00:51(+1)}{15:02}{13:01}
{\kalas{05:24 06:12 10:12 09:24 11:00 17:25 11:48 14:13 16:37 18:13 19:49 22:01 23:31 02:31(+1)}}}
{\tnykdata{\anga{\tithi{23}{कृष्ण-अष्टमी}}{\time{0-49}{07:20}}\hspace{1ex}\anga{\tithi{24}{कृष्ण-नवमी}}{\time{57-47}{06:07(+1)}}\hspace{1ex}\avamA{}}%
{\anga{आर्द्रा}{\time{39-51}{22:57}}\hspace{1ex}}{चन्द्रराशिः—\mbox{मिथुनम्}}%
{\anga{व्यतीपातः}{\time{2-6}{07:50}}\hspace{1ex}\anga{वरीयान्}{\time{56-54}{05:46(+1)}}\hspace{1ex}\uanga{परिघः}}%
{\anga{कौलवः}{\time{0-49}{07:20}}\hspace{1ex}\anga{तैतिलः}{\time{29-31}{18:49}}\hspace{1ex}\anga{गरः}{\time{57-47}{06:07(+1)}}\hspace{1ex}\uanga{वणिजः}}{}
}
{अविधवा-नवमी\eventsep अशोकाष्टमी-व्रत-आरम्भः\eventsep दुर्गा/गौरी-पूजा\eventsep जीमूतवाहन-पूजा\eventsep मध्याष्टमी\eventsep नभस्य-मासः/वर्षऋतुः\RIGHTarrow{}00:50(+1)\eventsep तुला-विषु-पुण्यकालः~20:50\RIGHTarrow{}04:50(+1)\eventsep वन-रक्षक-वैष्णव-हत्या~\#{२८९}}
{Sun} 
\cfoot{\rygdata{17:31--19:01}{13:01--14:31}{16:01--17:31}}
\caldata{SEPTEMBER}{23}{\sunmonth{कन्या}{7}{}{भाद्रपदः}{वर्षऋतुः}{सोमः}{विकारी}{दक्षिणायनम्}{वर्षऋतुः}}
{\sunmoonsrdata{07:01}{19:00}{01:58(+1)}{15:52}{13:00}
{\kalas{05:25 06:13 10:13 09:25 11:00 17:24 11:48 14:12 16:36 18:12 19:48 22:00 23:30 02:31(+1)}}}
{\tnykdata{\anga{\tithi{25}{कृष्ण-दशमी}}{\time{52-58}{04:12(+1)}}\hspace{1ex}}%
{\anga{पुनर्वसुः}{\time{37-23}{21:58}}\hspace{1ex}}{चन्द्रराशिः—\mbox{मिथुनम्\RIGHTarrow{16:17}}}%
{\anga{परिघः}{\time{50-15}{03:07(+1)}}\hspace{1ex}\uanga{शिवः}}%
{\anga{वणिजः}{\time{25-34}{17:15}}\hspace{1ex}\anga{विष्टिः}{\time{52-58}{04:12(+1)}}\hspace{1ex}\uanga{बवः}}{}
}
{दक्षिण-विषुव-दिनम्}
{Mon} 
\cfoot{\rygdata{08:31--10:01}{11:30--13:00}{14:30--16:00}}
\caldata{SEPTEMBER}{24}{\sunmonth{कन्या}{8}{}{भाद्रपदः}{वर्षऋतुः}{मङ्गलः}{विकारी}{दक्षिणायनम्}{वर्षऋतुः}}
{\sunmoonsrdata{07:02}{18:58}{03:09(+1)}{16:37}{13:00}
{\kalas{05:26 06:14 10:13 09:25 11:00 17:22 11:48 14:12 16:35 18:10 19:46 21:59 23:30 02:31(+1)}}}
{\tnykdata{\anga{\tithi{26}{कृष्ण-एकादशी}}{\time{46-32}{01:39(+1)}}\hspace{1ex}}%
{\anga{पुष्यः}{\time{33-15}{20:20}}\hspace{1ex}}{चन्द्रराशिः—\mbox{कर्कटः}}%
{\anga{शिवः}{\time{42-16}{23:56}}\hspace{1ex}\uanga{सिद्धः}}%
{\anga{बवः}{\time{19-55}{15:00}}\hspace{1ex}\anga{बालवः}{\time{46-32}{01:39(+1)}}\hspace{1ex}\uanga{कौलवः}}{}
}
{सर्व-इन्दिरा-एकादशी}
{Tue} 
\cfoot{\rygdata{15:59--17:28}{10:01--11:30}{13:00--14:29}}
\caldata{SEPTEMBER}{25}{\sunmonth{कन्या}{9}{}{भाद्रपदः}{वर्षऋतुः}{बुधः}{विकारी}{दक्षिणायनम्}{वर्षऋतुः}}
{\sunmoonsrdata{07:03}{18:56}{04:23(+1)}{17:17}{12:59}
{\kalas{05:26 06:14 10:13 09:25 11:01 17:21 11:48 14:11 16:34 18:09 19:45 21:58 23:29 02:31(+1)}}}
{\tnykdata{\anga{\tithi{27}{कृष्ण-द्वादशी}}{\time{38-44}{22:32}}\hspace{1ex}}%
{\anga{आश्रेषा}{\time{27-42}{18:07}}\hspace{1ex}}{चन्द्रराशिः—\mbox{कर्कटः\RIGHTarrow{18:07}}}%
{\anga{सिद्धः}{\time{33-9}{20:18}}\hspace{1ex}\uanga{साध्यः}}%
{\anga{कौलवः}{\time{12-46}{12:09}}\hspace{1ex}\anga{तैतिलः}{\time{38-44}{22:32}}\hspace{1ex}\uanga{गरः}}{}
}
{हरिवासरः\RIGHTarrow{}06:55\eventsep यति-महालयम्}
{Wed} 
\cfoot{\rygdata{13:00--14:29}{08:32--10:01}{11:30--13:00}}
\caldata{SEPTEMBER}{26}{\sunmonth{कन्या}{10}{}{भाद्रपदः}{वर्षऋतुः}{गुरुः}{विकारी}{दक्षिणायनम्}{वर्षऋतुः}}
{\sunmoonsrdata{07:03}{18:55}{05:37(+1)}{17:54}{12:59}
{\kalas{05:27 06:15 10:13 09:26 11:01 17:20 11:48 14:10 16:33 18:07 19:43 21:57 23:28 02:31(+1)}}}
{\tnykdata{\anga{\tithi{28}{कृष्ण-त्रयोदशी}}{\time{29-55}{19:02}}\hspace{1ex}}%
{\anga{मघा}{\time{21-1}{15:28}}\hspace{1ex}}{चन्द्रराशिः—\mbox{सिंहः}}%
{\anga{साध्यः}{\time{23-9}{16:19}}\hspace{1ex}\uanga{शुभः}}%
{\anga{गरः}{\time{4-24}{08:49}}\hspace{1ex}\anga{वणिजः}{\time{29-55}{19:02}}\hspace{1ex}\anga{विष्टिः}{\time{55-15}{05:10(+1)}}\hspace{1ex}\uanga{शकुनिः}}{}
}
{द्वापरयुगादिः\eventsep काञ्ची ४४ जगद्गुरु श्री-पूर्णबोधेन्द्र सरस्वती २ आराधना~\#{९८०}\eventsep मासशिवरात्रिः\eventsep पञ्च-पर्व-पूजा (चतुर्दशी)\eventsep प्रदोष-व्रतम्~18:55\RIGHTarrow{}19:44}
{Thu} 
\cfoot{\rygdata{14:28--15:57}{07:03--08:32}{10:01--11:30}}
\caldata{SEPTEMBER}{27}{\sunmonth{कन्या}{11}{}{भाद्रपदः}{वर्षऋतुः}{शुक्रः}{विकारी}{दक्षिणायनम्}{वर्षऋतुः}}
{\sunmoonsrdata{07:04}{18:53}{06:52(+1)}{18:29}{12:59}
{\kalas{05:27 06:16 10:13 09:26 11:01 17:19 11:48 14:10 16:32 18:06 19:42 21:56 23:28 02:31(+1)}}}
{\tnykdata{\anga{\tithi{29}{कृष्ण-चतुर्दशी}}{\time{20-28}{15:16}}\hspace{1ex}}%
{\anga{पूर्वफल्गुनी}{\time{13-38}{12:32}}\hspace{1ex}}{चन्द्रराशिः—\mbox{सिंहः\RIGHTarrow{17:47}}}%
{\anga{शुभः}{\time{12-34}{12:06}}\hspace{1ex}\uanga{शुक्लः}}%
{\anga{शकुनिः}{\time{20-28}{15:16}}\hspace{1ex}\anga{चतुष्पात्}{\time{45-40}{01:21(+1)}}\hspace{1ex}\uanga{नाग}}{}
}
{कात्यायनी-जयन्ती\eventsep महालय-पक्ष-समापनम्\eventsep पार्वणव्रतम् अमावास्यायाम्\eventsep पञ्च-पर्व-पूजा (अमावास्या)\eventsep सर्व-(भाद्रपद) महालय अमावास्या\eventsep शृङ्गेरी ३४ जगद्गुरु श्री-चन्द्रशेखर भारती आराधना\eventsep शस्त्रहतचतुर्दशी}
{Fri} 
\cfoot{\rygdata{11:30--12:59}{15:56--17:25}{08:33--10:02}}
\caldata{SEPTEMBER}{28}{\sunmonth{कन्या}{12}{}{भाद्रपदः}{वर्षऋतुः}{शनिः}{विकारी}{दक्षिणायनम्}{वर्षऋतुः}}
{\sunmoonsrdata{07:05}{18:52}{---}{19:03}{12:59}
{\kalas{05:28 06:16 10:14 09:27 11:01 17:18 11:48 14:09 16:31 18:05 19:41 21:55 23:27 02:31(+1)}}}
{\tnykdata{\anga{\tithi{30}{अमावास्या}}{\time{10-51}{11:26}}\hspace{1ex}}%
{\anga{उत्तरफल्गुनी}{\time{6-1}{09:30}}\hspace{1ex}\anga{हस्तः}{\time{58-41}{06:34(+1)}}\hspace{1ex}}{चन्द्रराशिः—\mbox{कन्या}}%
{\anga{शुक्लः}{\time{1-46}{07:48}}\hspace{1ex}\anga{ब्रह्म}{\time{51-12}{03:34(+1)}}\hspace{1ex}\uanga{इन्द्रः}}%
{\anga{नाग}{\time{10-51}{11:26}}\hspace{1ex}\anga{किंस्तुघ्नः}{\time{36-9}{21:33}}\hspace{1ex}\uanga{बवः}}{}
}
{अश्वशिरो-देव-पूजा\eventsep दर्शेष्टिः\eventsep गजच्छाया-योगः~09:30\RIGHTarrow{}11:26\eventsep पार्वण-प्रायश्चित्तावकाशः पौर्णमास्याम्\eventsep \tamil{புரட்டாசி~சனிக்கிழமை}\eventsep स्थालीपाकः\eventsep सुजन्मप्राप्ति-व्रतम्}
{Sat} 
\cfoot{\rygdata{10:02--11:30}{14:27--15:55}{07:05--08:34}}
\caldata{SEPTEMBER}{29}{\sunmonth{कन्या}{13}{}{आश्वयुजः}{शरदृतुः}{भानुः}{विकारी}{दक्षिणायनम्}{वर्षऋतुः}}
{\sunmoonrsdata{07:06}{18:50}{08:06}{19:38}{12:58}
{\kalas{05:28 06:17 10:14 09:27 11:01 17:16 11:48 14:09 16:29 18:03 19:39 21:54 23:27 02:31(+1)}}}
{\tnykdata{\anga{\tithi{1}{शुक्ल-प्रथमा}}{\time{1-32}{07:43}}\hspace{1ex}\anga{\tithi{2}{शुक्ल-द्वितीया}}{\time{53-2}{04:19(+1)}}\hspace{1ex}\avamA{}}%
{\anga{चित्रा}{\time{52-5}{03:56(+1)}}\hspace{1ex}}{चन्द्रराशिः—\mbox{कन्या\RIGHTarrow{17:12}}}%
{\anga{इन्द्रः}{\time{41-9}{23:34}}\hspace{1ex}\uanga{वैधृतिः}}%
{\anga{बवः}{\time{1-32}{07:43}}\hspace{1ex}\anga{बालवः}{\time{27-9}{17:58}}\hspace{1ex}\anga{कौलवः}{\time{53-2}{04:19(+1)}}\hspace{1ex}\uanga{तैतिलः}}{}
}
{अग्रसेन-महाराज-जयन्ती\eventsep चन्द्र-दर्शनम्~18:50\RIGHTarrow{}19:39\eventsep दौहित्र-प्रतिपत्\eventsep गृहदेवी-पूजा\eventsep स्तन्यवृद्धि-गौरी-व्रतम्\eventsep शरन्नवरात्र-आरम्भः}
{Sun} 
\cfoot{\rygdata{17:22--18:50}{12:58--14:26}{15:54--17:22}}
\caldata{SEPTEMBER}{30}{\sunmonth{कन्या}{14}{}{आश्वयुजः}{शरदृतुः}{सोमः}{विकारी}{दक्षिणायनम्}{वर्षऋतुः}}
{\sunmoonrsdata{07:07}{18:49}{09:20}{20:15}{12:58}
{\kalas{05:29 06:18 10:14 09:27 11:01 17:15 11:48 14:08 16:28 18:02 19:38 21:54 23:26 02:31(+1)}}}
{\tnykdata{\anga{\tithi{3}{शुक्ल-तृतीया}}{\time{45-43}{01:25(+1)}}\hspace{1ex}}%
{\anga{स्वाती}{\time{46-42}{01:48(+1)}}\hspace{1ex}}{चन्द्रराशिः—\mbox{तुला}}%
{\anga{वैधृतिः}{\time{32-2}{19:56}}\hspace{1ex}\uanga{विष्कम्भः}}%
{\anga{तैतिलः}{\time{19-10}{14:47}}\hspace{1ex}\anga{गरः}{\time{45-43}{01:25(+1)}}\hspace{1ex}\uanga{वणिजः}}{}
}
{मेघपालीय-तृतीया\eventsep वैधृति-श्राद्धम्}
{Mon} 
\cfoot{\rygdata{08:35--10:02}{11:30--12:58}{14:26--15:53}}
\caldata{OCTOBER}{1}{\sunmonth{कन्या}{15}{}{आश्वयुजः}{शरदृतुः}{मङ्गलः}{विकारी}{दक्षिणायनम्}{वर्षऋतुः}}
{\sunmoonrsdata{07:08}{18:47}{10:32}{20:56}{12:58}
{\kalas{05:29 06:19 10:14 09:28 11:01 17:14 11:48 14:07 16:27 18:01 19:37 21:52 23:25 02:31(+1)}}}
{\tnykdata{\anga{\tithi{4}{शुक्ल-चतुर्थी}}{\time{40-3}{23:09}}\hspace{1ex}}%
{\anga{विशाखा}{\time{42-57}{00:19(+1)}}\hspace{1ex}}{चन्द्रराशिः—\mbox{तुला\RIGHTarrow{18:37}}}%
{\anga{विष्कम्भः}{\time{24-12}{16:49}}\hspace{1ex}\uanga{प्रीतिः}}%
{\anga{वणिजः}{\time{12-39}{12:12}}\hspace{1ex}\anga{विष्टिः}{\time{40-3}{23:09}}\hspace{1ex}\uanga{बवः}}{}
}
{देवता-सुवासिनी-पूजा\eventsep सुखा-अङ्गारक-चतुर्थी}
{Tue} 
\cfoot{\rygdata{15:52--17:20}{10:03--11:30}{12:58--14:25}}
\caldata{OCTOBER}{2}{\sunmonth{कन्या}{16}{}{आश्वयुजः}{शरदृतुः}{बुधः}{विकारी}{दक्षिणायनम्}{वर्षऋतुः}}
{\sunmoonrsdata{07:09}{18:46}{11:41}{21:41}{12:57}
{\kalas{05:30 06:19 10:15 09:28 11:01 17:13 11:48 14:07 16:26 17:59 19:35 21:52 23:25 02:31(+1)}}}
{\tnykdata{\anga{\tithi{5}{शुक्ल-पञ्चमी}}{\time{36-21}{21:41}}\hspace{1ex}}%
{\anga{अनूराधा}{\time{41-11}{23:37}}\hspace{1ex}}{चन्द्रराशिः—\mbox{वृश्चिकः}}%
{\anga{प्रीतिः}{\time{17-52}{14:18}}\hspace{1ex}\uanga{आयुष्मान्}}%
{\anga{बवः}{\time{7-55}{10:19}}\hspace{1ex}\anga{बालवः}{\time{36-21}{21:41}}\hspace{1ex}\uanga{कौलवः}}{}
}
{बुधानुराधा-पुण्यकालः\eventsep ललिता-पञ्चमी\eventsep उपाङ्ग-ललिता-व्रतम्\eventsep शान्ति-पञ्चमी-व्रतम्}
{Wed} 
\cfoot{\rygdata{12:57--14:24}{08:36--10:03}{11:30--12:57}}
\caldata{OCTOBER}{3}{\sunmonth{कन्या}{17}{}{आश्वयुजः}{शरदृतुः}{गुरुः}{विकारी}{दक्षिणायनम्}{वर्षऋतुः}}
{\sunmoonrsdata{07:10}{18:44}{12:44}{22:31}{12:57}
{\kalas{05:30 06:20 10:15 09:29 11:01 17:12 11:47 14:06 16:25 17:58 19:34 21:51 23:24 02:31(+1)}}}
{\tnykdata{\anga{\tithi{6}{शुक्ल-षष्ठी}}{\time{34-48}{21:05}}\hspace{1ex}}%
{\anga{ज्येष्ठा}{\time{41-30}{23:46}}\hspace{1ex}}{चन्द्रराशिः—\mbox{वृश्चिकः\RIGHTarrow{23:46}}}%
{\anga{आयुष्मान्}{\time{13-14}{12:27}}\hspace{1ex}\uanga{सौभाग्यः}}%
{\anga{कौलवः}{\time{5-17}{09:16}}\hspace{1ex}\anga{तैतिलः}{\time{34-48}{21:05}}\hspace{1ex}\uanga{गरः}}{}
}
{षष्ठी-व्रतम्}
{Thu} 
\cfoot{\rygdata{14:24--15:50}{07:10--08:36}{10:03--11:30}}
\caldata{OCTOBER}{4}{\sunmonth{कन्या}{18}{}{आश्वयुजः}{शरदृतुः}{शुक्रः}{विकारी}{दक्षिणायनम्}{वर्षऋतुः}}
{\sunmoonrsdata{07:11}{18:43}{13:41}{23:25}{12:57}
{\kalas{05:31 06:21 10:15 09:29 11:01 17:10 11:47 14:06 16:24 17:56 19:33 21:50 23:23 02:31(+1)}}}
{\tnykdata{\anga{\tithi{7}{शुक्ल-सप्तमी}}{\time{35-24}{21:20}}\hspace{1ex}}%
{\anga{मूला}{\time{43-56}{00:45(+1)}}\hspace{1ex}}{चन्द्रराशिः—\mbox{धनुः}}%
{\anga{सौभाग्यः}{\time{10-19}{11:18}}\hspace{1ex}\uanga{शोभनः}}%
{\anga{गरः}{\time{4-48}{09:06}}\hspace{1ex}\anga{वणिजः}{\time{35-24}{21:20}}\hspace{1ex}\uanga{विष्टिः}}{}
}
{काञ्ची ४५ जगद्गुरु श्री-परमशिवेन्द्र सरस्वती १ आराधना~\#{९५९}\eventsep पत्रिका-प्रवेश-पूजा\eventsep सरस्वती-आवाहनम्\eventsep शुभ-सप्तमी}
{Fri} 
\cfoot{\rygdata{11:30--12:57}{15:50--17:16}{08:37--10:04}}
\caldata{OCTOBER}{5}{\sunmonth{कन्या}{19}{}{आश्वयुजः}{शरदृतुः}{शनिः}{विकारी}{दक्षिणायनम्}{वर्षऋतुः}}
{\sunmoonrsdata{07:12}{18:41}{14:30}{00:22(+1)}{12:56}
{\kalas{05:32 06:22 10:15 09:29 11:01 17:09 11:47 14:05 16:23 17:55 19:31 21:49 23:23 02:31(+1)}}}
{\tnykdata{\anga{\tithi{8}{शुक्ल-अष्टमी}}{\time{38-0}{22:24}}\hspace{1ex}}%
{\anga{पूर्वाषाढा}{\time{48-16}{02:30(+1)}}\hspace{1ex}}{चन्द्रराशिः—\mbox{धनुः}}%
{\anga{शोभनः}{\time{9-1}{10:48}}\hspace{1ex}\uanga{अतिगण्डः}}%
{\anga{विष्टिः}{\time{6-27}{09:46}}\hspace{1ex}\anga{बवः}{\time{38-0}{22:24}}\hspace{1ex}\uanga{बालवः}}{}
}
{भद्रकाळी-पूजा\eventsep दुर्गाष्टमी\eventsep काल-त्रिरात्रि-व्रतम्\eventsep \tamil{புரட்டாசி~சனிக்கிழமை}}
{Sat} 
\cfoot{\rygdata{10:04--11:30}{14:22--15:49}{07:12--08:38}}
\caldata{OCTOBER}{6}{\sunmonth{कन्या}{20}{}{आश्वयुजः}{शरदृतुः}{भानुः}{विकारी}{दक्षिणायनम्}{वर्षऋतुः}}
{\sunmoonrsdata{07:12}{18:40}{15:13}{01:19(+1)}{12:56}
{\kalas{05:32 06:22 10:16 09:30 11:01 17:08 11:47 14:05 16:22 17:54 19:30 21:48 23:22 02:31(+1)}}}
{\tnykdata{\anga{\tithi{9}{शुक्ल-नवमी}}{\time{42-18}{00:08(+1)}}\hspace{1ex}}%
{\anga{उत्तराषाढा}{\time{54-9}{04:52(+1)}}\hspace{1ex}}{चन्द्रराशिः—\mbox{धनुः\RIGHTarrow{09:03}}}%
{\anga{अतिगण्डः}{\time{9-8}{10:52}}\hspace{1ex}\uanga{सुकर्म}}%
{\anga{बालवः}{\time{9-57}{11:11}}\hspace{1ex}\anga{कौलवः}{\time{42-18}{00:08(+1)}}\hspace{1ex}\uanga{तैतिलः}}{}
}
{\tamil{ஏனாதிநாத நாயனார் (8) குருபூஜை}\eventsep भद्रकाळी-व्रतम्\eventsep काञ्ची १९ जगद्गुरु श्री-मार्ताण्ड विद्याघनेन्द्र सरस्वती आराधना~\#{१६२२}\eventsep महानवमी/सरस्वती-पूजा\eventsep मन्वादिः-(स्वायम्भुवः-[१])\eventsep शरन्नवरात्र-समापनम्}
{Sun} 
\cfoot{\rygdata{17:14--18:40}{12:56--14:22}{15:48--17:14}}
\caldata{OCTOBER}{7}{\sunmonth{कन्या}{21}{}{आश्वयुजः}{शरदृतुः}{सोमः}{विकारी}{दक्षिणायनम्}{वर्षऋतुः}}
{\sunmoonrsdata{07:13}{18:38}{15:49}{02:17(+1)}{12:56}
{\kalas{05:33 06:23 10:16 09:30 11:02 17:07 11:47 14:04 16:21 17:52 19:28 21:47 23:22 02:31(+1)}}}
{\tnykdata{\anga{\tithi{10}{शुक्ल-दशमी}}{\time{47-45}{02:20(+1)}}\hspace{1ex}}%
{\fullanga{श्रवणः}}{चन्द्रराशिः—\mbox{मकरः}}%
{\anga{सुकर्म}{\time{10-21}{11:22}}\hspace{1ex}\uanga{धृतिः}}%
{\anga{तैतिलः}{\time{14-53}{13:11}}\hspace{1ex}\anga{गरः}{\time{47-45}{02:20(+1)}}\hspace{1ex}\uanga{वणिजः}}{}
}
{आयुध-पूजा\eventsep बुद्ध-जयन्ती\eventsep दशहरा\eventsep दुर्गा-पूजा\eventsep गङ्गावतरणम्\eventsep कूष्माण्ड-दशमी\eventsep मध्वाचार्य-जयन्ती~\#{७८२}\eventsep सोमश्रावणी-पुण्यकालः\eventsep विजयादशमी\eventsep युद्धदेवता-आराधना\eventsep शमी-पूजा\eventsep श्रवण-व्रतम्}
{Mon} 
\cfoot{\rygdata{08:39--10:04}{11:30--12:56}{14:21--15:47}}
\caldata{OCTOBER}{8}{\sunmonth{कन्या}{22}{}{आश्वयुजः}{शरदृतुः}{मङ्गलः}{विकारी}{दक्षिणायनम्}{वर्षऋतुः}}
{\sunmoonrsdata{07:14}{18:37}{16:21}{03:15(+1)}{12:55}
{\kalas{05:33 06:24 10:16 09:31 11:02 17:06 11:47 14:04 16:20 17:51 19:27 21:46 23:21 02:31(+1)}}}
{\tnykdata{\anga{\tithi{11}{शुक्ल-एकादशी}}{\time{53-55}{04:48(+1)}}\hspace{1ex}}%
{\anga{श्रवणः}{\time{1-1}{07:39}}\hspace{1ex}}{चन्द्रराशिः—\mbox{मकरः\RIGHTarrow{21:08}}}%
{\anga{धृतिः}{\time{12-17}{12:09}}\hspace{1ex}\uanga{शूलः}}%
{\anga{वणिजः}{\time{20-46}{15:33}}\hspace{1ex}\anga{विष्टिः}{\time{53-55}{04:48(+1)}}\hspace{1ex}\uanga{बवः}}{}
}
{स्मार्त-पापाङ्कुशा-एकादशी}
{Tue} 
\cfoot{\rygdata{15:46--17:11}{10:05--11:30}{12:55--14:21}}
\caldata{OCTOBER}{9}{\sunmonth{कन्या}{23}{}{आश्वयुजः}{शरदृतुः}{बुधः}{विकारी}{दक्षिणायनम्}{वर्षऋतुः}}
{\sunmoonrsdata{07:15}{18:35}{16:50}{04:12(+1)}{12:55}
{\kalas{05:34 06:25 10:17 09:31 11:02 17:04 11:47 14:03 16:19 17:50 19:26 21:45 23:20 02:31(+1)}}}
{\tnykdata{\fulltithi{\tithi{12}{शुक्ल-द्वादशी}}}%
{\anga{श्रविष्ठा}{\time{8-29}{10:39}}\hspace{1ex}}{चन्द्रराशिः—\mbox{कुम्भः}}%
{\anga{शूलः}{\time{14-35}{13:05}}\hspace{1ex}\uanga{गण्डः}}%
{\anga{बवः}{\time{27-4}{18:05}}\hspace{1ex}\uanga{बालवः}}{}
}
{द्विदल-व्रत-आरम्भः\eventsep हरिवासरः\RIGHTarrow{}11:27\eventsep पयोव्रत-समापनम्\eventsep वैष्णव-पापाङ्कुशा-एकादशी\eventsep व्यञ्जुली-महाद्वादशी}
{Wed} 
\cfoot{\rygdata{12:55--14:20}{08:40--10:05}{11:30--12:55}}
\caldata{OCTOBER}{10}{\sunmonth{कन्या}{24}{}{आश्वयुजः}{शरदृतुः}{गुरुः}{विकारी}{दक्षिणायनम्}{वर्षऋतुः}}
{\sunmoonrsdata{07:16}{18:34}{17:16}{05:09(+1)}{12:55}
{\kalas{05:35 06:25 10:17 09:32 11:02 17:03 11:47 14:03 16:18 17:49 19:25 21:44 23:20 02:31(+1)}}}
{\tnykdata{\anga{\tithi{12}{शुक्ल-द्वादशी}}{\time{0-13}{07:21}}\hspace{1ex}}%
{\anga{शतभिषक्}{\time{16-1}{13:41}}\hspace{1ex}}{चन्द्रराशिः—\mbox{कुम्भः}}%
{\anga{गण्डः}{\time{16-55}{14:02}}\hspace{1ex}\uanga{वृद्धिः}}%
{\anga{बालवः}{\time{0-13}{07:21}}\hspace{1ex}\anga{कौलवः}{\time{33-21}{20:37}}\hspace{1ex}\uanga{तैतिलः}}{}
}
{\tamil{நரசிங்கமுனையரைய நாயனார் (40) குருபூஜை}\eventsep प्रदोष-व्रतम्~18:34\RIGHTarrow{}19:25}
{Thu} 
\cfoot{\rygdata{14:20--15:44}{07:16--08:41}{10:05--11:30}}
\caldata{OCTOBER}{11}{\sunmonth{कन्या}{25}{}{आश्वयुजः}{शरदृतुः}{शुक्रः}{विकारी}{दक्षिणायनम्}{वर्षऋतुः}}
{\sunmoonrsdata{07:17}{18:32}{17:42}{06:05(+1)}{12:55}
{\kalas{05:35 06:26 10:17 09:32 11:02 17:02 11:47 14:02 16:17 17:47 19:23 21:44 23:19 02:31(+1)}}}
{\tnykdata{\anga{\tithi{13}{शुक्ल-त्रयोदशी}}{\time{6-21}{09:50}}\hspace{1ex}}%
{\anga{पूर्वप्रोष्ठपदा}{\time{23-18}{16:36}}\hspace{1ex}}{चन्द्रराशिः—\mbox{कुम्भः\RIGHTarrow{09:53}}}%
{\anga{वृद्धिः}{\time{19-3}{14:54}}\hspace{1ex}\uanga{ध्रुवः}}%
{\anga{तैतिलः}{\time{6-21}{09:50}}\hspace{1ex}\anga{गरः}{\time{39-16}{23:00}}\hspace{1ex}\uanga{वणिजः}}{}
}
{\tamil{நடராஜர் மஹாபிஷேகம்}}
{Fri} 
\cfoot{\rygdata{11:30--12:55}{15:43--17:08}{08:41--10:06}}
\caldata{OCTOBER}{12}{\sunmonth{कन्या}{26}{}{आश्वयुजः}{शरदृतुः}{शनिः}{विकारी}{दक्षिणायनम्}{वर्षऋतुः}}
{\sunmoonrsdata{07:18}{18:31}{18:08}{07:03(+1)}{12:54}
{\kalas{05:36 06:27 10:17 09:32 11:02 17:01 11:47 14:02 16:16 17:46 19:22 21:43 23:19 02:31(+1)}}}
{\tnykdata{\anga{\tithi{14}{शुक्ल-चतुर्दशी}}{\time{12-0}{12:06}}\hspace{1ex}}%
{\anga{उत्तरप्रोष्ठपदा}{\time{30-3}{19:20}}\hspace{1ex}}{चन्द्रराशिः—\mbox{मीनः}}%
{\anga{ध्रुवः}{\time{20-46}{15:37}}\hspace{1ex}\uanga{व्याघातः}}%
{\anga{वणिजः}{\time{12-0}{12:06}}\hspace{1ex}\anga{विष्टिः}{\time{44-37}{01:09(+1)}}\hspace{1ex}\uanga{बवः}}{}
}
{काञ्ची ३६ जगद्गुरु श्री-चित्सुखानन्देन्द्र सरस्वती आराधना~\#{१२६२}\eventsep को-जागर्ति-व्रतम्\eventsep पञ्च-पर्व-पूजा (पूर्णिमा)\eventsep \tamil{புரட்டாசி~சனிக்கிழமை}\eventsep वेङ्कटाचले पूर्णिमा-गरुड-सेवा}
{Sat} 
\cfoot{\rygdata{10:06--11:30}{14:18--15:43}{07:18--08:42}}
\caldata{OCTOBER}{13}{\sunmonth{कन्या}{27}{}{आश्वयुजः}{शरदृतुः}{भानुः}{विकारी}{दक्षिणायनम्}{वर्षऋतुः}}
{\sunmoonrsdata{07:19}{18:29}{18:35}{---}{12:54}
{\kalas{05:36 06:28 10:18 09:33 11:02 17:00 11:47 14:01 16:15 17:45 19:21 21:42 23:18 02:31(+1)}}}
{\tnykdata{\anga{\tithi{15}{पौर्णमासी}}{\time{17-1}{14:07}}\hspace{1ex}}%
{\anga{रेवती}{\time{36-10}{21:47}}\hspace{1ex}}{चन्द्रराशिः—\mbox{मीनः\RIGHTarrow{21:47}}}%
{\anga{व्याघातः}{\time{21-59}{16:06}}\hspace{1ex}\uanga{हर्षणः}}%
{\anga{बवः}{\time{17-1}{14:07}}\hspace{1ex}\anga{बालवः}{\time{49-15}{03:01(+1)}}\hspace{1ex}\uanga{कौलवः}}{}
}
{कौमुदी-उत्सवः\eventsep कुमार-पूर्णिमा/महा-अश्विनी\eventsep कुन्ती-(पार्वती)-व्रतम्\eventsep लक्ष्मी-इन्द्र-कुबेर-पूजा\eventsep मीराबाई-जयन्ती~\#{५२२}\eventsep पार्वणव्रतम् पूर्णिमायाम्\eventsep पूर्णिमा-व्रतम्\eventsep सप्तम-अपरपक्ष-आरम्भः\eventsep वाल्मीकि-महर्षि-जयन्ती\eventsep शरद्-पूर्णिमा}
{Sun} 
\cfoot{\rygdata{17:06--18:29}{12:54--14:18}{15:42--17:06}}
\caldata{OCTOBER}{14}{\sunmonth{कन्या}{28}{}{आश्वयुजः}{शरदृतुः}{सोमः}{विकारी}{दक्षिणायनम्}{वर्षऋतुः}}
{\sunmoonsrdata{07:20}{18:28}{19:04}{08:01}{12:54}
{\kalas{05:37 06:28 10:18 09:33 11:02 16:59 11:47 14:01 16:14 17:43 19:19 21:41 23:18 02:31(+1)}}}
{\tnykdata{\anga{\tithi{16}{कृष्ण-प्रथमा}}{\time{21-16}{15:51}}\hspace{1ex}}%
{\anga{अश्विनी}{\time{41-33}{23:57}}\hspace{1ex}}{चन्द्रराशिः—\mbox{मेषः}}%
{\anga{हर्षणः}{\time{22-37}{16:23}}\hspace{1ex}\uanga{वज्रम्}}%
{\anga{कौलवः}{\time{21-16}{15:51}}\hspace{1ex}\anga{तैतिलः}{\time{53-8}{04:35(+1)}}\hspace{1ex}\uanga{गरः}}{}
}
{अप्पय्य-दीक्षित-जयन्ती~\#{५०१}\eventsep अशून्यशयन-व्रतम्\eventsep जयावाप्ति-व्रतम्\eventsep पार्वण-प्रायश्चित्तावकाशः दर्शे\eventsep पूर्णमासेष्टिः\eventsep \tamil{ருத்ர~பஶுபதி நாயனார் (16) குருபூஜை}\eventsep स्थालीपाकः}
{Mon} 
\cfoot{\rygdata{08:43--10:07}{11:30--12:54}{14:17--15:41}}
\caldata{OCTOBER}{15}{\sunmonth{कन्या}{29}{}{आश्वयुजः}{शरदृतुः}{मङ्गलः}{विकारी}{दक्षिणायनम्}{वर्षऋतुः}}
{\sunmoonsrdata{07:21}{18:27}{19:36}{09:01}{12:54}
{\kalas{05:38 06:29 10:18 09:34 11:03 16:58 11:47 14:00 16:13 17:42 19:18 21:40 23:17 02:31(+1)}}}
{\tnykdata{\anga{\tithi{17}{कृष्ण-द्वितीया}}{\time{24-44}{17:15}}\hspace{1ex}}%
{\anga{अपभरणी}{\time{46-8}{01:48(+1)}}\hspace{1ex}}{चन्द्रराशिः—\mbox{मेषः}}%
{\anga{वज्रम्}{\time{22-39}{16:24}}\hspace{1ex}\uanga{सिद्धिः}}%
{\anga{गरः}{\time{24-44}{17:15}}\hspace{1ex}\anga{वणिजः}{\time{56-10}{05:49(+1)}}\hspace{1ex}\uanga{विष्टिः}}{}
}
{चन्द्रोदय-गौरी-व्रतम्}
{Tue} 
\cfoot{\rygdata{15:40--17:03}{10:07--11:30}{12:54--14:17}}
\caldata{OCTOBER}{16}{\sunmonth{कन्या}{30}{}{आश्वयुजः}{शरदृतुः}{बुधः}{विकारी}{दक्षिणायनम्}{वर्षऋतुः}}
{\sunmoonsrdata{07:22}{18:25}{20:13}{10:02}{12:53}
{\kalas{05:38 06:30 10:19 09:34 11:03 16:57 11:47 14:00 16:13 17:41 19:17 21:40 23:17 02:31(+1)}}}
{\tnykdata{\anga{\tithi{18}{कृष्ण-तृतीया}}{\time{27-20}{18:18}}\hspace{1ex}}%
{\anga{कृत्तिका}{\time{49-51}{03:19(+1)}}\hspace{1ex}}{चन्द्रराशिः—\mbox{मेषः\RIGHTarrow{08:13}}}%
{\anga{सिद्धिः}{\time{22-0}{16:10}}\hspace{1ex}\uanga{व्यतीपातः}}%
{\anga{विष्टिः}{\time{27-20}{18:18}}\hspace{1ex}\anga{बवः}{\time{58-18}{06:41(+1)}}\hspace{1ex}\uanga{बालवः}}{}
}
{कृत्तिका-व्रतम्\eventsep कनक-गणेश-व्रतम्\eventsep करक-चतुर्थी\eventsep ललिता-गौरी-व्रतम्\eventsep वक्रतुण्ड-महागणपति सङ्कटहर-चतुर्थी-व्रतम्}
{Wed} 
\cfoot{\rygdata{12:54--14:16}{08:45--10:08}{11:31--12:54}}
\caldata{OCTOBER}{17}{\sunmonth{तुला}{1}{\mbox{कन्या{\tiny\RIGHTarrow}{12:01}}}{आश्वयुजः}{शरदृतुः}{गुरुः}{विकारी}{दक्षिणायनम्}{शरदृतुः}}
{\sunmoonsrdata{07:23}{18:24}{20:56}{11:02}{12:53}
{\kalas{05:39 06:31 10:19 09:35 11:03 16:56 11:47 13:59 16:12 17:40 19:16 21:39 23:16 02:31(+1)}}}
{\tnykdata{\anga{\tithi{19}{कृष्ण-चतुर्थी}}{\time{28-59}{18:58}}\hspace{1ex}}%
{\anga{रोहिणी}{\time{52-38}{04:26(+1)}}\hspace{1ex}}{चन्द्रराशिः—\mbox{वृषभः}}%
{\anga{व्यतीपातः}{\time{20-39}{15:38}}\hspace{1ex}\uanga{वरीयान्}}%
{\anga{बालवः}{\time{28-59}{18:58}}\hspace{1ex}\anga{कौलवः}{\time{59-26}{07:09(+1)}}\hspace{1ex}\uanga{तैतिलः}}{}
}
{आकाशदीप-आरम्भः\eventsep तुला-कावेरी-स्नान-आरम्भः\eventsep तुला-सङ्क्रमण-पुण्यकालः~08:01\RIGHTarrow{}16:01\eventsep व्यतीपात-श्राद्धम्}
{Thu} 
\cfoot{\rygdata{14:16--15:39}{07:23--08:45}{10:08--11:31}}
\caldata{OCTOBER}{18}{\sunmonth{तुला}{2}{}{आश्वयुजः}{शरदृतुः}{शुक्रः}{विकारी}{दक्षिणायनम्}{शरदृतुः}}
{\sunmoonsrdata{07:24}{18:22}{21:46}{12:01}{12:53}
{\kalas{05:40 06:32 10:19 09:36 11:03 16:55 11:47 13:59 16:11 17:38 19:15 21:38 23:16 02:31(+1)}}}
{\tnykdata{\anga{\tithi{20}{कृष्ण-पञ्चमी}}{\time{29-34}{19:13}}\hspace{1ex}}%
{\anga{मृगशीर्षम्}{\time{54-19}{05:07(+1)}}\hspace{1ex}}{चन्द्रराशिः—\mbox{वृषभः\RIGHTarrow{16:50}}}%
{\anga{वरीयान्}{\time{18-28}{14:47}}\hspace{1ex}\uanga{परिघः}}%
{\anga{तैतिलः}{\time{29-34}{19:13}}\hspace{1ex}\anga{गरः}{\time{59-26}{07:10(+1)}}\hspace{1ex}\uanga{वणिजः}}{}
}
{भृगुवार-सुब्रह्मण्य-व्रतम्\eventsep घोटक-पञ्चमी}
{Fri} 
\cfoot{\rygdata{11:31--12:53}{15:38--17:00}{08:46--10:08}}
\caldata{OCTOBER}{19}{\sunmonth{तुला}{3}{}{आश्वयुजः}{शरदृतुः}{शनिः}{विकारी}{दक्षिणायनम्}{शरदृतुः}}
{\sunmoonsrdata{07:25}{18:21}{22:43}{12:57}{12:53}
{\kalas{05:40 06:32 10:20 09:36 11:03 16:54 11:47 13:58 16:10 17:37 19:13 21:37 23:15 02:31(+1)}}}
{\tnykdata{\anga{\tithi{21}{कृष्ण-षष्ठी}}{\time{28-57}{19:00}}\hspace{1ex}}%
{\anga{आर्द्रा}{\time{54-46}{05:19(+1)}}\hspace{1ex}}{चन्द्रराशिः—\mbox{मिथुनम्}}%
{\anga{परिघः}{\time{15-22}{13:34}}\hspace{1ex}\uanga{शिवः}}%
{\anga{वणिजः}{\time{28-57}{19:00}}\hspace{1ex}\anga{विष्टिः}{\time{58-10}{06:41(+1)}}\hspace{1ex}\uanga{बवः}}{}
}
{सेङ्गालिपुरम् अनन्तराम-दीक्षित-आराधना~\#{५०}}
{Sat} 
\cfoot{\rygdata{10:09--11:31}{14:15--15:37}{07:25--08:47}}
\caldata{OCTOBER}{20}{\sunmonth{तुला}{4}{}{आश्वयुजः}{शरदृतुः}{भानुः}{विकारी}{दक्षिणायनम्}{शरदृतुः}}
{\sunmoonsrdata{07:26}{18:20}{23:46}{13:47}{12:53}
{\kalas{05:41 06:33 10:20 09:37 11:04 16:53 11:47 13:58 16:09 17:36 19:12 21:36 23:15 02:32(+1)}}}
{\tnykdata{\anga{\tithi{22}{कृष्ण-सप्तमी}}{\time{27-1}{18:14}}\hspace{1ex}}%
{\anga{पुनर्वसुः}{\time{53-53}{04:59(+1)}}\hspace{1ex}}{चन्द्रराशिः—\mbox{मिथुनम्\RIGHTarrow{23:07}}}%
{\anga{शिवः}{\time{11-15}{11:56}}\hspace{1ex}\uanga{सिद्धः}}%
{\anga{बवः}{\time{27-1}{18:14}}\hspace{1ex}\anga{बालवः}{\time{55-32}{05:39(+1)}}\hspace{1ex}\uanga{कौलवः}}{}
}
{भानुसप्तमी\eventsep पञ्च-पर्व-पूजा (अष्टमी)}
{Sun} 
\cfoot{\rygdata{16:58--18:20}{12:53--14:14}{15:36--16:58}}
\caldata{OCTOBER}{21}{\sunmonth{तुला}{5}{}{आश्वयुजः}{शरदृतुः}{सोमः}{विकारी}{दक्षिणायनम्}{शरदृतुः}}
{\sunmoonsrdata{07:27}{18:18}{00:53(+1)}{14:33}{12:52}
{\kalas{05:42 06:34 10:20 09:37 11:04 16:52 11:47 13:58 16:08 17:35 19:11 21:36 23:14 02:32(+1)}}}
{\tnykdata{\anga{\tithi{23}{कृष्ण-अष्टमी}}{\time{23-40}{16:55}}\hspace{1ex}}%
{\anga{पुष्यः}{\time{51-37}{04:06(+1)}}\hspace{1ex}}{चन्द्रराशिः—\mbox{कर्कटः}}%
{\anga{सिद्धः}{\time{6-1}{09:51}}\hspace{1ex}\anga{साध्यः}{\time{59-43}{07:20(+1)}}\hspace{1ex}\uanga{शुभः}}%
{\anga{कौलवः}{\time{23-40}{16:55}}\hspace{1ex}\anga{तैतिलः}{\time{51-30}{04:03(+1)}}\hspace{1ex}\uanga{गरः}}{}
}
{जीमूतवाहन-पूजा\eventsep जीवपुत्रिकाष्टमी\eventsep कालाष्टमी\eventsep मङ्गल-व्रतम्\eventsep महालक्ष्मी-व्रतम्}
{Mon} 
\cfoot{\rygdata{08:48--10:10}{11:31--12:52}{14:14--15:36}}
\caldata{OCTOBER}{22}{\sunmonth{तुला}{6}{}{आश्वयुजः}{शरदृतुः}{मङ्गलः}{विकारी}{दक्षिणायनम्}{शरदृतुः}}
{\sunmoonsrdata{07:28}{18:17}{02:03(+1)}{15:13}{12:52}
{\kalas{05:42 06:35 10:21 09:38 11:04 16:51 11:47 13:57 16:07 17:34 19:10 21:35 23:14 02:32(+1)}}}
{\tnykdata{\anga{\tithi{24}{कृष्ण-नवमी}}{\time{18-56}{15:02}}\hspace{1ex}}%
{\anga{आश्रेषा}{\time{48-0}{02:40(+1)}}\hspace{1ex}}{चन्द्रराशिः—\mbox{कर्कटः\RIGHTarrow{02:40(+1)}}}%
{\anga{शुभः}{\time{52-16}{04:22(+1)}}\hspace{1ex}\uanga{शुक्लः}}%
{\anga{गरः}{\time{18-56}{15:02}}\hspace{1ex}\anga{वणिजः}{\time{46-6}{01:54(+1)}}\hspace{1ex}\uanga{विष्टिः}}{}
}
{भीमसेन-जयन्ती}
{Tue} 
\cfoot{\rygdata{15:35--16:56}{10:10--11:31}{12:52--14:14}}
\caldata{OCTOBER}{23}{\sunmonth{तुला}{7}{}{आश्वयुजः}{शरदृतुः}{बुधः}{विकारी}{दक्षिणायनम्}{शरदृतुः}}
{\sunmoonsrdata{07:29}{18:16}{03:14(+1)}{15:50}{12:52}
{\kalas{05:43 06:36 10:21 09:38 11:04 16:50 11:48 13:57 16:06 17:33 19:09 21:34 23:13 02:32(+1)}}}
{\tnykdata{\anga{\tithi{25}{कृष्ण-दशमी}}{\time{12-54}{12:39}}\hspace{1ex}}%
{\anga{मघा}{\time{43-11}{00:45(+1)}}\hspace{1ex}}{चन्द्रराशिः—\mbox{सिंहः}}%
{\anga{शुक्लः}{\time{43-51}{01:01(+1)}}\hspace{1ex}\uanga{ब्रह्म}}%
{\anga{विष्टिः}{\time{12-54}{12:39}}\hspace{1ex}\anga{बवः}{\time{39-29}{23:16}}\hspace{1ex}\uanga{बालवः}}{}
}
{इष-मासः\RIGHTarrow{}10:19\eventsep स्मार्त-रमा-एकादशी (गृहस्थ)\eventsep विष्णुपदी-पुण्यकालः~03:55\RIGHTarrow{}16:43}
{Wed} 
\cfoot{\rygdata{12:52--14:13}{08:50--10:10}{11:31--12:52}}
\caldata{OCTOBER}{24}{\sunmonth{तुला}{8}{}{आश्वयुजः}{शरदृतुः}{गुरुः}{विकारी}{दक्षिणायनम्}{शरदृतुः}}
{\sunmoonsrdata{07:30}{18:14}{04:27(+1)}{16:24}{12:52}
{\kalas{05:44 06:37 10:22 09:39 11:05 16:49 11:48 13:57 16:06 17:32 19:08 21:33 23:13 02:32(+1)}}}
{\tnykdata{\anga{\tithi{26}{कृष्ण-एकादशी}}{\time{5-46}{09:48}}\hspace{1ex}\anga{\tithi{27}{कृष्ण-द्वादशी}}{\time{57-50}{06:38(+1)}}\hspace{1ex}\avamA{}}%
{\anga{पूर्वफल्गुनी}{\time{37-24}{22:27}}\hspace{1ex}}{चन्द्रराशिः—\mbox{सिंहः\RIGHTarrow{03:50(+1)}}}%
{\anga{ब्रह्म}{\time{34-39}{21:21}}\hspace{1ex}\uanga{इन्द्रः}}%
{\anga{बालवः}{\time{5-46}{09:48}}\hspace{1ex}\anga{कौलवः}{\time{31-53}{20:15}}\hspace{1ex}\anga{तैतिलः}{\time{57-50}{06:38(+1)}}\hspace{1ex}\uanga{गरः}}{}
}
{\tamil{சத்தி நாயனார் (44) குருபூஜை}\eventsep गोवत्स-द्वादशी\eventsep हरिवासरः\RIGHTarrow{}15:02\eventsep स्मार्त-रमा-एकादशी (सन्न्यस्थ)\eventsep ताम्रपर्णी-अन्त्य-पुष्कर-आरम्भः\eventsep त्रिस्पर्शा-महाद्वादशी\eventsep वैष्णव-रमा-एकादशी\eventsep वसुदेव-पूजा\eventsep व्याघ्र-द्वादशी\eventsep शृङ्गेरी ३४ जगद्गुरु श्री-चन्द्रशेखर भारती-३ जयन्ती}
{Thu} 
\cfoot{\rygdata{14:13--15:33}{07:30--08:50}{10:11--11:32}}
\caldata{OCTOBER}{25}{\sunmonth{तुला}{9}{}{आश्वयुजः}{शरदृतुः}{शुक्रः}{विकारी}{दक्षिणायनम्}{शरदृतुः}}
{\sunmoonsrdata{07:31}{18:13}{05:40(+1)}{16:57}{12:52}
{\kalas{05:45 06:38 10:22 09:39 11:05 16:48 11:48 13:56 16:05 17:30 19:06 21:33 23:13 02:32(+1)}}}
{\tnykdata{\anga{\tithi{28}{कृष्ण-त्रयोदशी}}{\time{49-23}{03:16(+1)}}\hspace{1ex}}%
{\anga{उत्तरफल्गुनी}{\time{31-0}{19:55}}\hspace{1ex}}{चन्द्रराशिः—\mbox{कन्या}}%
{\anga{इन्द्रः}{\time{24-55}{17:29}}\hspace{1ex}\uanga{वैधृतिः}}%
{\anga{गरः}{\time{23-37}{16:58}}\hspace{1ex}\anga{वणिजः}{\time{49-23}{03:16(+1)}}\hspace{1ex}\uanga{विष्टिः}}{}
}
{(यम)-दीप-त्रयोदशी\eventsep धन-त्रयोदशी\eventsep धन्वन्तरि-जयन्ती\eventsep गो-त्रिरात्रि-व्रतम्\eventsep मासशिवरात्रिः\eventsep प्रदोष-व्रतम्~18:13\RIGHTarrow{}19:07}
{Fri} 
\cfoot{\rygdata{11:32--12:52}{15:33--16:53}{08:51--10:11}}
\caldata{OCTOBER}{26}{\sunmonth{तुला}{10}{}{आश्वयुजः}{शरदृतुः}{शनिः}{विकारी}{दक्षिणायनम्}{शरदृतुः}}
{\sunmoonsrdata{07:32}{18:12}{06:53(+1)}{17:31}{12:52}
{\kalas{05:45 06:39 10:22 09:40 11:05 16:47 11:48 13:56 16:04 17:29 19:05 21:32 23:12 02:32(+1)}}}
{\tnykdata{\anga{\tithi{29}{कृष्ण-चतुर्दशी}}{\time{40-51}{23:53}}\hspace{1ex}}%
{\anga{हस्तः}{\time{24-21}{17:17}}\hspace{1ex}}{चन्द्रराशिः—\mbox{कन्या\RIGHTarrow{03:59(+1)}}}%
{\anga{वैधृतिः}{\time{14-57}{13:31}}\hspace{1ex}\uanga{विष्कम्भः}}%
{\anga{विष्टिः}{\time{15-5}{13:34}}\hspace{1ex}\anga{शकुनिः}{\time{40-51}{23:53}}\hspace{1ex}\uanga{चतुष्पात्}}{}
}
{बोधायन आश्वयुज-अमावास्या\eventsep दीपोत्सव-चतुर्दशी/यम-तर्पणम्\eventsep देवी-पर्व-७\eventsep नरक-चतुर्दशी\eventsep पञ्च-पर्व-पूजा (अमावास्या)\eventsep पञ्च-पर्व-पूजा (चतुर्दशी)\eventsep प्रेत-चतुर्दशी\eventsep वैधृति-श्राद्धम्\eventsep शृङ्गेरी ३५ जगद्गुरु श्री-अभिनव विद्यातीर्थ महास्वामी जयन्ती}
{Sat} 
\cfoot{\rygdata{10:12--11:32}{14:12--15:32}{07:32--08:52}}
\caldata{OCTOBER}{27}{\sunmonth{तुला}{11}{}{आश्वयुजः}{शरदृतुः}{भानुः}{विकारी}{दक्षिणायनम्}{शरदृतुः}}
{\sunmoonsrdata{07:33}{18:11}{---}{18:07}{12:52}
{\kalas{05:46 06:39 10:23 09:40 11:06 16:46 11:48 13:56 16:03 17:28 19:04 21:32 23:12 02:33(+1)}}}
{\tnykdata{\anga{\tithi{30}{अमावास्या}}{\time{32-42}{20:38}}\hspace{1ex}}%
{\anga{चित्रा}{\time{17-57}{14:44}}\hspace{1ex}}{चन्द्रराशिः—\mbox{तुला}}%
{\anga{विष्कम्भः}{\time{5-6}{09:36}}\hspace{1ex}\anga{प्रीतिः}{\time{55-46}{05:51(+1)}}\hspace{1ex}\uanga{आयुष्मान्}}%
{\anga{चतुष्पात्}{\time{6-41}{10:14}}\hspace{1ex}\anga{नाग}{\time{32-42}{20:38}}\hspace{1ex}\anga{किंस्तुघ्नः}{\time{58-56}{07:07(+1)}}\hspace{1ex}\uanga{बवः}}{}
}
{आग्रयण-होमः द्राविडेषु\eventsep आश्वयुज-अमावास्या (अलभ्यम्–स्वाती)\eventsep बोधायन-इष्टिः\eventsep दीपावली/लक्ष्मी-कुबेर-पूजा\eventsep केदार-गौरी-व्रतम्\eventsep कमला-जयन्ती\eventsep पार्वणव्रतम् अमावास्यायाम्\eventsep विक्रमादित्य-पट्टाभिषेकः\eventsep श्रीराम-पट्टाभिषेकः}
{Sun} 
\cfoot{\rygdata{16:51--18:11}{12:52--14:12}{15:31--16:51}}
\caldata{OCTOBER}{28}{\sunmonth{तुला}{12}{}{कार्त्तिकः}{शरदृतुः}{सोमः}{विकारी}{दक्षिणायनम्}{शरदृतुः}}
{\sunmoonrsdata{07:34}{18:10}{08:07}{18:46}{12:52}
{\kalas{05:47 06:40 10:23 09:41 11:06 16:45 11:48 13:55 16:02 17:27 19:03 21:31 23:12 02:33(+1)}}}
{\tnykdata{\anga{\tithi{1}{शुक्ल-प्रथमा}}{\time{25-22}{17:43}}\hspace{1ex}}%
{\anga{स्वाती}{\time{12-14}{12:28}}\hspace{1ex}}{चन्द्रराशिः—\mbox{तुला\RIGHTarrow{05:03(+1)}}}%
{\anga{आयुष्मान्}{\time{47-13}{02:27(+1)}}\hspace{1ex}\uanga{सौभाग्यः}}%
{\anga{बवः}{\time{25-22}{17:43}}\hspace{1ex}\anga{बालवः}{\time{52-9}{04:26(+1)}}\hspace{1ex}\uanga{कौलवः}}{}
}
{चन्द्र-दर्शनम्~18:10\RIGHTarrow{}19:03\eventsep दर्शेष्टिः\eventsep गोवर्धन-पूजा\eventsep कृत्तिका-सोमवासरः\eventsep पार्वण-प्रायश्चित्तावकाशः पौर्णमास्याम्\eventsep स्थालीपाकः}
{Mon} 
\cfoot{\rygdata{08:53--10:13}{11:32--12:52}{14:11--15:31}}
\caldata{OCTOBER}{29}{\sunmonth{तुला}{13}{}{कार्त्तिकः}{शरदृतुः}{मङ्गलः}{विकारी}{दक्षिणायनम्}{शरदृतुः}}
{\sunmoonrsdata{07:35}{18:08}{09:18}{19:30}{12:52}
{\kalas{05:48 06:41 10:24 09:42 11:06 16:44 11:48 13:55 16:02 17:26 19:02 21:30 23:11 02:33(+1)}}}
{\tnykdata{\anga{\tithi{2}{शुक्ल-द्वितीया}}{\time{19-16}{15:17}}\hspace{1ex}}%
{\anga{विशाखा}{\time{7-39}{10:38}}\hspace{1ex}}{चन्द्रराशिः—\mbox{वृश्चिकः}}%
{\anga{सौभाग्यः}{\time{39-48}{23:30}}\hspace{1ex}\uanga{शोभनः}}%
{\anga{कौलवः}{\time{19-16}{15:17}}\hspace{1ex}\anga{तैतिलः}{\time{46-50}{02:19(+1)}}\hspace{1ex}\uanga{गरः}}{}
}
{\tamil{பூசலார் நாயனார் (56) குருபூஜை}\eventsep यम/भ्रातृ-द्वितीया}
{Tue} 
\cfoot{\rygdata{15:30--16:49}{10:13--11:32}{12:52--14:11}}
\caldata{OCTOBER}{30}{\sunmonth{तुला}{14}{}{कार्त्तिकः}{शरदृतुः}{बुधः}{विकारी}{दक्षिणायनम्}{शरदृतुः}}
{\sunmoonrsdata{07:36}{18:07}{10:26}{20:19}{12:52}
{\kalas{05:48 06:42 10:24 09:42 11:06 16:43 11:48 13:55 16:01 17:25 19:01 21:30 23:11 02:33(+1)}}}
{\tnykdata{\anga{\tithi{3}{शुक्ल-तृतीया}}{\time{14-48}{13:31}}\hspace{1ex}}%
{\anga{अनूराधा}{\time{4-35}{09:26}}\hspace{1ex}}{चन्द्रराशिः—\mbox{वृश्चिकः}}%
{\anga{शोभनः}{\time{33-47}{21:07}}\hspace{1ex}\uanga{अतिगण्डः}}%
{\anga{गरः}{\time{14-48}{13:31}}\hspace{1ex}\anga{वणिजः}{\time{43-17}{00:55(+1)}}\hspace{1ex}\uanga{विष्टिः}}{}
}
{बुधानुराधा-पुण्यकालः\RIGHTarrow{}09:26}
{Wed} 
\cfoot{\rygdata{12:52--14:11}{08:55--10:14}{11:33--12:52}}
\caldata{OCTOBER}{31}{\sunmonth{तुला}{15}{}{कार्त्तिकः}{शरदृतुः}{गुरुः}{विकारी}{दक्षिणायनम्}{शरदृतुः}}
{\sunmoonrsdata{07:37}{18:06}{11:28}{21:13}{12:52}
{\kalas{05:49 06:43 10:25 09:43 11:07 16:42 11:49 13:54 16:00 17:24 19:00 21:29 23:11 02:34(+1)}}}
{\tnykdata{\anga{\tithi{4}{शुक्ल-चतुर्थी}}{\time{12-14}{12:31}}\hspace{1ex}}%
{\anga{ज्येष्ठा}{\time{3-23}{08:58}}\hspace{1ex}}{चन्द्रराशिः—\mbox{वृश्चिकः\RIGHTarrow{08:58}}}%
{\anga{अतिगण्डः}{\time{29-21}{19:22}}\hspace{1ex}\uanga{सुकर्म}}%
{\anga{विष्टिः}{\time{12-14}{12:31}}\hspace{1ex}\anga{बवः}{\time{41-45}{00:19(+1)}}\hspace{1ex}\uanga{बालवः}}{}
}
{\tamil{ஐயடிகள் காடவர்கோன் நாயனார் (45) குருபூஜை}\eventsep देवसेना-पञ्चमी\eventsep पाण्डव-(लाभ)-पञ्चमी}
{Thu} 
\cfoot{\rygdata{14:10--15:29}{07:37--08:56}{10:14--11:33}}
\caldata{NOVEMBER}{1}{\sunmonth{तुला}{16}{}{कार्त्तिकः}{शरदृतुः}{शुक्रः}{विकारी}{दक्षिणायनम्}{शरदृतुः}}
{\sunmoonrsdata{07:38}{18:05}{12:22}{22:10}{12:51}
{\kalas{05:50 06:44 10:25 09:43 11:07 16:41 11:49 13:54 16:00 17:23 18:59 21:29 23:10 02:34(+1)}}}
{\tnykdata{\anga{\tithi{5}{शुक्ल-पञ्चमी}}{\time{11-46}{12:21}}\hspace{1ex}}%
{\anga{मूला}{\time{4-12}{09:19}}\hspace{1ex}}{चन्द्रराशिः—\mbox{धनुः}}%
{\anga{सुकर्म}{\time{26-34}{18:16}}\hspace{1ex}\uanga{धृतिः}}%
{\anga{बालवः}{\time{11-46}{12:21}}\hspace{1ex}\anga{कौलवः}{\time{42-21}{00:34(+1)}}\hspace{1ex}\uanga{तैतिलः}}{}
}
{सर्प-पूजा\eventsep स्कन्दषष्ठी-व्रतम्}
{Fri} 
\cfoot{\rygdata{11:33--12:51}{15:28--16:47}{08:56--10:15}}
\caldata{NOVEMBER}{2}{\sunmonth{तुला}{17}{}{कार्त्तिकः}{शरदृतुः}{शनिः}{विकारी}{दक्षिणायनम्}{शरदृतुः}}
{\sunmoonrsdata{07:39}{18:04}{13:08}{23:09}{12:51}
{\kalas{05:51 06:45 10:26 09:44 11:07 16:41 11:49 13:54 15:59 17:22 18:58 21:28 23:10 02:34(+1)}}}
{\tnykdata{\anga{\tithi{6}{शुक्ल-षष्ठी}}{\time{13-23}{13:01}}\hspace{1ex}}%
{\anga{पूर्वाषाढा}{\time{7-3}{10:28}}\hspace{1ex}}{चन्द्रराशिः—\mbox{धनुः\RIGHTarrow{16:53}}}%
{\anga{धृतिः}{\time{25-22}{17:48}}\hspace{1ex}\uanga{शूलः}}%
{\anga{तैतिलः}{\time{13-23}{13:01}}\hspace{1ex}\anga{गरः}{\time{44-57}{01:38(+1)}}\hspace{1ex}\uanga{वणिजः}}{}
}
{सावित्र्य-कल्पादिः}
{Sat} 
\cfoot{\rygdata{10:15--11:33}{14:10--15:28}{07:39--08:57}}
\caldata{NOVEMBER}{3}{\sunmonth{तुला}{18}{}{कार्त्तिकः}{शरदृतुः}{भानुः}{विकारी}{दक्षिणायनम्}{शरदृतुः}}
{\sunmoonrsdata{07:40}{18:03}{13:48}{00:08(+1)}{12:52}
{\kalas{05:51 06:46 10:26 09:45 11:08 16:40 11:49 13:54 15:58 17:21 18:57 21:27 23:10 02:34(+1)}}}
{\tnykdata{\anga{\tithi{7}{शुक्ल-सप्तमी}}{\time{16-53}{14:26}}\hspace{1ex}}%
{\anga{उत्तराषाढा}{\time{11-43}{12:22}}\hspace{1ex}}{चन्द्रराशिः—\mbox{मकरः}}%
{\anga{शूलः}{\time{25-32}{17:53}}\hspace{1ex}\uanga{गण्डः}}%
{\anga{वणिजः}{\time{16-53}{14:26}}\hspace{1ex}\anga{विष्टिः}{\time{49-15}{03:22(+1)}}\hspace{1ex}\uanga{बवः}}{}
}
{विजया-भानुसप्तमी}
{Sun} 
\cfoot{\rygdata{16:45--18:03}{12:51--14:09}{15:27--16:45}}
\caldata{NOVEMBER}{4}{\sunmonth{तुला}{19}{}{कार्त्तिकः}{शरदृतुः}{सोमः}{विकारी}{दक्षिणायनम्}{शरदृतुः}}
{\sunmoonrsdata{06:41}{17:02}{13:22}{00:06(+1)}{11:52}
{\kalas{04:52 05:47 09:27 08:45 10:08 15:39 10:50 12:54 14:58 16:21 17:57 20:27 22:10 01:35(+1)}}}
{\tnykdata{\anga{\tithi{8}{शुक्ल-अष्टमी}}{\time{21-54}{15:27}}\hspace{1ex}}%
{\anga{श्रवणः}{\time{17-52}{13:50}}\hspace{1ex}}{चन्द्रराशिः—\mbox{मकरः\RIGHTarrow{03:14(+1)}}}%
{\anga{गण्डः}{\time{26-48}{17:25}}\hspace{1ex}\uanga{वृद्धिः}}%
{\anga{बवः}{\time{21-54}{15:27}}\hspace{1ex}\anga{बालवः}{\time{54-49}{04:37(+1)}}\hspace{1ex}\uanga{कौलवः}}{}
}
{देवी-पर्व-८\eventsep गोपाष्टमी\eventsep कार्तवीर्यार्जुन-जयन्ती\eventsep कृत्तिका-सोमवासरः\eventsep \tamil{பொய்கையாழ்வார் திருநக்ஷத்திரம்}\eventsep सोमश्रावणी-पुण्यकालः\RIGHTarrow{}13:50\eventsep ताम्रपर्णी-अन्त्य-पुष्कर-समापनम्\eventsep श्रवण-व्रतम्}
{Mon} 
\cfoot{\rygdata{07:59--09:16}{10:34--11:52}{13:09--14:27}}
\caldata{NOVEMBER}{5}{\sunmonth{तुला}{20}{}{कार्त्तिकः}{शरदृतुः}{मङ्गलः}{विकारी}{दक्षिणायनम्}{शरदृतुः}}
{\sunmoonrsdata{06:42}{17:01}{13:52}{01:04(+1)}{11:52}
{\kalas{04:53 05:48 09:27 08:46 10:08 15:38 10:50 12:54 14:57 16:20 17:56 20:27 22:09 01:35(+1)}}}
{\tnykdata{\anga{\tithi{9}{शुक्ल-नवमी}}{\time{27-51}{17:51}}\hspace{1ex}}%
{\anga{श्रविष्ठा}{\time{24-57}{16:42}}\hspace{1ex}}{चन्द्रराशिः—\mbox{कुम्भः}}%
{\anga{वृद्धिः}{\time{28-44}{18:12}}\hspace{1ex}\uanga{ध्रुवः}}%
{\anga{कौलवः}{\time{27-51}{17:51}}\hspace{1ex}\uanga{तैतिलः}}{}
}
{अक्षया-नवमी\eventsep \tamil{பூதத்தாழ்வார் திருநக்ஷத்திரம்}\eventsep गुरु-सङ्क्रान्तिः~(वृश्चिकः\To{}धनुः)\eventsep जगद्धात्री-पूजा\eventsep काञ्ची २२ जगद्गुरु श्री-परिपूर्णबोधेन्द्र सरस्वती आराधना~\#{१५३९}\eventsep सिन्धु-आदि-पुष्कर-आरम्भः\eventsep त्रेतायुगादिः}
{Tue} 
\cfoot{\rygdata{14:26--15:44}{09:17--10:34}{11:52--13:09}}
\caldata{NOVEMBER}{6}{\sunmonth{तुला}{21}{}{कार्त्तिकः}{शरदृतुः}{बुधः}{विकारी}{दक्षिणायनम्}{शरदृतुः}}
{\sunmoonrsdata{06:43}{17:00}{14:19}{02:00(+1)}{11:52}
{\kalas{04:54 05:49 09:28 08:47 10:09 15:38 10:50 12:53 14:57 16:19 17:55 20:26 22:09 01:35(+1)}}}
{\tnykdata{\anga{\tithi{10}{शुक्ल-दशमी}}{\time{34-13}{20:25}}\hspace{1ex}}%
{\anga{शतभिषक्}{\time{32-26}{19:42}}\hspace{1ex}}{चन्द्रराशिः—\mbox{कुम्भः}}%
{\anga{ध्रुवः}{\time{30-57}{19:06}}\hspace{1ex}\uanga{व्याघातः}}%
{\anga{तैतिलः}{\time{1-0}{07:08}}\hspace{1ex}\anga{गरः}{\time{34-13}{20:25}}\hspace{1ex}\uanga{वणिजः}}{}
}
{कंस-वधः\eventsep \tamil{பேயாழ்வார் திருநக்ஷத்திரம்}}
{Wed} 
\cfoot{\rygdata{11:52--13:09}{08:00--09:18}{10:35--11:52}}
\caldata{NOVEMBER}{7}{\sunmonth{तुला}{22}{}{कार्त्तिकः}{शरदृतुः}{गुरुः}{विकारी}{दक्षिणायनम्}{शरदृतुः}}
{\sunmoonrsdata{06:44}{16:59}{14:45}{02:57(+1)}{11:52}
{\kalas{04:55 05:50 09:28 08:47 10:09 15:37 10:50 12:53 14:56 16:18 17:54 20:26 22:09 01:36(+1)}}}
{\tnykdata{\anga{\tithi{11}{शुक्ल-एकादशी}}{\time{40-24}{22:54}}\hspace{1ex}}%
{\anga{पूर्वप्रोष्ठपदा}{\time{39-47}{22:39}}\hspace{1ex}}{चन्द्रराशिः—\mbox{कुम्भः\RIGHTarrow{15:56}}}%
{\anga{व्याघातः}{\time{33-3}{19:58}}\hspace{1ex}\uanga{हर्षणः}}%
{\anga{वणिजः}{\time{7-20}{09:41}}\hspace{1ex}\anga{विष्टिः}{\time{40-24}{22:54}}\hspace{1ex}\uanga{बवः}}{}
}
{आदि-शङ्कर मानसिक-सन्न्यास-दिनम्\eventsep भीष्म-पञ्चक-व्रत-आरम्भः\eventsep सर्व-उत्थान-एकादशी}
{Thu} 
\cfoot{\rygdata{13:08--14:25}{06:44--08:01}{09:18--10:35}}
\caldata{NOVEMBER}{8}{\sunmonth{तुला}{23}{}{कार्त्तिकः}{शरदृतुः}{शुक्रः}{विकारी}{दक्षिणायनम्}{शरदृतुः}}
{\sunmoonrsdata{06:46}{16:58}{15:11}{03:54(+1)}{11:52}
{\kalas{04:55 05:50 09:29 08:48 10:10 15:36 10:51 12:53 14:55 16:17 17:53 20:25 22:09 01:36(+1)}}}
{\tnykdata{\anga{\tithi{12}{शुक्ल-द्वादशी}}{\time{45-59}{01:09(+1)}}\hspace{1ex}}%
{\anga{उत्तरप्रोष्ठपदा}{\time{46-32}{01:23(+1)}}\hspace{1ex}}{चन्द्रराशिः—\mbox{मीनः}}%
{\anga{हर्षणः}{\time{34-44}{20:39}}\hspace{1ex}\uanga{वज्रम्}}%
{\anga{बवः}{\time{13-15}{12:04}}\hspace{1ex}\anga{बालवः}{\time{45-59}{01:09(+1)}}\hspace{1ex}\uanga{कौलवः}}{}
}
{बृन्दावन-द्वादशी\eventsep द्विदल-व्रत-समापनम्\eventsep गोपद्म-व्रत-समापनम्\eventsep हरिवासरः\RIGHTarrow{}05:30\eventsep मन्वादिः-(स्वारोचिषः-[२])\eventsep प्रबोधोत्सवः\eventsep तुलसी-विवाहः\eventsep याज्ञवल्क्य-जयन्ती}
{Fri} 
\cfoot{\rygdata{10:35--11:52}{14:25--15:41}{08:02--09:19}}
\caldata{NOVEMBER}{9}{\sunmonth{तुला}{24}{}{कार्त्तिकः}{शरदृतुः}{शनिः}{विकारी}{दक्षिणायनम्}{शरदृतुः}}
{\sunmoonrsdata{06:47}{16:57}{15:37}{04:52(+1)}{11:52}
{\kalas{04:56 05:51 09:29 08:49 10:10 15:36 10:51 12:53 14:55 16:16 17:52 20:25 22:09 01:36(+1)}}}
{\tnykdata{\anga{\tithi{13}{शुक्ल-त्रयोदशी}}{\time{50-40}{03:03(+1)}}\hspace{1ex}}%
{\anga{रेवती}{\time{52-27}{03:45(+1)}}\hspace{1ex}}{चन्द्रराशिः—\mbox{मीनः\RIGHTarrow{03:45(+1)}}}%
{\anga{वज्रम्}{\time{35-47}{21:05}}\hspace{1ex}\uanga{सिद्धिः}}%
{\anga{कौलवः}{\time{18-25}{14:09}}\hspace{1ex}\anga{तैतिलः}{\time{50-40}{03:03(+1)}}\hspace{1ex}\uanga{गरः}}{}
}
{कार्त्तिक-मास-अन्तिमत्रयतिथि-व्रत-आरम्भः\eventsep शनि-प्रदोष-व्रतम्~16:57\RIGHTarrow{}17:52}
{Sat} 
\cfoot{\rygdata{09:19--10:36}{13:08--14:25}{06:47--08:03}}
\caldata{NOVEMBER}{10}{\sunmonth{तुला}{25}{}{कार्त्तिकः}{शरदृतुः}{भानुः}{विकारी}{दक्षिणायनम्}{शरदृतुः}}
{\sunmoonrsdata{06:48}{16:56}{16:05}{05:52(+1)}{11:52}
{\kalas{04:57 05:52 09:30 08:49 10:10 15:35 10:51 12:53 14:54 16:16 17:52 20:24 22:08 01:37(+1)}}}
{\tnykdata{\anga{\tithi{14}{शुक्ल-चतुर्दशी}}{\time{54-19}{04:31(+1)}}\hspace{1ex}}%
{\anga{अश्विनी}{\time{57-21}{05:44(+1)}}\hspace{1ex}}{चन्द्रराशिः—\mbox{मेषः}}%
{\anga{सिद्धिः}{\time{36-2}{21:13}}\hspace{1ex}\uanga{व्यतीपातः}}%
{\anga{गरः}{\time{22-36}{15:50}}\hspace{1ex}\anga{वणिजः}{\time{54-19}{04:31(+1)}}\hspace{1ex}\uanga{विष्टिः}}{}
}
{मणिकर्णिका-स्नानम्/वैकुण्ठ-चतुर्दशी\eventsep \tamil{திருமூல நாயனார் (29) குருபூஜை}}
{Sun} 
\cfoot{\rygdata{15:40--16:56}{11:52--13:08}{14:24--15:40}}
\caldata{NOVEMBER}{11}{\sunmonth{तुला}{26}{}{कार्त्तिकः}{शरदृतुः}{सोमः}{विकारी}{दक्षिणायनम्}{शरदृतुः}}
{\sunmoonrsdata{06:49}{16:55}{16:37}{---}{11:52}
{\kalas{04:58 05:53 09:31 08:50 10:11 15:34 10:51 12:53 14:54 16:15 17:51 20:24 22:08 01:37(+1)}}}
{\tnykdata{\anga{\tithi{15}{पौर्णमासी}}{\time{56-52}{05:34(+1)}}\hspace{1ex}}%
{\fullanga{अपभरणी}}{चन्द्रराशिः—\mbox{मेषः}}%
{\anga{व्यतीपातः}{\time{35-29}{21:01}}\hspace{1ex}\uanga{वरीयान्}}%
{\anga{विष्टिः}{\time{25-42}{17:06}}\hspace{1ex}\anga{बवः}{\time{56-52}{05:34(+1)}}\hspace{1ex}\uanga{बालवः}}{}
}
{आग्रयण-होमः द्राविडेषु\eventsep भीष्म-पञ्चक-व्रत-समापनम्\eventsep चातुर्मास्यव्रत-समापनम्\eventsep कार्त्तिक-मास-अन्तिमत्रयतिथि-व्रत-समापनम्\eventsep कार्त्तिक-पूर्णिमा-स्नानम्\eventsep कृत्तिका-सोमवासरः\eventsep महा-अन्नाभिषेकः\eventsep मन्वादिः-(धर्मः-[११])\eventsep \tamil{நின்றசீர் நெடுமாற நாயனார் (48) குருபூஜை}\eventsep पार्वणव्रतम् पूर्णिमायाम्\eventsep पूर्णिमा-व्रतम्\eventsep पञ्च-पर्व-पूजा (पूर्णिमा)\eventsep त्रिपुरोत्सवः\eventsep वेङ्कटाचले पूर्णिमा-गरुड-सेवा\eventsep व्यतीपात-श्राद्धम्\eventsep श्री-गोविन्द भगवत्पाद आराधना}
{Mon} 
\cfoot{\rygdata{08:05--09:20}{10:36--11:52}{13:08--14:24}}
\caldata{NOVEMBER}{12}{\sunmonth{तुला}{27}{}{कार्त्तिकः}{शरदृतुः}{मङ्गलः}{विकारी}{दक्षिणायनम्}{शरदृतुः}}
{\sunmoonsrdata{06:50}{16:55}{17:13}{06:53}{11:52}
{\kalas{04:59 05:54 09:31 08:51 10:12 15:34 10:52 12:53 14:54 16:14 17:50 20:24 22:08 01:37(+1)}}}
{\tnykdata{\anga{\tithi{16}{कृष्ण-प्रथमा}}{\time{58-23}{06:11(+1)}}\hspace{1ex}}%
{\anga{अपभरणी}{\time{1-10}{07:18}}\hspace{1ex}}{चन्द्रराशिः—\mbox{मेषः\RIGHTarrow{13:38}}}%
{\anga{वरीयान्}{\time{34-6}{20:29}}\hspace{1ex}\uanga{परिघः}}%
{\anga{बालवः}{\time{27-44}{17:56}}\hspace{1ex}\anga{कौलवः}{\time{58-23}{06:11(+1)}}\hspace{1ex}\uanga{तैतिलः}}{}
}
{\tamil{இடங்கழி நாயனார் (52) குருபூஜை}\eventsep काञ्ची ६४ जगद्गुरु श्री-चन्द्रशेखरेन्द्र सरस्वती ५ आराधना~\#{१६९}\eventsep कृत्तिका-व्रतम्\eventsep नवम-अपरपक्ष-आरम्भः\eventsep पार्वण-प्रायश्चित्तावकाशः दर्शे\eventsep पूर्णमासेष्टिः\eventsep पद्मक-योगः\eventsep स्थालीपाकः}
{Tue} 
\cfoot{\rygdata{14:23--15:39}{09:21--10:37}{11:52--13:08}}
\caldata{NOVEMBER}{13}{\sunmonth{तुला}{28}{}{कार्त्तिकः}{शरदृतुः}{बुधः}{विकारी}{दक्षिणायनम्}{शरदृतुः}}
{\sunmoonsrdata{06:51}{16:54}{17:54}{07:55}{11:52}
{\kalas{04:59 05:55 09:32 08:52 10:12 15:33 10:52 12:53 14:53 16:14 17:50 20:23 22:08 01:38(+1)}}}
{\tnykdata{\anga{\tithi{17}{कृष्ण-द्वितीया}}{\time{58-54}{06:25(+1)}}\hspace{1ex}}%
{\anga{कृत्तिका}{\time{4-2}{08:28}}\hspace{1ex}}{चन्द्रराशिः—\mbox{वृषभः}}%
{\anga{परिघः}{\time{31-57}{19:38}}\hspace{1ex}\uanga{शिवः}}%
{\anga{तैतिलः}{\time{28-44}{18:21}}\hspace{1ex}\anga{गरः}{\time{58-54}{06:25(+1)}}\hspace{1ex}\uanga{वणिजः}}{}
}
{अशून्यशयन-व्रतम्}
{Wed} 
\cfoot{\rygdata{11:52--13:08}{08:06--09:22}{10:37--11:52}}
\caldata{NOVEMBER}{14}{\sunmonth{तुला}{29}{}{कार्त्तिकः}{शरदृतुः}{गुरुः}{विकारी}{दक्षिणायनम्}{शरदृतुः}}
{\sunmoonsrdata{06:52}{16:53}{18:42}{08:55}{11:53}
{\kalas{05:00 05:56 09:32 08:52 10:12 15:33 10:52 12:53 14:53 16:13 17:49 20:23 22:08 01:38(+1)}}}
{\tnykdata{\anga{\tithi{18}{कृष्ण-तृतीया}}{\time{58-28}{06:15(+1)}}\hspace{1ex}}%
{\anga{रोहिणी}{\time{5-55}{09:14}}\hspace{1ex}}{चन्द्रराशिः—\mbox{वृषभः\RIGHTarrow{21:29}}}%
{\anga{शिवः}{\time{29-2}{18:29}}\hspace{1ex}\uanga{सिद्धः}}%
{\anga{वणिजः}{\time{28-46}{18:23}}\hspace{1ex}\anga{विष्टिः}{\time{58-28}{06:15(+1)}}\hspace{1ex}\uanga{बवः}}{}
}
{काञ्ची ९ जगद्गुरु श्री-कृपाशङ्करेन्द्र सरस्वती आराधना~\#{१९५१}\eventsep सौभाग्य-सुन्दरी-व्रतम्}
{Thu} 
\cfoot{\rygdata{13:08--14:23}{06:52--08:07}{09:22--10:37}}
\caldata{NOVEMBER}{15}{\sunmonth{तुला}{30}{}{कार्त्तिकः}{शरदृतुः}{शुक्रः}{विकारी}{दक्षिणायनम्}{शरदृतुः}}
{\sunmoonsrdata{06:53}{16:52}{19:38}{09:53}{11:53}
{\kalas{05:01 05:57 09:33 08:53 10:13 15:32 10:53 12:53 14:52 16:12 17:48 20:23 22:08 01:38(+1)}}}
{\tnykdata{\anga{\tithi{19}{कृष्ण-चतुर्थी}}{\time{57-9}{05:45(+1)}}\hspace{1ex}}%
{\anga{मृगशीर्षम्}{\time{6-55}{09:39}}\hspace{1ex}}{चन्द्रराशिः—\mbox{मिथुनम्}}%
{\anga{सिद्धः}{\time{25-25}{17:03}}\hspace{1ex}\uanga{साध्यः}}%
{\anga{बवः}{\time{27-54}{18:03}}\hspace{1ex}\anga{बालवः}{\time{57-9}{05:45(+1)}}\hspace{1ex}\uanga{कौलवः}}{}
}
{आकाशदीप-समापनम्\eventsep गणाधिप-महागणपति सङ्कटहर-चतुर्थी-व्रतम्\eventsep तुला-कावेरी-स्नान-समापनम्}
{Fri} 
\cfoot{\rygdata{10:38--11:53}{14:22--15:37}{08:08--09:23}}
\caldata{NOVEMBER}{16}{\sunmonth{वृश्चिकः}{1}{\mbox{तुला{\tiny\RIGHTarrow}{10:51}}}{कार्त्तिकः}{शरदृतुः}{शनिः}{विकारी}{दक्षिणायनम्}{शरदृतुः}}
{\sunmoonsrdata{06:54}{16:52}{20:39}{10:45}{11:53}
{\kalas{05:02 05:58 09:33 08:54 10:13 15:32 10:53 12:53 14:52 16:12 17:48 20:22 22:08 01:39(+1)}}}
{\tnykdata{\anga{\tithi{20}{कृष्ण-पञ्चमी}}{\time{54-56}{04:53(+1)}}\hspace{1ex}}%
{\anga{आर्द्रा}{\time{7-1}{09:43}}\hspace{1ex}}{चन्द्रराशिः—\mbox{मिथुनम्\RIGHTarrow{03:32(+1)}}}%
{\anga{साध्यः}{\time{21-5}{15:21}}\hspace{1ex}\uanga{शुभः}}%
{\anga{कौलवः}{\time{26-7}{17:21}}\hspace{1ex}\anga{तैतिलः}{\time{54-56}{04:53(+1)}}\hspace{1ex}\uanga{गरः}}{}
}
{कृत्तिका-मण्डल-पारायण-आरम्भः\eventsep सिन्धु-आदि-पुष्कर-समापनम्\eventsep तिरुविशलूर् गङ्गाकर्षण-महोत्सव-आरम्भः\eventsep वृश्चिक-रवि-सङ्क्रमण-विष्णुपदी-पुण्यकालः~04:27\RIGHTarrow{}17:15}
{Sat} 
\cfoot{\rygdata{09:24--10:38}{13:07--14:22}{06:54--08:09}}
\caldata{NOVEMBER}{17}{\sunmonth{वृश्चिकः}{2}{}{कार्त्तिकः}{शरदृतुः}{भानुः}{विकारी}{दक्षिणायनम्}{शरदृतुः}}
{\sunmoonsrdata{06:55}{16:51}{21:45}{11:32}{11:53}
{\kalas{05:03 05:59 09:34 08:54 10:14 15:31 10:54 12:53 14:52 16:11 17:47 20:22 22:08 01:39(+1)}}}
{\tnykdata{\anga{\tithi{21}{कृष्ण-षष्ठी}}{\time{51-50}{03:39(+1)}}\hspace{1ex}}%
{\anga{पुनर्वसुः}{\time{6-16}{09:26}}\hspace{1ex}}{चन्द्रराशिः—\mbox{कर्कटः}}%
{\anga{शुभः}{\time{16-4}{13:21}}\hspace{1ex}\uanga{शुक्लः}}%
{\anga{गरः}{\time{23-28}{16:19}}\hspace{1ex}\anga{वणिजः}{\time{51-50}{03:39(+1)}}\hspace{1ex}\uanga{विष्टिः}}{}
}
{\tamil{கார்த்திகை~ஞாயிற்றுக்கிழமை}\eventsep \tamil{முடவன் முழுக்கு}\eventsep रविपुष्ययोग-पुण्यकालः~09:26\RIGHTarrow{}}
{Sun} 
\cfoot{\rygdata{15:36--16:51}{11:53--13:07}{14:22--15:36}}
\caldata{NOVEMBER}{18}{\sunmonth{वृश्चिकः}{3}{}{कार्त्तिकः}{शरदृतुः}{सोमः}{विकारी}{दक्षिणायनम्}{शरदृतुः}}
{\sunmoonsrdata{06:56}{16:50}{22:53}{12:14}{11:53}
{\kalas{05:04 06:00 09:35 08:55 10:14 15:31 10:54 12:53 14:51 16:10 17:47 20:22 22:08 01:40(+1)}}}
{\tnykdata{\anga{\tithi{22}{कृष्ण-सप्तमी}}{\time{47-52}{02:05(+1)}}\hspace{1ex}}%
{\anga{पुष्यः}{\time{4-39}{08:48}}\hspace{1ex}}{चन्द्रराशिः—\mbox{कर्कटः}}%
{\anga{शुक्लः}{\time{10-20}{11:05}}\hspace{1ex}\uanga{ब्रह्म}}%
{\anga{विष्टिः}{\time{19-56}{14:55}}\hspace{1ex}\anga{बवः}{\time{47-52}{02:05(+1)}}\hspace{1ex}\uanga{बालवः}}{}
}
{कृत्तिका-सोमवासरः}
{Mon} 
\cfoot{\rygdata{08:11--09:25}{10:39--11:53}{13:07--14:22}}
\caldata{NOVEMBER}{19}{\sunmonth{वृश्चिकः}{4}{}{कार्त्तिकः}{शरदृतुः}{मङ्गलः}{विकारी}{दक्षिणायनम्}{शरदृतुः}}
{\sunmoonsrdata{06:57}{16:50}{00:02(+1)}{12:51}{11:53}
{\kalas{05:04 06:01 09:35 08:56 10:15 15:31 10:54 12:53 14:51 16:10 17:46 20:22 22:08 01:40(+1)}}}
{\tnykdata{\anga{\tithi{23}{कृष्ण-अष्टमी}}{\time{43-3}{00:11(+1)}}\hspace{1ex}}%
{\anga{आश्रेषा}{\time{2-11}{07:50}}\hspace{1ex}\anga{मघा}{\time{58-56}{06:32(+1)}}\hspace{1ex}}{चन्द्रराशिः—\mbox{कर्कटः\RIGHTarrow{07:50}}}%
{\anga{ब्रह्म}{\time{3-55}{08:32}}\hspace{1ex}\anga{इन्द्रः}{\time{56-53}{05:43(+1)}}\hspace{1ex}\uanga{वैधृतिः}}%
{\anga{बालवः}{\time{15-32}{13:10}}\hspace{1ex}\anga{कौलवः}{\time{43-3}{00:11(+1)}}\hspace{1ex}\uanga{तैतिलः}}{}
}
{काञ्ची ४२ जगद्गुरु श्री-ब्रह्मानन्दघनेन्द्र सरस्वती २ आराधना~\#{१०४२}\eventsep काञ्ची ४९ जगद्गुरु श्री-महादेवेन्द्र सरस्वती ३ आराधना~\#{७७३}\eventsep काञ्ची ५८ जगद्गुरु श्री-आत्मबोधेन्द्र सरस्वती आराधना~\#{३८२}\eventsep कालभैरवाष्टमी\eventsep महादेवाष्टमी\eventsep पञ्च-पर्व-पूजा (अष्टमी)}
{Tue} 
\cfoot{\rygdata{14:21--15:36}{09:25--10:39}{11:53--13:07}}
\caldata{NOVEMBER}{20}{\sunmonth{वृश्चिकः}{5}{}{कार्त्तिकः}{शरदृतुः}{बुधः}{विकारी}{दक्षिणायनम्}{शरदृतुः}}
{\sunmoonsrdata{06:59}{16:49}{01:12(+1)}{13:24}{11:54}
{\kalas{05:05 06:02 09:36 08:57 10:15 15:30 10:55 12:53 14:51 16:10 17:46 20:22 22:08 01:41(+1)}}}
{\tnykdata{\anga{\tithi{24}{कृष्ण-नवमी}}{\time{37-29}{21:58}}\hspace{1ex}}%
{\anga{पूर्वफल्गुनी}{\time{54-55}{04:57(+1)}}\hspace{1ex}}{चन्द्रराशिः—\mbox{सिंहः}}%
{\anga{वैधृतिः}{\time{49-13}{02:40(+1)}}\hspace{1ex}\uanga{विष्कम्भः}}%
{\anga{तैतिलः}{\time{10-20}{11:07}}\hspace{1ex}\anga{गरः}{\time{37-29}{21:58}}\hspace{1ex}\uanga{वणिजः}}{}
}
{वैधृति-श्राद्धम्}
{Wed} 
\cfoot{\rygdata{11:54--13:07}{08:12--09:26}{10:40--11:54}}
\caldata{NOVEMBER}{21}{\sunmonth{वृश्चिकः}{6}{}{कार्त्तिकः}{शरदृतुः}{गुरुः}{विकारी}{दक्षिणायनम्}{शरदृतुः}}
{\sunmoonsrdata{07:00}{16:48}{02:22(+1)}{13:57}{11:54}
{\kalas{05:06 06:03 09:37 08:57 10:16 15:30 10:55 12:53 14:51 16:09 17:45 20:21 22:08 01:41(+1)}}}
{\tnykdata{\anga{\tithi{25}{कृष्ण-दशमी}}{\time{31-18}{19:31}}\hspace{1ex}}%
{\anga{उत्तरफल्गुनी}{\time{50-22}{03:08(+1)}}\hspace{1ex}}{चन्द्रराशिः—\mbox{सिंहः\RIGHTarrow{10:31}}}%
{\anga{विष्कम्भः}{\time{41-4}{23:25}}\hspace{1ex}\uanga{प्रीतिः}}%
{\anga{वणिजः}{\time{4-26}{08:46}}\hspace{1ex}\anga{विष्टिः}{\time{31-18}{19:31}}\hspace{1ex}\anga{बवः}{\time{58-4}{06:13(+1)}}\hspace{1ex}\uanga{बालवः}}{}
}
{ऊर्ज-मासः/शरदृतुः\RIGHTarrow{}06:58(+1)\eventsep काञ्ची २८ जगद्गुरु श्री-महादेवेन्द्र सरस्वती १ आराधना~\#{१४१९}\eventsep \tamil{மெய்ப்பொருள் நாயனார் (4) குருபூஜை}}
{Thu} 
\cfoot{\rygdata{13:08--14:21}{07:00--08:13}{09:27--10:40}}
\caldata{NOVEMBER}{22}{\sunmonth{वृश्चिकः}{7}{}{कार्त्तिकः}{शरदृतुः}{शुक्रः}{विकारी}{दक्षिणायनम्}{शरदृतुः}}
{\sunmoonsrdata{07:01}{16:48}{03:32(+1)}{14:29}{11:54}
{\kalas{05:07 06:04 09:37 08:58 10:16 15:30 10:56 12:53 14:50 16:09 17:45 20:21 22:08 01:41(+1)}}}
{\tnykdata{\anga{\tithi{26}{कृष्ण-एकादशी}}{\time{24-42}{16:54}}\hspace{1ex}}%
{\anga{हस्तः}{\time{45-28}{01:12(+1)}}\hspace{1ex}}{चन्द्रराशिः—\mbox{कन्या}}%
{\anga{प्रीतिः}{\time{32-38}{20:04}}\hspace{1ex}\uanga{आयुष्मान्}}%
{\anga{बालवः}{\time{24-42}{16:54}}\hspace{1ex}\anga{कौलवः}{\time{51-21}{03:33(+1)}}\hspace{1ex}\uanga{तैतिलः}}{}
}
{\tamil{ஆனாய நாயனார் (13) குருபூஜை}\eventsep षडशीति-पुण्यकालः~06:58\RIGHTarrow{}06:58(+1)\eventsep हरिवासरः\RIGHTarrow{}22:14\eventsep सर्व-उत्पन्ना-एकादशी}
{Fri} 
\cfoot{\rygdata{10:41--11:54}{14:21--15:34}{08:14--09:27}}
\caldata{NOVEMBER}{23}{\sunmonth{वृश्चिकः}{8}{}{कार्त्तिकः}{शरदृतुः}{शनिः}{विकारी}{दक्षिणायनम्}{शरदृतुः}}
{\sunmoonsrdata{07:02}{16:47}{04:44(+1)}{15:02}{11:55}
{\kalas{05:08 06:05 09:38 08:59 10:17 15:29 10:56 12:53 14:50 16:08 17:44 20:21 22:08 01:42(+1)}}}
{\tnykdata{\anga{\tithi{27}{कृष्ण-द्वादशी}}{\time{17-57}{14:13}}\hspace{1ex}}%
{\anga{चित्रा}{\time{40-32}{23:15}}\hspace{1ex}}{चन्द्रराशिः—\mbox{कन्या\RIGHTarrow{12:13}}}%
{\anga{आयुष्मान्}{\time{24-5}{16:40}}\hspace{1ex}\uanga{सौभाग्यः}}%
{\anga{तैतिलः}{\time{17-57}{14:13}}\hspace{1ex}\anga{गरः}{\time{44-38}{00:53(+1)}}\hspace{1ex}\uanga{वणिजः}}{}
}
{शनि-प्रदोष-व्रतम्~16:47\RIGHTarrow{}17:44}
{Sat} 
\cfoot{\rygdata{09:28--10:41}{13:08--14:21}{07:02--08:15}}
\caldata{NOVEMBER}{24}{\sunmonth{वृश्चिकः}{9}{}{कार्त्तिकः}{शरदृतुः}{भानुः}{विकारी}{दक्षिणायनम्}{शरदृतुः}}
{\sunmoonsrdata{07:03}{16:47}{05:55(+1)}{15:39}{11:55}
{\kalas{05:09 06:06 09:39 09:00 10:17 15:29 10:56 12:53 14:50 16:08 17:44 20:21 22:08 01:42(+1)}}}
{\tnykdata{\anga{\tithi{28}{कृष्ण-त्रयोदशी}}{\time{11-21}{11:35}}\hspace{1ex}}%
{\anga{स्वाती}{\time{35-54}{21:25}}\hspace{1ex}}{चन्द्रराशिः—\mbox{तुला}}%
{\anga{सौभाग्यः}{\time{15-42}{13:20}}\hspace{1ex}\uanga{शोभनः}}%
{\anga{वणिजः}{\time{11-21}{11:35}}\hspace{1ex}\anga{विष्टिः}{\time{38-15}{22:21}}\hspace{1ex}\uanga{शकुनिः}}{}
}
{\tamil{கார்த்திகை~ஞாயிற்றுக்கிழமை}\eventsep मासशिवरात्रिः\eventsep पञ्च-पर्व-पूजा (चतुर्दशी)}
{Sun} 
\cfoot{\rygdata{15:34--16:47}{11:55--13:08}{14:21--15:34}}
\caldata{NOVEMBER}{25}{\sunmonth{वृश्चिकः}{10}{}{कार्त्तिकः}{शरदृतुः}{सोमः}{विकारी}{दक्षिणायनम्}{शरदृतुः}}
{\sunmoonsrdata{07:04}{16:46}{07:05(+1)}{16:19}{11:55}
{\kalas{05:10 06:07 09:39 09:00 10:18 15:29 10:57 12:53 14:50 16:08 17:44 20:21 22:08 01:43(+1)}}}
{\tnykdata{\anga{\tithi{29}{कृष्ण-चतुर्दशी}}{\time{5-16}{09:10}}\hspace{1ex}}%
{\anga{विशाखा}{\time{31-55}{19:50}}\hspace{1ex}}{चन्द्रराशिः—\mbox{तुला\RIGHTarrow{14:12}}}%
{\anga{शोभनः}{\time{7-45}{10:10}}\hspace{1ex}\uanga{अतिगण्डः}}%
{\anga{शकुनिः}{\time{5-16}{09:10}}\hspace{1ex}\anga{चतुष्पात्}{\time{32-32}{20:05}}\hspace{1ex}\uanga{नाग}}{}
}
{कृत्तिका-सोमवासरः\eventsep नवम-अपरपक्ष-समापनम्\eventsep पार्वणव्रतम् अमावास्यायाम्\eventsep पञ्च-पर्व-पूजा (अमावास्या)\eventsep सर्व-कार्त्तिक-अमावास्या (अलभ्यम्–विशाखा, पुष्कला)\eventsep तिरुविशलूर् गङ्गाकर्षण-महोत्सव-समापनम्}
{Mon} 
\cfoot{\rygdata{08:17--09:29}{10:42--11:55}{13:08--14:21}}
\caldata{NOVEMBER}{26}{\sunmonth{वृश्चिकः}{11}{}{कार्त्तिकः}{शरदृतुः}{मङ्गलः}{विकारी}{दक्षिणायनम्}{शरदृतुः}}
{\sunmoonsrdata{07:05}{16:46}{---}{17:06}{11:55}
{\kalas{05:10 06:08 09:40 09:01 10:19 15:29 10:57 12:54 14:50 16:07 17:43 20:21 22:08 01:43(+1)}}}
{\tnykdata{\anga{\tithi{30}{अमावास्या}}{\time{0-1}{07:05}}\hspace{1ex}\anga{\tithi{1}{शुक्ल-प्रथमा}}{\time{56-0}{05:29(+1)}}\hspace{1ex}\avamA{}}%
{\anga{अनूराधा}{\time{28-57}{18:40}}\hspace{1ex}}{चन्द्रराशिः—\mbox{वृश्चिकः}}%
{\anga{अतिगण्डः}{\time{0-29}{07:17}}\hspace{1ex}\anga{सुकर्म}{\time{54-13}{04:46(+1)}}\hspace{1ex}\uanga{धृतिः}}%
{\anga{नाग}{\time{0-1}{07:05}}\hspace{1ex}\anga{किंस्तुघ्नः}{\time{27-50}{18:13}}\hspace{1ex}\anga{बवः}{\time{56-0}{05:29(+1)}}\hspace{1ex}\uanga{बालवः}}{}
}
{आग्रयण-होमः द्राविडेषु\eventsep दर्शेष्टिः\eventsep काञ्ची १८ जगद्गुरु श्री-योगतिलक सुरेन्द्र सरस्वती आराधना~\#{१६३५}\eventsep कार्त्तिक-स्नानपूर्तिः\eventsep पार्वण-प्रायश्चित्तावकाशः पौर्णमास्याम्\eventsep स्थालीपाकः\eventsep वनदुर्गानवरात्र-आरम्भः}
{Tue} 
\cfoot{\rygdata{14:21--15:33}{09:30--10:43}{11:55--13:08}}
\caldata{NOVEMBER}{27}{\sunmonth{वृश्चिकः}{12}{}{मार्गशीर्षः}{हेमन्तऋतुः}{बुधः}{विकारी}{दक्षिणायनम्}{शरदृतुः}}
{\sunmoonrsdata{07:06}{16:46}{08:10}{17:57}{11:56}
{\kalas{05:11 06:08 09:40 09:02 10:19 15:28 10:58 12:54 14:50 16:07 17:43 20:21 22:09 01:44(+1)}}}
{\tnykdata{\anga{\tithi{2}{शुक्ल-द्वितीया}}{\time{53-26}{04:28(+1)}}\hspace{1ex}}%
{\anga{ज्येष्ठा}{\time{27-18}{18:01}}\hspace{1ex}}{चन्द्रराशिः—\mbox{वृश्चिकः\RIGHTarrow{18:01}}}%
{\anga{धृतिः}{\time{49-5}{02:44(+1)}}\hspace{1ex}\uanga{शूलः}}%
{\anga{बालवः}{\time{24-29}{16:54}}\hspace{1ex}\anga{कौलवः}{\time{53-26}{04:28(+1)}}\hspace{1ex}\uanga{तैतिलः}}{}
}
{चन्द्र-दर्शनम्~16:46\RIGHTarrow{}17:43\eventsep तिन्त्रिणी-गौरी-व्रतम्}
{Wed} 
\cfoot{\rygdata{11:56--13:08}{08:18--09:31}{10:43--11:56}}
\caldata{NOVEMBER}{28}{\sunmonth{वृश्चिकः}{13}{}{मार्गशीर्षः}{हेमन्तऋतुः}{गुरुः}{विकारी}{दक्षिणायनम्}{शरदृतुः}}
{\sunmoonrsdata{07:07}{16:45}{09:09}{18:54}{11:56}
{\kalas{05:12 06:09 09:41 09:03 10:20 15:28 10:58 12:54 14:50 16:07 17:43 20:21 22:09 01:44(+1)}}}
{\tnykdata{\anga{\tithi{3}{शुक्ल-तृतीया}}{\time{52-36}{04:09(+1)}}\hspace{1ex}}%
{\anga{मूला}{\time{27-15}{18:01}}\hspace{1ex}}{चन्द्रराशिः—\mbox{धनुः}}%
{\anga{शूलः}{\time{45-17}{01:14(+1)}}\hspace{1ex}\uanga{गण्डः}}%
{\anga{तैतिलः}{\time{22-46}{16:13}}\hspace{1ex}\anga{गरः}{\time{52-36}{04:09(+1)}}\hspace{1ex}\uanga{वणिजः}}{}
}
{\tamil{மூர்க்க நாயனார் (31) குருபூஜை}}
{Thu} 
\cfoot{\rygdata{13:08--14:21}{07:07--08:19}{09:31--10:44}}
\caldata{NOVEMBER}{29}{\sunmonth{वृश्चिकः}{14}{}{मार्गशीर्षः}{हेमन्तऋतुः}{शुक्रः}{विकारी}{दक्षिणायनम्}{शरदृतुः}}
{\sunmoonrsdata{07:08}{16:45}{10:00}{19:54}{11:56}
{\kalas{05:13 06:10 09:42 09:03 10:20 15:28 10:59 12:54 14:50 16:07 17:43 20:21 22:09 01:45(+1)}}}
{\tnykdata{\anga{\tithi{4}{शुक्ल-चतुर्थी}}{\time{53-36}{04:35(+1)}}\hspace{1ex}}%
{\anga{पूर्वाषाढा}{\time{28-57}{18:43}}\hspace{1ex}}{चन्द्रराशिः—\mbox{धनुः\RIGHTarrow{01:00(+1)}}}%
{\anga{गण्डः}{\time{42-55}{00:18(+1)}}\hspace{1ex}\uanga{वृद्धिः}}%
{\anga{वणिजः}{\time{22-51}{16:16}}\hspace{1ex}\anga{विष्टिः}{\time{53-36}{04:35(+1)}}\hspace{1ex}\uanga{बवः}}{}
}
{बदरी-गौरी-व्रतम्\eventsep \tamil{சிறப்புலி நாயனார் (34) குருபூஜை}}
{Fri} 
\cfoot{\rygdata{10:44--11:56}{14:21--15:33}{08:20--09:32}}
\caldata{NOVEMBER}{30}{\sunmonth{वृश्चिकः}{15}{}{मार्गशीर्षः}{हेमन्तऋतुः}{शनिः}{विकारी}{दक्षिणायनम्}{शरदृतुः}}
{\sunmoonrsdata{07:09}{16:45}{10:43}{20:54}{11:57}
{\kalas{05:14 06:11 09:42 09:04 10:21 15:28 10:59 12:54 14:50 16:06 17:43 20:21 22:09 01:45(+1)}}}
{\tnykdata{\anga{\tithi{5}{शुक्ल-पञ्चमी}}{\time{56-25}{05:43(+1)}}\hspace{1ex}}%
{\anga{उत्तराषाढा}{\time{32-25}{20:07}}\hspace{1ex}}{चन्द्रराशिः—\mbox{मकरः}}%
{\anga{वृद्धिः}{\time{41-55}{23:55}}\hspace{1ex}\uanga{ध्रुवः}}%
{\anga{बवः}{\time{24-46}{17:03}}\hspace{1ex}\anga{बालवः}{\time{56-25}{05:43(+1)}}\hspace{1ex}\uanga{कौलवः}}{}
}
{देवी-पर्व-९}
{Sat} 
\cfoot{\rygdata{09:33--10:45}{13:09--14:21}{07:09--08:21}}
\caldata{DECEMBER}{1}{\sunmonth{वृश्चिकः}{16}{}{मार्गशीर्षः}{हेमन्तऋतुः}{भानुः}{विकारी}{दक्षिणायनम्}{शरदृतुः}}
{\sunmoonrsdata{07:10}{16:45}{11:20}{21:54}{11:57}
{\kalas{05:14 06:12 09:43 09:05 10:21 15:28 11:00 12:55 14:50 16:06 17:42 20:21 22:09 01:46(+1)}}}
{\tnykdata{\fulltithi{\tithi{6}{शुक्ल-षष्ठी}}}%
{\anga{श्रवणः}{\time{37-30}{22:10}}\hspace{1ex}}{चन्द्रराशिः—\mbox{मकरः}}%
{\anga{ध्रुवः}{\time{42-11}{00:02(+1)}}\hspace{1ex}\uanga{व्याघातः}}%
{\anga{कौलवः}{\time{28-24}{18:32}}\hspace{1ex}\uanga{तैतिलः}}{}
}
{काञ्ची ३२ जगद्गुरु श्री-चिदानन्दघनेन्द्र सरस्वती आराधना~\#{१३४८}\eventsep \tamil{கார்த்திகை~ஞாயிற்றுக்கிழமை}\eventsep मार्गशीर्ष-शिवलिङ्ग-षष्ठी\eventsep सुब्रह्मण्य-षष्ठी-व्रतम्\eventsep श्रवण-व्रतम्}
{Sun} 
\cfoot{\rygdata{15:33--16:45}{11:57--13:09}{14:21--15:33}}
\caldata{DECEMBER}{2}{\sunmonth{वृश्चिकः}{17}{}{मार्गशीर्षः}{हेमन्तऋतुः}{सोमः}{विकारी}{दक्षिणायनम्}{शरदृतुः}}
{\sunmoonrsdata{07:11}{16:44}{11:52}{22:52}{11:58}
{\kalas{05:15 06:13 09:44 09:05 10:22 15:28 11:00 12:55 14:50 16:06 17:42 20:21 22:10 01:46(+1)}}}
{\tnykdata{\anga{\tithi{6}{शुक्ल-षष्ठी}}{\time{0-46}{07:29}}\hspace{1ex}}%
{\anga{श्रविष्ठा}{\time{43-52}{00:44(+1)}}\hspace{1ex}}{चन्द्रराशिः—\mbox{मकरः\RIGHTarrow{11:24}}}%
{\anga{व्याघातः}{\time{43-25}{00:33(+1)}}\hspace{1ex}\uanga{हर्षणः}}%
{\anga{तैतिलः}{\time{0-46}{07:29}}\hspace{1ex}\anga{गरः}{\time{33-27}{20:34}}\hspace{1ex}\uanga{वणिजः}}{}
}
{काञ्ची ५ जगद्गुरु श्री-ज्ञानानन्देन्द्र सरस्वती आराधना~\#{२२२४}\eventsep मित्र-सप्तमी\eventsep नन्दा-सप्तमी}
{Mon} 
\cfoot{\rygdata{08:22--09:34}{10:46--11:58}{13:09--14:21}}
\caldata{DECEMBER}{3}{\sunmonth{वृश्चिकः}{18}{}{मार्गशीर्षः}{हेमन्तऋतुः}{मङ्गलः}{विकारी}{दक्षिणायनम्}{शरदृतुः}}
{\sunmoonrsdata{07:12}{16:44}{12:21}{23:50}{11:58}
{\kalas{05:16 06:14 09:44 09:06 10:23 15:28 11:01 12:55 14:50 16:06 17:42 20:21 22:10 01:47(+1)}}}
{\tnykdata{\anga{\tithi{7}{शुक्ल-सप्तमी}}{\time{6-20}{09:44}}\hspace{1ex}}%
{\anga{शतभिषक्}{\time{51-1}{03:36(+1)}}\hspace{1ex}}{चन्द्रराशिः—\mbox{कुम्भः}}%
{\anga{हर्षणः}{\time{45-16}{01:18(+1)}}\hspace{1ex}\uanga{वज्रम्}}%
{\anga{वणिजः}{\time{6-20}{09:44}}\hspace{1ex}\anga{विष्टिः}{\time{39-25}{22:58}}\hspace{1ex}\uanga{बवः}}{}
}
{}
{Tue} 
\cfoot{\rygdata{14:21--15:33}{09:35--10:46}{11:58--13:09}}
\caldata{DECEMBER}{4}{\sunmonth{वृश्चिकः}{19}{}{मार्गशीर्षः}{हेमन्तऋतुः}{बुधः}{विकारी}{दक्षिणायनम्}{शरदृतुः}}
{\sunmoonrsdata{07:13}{16:44}{12:47}{00:46(+1)}{11:58}
{\kalas{05:17 06:15 09:45 09:07 10:23 15:28 11:01 12:56 14:50 16:06 17:42 20:21 22:10 01:48(+1)}}}
{\tnykdata{\anga{\tithi{8}{शुक्ल-अष्टमी}}{\time{12-33}{12:14}}\hspace{1ex}}%
{\anga{पूर्वप्रोष्ठपदा}{\time{58-24}{06:34(+1)}}\hspace{1ex}}{चन्द्रराशिः—\mbox{कुम्भः\RIGHTarrow{23:50}}}%
{\anga{वज्रम्}{\time{47-21}{02:09(+1)}}\hspace{1ex}\uanga{सिद्धिः}}%
{\anga{बवः}{\time{12-33}{12:14}}\hspace{1ex}\anga{बालवः}{\time{45-44}{01:30(+1)}}\hspace{1ex}\uanga{कौलवः}}{}
}
{बुधाष्टमी\eventsep प्रलय-कल्पादिः}
{Wed} 
\cfoot{\rygdata{11:58--13:10}{08:24--09:36}{10:47--11:58}}
\caldata{DECEMBER}{5}{\sunmonth{वृश्चिकः}{20}{}{मार्गशीर्षः}{हेमन्तऋतुः}{गुरुः}{विकारी}{दक्षिणायनम्}{शरदृतुः}}
{\sunmoonrsdata{07:14}{16:44}{13:13}{01:43(+1)}{11:59}
{\kalas{05:18 06:15 09:46 09:08 10:24 15:28 11:02 12:56 14:50 16:06 17:42 20:22 22:10 01:48(+1)}}}
{\tnykdata{\anga{\tithi{9}{शुक्ल-नवमी}}{\time{18-48}{14:45}}\hspace{1ex}}%
{\fullanga{उत्तरप्रोष्ठपदा}}{चन्द्रराशिः—\mbox{मीनः}}%
{\anga{सिद्धिः}{\time{49-14}{02:55(+1)}}\hspace{1ex}\uanga{व्यतीपातः}}%
{\anga{कौलवः}{\time{18-48}{14:45}}\hspace{1ex}\anga{तैतिलः}{\time{51-48}{03:57(+1)}}\hspace{1ex}\uanga{गरः}}{}
}
{वनदुर्गानवरात्र-समापनम्}
{Thu} 
\cfoot{\rygdata{13:10--14:21}{07:14--08:25}{09:36--10:48}}
\caldata{DECEMBER}{6}{\sunmonth{वृश्चिकः}{21}{}{मार्गशीर्षः}{हेमन्तऋतुः}{शुक्रः}{विकारी}{दक्षिणायनम्}{शरदृतुः}}
{\sunmoonrsdata{07:14}{16:44}{13:38}{02:40(+1)}{11:59}
{\kalas{05:18 06:16 09:46 09:08 10:24 15:28 11:02 12:56 14:50 16:06 17:42 20:22 22:11 01:49(+1)}}}
{\tnykdata{\anga{\tithi{10}{शुक्ल-दशमी}}{\time{24-33}{17:04}}\hspace{1ex}}%
{\anga{उत्तरप्रोष्ठपदा}{\time{5-25}{09:24}}\hspace{1ex}}{चन्द्रराशिः—\mbox{मीनः}}%
{\anga{व्यतीपातः}{\time{50-34}{03:28(+1)}}\hspace{1ex}\uanga{वरीयान्}}%
{\anga{गरः}{\time{24-33}{17:04}}\hspace{1ex}\anga{वणिजः}{\time{57-6}{06:05(+1)}}\hspace{1ex}\uanga{विष्टिः}}{}
}
{भृगुरेवती-पुण्यकालः~09:24\RIGHTarrow{}\eventsep व्यतीपात-श्राद्धम्}
{Fri} 
\cfoot{\rygdata{10:48--11:59}{14:22--15:33}{08:26--09:37}}
\caldata{DECEMBER}{7}{\sunmonth{वृश्चिकः}{22}{}{मार्गशीर्षः}{हेमन्तऋतुः}{शनिः}{विकारी}{दक्षिणायनम्}{शरदृतुः}}
{\sunmoonrsdata{07:15}{16:44}{14:06}{03:39(+1)}{12:00}
{\kalas{05:19 06:17 09:47 09:09 10:25 15:28 11:03 12:57 14:50 16:06 17:42 20:22 22:11 01:49(+1)}}}
{\tnykdata{\anga{\tithi{11}{शुक्ल-एकादशी}}{\time{29-19}{18:59}}\hspace{1ex}}%
{\anga{रेवती}{\time{11-39}{11:55}}\hspace{1ex}}{चन्द्रराशिः—\mbox{मीनः\RIGHTarrow{11:55}}}%
{\anga{वरीयान्}{\time{51-3}{03:41(+1)}}\hspace{1ex}\uanga{परिघः}}%
{\anga{विष्टिः}{\time{29-19}{18:59}}\hspace{1ex}\uanga{बवः}}{}
}
{गीता-जयन्ती\eventsep गुरुवायुपुर-एकादशी\eventsep कैशिक-एकादशी\eventsep सर्व-मोक्षदा-एकादशी}
{Sat} 
\cfoot{\rygdata{09:37--10:49}{13:11--14:22}{07:15--08:26}}
\caldata{DECEMBER}{8}{\sunmonth{वृश्चिकः}{23}{}{मार्गशीर्षः}{हेमन्तऋतुः}{भानुः}{विकारी}{दक्षिणायनम्}{शरदृतुः}}
{\sunmoonrsdata{07:16}{16:44}{14:35}{04:40(+1)}{12:00}
{\kalas{05:20 06:18 09:48 09:10 10:25 15:28 11:03 12:57 14:50 16:06 17:42 20:22 22:11 01:50(+1)}}}
{\tnykdata{\anga{\tithi{12}{शुक्ल-द्वादशी}}{\time{32-48}{20:24}}\hspace{1ex}}%
{\anga{अश्विनी}{\time{16-43}{13:57}}\hspace{1ex}}{चन्द्रराशिः—\mbox{मेषः}}%
{\anga{परिघः}{\time{50-31}{03:29(+1)}}\hspace{1ex}\uanga{शिवः}}%
{\anga{बवः}{\time{1-13}{07:45}}\hspace{1ex}\anga{बालवः}{\time{32-48}{20:24}}\hspace{1ex}\uanga{कौलवः}}{}
}
{हरिवासरः\RIGHTarrow{}01:23\eventsep \tamil{கார்த்திகை~ஞாயிற்றுக்கிழமை}\eventsep कैशिक-द्वादशी}
{Sun} 
\cfoot{\rygdata{15:33--16:44}{12:00--13:11}{14:22--15:33}}
\caldata{DECEMBER}{9}{\sunmonth{वृश्चिकः}{24}{}{मार्गशीर्षः}{हेमन्तऋतुः}{सोमः}{विकारी}{दक्षिणायनम्}{शरदृतुः}}
{\sunmoonrsdata{07:17}{16:44}{15:09}{05:42(+1)}{12:01}
{\kalas{05:20 06:19 09:48 09:10 10:26 15:29 11:04 12:57 14:51 16:06 17:42 20:22 22:12 01:50(+1)}}}
{\tnykdata{\anga{\tithi{13}{शुक्ल-त्रयोदशी}}{\time{34-51}{21:13}}\hspace{1ex}}%
{\anga{अपभरणी}{\time{20-26}{15:28}}\hspace{1ex}}{चन्द्रराशिः—\mbox{मेषः\RIGHTarrow{21:45}}}%
{\anga{शिवः}{\time{48-52}{02:50(+1)}}\hspace{1ex}\uanga{सिद्धः}}%
{\anga{कौलवः}{\time{3-59}{08:53}}\hspace{1ex}\anga{तैतिलः}{\time{34-51}{21:13}}\hspace{1ex}\uanga{गरः}}{}
}
{\tamil{பரணீ~தீபம்}\eventsep \tamil{கார்த்திகை}\eventsep कृत्तिका-व्रतम्\eventsep सोम-प्रदोष-व्रतम्~16:44\RIGHTarrow{}17:42\eventsep \tamil{திருவண்ணாமலை~தீபம்}}
{Mon} 
\cfoot{\rygdata{08:28--09:39}{10:50--12:01}{13:12--14:22}}
\caldata{DECEMBER}{10}{\sunmonth{वृश्चिकः}{25}{}{मार्गशीर्षः}{हेमन्तऋतुः}{मङ्गलः}{विकारी}{दक्षिणायनम्}{शरदृतुः}}
{\sunmoonrsdata{07:18}{16:44}{15:49}{06:44(+1)}{12:01}
{\kalas{05:21 06:20 09:49 09:11 10:27 15:29 11:04 12:58 14:51 16:06 17:43 20:23 22:12 01:51(+1)}}}
{\tnykdata{\anga{\tithi{14}{शुक्ल-चतुर्दशी}}{\time{35-27}{21:29}}\hspace{1ex}}%
{\anga{कृत्तिका}{\time{22-46}{16:24}}\hspace{1ex}}{चन्द्रराशिः—\mbox{वृषभः}}%
{\anga{सिद्धः}{\time{46-9}{01:45(+1)}}\hspace{1ex}\uanga{साध्यः}}%
{\anga{गरः}{\time{5-19}{09:25}}\hspace{1ex}\anga{वणिजः}{\time{35-27}{21:29}}\hspace{1ex}\uanga{विष्टिः}}{}
}
{\tamil{கணம்புல்ல நாயனார் (46) குருபூஜை}\eventsep पञ्च-पर्व-पूजा (पूर्णिमा)\eventsep \tamil{ஸர்வாலய~தீபம்}\eventsep \tamil{திருமங்கையாழ்வார் திருநக்ஷத்திரம்}}
{Tue} 
\cfoot{\rygdata{14:23--15:33}{09:39--10:50}{12:01--13:12}}
\caldata{DECEMBER}{11}{\sunmonth{वृश्चिकः}{26}{}{मार्गशीर्षः}{हेमन्तऋतुः}{बुधः}{विकारी}{दक्षिणायनम्}{शरदृतुः}}
{\sunmoonrsdata{07:19}{16:44}{16:35}{---}{12:02}
{\kalas{05:22 06:20 09:49 09:12 10:27 15:29 11:05 12:58 14:51 16:07 17:43 20:23 22:13 01:51(+1)}}}
{\tnykdata{\anga{\tithi{15}{पौर्णमासी}}{\time{34-42}{21:12}}\hspace{1ex}}%
{\anga{रोहिणी}{\time{23-47}{16:49}}\hspace{1ex}}{चन्द्रराशिः—\mbox{वृषभः\RIGHTarrow{04:51(+1)}}}%
{\anga{साध्यः}{\time{42-23}{00:16(+1)}}\hspace{1ex}\uanga{शुभः}}%
{\anga{विष्टिः}{\time{5-13}{09:24}}\hspace{1ex}\anga{बवः}{\time{34-42}{21:12}}\hspace{1ex}\uanga{बालवः}}{}
}
{आग्रयण-होमः द्राविडेषु\eventsep अन्नपूर्णा-जयन्ती\eventsep दत्तात्रेय-जयन्ती\eventsep मार्गशीर्ष-पूर्णिमा\eventsep पार्वणव्रतम् पूर्णिमायाम्\eventsep पूर्णिमा-व्रतम्\eventsep सर्प-बल्युत्सर्जनम्\eventsep त्रिपुर-भैरवी-जयन्ती\eventsep वेङ्कटाचले पूर्णिमा-गरुड-सेवा}
{Wed} 
\cfoot{\rygdata{12:02--13:12}{08:29--09:40}{10:51--12:02}}
\caldata{DECEMBER}{12}{\sunmonth{वृश्चिकः}{27}{}{मार्गशीर्षः}{हेमन्तऋतुः}{गुरुः}{विकारी}{दक्षिणायनम्}{शरदृतुः}}
{\sunmoonsrdata{07:19}{16:45}{17:29}{07:44}{12:02}
{\kalas{05:23 06:21 09:50 09:12 10:28 15:29 11:05 12:58 14:52 16:07 17:43 20:24 22:13 01:52(+1)}}}
{\tnykdata{\anga{\tithi{16}{कृष्ण-प्रथमा}}{\time{32-47}{20:26}}\hspace{1ex}}%
{\anga{मृगशीर्षम्}{\time{23-36}{16:46}}\hspace{1ex}}{चन्द्रराशिः—\mbox{मिथुनम्}}%
{\anga{शुभः}{\time{37-43}{22:25}}\hspace{1ex}\uanga{शुक्लः}}%
{\anga{बालवः}{\time{3-52}{08:52}}\hspace{1ex}\anga{कौलवः}{\time{32-47}{20:26}}\hspace{1ex}\uanga{तैतिलः}}{}
}
{काञ्ची १३ जगद्गुरु श्री-सच्चिद्घनेन्द्र सरस्वती आराधना~\#{१७४८}\eventsep पार्वण-प्रायश्चित्तावकाशः दर्शे\eventsep पूर्णमासेष्टिः\eventsep स्थालीपाकः}
{Thu} 
\cfoot{\rygdata{13:13--14:23}{07:19--08:30}{09:41--10:51}}
\caldata{DECEMBER}{13}{\sunmonth{वृश्चिकः}{28}{}{मार्गशीर्षः}{हेमन्तऋतुः}{शुक्रः}{विकारी}{दक्षिणायनम्}{शरदृतुः}}
{\sunmoonsrdata{07:20}{16:45}{18:30}{08:40}{12:02}
{\kalas{05:23 06:22 09:51 09:13 10:28 15:29 11:06 12:59 14:52 16:07 17:43 20:24 22:13 01:52(+1)}}}
{\tnykdata{\anga{\tithi{17}{कृष्ण-द्वितीया}}{\time{29-52}{19:17}}\hspace{1ex}}%
{\anga{आर्द्रा}{\time{22-24}{16:18}}\hspace{1ex}}{चन्द्रराशिः—\mbox{मिथुनम्}}%
{\anga{शुक्लः}{\time{32-17}{20:15}}\hspace{1ex}\uanga{ब्रह्म}}%
{\anga{तैतिलः}{\time{1-25}{07:54}}\hspace{1ex}\anga{गरः}{\time{29-52}{19:17}}\hspace{1ex}\anga{वणिजः}{\time{58-6}{06:34(+1)}}\hspace{1ex}\uanga{विष्टिः}}{}
}
{नारायणीयं-जयन्ती~\#{४३४}\eventsep परशुराम-जयन्ती}
{Fri} 
\cfoot{\rygdata{10:52--12:02}{14:24--15:34}{08:31--09:41}}
\caldata{DECEMBER}{14}{\sunmonth{वृश्चिकः}{29}{}{मार्गशीर्षः}{हेमन्तऋतुः}{शनिः}{विकारी}{दक्षिणायनम्}{शरदृतुः}}
{\sunmoonsrdata{07:21}{16:45}{19:36}{09:30}{12:03}
{\kalas{05:24 06:22 09:51 09:14 10:29 15:30 11:07 12:59 14:52 16:07 17:44 20:24 22:14 01:53(+1)}}}
{\tnykdata{\anga{\tithi{18}{कृष्ण-तृतीया}}{\time{26-8}{17:48}}\hspace{1ex}}%
{\anga{पुनर्वसुः}{\time{20-24}{15:31}}\hspace{1ex}}{चन्द्रराशिः—\mbox{मिथुनम्\RIGHTarrow{09:44}}}%
{\anga{ब्रह्म}{\time{26-13}{17:50}}\hspace{1ex}\uanga{इन्द्रः}}%
{\anga{विष्टिः}{\time{26-8}{17:48}}\hspace{1ex}\anga{बवः}{\time{54-2}{04:58(+1)}}\hspace{1ex}\uanga{बालवः}}{}
}
{आखुरथ-महागणपति सङ्कटहर-चतुर्थी-व्रतम्}
{Sat} 
\cfoot{\rygdata{09:42--10:52}{13:13--14:24}{07:21--08:31}}
\caldata{DECEMBER}{15}{\sunmonth{वृश्चिकः}{30}{\mbox{वृश्चिकः{\tiny\RIGHTarrow}{01:31(+1)}}}{मार्गशीर्षः}{हेमन्तऋतुः}{भानुः}{विकारी}{दक्षिणायनम्}{शरदृतुः}}
{\sunmoonsrdata{07:21}{16:45}{20:45}{10:14}{12:03}
{\kalas{05:25 06:23 09:52 09:14 10:29 15:30 11:07 13:00 14:53 16:08 17:44 20:25 22:14 01:53(+1)}}}
{\tnykdata{\anga{\tithi{19}{कृष्ण-चतुर्थी}}{\time{21-46}{16:04}}\hspace{1ex}}%
{\anga{पुष्यः}{\time{17-46}{14:28}}\hspace{1ex}}{चन्द्रराशिः—\mbox{कर्कटः}}%
{\anga{इन्द्रः}{\time{19-39}{15:13}}\hspace{1ex}\uanga{वैधृतिः}}%
{\anga{बालवः}{\time{21-46}{16:04}}\hspace{1ex}\anga{कौलवः}{\time{49-26}{03:08(+1)}}\hspace{1ex}\uanga{तैतिलः}}{}
}
{धनूरवि-सङ्क्रमण-षडशीति-पुण्यकालः~01:31(+1)\RIGHTarrow{}25:31(+1)\eventsep \tamil{கார்த்திகை~ஞாயிற்றுக்கிழமை}\eventsep रविपुष्ययोग-पुण्यकालः\RIGHTarrow{}14:28}
{Sun} 
\cfoot{\rygdata{15:35--16:45}{12:03--13:14}{14:24--15:35}}
\caldata{DECEMBER}{16}{\sunmonth{धनुः}{1}{}{मार्गशीर्षः}{हेमन्तऋतुः}{सोमः}{विकारी}{दक्षिणायनम्}{हेमन्तऋतुः}}
{\sunmoonsrdata{07:22}{16:46}{21:54}{10:53}{12:04}
{\kalas{05:25 06:24 09:52 09:15 10:30 15:30 11:08 13:00 14:53 16:08 17:44 20:25 22:15 01:54(+1)}}}
{\tnykdata{\anga{\tithi{20}{कृष्ण-पञ्चमी}}{\time{16-58}{14:09}}\hspace{1ex}}%
{\anga{आश्रेषा}{\time{14-41}{13:15}}\hspace{1ex}}{चन्द्रराशिः—\mbox{कर्कटः\RIGHTarrow{13:15}}}%
{\anga{वैधृतिः}{\time{12-42}{12:27}}\hspace{1ex}\uanga{विष्कम्भः}}%
{\anga{तैतिलः}{\time{16-58}{14:09}}\hspace{1ex}\anga{गरः}{\time{44-26}{01:09(+1)}}\hspace{1ex}\uanga{वणिजः}}{}
}
{वैधृति-श्राद्धम्}
{Mon} 
\cfoot{\rygdata{08:33--09:43}{10:53--12:04}{13:14--14:25}}
\caldata{DECEMBER}{17}{\sunmonth{धनुः}{2}{}{मार्गशीर्षः}{हेमन्तऋतुः}{मङ्गलः}{विकारी}{दक्षिणायनम्}{हेमन्तऋतुः}}
{\sunmoonsrdata{07:23}{16:46}{23:03}{11:27}{12:04}
{\kalas{05:26 06:24 09:53 09:15 10:31 15:31 11:08 13:01 14:53 16:08 17:45 20:25 22:15 01:54(+1)}}}
{\tnykdata{\anga{\tithi{21}{कृष्ण-षष्ठी}}{\time{11-51}{12:07}}\hspace{1ex}}%
{\anga{मघा}{\time{11-16}{11:53}}\hspace{1ex}}{चन्द्रराशिः—\mbox{सिंहः}}%
{\anga{विष्कम्भः}{\time{5-31}{09:35}}\hspace{1ex}\anga{प्रीतिः}{\time{58-11}{06:39(+1)}}\hspace{1ex}\uanga{आयुष्मान्}}%
{\anga{वणिजः}{\time{11-51}{12:07}}\hspace{1ex}\anga{विष्टिः}{\time{39-13}{23:04}}\hspace{1ex}\uanga{बवः}}{}
}
{मार्गशीर्ष-अष्टका-पूर्वेद्युः}
{Tue} 
\cfoot{\rygdata{14:25--15:36}{09:44--10:54}{12:04--13:15}}
\caldata{DECEMBER}{18}{\sunmonth{धनुः}{3}{}{मार्गशीर्षः}{हेमन्तऋतुः}{बुधः}{विकारी}{दक्षिणायनम्}{हेमन्तऋतुः}}
{\sunmoonsrdata{07:23}{16:46}{00:12(+1)}{11:59}{12:05}
{\kalas{05:26 06:25 09:54 09:16 10:31 15:31 11:09 13:01 14:54 16:09 17:45 20:26 22:15 01:55(+1)}}}
{\tnykdata{\anga{\tithi{22}{कृष्ण-सप्तमी}}{\time{6-33}{10:01}}\hspace{1ex}}%
{\anga{पूर्वफल्गुनी}{\time{7-41}{10:28}}\hspace{1ex}}{चन्द्रराशिः—\mbox{सिंहः\RIGHTarrow{16:06}}}%
{\anga{आयुष्मान्}{\time{50-47}{03:42(+1)}}\hspace{1ex}\uanga{सौभाग्यः}}%
{\anga{बवः}{\time{6-33}{10:01}}\hspace{1ex}\anga{बालवः}{\time{33-53}{20:57}}\hspace{1ex}\uanga{कौलवः}}{}
}
{आयुष्मद्-बव-सौम्य-संयॊगः\eventsep \tamil{இயற்பகை நாயனார் (2) குருபூஜை}\eventsep काञ्ची ४ जगद्गुरु श्री-सत्यबोधेन्द्र सरस्वती आराधना~\#{२२८७}\eventsep कुचेल-दिनम्\eventsep मार्गशीर्ष-अष्टका-श्राद्धम्\eventsep पञ्च-पर्व-पूजा (अष्टमी)}
{Wed} 
\cfoot{\rygdata{12:05--13:15}{08:34--09:44}{10:54--12:05}}
\caldata{DECEMBER}{19}{\sunmonth{धनुः}{4}{}{मार्गशीर्षः}{हेमन्तऋतुः}{गुरुः}{विकारी}{दक्षिणायनम्}{हेमन्तऋतुः}}
{\sunmoonsrdata{07:24}{16:47}{01:21(+1)}{12:31}{12:05}
{\kalas{05:27 06:25 09:54 09:16 10:32 15:32 11:09 13:02 14:54 16:09 17:45 20:26 22:16 01:55(+1)}}}
{\tnykdata{\anga{\tithi{23}{कृष्ण-अष्टमी}}{\time{1-12}{07:53}}\hspace{1ex}\anga{\tithi{24}{कृष्ण-नवमी}}{\time{55-56}{05:46(+1)}}\hspace{1ex}\avamA{}}%
{\anga{उत्तरफल्गुनी}{\time{4-3}{09:01}}\hspace{1ex}}{चन्द्रराशिः—\mbox{कन्या}}%
{\anga{सौभाग्यः}{\time{43-24}{00:46(+1)}}\hspace{1ex}\uanga{शोभनः}}%
{\anga{कौलवः}{\time{1-12}{07:53}}\hspace{1ex}\anga{तैतिलः}{\time{28-33}{18:49}}\hspace{1ex}\anga{गरः}{\time{55-56}{05:46(+1)}}\hspace{1ex}\uanga{वणिजः}}{}
}
{मार्गशीर्ष-अन्वष्टका-श्राद्धम्\eventsep श्री-शेषाद्रि-स्वामि-आराधना~\#{९१}}
{Thu} 
\cfoot{\rygdata{13:16--14:26}{07:24--08:34}{09:45--10:55}}
\caldata{DECEMBER}{20}{\sunmonth{धनुः}{5}{}{मार्गशीर्षः}{हेमन्तऋतुः}{शुक्रः}{विकारी}{दक्षिणायनम्}{हेमन्तऋतुः}}
{\sunmoonsrdata{07:24}{16:47}{02:30(+1)}{13:03}{12:06}
{\kalas{05:28 06:26 09:55 09:17 10:32 15:32 11:10 13:02 14:55 16:10 17:46 20:27 22:16 01:56(+1)}}}
{\tnykdata{\anga{\tithi{25}{कृष्ण-दशमी}}{\time{50-51}{03:45(+1)}}\hspace{1ex}}%
{\anga{हस्तः}{\time{0-29}{07:36}}\hspace{1ex}\anga{चित्रा}{\time{57-10}{06:16(+1)}}\hspace{1ex}}{चन्द्रराशिः—\mbox{कन्या\RIGHTarrow{18:56}}}%
{\anga{शोभनः}{\time{36-9}{21:52}}\hspace{1ex}\uanga{अतिगण्डः}}%
{\anga{वणिजः}{\time{23-20}{16:45}}\hspace{1ex}\anga{विष्टिः}{\time{50-51}{03:45(+1)}}\hspace{1ex}\uanga{बवः}}{}
}
{}
{Fri} 
\cfoot{\rygdata{10:56--12:06}{14:27--15:37}{08:35--09:45}}
\caldata{DECEMBER}{21}{\sunmonth{धनुः}{6}{}{मार्गशीर्षः}{हेमन्तऋतुः}{शनिः}{विकारी}{दक्षिणायनम्}{हेमन्तऋतुः}}
{\sunmoonsrdata{07:25}{16:48}{03:40(+1)}{13:37}{12:06}
{\kalas{05:28 06:27 09:55 09:18 10:33 15:33 11:10 13:03 14:55 16:10 17:46 20:27 22:17 01:56(+1)}}}
{\tnykdata{\anga{\tithi{26}{कृष्ण-एकादशी}}{\time{46-7}{01:52(+1)}}\hspace{1ex}}%
{\anga{स्वाती}{\time{54-10}{05:05(+1)}}\hspace{1ex}}{चन्द्रराशिः—\mbox{तुला}}%
{\anga{अतिगण्डः}{\time{29-9}{19:05}}\hspace{1ex}\uanga{सुकर्म}}%
{\anga{बवः}{\time{18-25}{14:47}}\hspace{1ex}\anga{बालवः}{\time{46-7}{01:52(+1)}}\hspace{1ex}\uanga{कौलवः}}{}
}
{\tamil{மானக்கஞ்சாற நாயனார் (11) குருபூஜை}\eventsep सहो-मासः/दक्षिणायनम्\RIGHTarrow{}20:19\eventsep सर्व-सफला-एकादशी\eventsep उत्तरायण-पुण्यकालः~20:19\RIGHTarrow{}04:19(+1)}
{Sat} 
\cfoot{\rygdata{09:46--10:56}{13:17--14:27}{07:25--08:35}}
\caldata{DECEMBER}{22}{\sunmonth{धनुः}{7}{}{मार्गशीर्षः}{हेमन्तऋतुः}{भानुः}{विकारी}{दक्षिणायनम्}{हेमन्तऋतुः}}
{\sunmoonsrdata{07:25}{16:48}{04:48(+1)}{14:14}{12:07}
{\kalas{05:29 06:27 09:56 09:18 10:33 15:33 11:11 13:03 14:56 16:11 17:47 20:28 22:17 01:57(+1)}}}
{\tnykdata{\anga{\tithi{27}{कृष्ण-द्वादशी}}{\time{41-55}{00:12(+1)}}\hspace{1ex}}%
{\anga{विशाखा}{\time{51-44}{04:07(+1)}}\hspace{1ex}}{चन्द्रराशिः—\mbox{तुला\RIGHTarrow{22:20}}}%
{\anga{सुकर्म}{\time{22-32}{16:27}}\hspace{1ex}\uanga{धृतिः}}%
{\anga{कौलवः}{\time{13-55}{13:00}}\hspace{1ex}\anga{तैतिलः}{\time{41-55}{00:12(+1)}}\hspace{1ex}\uanga{गरः}}{}
}
{गणितज्ञ-रामानुज-जन्म~\#{१३२}\eventsep हरिवासरः\RIGHTarrow{}07:25\eventsep काञ्ची ६८ जगद्गुरु श्री-चन्द्रशेखरेन्द्र सरस्वती ७ आराधना~\#{२६}\eventsep उत्तरायणारम्भः}
{Sun} 
\cfoot{\rygdata{15:38--16:48}{12:07--13:17}{14:28--15:38}}
\caldata{DECEMBER}{23}{\sunmonth{धनुः}{8}{}{मार्गशीर्षः}{हेमन्तऋतुः}{सोमः}{विकारी}{दक्षिणायनम्}{हेमन्तऋतुः}}
{\sunmoonsrdata{07:26}{16:49}{05:54(+1)}{14:57}{12:07}
{\kalas{05:29 06:27 09:56 09:19 10:34 15:34 11:11 13:04 14:56 16:11 17:47 20:28 22:18 01:57(+1)}}}
{\tnykdata{\anga{\tithi{28}{कृष्ण-त्रयोदशी}}{\time{38-25}{22:48}}\hspace{1ex}}%
{\anga{अनूराधा}{\time{50-1}{03:27(+1)}}\hspace{1ex}}{चन्द्रराशिः—\mbox{वृश्चिकः}}%
{\anga{धृतिः}{\time{16-26}{14:01}}\hspace{1ex}\uanga{शूलः}}%
{\anga{गरः}{\time{10-3}{11:28}}\hspace{1ex}\anga{वणिजः}{\time{38-25}{22:48}}\hspace{1ex}\uanga{विष्टिः}}{}
}
{मासशिवरात्रिः\eventsep सोम-प्रदोष-व्रतम्~16:49\RIGHTarrow{}17:47}
{Mon} 
\cfoot{\rygdata{08:36--09:47}{10:57--12:07}{13:18--14:28}}
\caldata{DECEMBER}{24}{\sunmonth{धनुः}{9}{}{मार्गशीर्षः}{हेमन्तऋतुः}{मङ्गलः}{विकारी}{दक्षिणायनम्}{हेमन्तऋतुः}}
{\sunmoonsrdata{07:26}{16:49}{06:55(+1)}{15:46}{12:08}
{\kalas{05:29 06:28 09:57 09:19 10:34 15:34 11:12 13:04 14:57 16:12 17:48 20:29 22:18 01:58(+1)}}}
{\tnykdata{\anga{\tithi{29}{कृष्ण-चतुर्दशी}}{\time{35-51}{21:47}}\hspace{1ex}}%
{\anga{ज्येष्ठा}{\time{49-15}{03:09(+1)}}\hspace{1ex}}{चन्द्रराशिः—\mbox{वृश्चिकः\RIGHTarrow{03:09(+1)}}}%
{\anga{शूलः}{\time{10-59}{11:50}}\hspace{1ex}\uanga{गण्डः}}%
{\anga{विष्टिः}{\time{7-0}{10:15}}\hspace{1ex}\anga{शकुनिः}{\time{35-51}{21:47}}\hspace{1ex}\uanga{चतुष्पात्}}{}
}
{कृष्णाङ्गारक-चतुर्दशी-पुण्यकालः/यमतर्पणम्\eventsep पञ्च-पर्व-पूजा (चतुर्दशी)\eventsep \tamil{தொண்டரடிப்பொடியாழ்வார் திருநக்ஷத்திரம்}}
{Tue} 
\cfoot{\rygdata{14:29--15:39}{09:47--10:58}{12:08--13:18}}
\caldata{DECEMBER}{25}{\sunmonth{धनुः}{10}{}{मार्गशीर्षः}{हेमन्तऋतुः}{बुधः}{विकारी}{दक्षिणायनम्}{हेमन्तऋतुः}}
{\sunmoonsrdata{07:27}{16:50}{---}{16:40}{12:08}
{\kalas{05:30 06:28 09:57 09:20 10:35 15:35 11:12 13:05 14:57 16:12 17:48 20:29 22:19 01:58(+1)}}}
{\tnykdata{\anga{\tithi{30}{अमावास्या}}{\time{34-24}{21:13}}\hspace{1ex}}%
{\anga{मूला}{\time{49-36}{03:18(+1)}}\hspace{1ex}}{चन्द्रराशिः—\mbox{धनुः}}%
{\anga{गण्डः}{\time{6-21}{10:00}}\hspace{1ex}\uanga{वृद्धिः}}%
{\anga{चतुष्पात्}{\time{4-58}{09:26}}\hspace{1ex}\anga{नाग}{\time{34-24}{21:13}}\hspace{1ex}\uanga{किंस्तुघ्नः}}{}
}
{काञ्ची १४ जगद्गुरु श्री-विद्याघनेन्द्र सरस्वती आराधना~\#{१७०३}\eventsep काञ्ची ३४ जगद्गुरु श्री-चन्द्रशेखरेन्द्र सरस्वती २ आराधना~\#{१३१०}\eventsep पार्वणव्रतम् अमावास्यायाम्\eventsep पञ्च-पर्व-पूजा (अमावास्या)\eventsep सर्व-मार्गशीर्ष-अमावास्या\eventsep श्री-हनूमत्-जयन्ती}
{Wed} 
\cfoot{\rygdata{12:08--13:19}{08:37--09:48}{10:58--12:08}}
\caldata{DECEMBER}{26}{\sunmonth{धनुः}{11}{}{पौषः}{हेमन्तऋतुः}{गुरुः}{विकारी}{दक्षिणायनम्}{हेमन्तऋतुः}}
{\sunmoonrsdata{07:27}{16:51}{07:49}{17:38}{12:09}
{\kalas{05:30 06:29 09:57 09:20 10:35 15:35 11:13 13:05 14:58 16:13 17:49 20:30 22:19 01:59(+1)}}}
{\tnykdata{\anga{\tithi{1}{शुक्ल-प्रथमा}}{\time{34-14}{21:09}}\hspace{1ex}}%
{\anga{पूर्वाषाढा}{\time{51-15}{03:57(+1)}}\hspace{1ex}}{चन्द्रराशिः—\mbox{धनुः}}%
{\anga{वृद्धिः}{\time{2-40}{08:31}}\hspace{1ex}\uanga{ध्रुवः}}%
{\anga{किंस्तुघ्नः}{\time{4-9}{09:07}}\hspace{1ex}\anga{बवः}{\time{34-14}{21:09}}\hspace{1ex}\uanga{बालवः}}{}
}
{\tamil{சாக்கிய நாயனார் (33) குருபூஜை}\eventsep दर्शेष्टिः\eventsep पार्वण-प्रायश्चित्तावकाशः पौर्णमास्याम्\eventsep स्थालीपाकः}
{Thu} 
\cfoot{\rygdata{13:19--14:30}{07:27--08:38}{09:48--10:58}}
\caldata{DECEMBER}{27}{\sunmonth{धनुः}{12}{}{पौषः}{हेमन्तऋतुः}{शुक्रः}{विकारी}{दक्षिणायनम्}{हेमन्तऋतुः}}
{\sunmoonrsdata{07:28}{16:51}{08:36}{18:39}{12:09}
{\kalas{05:31 06:29 09:58 09:20 10:35 15:36 11:13 13:06 14:58 16:14 17:50 20:30 22:20 01:59(+1)}}}
{\tnykdata{\anga{\tithi{2}{शुक्ल-द्वितीया}}{\time{35-30}{21:40}}\hspace{1ex}}%
{\anga{उत्तराषाढा}{\time{54-17}{05:11(+1)}}\hspace{1ex}}{चन्द्रराशिः—\mbox{धनुः\RIGHTarrow{10:13}}}%
{\anga{ध्रुवः}{\time{0-2}{07:28}}\hspace{1ex}\anga{व्याघातः}{\time{58-31}{06:52(+1)}}\hspace{1ex}\uanga{हर्षणः}}%
{\anga{बालवः}{\time{4-41}{09:20}}\hspace{1ex}\anga{कौलवः}{\time{35-30}{21:40}}\hspace{1ex}\uanga{तैतिलः}}{}
}
{चन्द्र-दर्शनम्~16:51\RIGHTarrow{}17:50}
{Fri} 
\cfoot{\rygdata{10:59--12:09}{14:30--15:41}{08:38--09:48}}
\caldata{DECEMBER}{28}{\sunmonth{धनुः}{13}{}{पौषः}{हेमन्तऋतुः}{शनिः}{विकारी}{दक्षिणायनम्}{हेमन्तऋतुः}}
{\sunmoonrsdata{07:28}{16:52}{09:16}{19:39}{12:10}
{\kalas{05:31 06:29 09:58 09:21 10:36 15:37 11:13 13:06 14:59 16:14 17:50 20:31 22:20 02:00(+1)}}}
{\tnykdata{\anga{\tithi{3}{शुक्ल-तृतीया}}{\time{38-13}{22:45}}\hspace{1ex}}%
{\anga{श्रवणः}{\time{58-42}{06:57(+1)}}\hspace{1ex}}{चन्द्रराशिः—\mbox{मकरः}}%
{\anga{हर्षणः}{\time{58-5}{06:42(+1)}}\hspace{1ex}\uanga{वज्रम्}}%
{\anga{तैतिलः}{\time{6-40}{10:08}}\hspace{1ex}\anga{गरः}{\time{38-13}{22:45}}\hspace{1ex}\uanga{वणिजः}}{}
}
{श्रवण-व्रतम्}
{Sat} 
\cfoot{\rygdata{09:49--10:59}{13:20--14:31}{07:28--08:38}}
\caldata{DECEMBER}{29}{\sunmonth{धनुः}{14}{}{पौषः}{हेमन्तऋतुः}{भानुः}{विकारी}{दक्षिणायनम्}{हेमन्तऋतुः}}
{\sunmoonrsdata{07:28}{16:53}{09:51}{20:39}{12:10}
{\kalas{05:31 06:30 09:59 09:21 10:36 15:37 11:14 13:07 15:00 16:15 17:51 20:32 22:21 02:00(+1)}}}
{\tnykdata{\anga{\tithi{4}{शुक्ल-चतुर्थी}}{\time{42-20}{00:24(+1)}}\hspace{1ex}}%
{\fullanga{श्रविष्ठा}}{चन्द्रराशिः—\mbox{मकरः\RIGHTarrow{20:02}}}%
{\anga{वज्रम्}{\time{58-39}{06:56(+1)}}\hspace{1ex}\uanga{सिद्धिः}}%
{\anga{वणिजः}{\time{10-6}{11:31}}\hspace{1ex}\anga{विष्टिः}{\time{42-20}{00:24(+1)}}\hspace{1ex}\uanga{बवः}}{}
}
{}
{Sun} 
\cfoot{\rygdata{15:42--16:53}{12:10--13:21}{14:31--15:42}}
\caldata{DECEMBER}{30}{\sunmonth{धनुः}{15}{}{पौषः}{हेमन्तऋतुः}{सोमः}{विकारी}{दक्षिणायनम्}{हेमन्तऋतुः}}
{\sunmoonrsdata{07:28}{16:53}{10:21}{21:38}{12:11}
{\kalas{05:32 06:30 09:59 09:21 10:37 15:38 11:14 13:07 15:00 16:16 17:52 20:32 22:22 02:00(+1)}}}
{\tnykdata{\anga{\tithi{5}{शुक्ल-पञ्चमी}}{\time{47-37}{02:31(+1)}}\hspace{1ex}}%
{\anga{श्रविष्ठा}{\time{4-24}{09:14}}\hspace{1ex}}{चन्द्रराशिः—\mbox{कुम्भः}}%
{\fullanga{सिद्धिः}}%
{\anga{बवः}{\time{14-51}{13:25}}\hspace{1ex}\anga{बालवः}{\time{47-37}{02:31(+1)}}\hspace{1ex}\uanga{कौलवः}}{}
}
{}
{Mon} 
\cfoot{\rygdata{08:39--09:50}{11:00--12:11}{13:21--14:32}}
\caldata{DECEMBER}{31}{\sunmonth{धनुः}{16}{}{पौषः}{हेमन्तऋतुः}{मङ्गलः}{विकारी}{दक्षिणायनम्}{हेमन्तऋतुः}}
{\sunmoonrsdata{07:29}{16:54}{10:48}{22:35}{12:11}
{\kalas{05:32 06:30 09:59 09:22 10:37 15:39 11:15 13:08 15:01 16:16 17:52 20:33 22:22 02:01(+1)}}}
{\tnykdata{\anga{\tithi{6}{शुक्ल-षष्ठी}}{\time{53-41}{04:57(+1)}}\hspace{1ex}}%
{\anga{शतभिषक्}{\time{11-6}{11:55}}\hspace{1ex}}{चन्द्रराशिः—\mbox{कुम्भः}}%
{\anga{सिद्धिः}{\time{0-2}{07:30}}\hspace{1ex}\uanga{व्यतीपातः}}%
{\anga{कौलवः}{\time{20-34}{15:43}}\hspace{1ex}\anga{तैतिलः}{\time{53-41}{04:57(+1)}}\hspace{1ex}\uanga{गरः}}{}
}
{षष्ठी-व्रतम्\eventsep महाधनुर्व्यतीपात-श्राद्धम्}
{Tue} 
\cfoot{\rygdata{14:33--15:43}{09:50--11:01}{12:11--13:22}}
\end{document}
